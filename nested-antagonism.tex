\subsection{Antagonism as a Nested (Fractal) Principle from Redox Metabolism to SMN}
\label{subsec:antagonism_nested}

\paragraph{Thesis.}
Across evolution and development, \emph{antagonism} recurs as a formal pattern for stabilizing and steering living systems under persistent environmental forces (chemical gradients, fluid drag, and gravity). The pattern is \emph{nested}: at each scale, coupled opponent processes---push--pull, agonist--antagonist, excitation--inhibition, approach--avoid---create controllable degrees of freedom and adaptive sensitivity (attunement). From redox metabolism and gene-regulatory patterning to biomechanics and neural control in the Sensation-/Sensorimotor-Modulating Network (SMN), the same motifs are reused with scale-appropriate substrates. Classic physiological regulation (homeostasis $\to$ allostasis) frames the role of opponent balances in prediction and control \citep{Cannon1932_WisdomBody,Sterling2012_Allostasis}.

\paragraph{Ontological substrate: redox opponency and ecological closure.}
We treat \emph{oxidation--reduction} as the deep ontological instance of biological antagonism, with several intermediary metabolites between the redox spectrum. Living systems couple internal electron flows to external redox opportunities, closing organism--habitat loops: electrons are abstracted from reduced substrates and ultimately delivered to environmental acceptors via chemiosmotic coupling \citep{GN2005_TowardsModelLifeCognition,Mitchell1961_Chemiosmotic,NichollsFerguson2013_Bioenergetics}. Reductive pressure (high NADH/NAD$^+$ or NADPH/NADP$^+$ ratios) and oxidative pressure (high NAD$^+$/NADH) form a push--pull control knob (\emph{redox poise}) that tunes flux through competing pathways (respiration/fermentation, anabolism/catabolism), analogous to later neural E/I balance.

\paragraph{A metabolic spectrum: aldehydes $\leftrightarrow$ ketones and canonical redox pairs.}
Core carbon metabolism itself exhibits an intrinsic opponent geometry across carbonyl states. At the triose node, \emph{glyceraldehyde-3-phosphate} (an aldehyde) and \emph{dihydroxyacetone phosphate} (a ketone) interconvert via triose-phosphate isomerase, straddling redox and rearrangement constraints; downstream, \emph{pyruvate} (an $\alpha$-ketoacid) sits at a redox fork whose opponent pairs (\emph{lactate}$\leftrightarrow$\emph{pyruvate}, \emph{ethanol}$\leftrightarrow$\emph{acetaldehyde}, \emph{$\beta$-hydroxybutyrate}$\leftrightarrow$\emph{acetoacetate}, \emph{malate}$\leftrightarrow$\emph{oxaloacetate}) buffer the NADH/NAD$^+$ couple and route flux by redox demand \citep{Berg2015_Biochemistry,Alberty2003_ThermoBiochem}. These conjugate pairs are biochemical realizations of a general \emph{opponent} control: when one half-cycle is driven (e.g., reduction of pyruvate to lactate), the conjugate half is correspondingly suppressed, achieving stabilized operation around a moving set point (allostatic redox attunement).

\paragraph{Ecological connection: redox ladders and habitat forces.}
At ecosystem scale, organisms inhabit a \emph{redox ladder} of electron donors/acceptors (e.g., O$_2$, nitrate, sulfate, CO$_2$), and metabolic strategies are distributed across this gradient \citep{Madigan2018_Brock,SternerElser2002_EcoStoichiometry}. Aquatic habitats impose buoyancy/drag constraints with variable O$_2$ availability; terrestrial niches impose gravitational support with abundant O$_2$. Across these transitions, the same redox opponency re-parameterizes internal control (e.g., switching between fermentation and respiration) while body-plan antagonisms (agonist--antagonist musculature) tune impedance against the dominant physical load. Thus, chemical antagonism closes the organism--habitat circuit that higher-layer antagonisms exploit.

\paragraph{A. Metabolic \& evo--devo opponency: carving axes and fields.}
At the deepest developmental layer, reaction--diffusion and gene-regulatory antagonism generate spatial order via activator--inhibitor coupling and mutual repression. Turing instabilities and the Gierer--Meinhardt scheme formalize short-range activation with long-range inhibition \citep{Turing1952_Morphogenesis,GiererMeinhardt1972_Pattern}. Concrete systems instantiate these logics: BMP--Chordin for dorsal--ventral patterning and neural induction, Nodal--Lefty for axis formation, and Notch--Delta for lateral inhibition \citep{DeRobertisKuroda2004_DVPatterning,Muller2012_NodalLefty,Collier1996_LateralInhibition}. Bistable switching, a canonical outcome of mutual antagonism, provides decisive state control \citep{FerrellMachleder1998_Bistability,Gardner2000_ToggleSwitch}.

\paragraph{B. Anatomy \& biomechanics: agonist--antagonist embodiment.}
In bilaterians, muscles can contract but not push, making antagonistic pairs the minimal mechanism for reversible control and compliance. Antagonism enables impedance regulation and stability against habitat-specific forces (drag in aquatic media; weight support and impact mitigation on land). Control-theoretic accounts model co-activation of opponents to tune stiffness and to shift equilibrium postures \citep{Hogan1984_ImpedanceControl,Feldman1986_EquilibriumPoint}. Physiological regulation remains opponent-balanced in service of homeostatic/allostatic goals \citep{Cannon1932_WisdomBody,Sterling2012_Allostasis}.

\paragraph{C. Neural \& SMN motifs: reciprocal inhibition and zonal routing.}
Segmental central pattern generators (CPGs) implement alternating activation of antagonistic muscles via reciprocal inhibition---the neural expression of mechanical opponency \citep{Sherrington1906_IntegrativeAction,Brown1911_Stepping,Grillner2006_LocomotorNetworks}. At supra-segmental levels, action selection follows an \emph{affordance competition} logic: concurrently represented actions inhibit competitors until a winner emerges \citep{Cisek2007_AffordanceCompetition,CisekKalaska2010_DecisionsAction}. Across SMN zones (fast excitable vs.\ slow viscoelastic or diffuse substrates), opponent couplings route, gate, and stabilize multi-timescale action patterns, extending the push--pull motif from reflex arcs to whole-body control. Balanced excitation--inhibition (E/I) and inhibitory plasticity provide network-level self-tuning of opponency for stable, efficient codes \citep{Vogels2011_InhibitoryPlasticity,DeneveMachens2016_EfficientCodes}.

\paragraph{D. Perception \& valuation: opponent codes for sensitivity.}
Opponent processing recurs in perception and affect: color vision (red--green, blue--yellow) and direction-selective motion employ antagonistic channels that maintain sensitivity around operating points \citep{Hering1964_ColorOpponency,Reichardt1957_MotionDetector}. In motivation and affect, the opponent-process theory explains after-effects and adaptation as countervailing dynamics that restore balance \citep{SolomonCorbit1974_OpponentProcess}. These codes mirror E/I balance and competition in action selection, linking sensing and acting through shared antagonistic design.

\paragraph{E. Formal dynamical motif: why the theme is ``fractal.''}
Mutual inhibition, activator--inhibitor diffusion, winner-take-all competition, and coupled oscillators are generic circuit logics that recur across scales. They yield bistability, alternation, and regulated variability---precisely the behaviors needed for reversible control and robust attunement. Coordination dynamics shows how anti-phase vs.\ in-phase patterns (antagonistic coordinations) emerge and switch with parameter changes \citep{Kelso1995_DynamicPatterns}. Physiological time series often display scale-free (fractal) fluctuations that degrade with disease and aging, consistent with nested, self-similar control \citep{Goldberger2002_FracPhysiology}; cross-scale transport and branching constraints further support fractal-like organization \citep{West1997_ScalingBiology}. Thus, the ``fractal'' label is not merely metaphorical: it captures the reuse of opponent couplings, with substrate-specific realizations, from redox to movement to minds.

\paragraph{Implication for SMN.}
The SMN's multi-zonal action architecture continues this pattern: each zone provides an opponent-capable substrate (fast/slow; excitable/viscoelastic), and cross-zone couplings implement adaptive impedance, gating, and competition. Antagonism is therefore the \emph{formal bridge} by which bodies attune to chemical, aquatic, and terrestrial constraints while enabling flexible control across layers.

%  A figure placeholder we can replace later.
\begin{figure}[t]
  \centering
   \fbox{\begin{minipage}{0.92\linewidth}
   \small \textbf{Sketch:} Nested antagonism across layers (chemistry $\to$ tissue $\to$ CPG $\to$ cortical competition), showing repeated mutual-inhibition/activation--inhibition motifs and their habitat couplings (drag/gravity).
   \end{minipage}}
   \caption{Schematic of the nested (fractal) antagonism motif from metabolism to SMN.}
   \label{fig:nested_antagonism}
\end{figure}


\paragraph{Homeostasis as evaluation: Damasio's bridge from regulation to cognition.}
Damasio’s central claim is that \emph{homeostasis is not merely physiology but an evaluative imperative}: organisms continuously appraise internal states relative to viability bounds, and this appraisal is expressed as \emph{valence-laden feeling} (the mental aspect of homeostatic regulation). On this view, \emph{feelings are the subjective readout of homeostatic (and allostatic) control}, integrating interoceptive signals with situational contingencies to bias action selection \citep{Damasio1994_DescartesError,Damasio1999_Feeling,Damasio2010_Self,Damasio2018_StrangeOrder}. The \emph{somatic marker} framework makes this concrete: body-based signals (viscero-motor/interoceptive patterns) tag options with positive/negative value and guide decisions under uncertainty \citep{Damasio1994_DescartesError,Bechara1997_SomaticMarkers}. 

This dovetails with our \emph{nested antagonism} thesis. At every layer, opponent dynamics realize evaluative control around viability set-points: 
(i) \textbf{Redox antagonism} (NADH/NAD$^+$; aldehyde$\leftrightarrow$ketone pairs) sets biochemical ``good/bad'' directions for flux; 
(ii) \textbf{Biomechanical antagonists} (agonist--antagonist pairs) tune impedance to keep the body within safe interaction envelopes; 
(iii) \textbf{Neural opponency} (E/I balance; reciprocal inhibition; affordance competition) implements value-guided selection pressures; 
(iv) \textbf{Affective opponency} provides felt valence that summarizes the organism’s distance to/allocation within viability ranges. 
Thus, Damasio’s account grounds the \emph{cognitive} role of antagonism: opponent pairs are not only control mechanisms but also \emph{valuation mechanisms}, whose composite outputs (feelings) inform SMN policy at multiple timescales. In short, \emph{antagonism $\to$ appraisal $\to$ action} is the repeating, ``fractal'' schema linking metabolism, body, brain, and mind \citep{Damasio2018_StrangeOrder,Damasio2021_FeelingKnowing}.

\subsubsection{Categorization as Evaluative Attunement (Dissolving the Fact/Value Split)}
\label{subsec:categorization_evaluative}

\paragraph{Claim.}
The \emph{basic cognitive act of categorization is not separable from evaluation}. Organisms do not encounter a value-neutral world and then add valence; rather, they \emph{taste} the world—sensing it through homeostatic/allostatic appraisal that partitions experience into action-relevant kinds. On this view, the received fact/value dichotomy obscures the operational reality of cognition: categories are \emph{evidence‐and‐loss} dependent partitions that track what sustains or threatens viability.

\paragraph{Ecology first: affordances are already valued kinds.}
In an ecological stance, perception is directly of \emph{affordances}—opportunities for action relative to bodily capacities \citep{Gibson1979_Ecological}. Affordances are intrinsically valenced (edible/inedible, graspable/ungraspable). Hence the first categorical carve is \emph{evaluative}: worlds show up as gradients of suitability rather than neutral properties.

\paragraph{Homeostatic appraisal as the engine of category formation.}
If feelings are the mental face of homeostatic regulation \citep{Damasio1999_Feeling,Damasio2018_StrangeOrder}, then category learning recruits \emph{somatic markers} to tag options with value \citep{Damasio1994_DescartesError,Bechara1997_SomaticMarkers}. This ties our layered antagonism (redox $\to$ biomechanical $\to$ E/I opponency) to concept formation: evaluative signals bias which sensory regularities become stable kinds for an agent.

\paragraph{Decision‐theoretic backbone: no categories without losses.}
In statistical decision theory and signal detection, classification requires a \emph{loss/utility} (or cost) structure; the optimal boundary depends on both evidence and stakes \citep{Wald1950_StatisticalDecision,GreenSwets1966_SDT}. Empirically, human perceptual categories shift with payoffs and priors \citep{KnillPouget2004_BayesianBrain,GoldShadlen2007_DecisionNeuro}. Thus, classification is evaluation-laden by construction.

\paragraph{Active inference: preferences as priors over outcomes.}
Under the free-energy/active‐inference account, organisms act to minimize expected surprise relative to \emph{prior preferences} (viability ranges); perceptual categorization is tuned to those preferences and to policies that realize them \citep{Friston2010_FreeEnergy}. Categories are therefore \emph{policies’ partitions of the world} under homeostatic value.

\paragraph{Concepts as situated, graded, and task-bound.}
Grounded cognition shows that concepts and categories are built from sensorimotor/affective systems and remain \emph{graded} and \emph{task contingent}, not crisp taxonomies detached from use \citep{Barsalou2008_Grounded}. The familiar classificatory-naming paradigm shadows this mechanism by reifying linguistic labels over evaluative attunements.

\paragraph{UpShot.}
Cognition without evaluation lacks epistemic value because it cannot guide adaptive action. Fact and value are not two independent domains to be later bridged; rather, \emph{what counts as a fact for an organism is already shaped by value}. This dissolves the dichotomy in practice \citep{Putnam2002_Collapse}, and aligns with our SMN thesis: nested antagonisms implement the evaluative gradients that make categories actionable.


\subsection{Evaluation-first categorization: dissolving the fact/value split}
\label{subsec:evaluation_first}

\paragraph{Claim.}
We propose that the \emph{basic cognitive act of categorization is impossible without evaluation}. Organisms do not first receive neutral “facts” and then add value; rather, they \emph{encounter} the world as gradients of \emph{better/worse for viability}, sampled through multi-layered opponencies (redox poise, agonist--antagonist mechanics, neural E/I balance) that constitute an \emph{evaluative spectrum}. Categories are thus carved by \emph{attunement} to these spectra, not by value-free partitions. This stance places the received \emph{fact/value} dichotomy under strain \citep{Putnam2002_FactValue,Dewey1939_TheoryValuation}.

\paragraph{From antagonism to appraisal to kind-making.}
Nested antagonisms (redox, biomechanical, neural, affective) compute \emph{directional information} relative to viability bounds (\S\ref{subsec:antagonism_nested}). In Damasio’s terms, homeostatic/allostatic regulation \emph{is} evaluation, with feelings serving as the subjective readout that biases selection among actions \citep{Damasio1999_Feeling,Damasio2018_StrangeOrder}. On ecological grounds, the organism perceives \emph{affordances}—opportunities and threats—rather than value-neutral objects \citep{Gibson1979_Ecological}. Categories (edible, perchable, avoidable; safe/unsafe grasp) emerge as \emph{use-oriented kinds} stabilized by recurrent evaluative couplings in the SMN.

\paragraph{Spectra, not binaries.}
Operationally, many physical dimensions arrive as continuous evaluative axes (nutritional payoff, mechanical yield, friction, toxicity, thermal comfort). Classical psychophysiology already encodes this: in signal detection theory, the decision criterion is set by utilities/payoffs, not ‘‘pure’’ sensation \citep{GreenSwets1966_SDT}. In semantic cognition, Osgood’s \emph{evaluation–potency–activity} factors reveal an irreducible evaluative dimension even in linguistic meaning \citep{Osgood1957_MeasurementMeaning}. Olfaction and taste make this vivid: perceptual spaces are organized around valence/pleasantness gradients, with metabolic and ecological grounding \citep{Barwich2020_Smellosophy}.

\paragraph{Value-based learning sculpts representational kinds.}
Reinforcement learning ties categorization to outcome value: task representations and state abstractions compress the world along \emph{expected utility} structure \citep{Niv2009_RLReview}. Orbitofrontal cortex maintains a \emph{cognitive map of task space} that organizes latent categories by current goals and values \citep{Wilson2014_OFCMap}, aligning with affordance competition in action selection \citep{Cisek2007_AffordanceCompetition}. Teleosemantic accounts generalize this to content: biological ‘‘proper functions’’ couple categories to success conditions \citep{Millikan1984_LanguageThought}.

\paragraph{Conclusion.}
If cognition without evaluation cannot guide action, it has no \emph{epistemic} value for organisms like us. On our view, \emph{antagonism $\to$ appraisal $\to$ categorization} is the repeating schema: evaluative opponencies produce gradients; gradients induce partitions; partitions become names. The classificatory naming paradigm overlays, but does not originate, the underlying evaluative mechanism.

\subsection{Dialectical systems, inversion heuristics, and why antagonism scales}
\label{subsec:dialectical_inversion}

\paragraph{Position.}
A long systems tradition converges with dialectical ontology on a common thesis: \emph{organized wholes are coherent compositions of opposites}. Antagonistic couplings (activation--inhibition, agonist--antagonist, excitation--inhibition, approach--avoid) are not contradictions to be eliminated by classical logic; they are \emph{structural duals} whose joint dynamics generate order in open, far-from-equilibrium systems. The “parallelogram law of forces” offers a physical metaphor: opposed vectors do not reduce to inconsistency; they \emph{compose} into resultant trajectories in state space. In living systems, that resultant is the \emph{attuned} course of action.

\paragraph{Dialectical systems perspective.}
From general system theory and cybernetics to nonequilibrium thermodynamics, scholars model regulation as balanced opposition: negative feedback stabilizes deviation; requisite variety matches disturbances; dissipative structures harness flux; ecological couplings close organism--environment loops \citep{vonBertalanffy1968_GST,Ashby1956_Cybernetics,PrigogineStengers1984_OrderChaos,Bateson1979_MindNature,LevinsLewontin1985_DialecticalBiologist}. On this view, the antagonistic motifs we traced (redox, biomechanical, neural, affective) instantiate a \emph{dialectical composition}: each pole shapes and is shaped by its other through circular causality.

\paragraph{Opposition without contradiction.}
We do not claim that admissible belief sets tolerate $p$ and $\neg p$. Rather, \emph{physical} and \emph{biological} models make opposites co-real: dual variables and conjugate forces (e.g., pressure--volume, temperature--entropy; velocity--momentum) are coupled by constitutive laws. Vector addition, impedance matching, and E/I balance show how “opposites” are \emph{coordinated} to yield stable, purposive behavior without logical inconsistency.

\paragraph{Inversion as a heuristic principle.}
Building on our earlier thesis \citep{Nagarjuna2004_LogicInversion}, scientific and mathematical practice repeatedly exploit \emph{inversion} to construct models: Fourier and Laplace transforms (time/frequency duals), Legendre transforms (energy/entropy duals), control-theoretic inverse models, and geometric/algebraic dualities. In biology, redox couples (NADH/NAD$^+$), mutually inhibitory gene circuits, and agonist--antagonist muscles are \emph{natural inversions}. Inversion is thus a design heuristic for discovering the “other side” of a process that, when coupled back, closes the control loop.

\paragraph{From dialectic to attunement across knowledge regimes.}
Stochastic and schematic cognition (\S\ref{subsec:evaluation_first}) relate as regimes connected by \emph{phase transitions} in coordination dynamics: under task and resource constraints, systems reorganize their effective degrees of freedom (e.g., exploration $\to$ rule-governed composition) \citep{Haken1983_Synergetics,Kelso1995_DynamicPatterns}. The possibility of \emph{attunement}—assimilating and accommodating the world—arises because antagonistic/inversion pairs are preserved across layers: redox $\leftrightarrow$ flux control, muscle $\leftrightarrow$ load, E/I $\leftrightarrow$ code stability, affect $\leftrightarrow$ valuation. These duals ground both evaluation and category formation: \emph{antagonism $\to$ appraisal $\to$ categorization} recurs as a dialectical schema that is physically realizable and cognitively normative.

\paragraph{Implication for the SMN.}
The SMN is a dialectical engine: multi-zonal opponent substrates (fast/slow, excitable/viscoelastic, local/diffuse) implement inversion pairs whose compositions yield stable yet reconfigurable actions. This explains how agents remain coupled to habitat constraints while navigating between stochastic exploration and schematic construction.


\subsection{SMN as forward-model architecture and predictive engine}
\label{subsec:smn_forward_predictive}

\paragraph{Bridge.}
The SMN’s multi-zonal, antagonistic control can be read as a \emph{forward-model stack} embedded in a \emph{predictive-processing} loop. Each zone carries a substrate-specific internal model (fast excitable trajectories; slower viscoelastic/mechanical envelopes; diffuse metabolic constraints), while cross-zone couplings supply efference copies, sensory predictions, and precision control. Antagonism (agonist--antagonist; excitation--inhibition) implements the push--pull computations that both \emph{predict} and \emph{explain away} incoming signals.

\paragraph{Efference copy and internal models.}
Classical corollary-discharge/efference-copy mechanisms \citep{vonHolst1950_Reafference,Sperry1950_Corollary} already realize a minimal forward model: motor outflow predicts reafferent consequences and subtracts them from sensation. Modern internal-model theory refines this into paired \emph{forward} and \emph{inverse} models \citep{JordanRumelhart1992_ForwardInverse,WolpertKawato1998_MultipleModels}. In the SMN, antagonistic muscle pairs and reciprocal inhibition provide the plant/circuitry in which forward predictions (e.g., expected spindle and cutaneous signals) attenuate predictable input and expose task-relevant residuals (prediction errors).

\paragraph{Predictive coding and precision (gain) control.}
Predictive-coding formulations treat perception and action as hierarchical inference: descending predictions meet ascending errors, weighted by \emph{precision} (inverse variance) \citep{RaoBallard1999_PredictiveCoding,Friston2005_HierarchicalInference,Friston2010_FreeEnergy}. SMN antagonism supplies a biophysically grounded mechanism for precision control: co-activation of opponents (and E/I balance) tunes impedance and neural gain, thereby modulating the impact of error on belief and policy updates. Viscoelastic zones function as low-pass priors; fast excitable zones implement error units; neuromodulators adjust precision weights (attention) over zones.

\paragraph{Action as inference (active inference / optimal control).}
On active-inference views, organisms minimize expected free energy by either changing internal states (perception) or acting to make sensations conform to predictions \citep{AdamsShippFriston2013_ActiveInference,Friston2011_Agent}. SMN affordance-routing provides the actuator side of this loop: competing action policies (affordance competition) inhibit one another until a precision-weighted winner emerges \citep{Cisek2007_AffordanceCompetition}. This aligns with optimal/adaptive control accounts of motor behavior where internal models and impedance control stabilize interaction with uncertain loads \citep{TodorovJordan2002_OCShaping,Hogan1984_ImpedanceControl,ShadmehrMussaIvaldi1994_InternalModels}.

\paragraph{Bayesian fusion, learning, and zones as priors.}
Empirically, the brain integrates cues in a near-Bayesian fashion \citep{ErnstBanks2002_Bayesian,KordingWolpert2004_Bayesian,KnillPouget2004_Bayes}. In SMN terms, slowly adapting zones (mechanical compliance, posture fields) instantiate structural priors; faster zones carry likelihoods; antagonistic co-contraction encodes \emph{confidence} (precision). Learning reshapes both: forward-model calibration shifts with error-driven plasticity and changes in zone couplings.

\paragraph{Added value of the SMN view.}
(i) \emph{Material grounding:} internal models are not just neural—they are partly realized in body mechanics (impedance, morphology) and metabolic redox constraints that shape feasible predictions. (ii) \emph{Antagonism as computation:} opponent couplings implement both subtraction (prediction error) and gain control (precision). (iii) \emph{Affordance integration:} predictive hierarchies are routed through multi-zonal action substrates, making “what to predict” and “what to do” jointly resolved.
