
\subsection{Polarity as the First Cognitive Coordinate System}
\label{subsec:polarity}

\marginpar{Polarity furnishes a \emph{coordinate frame} for control.}
We take \emph{polarity} to mean oriented asymmetries at multiple scales: cellular (apico--basal and planar cell polarity), tissue/organ polarity, and whole-body axes (anterior--posterior, dorsal--ventral, and left--right).
On the evo--devo path, polarity is among the earliest and most conserved specifications of the body plan, providing the geometric scaffolding upon which segmentation, antagonistic actuation, and bilateral comparison later operate.
In the SMN, polarity is not merely anatomical: it is the \emph{computational resource} that gives actions a direction for prediction, comparison, and interruption.

\paragraph{Developmental grounding.}
Classically, polarity is formalized through \emph{positional information}: graded cues and boundary conditions that endow cells with interpretable coordinates \cite{Wolpert1969PositionalInformation}.
Axial patterning (e.g., anterior--posterior via Hox codes) further refines these coordinates into modular programs \cite{McGinnisKrumlauf1992Hox}.
Dorsal--ventral polarity is implemented by conserved BMP/Chordin antagonism across bilaterians \cite{DeRobertisSasai1996CommonPlan}, while \emph{planar cell polarity} (PCP) aligns subcellular structures across epithelia to produce coherent tissue-scale orientation \cite{GoodrichStrutt2011PCP}.
These mechanisms show that polarity is multi-level, redundant, and robust---properties that the SMN exploits for resilient control.

\paragraph{Bioelectric implementations and pattern memory.}
Endogenous bioelectric networks provide a \emph{physiological encoding} of polarity and target morphology.
Voltage gradients and gap-junction networks can bias axis specification and organ identity, and---crucially---they support \emph{multi-stable pattern memories} that persist across morphological change \cite{Levin2012MorphogeneticFields,Levin2014MolecularBioelectricity}.
This non-neural, distributed substrate directly grounds the SMN's claim that control states need not be stored as data in a central location; they can be \emph{maintained as attractors} in tissue-level dynamics.

\paragraph{From polarity to control.}
For the SMN, polarity anchors three control advantages:
\begin{enumerate}
    \item \textbf{Predictive orientation:} An oriented axis makes future sensory flow and mechanical contingencies more predictable along canonical directions (forward/back, dorsal/ventral).
    \item \textbf{Antagonistic haltability:} Opposed effectors aligned to an axis (flexor/extensor; ad-/abductor) implement \emph{interruptibility} by design; halting is a switch along the same axis.
    \item \textbf{Comparative inference:} Left--right polarity combined with bilateral symmetry enables \emph{counter-variation} and differencing---the computational basis of comparison and error signals.
\end{enumerate}
\marginpar{Interruptibility is \emph{easier} when opposition is built-in.}

\paragraph{Coupling to habitat: gravity and fluid.}
Polarity is co-constructed with the medium.
Gravity supplies a constant vector field that organisms internalize via vestibular and proprioceptive systems \cite{AngelakiCullen2008VestibularMultisensory}, turning dorsal--ventral and head--tail distinctions into control priors for posture, balance, and locomotion.
In aquatic habitats, viscosity and buoyancy make oscillatory control stable and efficient; oriented bodies leverage wake interactions and boundary layers to \emph{predict} flow and economize effort \cite{Vogel1994LifeInMovingFluids,Alexander2003PrinciplesLocomotion}.
Thus, the habitat directly participates in computation by reducing uncertainty along polarized axes.

\paragraph{Thermodynamic economy.}
Because polarity supplies reliable directions and comparisons, the SMN can store \emph{action patterns} rather than large internal data structures.
Under Landauer's bound, logically irreversible updates (policy resets, overwrites) dissipate heat; by coupling to stable axes (gravity) and medium regularities (fluid flow), organisms \emph{lower the frequency of costly resets} while retaining flexible control \cite{Landauer1961Irreversibility,Bennett2003LandauerNotes,StillEtAl2012ThermoPrediction}.
\marginpar{Action patterns index lawful structure in the world.}

\todo{Figure suggestion: a tri-axial schematic (A--P, D--V, L--R) overlaid with (i) antagonistic pairs, (ii) vestibular gravity vector, (iii) fluid streamlines indicating predictable sensorimotor contingencies. Label links to \cref{subsec:habitat} and \cref{subsec:bio-arch}.}

\paragraph{Cognitive function of polarity (tokens and vectors).}
Polarity converts undifferentiated space into a usable \emph{code}: oriented axes act as \emph{vectors} for prediction and control, while antagonistic actuation and gated halts yield discrete \emph{tokens}—countable, sequenceable units of action. In this way, polarity provides the SMN with the minimal representational kit for spatial decoding: vectors to project, tokens to compose. Subsequent sections (tubes, segmentation, bilateral comparison) elaborate how this kit scales into navigable, interruptible programs in complex habitats.


\paragraph{SMN vs Habitat (capsule).}
\textbf{SMN:} polarity orients local controllers and antagonistic pairs, enabling interruptible actions along canonical axes.
\textbf{Habitat:} gravity provides a constant vector; fluids provide directionally predictable flows—together supplying priors/vectors that the SMN exploits to decode space with minimal internal storage.
