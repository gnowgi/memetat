\section{Explaining Cognitive Phenomena and Reinterpreting Evidence}
\label{sec:phenomena}
The explanatory power of the Sensation-Modulating Network (SMN) lies in its capacity to reframe fundamental cognitive questions in terms of action dynamics. By grounding cognition in a specific, yet generalizable, bodily architecture, the model offers a unified account of phenomenology, representation, and syntax, while reinterpreting existing empirical evidence.

\subsection{From Action Modulation to Phenomenology}
\label{subsec:phenomenology}
The SMN model addresses the "hard problem" of consciousness by proposing that subjective experience is the agent's perception of its own modulated actions. The constant, rhythmic hum of the Fixed Action Patterns (FAPs) creates a stable, non-conscious background—a "cognitive canvas." A phenomenological event occurs when a Haltable Action Pattern (HAP) is initiated, modulated, or, most critically, paused. This change against the stable background *is* the experience. The "what it is like" to see red is the specific pattern of oculomotor and neural HAPs enacted to foveate on a red object; the feeling of thirst is the modulation of interoceptive patterns against the homeostatic background. Consciousness is not a substance or a passive "theater" but an active process of self-differentiation through action.

\subsubsection{Action as the Origin of Phenomenal Experience}
\label{ssubsec:action_origin}
In the Sensation-Modulating Network (SMN) framework, a phenomenological experience is not a passive event, such as the mere reception of sensory stimuli. Instead, it is an active construct, a direct consequence of the agent's own self-initiated actions. The process begins when the agent deploys a Haltable Action Pattern (HAP)—a voluntary, modulated action directed towards an affordance in the environment. This action, be it a saccade of the eye, a turn of the head, or the extension of a hand, is not a reaction to a stimulus but is itself the stimulus that generates the experience.

When the agent acts, it necessarily alters the flow of sensory information across its entire body. The HAP actively modulates this sensory stream, creating a specific, transient pattern of differentiation against the stable, homeostatic background provided by the Fixed Action Patterns (FAPs). The "phenomenological response" is precisely this internally-generated, action-driven pattern. For example, the experience of touching a rough surface is not caused by the surface's texture alone; it is constructed by the specific HAP of moving one's fingers across it, which generates a unique pattern of vibrations and pressures. The action and the sensation are inextricable.

Therefore, the agent's world is not something that is revealed to it, but something it brings forth through its own activity. Each self-initiated action is a question posed to the environment, and the resulting modulation of the agent's own sensory state is the answer. The senses provide a continuous, undifferentiated stream of differences; it is the motor act that breaks this stream into discrete "snapshots," thereby differentiating the difference in a kind of double derivative. Our mental world is this stream of action-generated snapshots, a process that ceases the moment we stop acting. This continuous loop of action-based self-stimulation is the very engine of conscious experience, grounding phenomenology not in a mysterious inner theater, but in the concrete, dynamic process of a body actively engaging with its world.

\subsection{Constructing a Geometric and Semiotic World}
\label{subsec:semiotic_world}
The agent's world is not a pre-given, objective space that is passively perceived, but a geometric and semiotic habitat that is constructed through action. The SMN computes this geometry through the coordination of its multiple action zones. For instance, the distance to an object is not calculated from retinal size alone, but is enacted through a fusion of HAPs: the proprioceptive feel of reaching a hand, the muscular strain of focusing the eyes, and the time delay between a sound and its echo. The world's geometry is mapped onto the agent's own bodily geometry and action capabilities.

Meaning arises as this geometric world becomes populated with signs. An object's affordances are the saturated HAPs it invites. A rock affords throwing, sitting, or striking. These saturated HAPs are the object's initial meaning for the agent. When these actions are tokenized as unsaturated HAPs (USHAPs), they become concepts that can be manipulated internally. When they are shared and imitated as Transactional Action Patterns (TAPs), they become external, public symbols. This provides a direct, embodied route from perception to semiotics, grounding the Peircean triad of object, representamen (the action pattern), and interpretant (the resulting experience or subsequent action) in the dynamics of the SMN.

\begin{figure}[ht]
    \centering
    % Placeholder for the actual graphic.
    \fbox{\parbox[c][12cm][c]{12cm}{\centering \Large \textbf{Figure 2: From Saturated Action to Unsaturated Concept} \ \vspace{1cm} \large (A) Saturated HAP: Agent grasps a real cup. \ \textit{Action is constrained by object.} \ \vspace{1cm} (B) Unsaturated HAP (USHAP): Agent mimes grasping. \ \textit{Action pattern is delinked from object.} \ \vspace{1cm} (C) Transactional Action Pattern (TAP): Agent uses gesture to signify "drink" to another agent. \ \textit{USHAP becomes a shared symbol.}}}
    \caption{\textbf{The Grounding of Concepts in Action.} This figure illustrates the progression from concrete action to abstract representation. \textbf{(A)} A **Saturated HAP** is a direct, physical interaction with an object. \textbf{(B)} An **Unsaturated HAP (USHAP)** is the re-enactment of the action's *pattern* without the object, forming the basis of an internal concept. \textbf{(C)} A **Transactional Action Pattern (TAP)** occurs when the USHAP is externalized and used in a social context, becoming a shared symbol.}
    \label{fig:saturation_spectrum}
\end{figure}

\subsubsection{Tokenization through Action}
\label{ssubsec:tokenization}
For any symbolic or computational system to emerge, there must be a process for creating discrete, stable units—tokens—that can be reliably identified and manipulated. In a purely dynamic, continuous system, this is a non-trivial problem. The Sensation-Modulating Network (SMN) solves this by leveraging the core faculty of haltability. A Haltable Action Pattern (HAP) is not merely a continuous movement; it is a bounded, repeatable, and recursive event that functions as a proto-symbolic token.

The creation of a token begins with the agent's ability to initiate and terminate an action at will. This act of "framing" an action—giving it a clear beginning and end—is what carves it out from the undifferentiated stream of activity. A grasp, a step, or a vocalization becomes a discrete unit precisely because it can be started, stopped, and, crucially, repeated. This repeatability gives the action-token a stable identity. The agent can perform the "grasp" action now, and then perform the "same" action again later, treating it as an instance of a type.

Furthermore, these action-tokens are inherently recursive and modulatable. A HAP can be embedded within another, as the HAP of wiggling a finger is part of the larger HAP of grasping an object. This nested, part-whole structure is a foundational precursor to syntactic recursion. The segmented architecture of the SMN provides a natural "lexicon" of potential tokens (the set of actions afforded by hands, limbs, mouth, etc.), while the agent's fine motor control allows for subtle modulations of these tokens—a gentle grasp versus a firm grasp—which function like inflections of a core verb.

Through this process, the agent transforms its continuous, dynamic engagement with the world into a set of discrete, manipulable, and meaningful units. These are not abstract, amodal symbols that stand *for* the world; they are the very patterns of the agent's meaningful interaction *with* the world, packaged into a form that can be used for the higher-order cognitive processes that follow.

\subsubsection{Grounding Tokens in the Graspable Situation}
\label{ssubsec:grounding}
An action-token, as a discrete and repeatable pattern, does not possess meaning in isolation. Its significance is derived from its context—the "graspable situation" in which it is performed. This grounding is not a mysterious semantic process but a physical one, rooted in how action schemas reconfigure the agent's neural network. Following a Hebbian model \cite{hebb1949organization}, the more an action is repeated in response to a specific ecological affordance, the more the neural pathways involved in that action-perception loop are strengthened. "Neurons that fire together, wire together," creating a durable linkage between the agent, the action, and the object.

These learned, reconfigured neural pathways are the very embodiment of conceptual schemes. In the parlance of computer science, the agent is not storing "data" about the world; it is forging the "data structures" through which the world can be understood. Each of these structures is inherently propositional. The agent (the subject) executes a differentiating action pattern (the predicate or verb phrase) upon an affordance (the object). This interaction naturally constructs a `subject-predicate-object` triple, the fundamental atom of meaning. For instance, the agent (`I`) performs the action (`grasp`) on the object (`cup`).

Crucially, this process does not result in a list of detached, atomic sentences. Because the same agent performs many actions, and the same object affords multiple actions, these triples share nodes. The "cup" node is linked not only to "grasp" but also to "lift," "drink from," and "see." The "I" node is linked to every action the agent can perform. The result is a vast, interconnected, graph-theoretic data model. In this framework, knowledge is not a collection of facts to be processed but *is* the very structure of this network. It is a dynamic, relational, and embodied web of possibilities, continuously shaped by the agent's ongoing engagement with its world.

\subsubsection{Musculature as the Organ of Spatial Computation}
\label{ssubsec:muscles_space}
Just as the motor system is the primary organ of differentiation, it is also the primary organ of spatial computation. Space is not a pre-existing, absolute container that the brain passively represents, but a fundamental category of experience that is actively constructed by the musculature. The body's network of muscles functions as a biological Global Positioning System (GPS), continuously resolving questions of distance, displacement, and orientation through action.

This framework leads to a strong, falsifiable claim: in the absence of movement, or at least the potential for movement, visuo-spatial construction of the world is impossible. An immobilized agent cannot truly perceive depth or distance because it cannot perform the epistemic acts—the reaching, walking, and turning of the head—that measure the world. The proprioceptive feedback from a stretched arm and the vestibular sensations from a tilted head are not secondary data points that confirm a visual hypothesis; they are the very substance of the spatial calculation. It is the felt effort of muscular action that provides the fundamental metric for space. Therefore, the category of space is not provided by purely neuronal means but is a direct phenomenal consequence of having a body with a specific musculoskeletal architecture acting in the world.

\subsection{The Embodied Origins of Generative Syntax}
\label{subsec:syntax}
A longstanding challenge in cognitive science is to explain the origin of generative syntax. The SMN model posits that syntax is not a unique, brain-based module for language, but is exapted from the inherent combinatorial structure of the body's action zones. The segmented nature of the SMN provides a finite set of action "lexemes" (the HAPs of different zones). The ability to halt and serially chain these actions provides a natural syntax. 

A complex goal, like eating a fruit, is achieved by a syntactic sequence of HAPs: `[see fruit] + [reach for fruit] + [grasp fruit] + [bring to mouth]`. Each element is a discrete action pattern, and the pauses between them act as syntactic boundaries, allowing for substitution (e.g., grasp a different fruit) or recursion (e.g., grasp another fruit). This action-syntax is the scaffold upon which spoken language is built. The rules of grammar are not abstract and amodal, but are deeply homologous with the rules of combining bodily actions. This reinterprets Chomsky's "universal grammar" not as an innate linguistic module, but as a reflection of the universal architecture of the vertebrate body plan.

\subsubsection{Combinatorial Action and the Explosion of Tokens}
\label{ssubsec:combinatorial}
A significant limitation of traditional models of generative syntax, such as those proposed by Chomsky, is the implicit assumption that actions are monotonic or linear sequences controlled by a central authority. This view fails to capture the rich, multi-layered dexterity of a biological agent. The Sensation-Modulating Network (SMN) reframes the body as a dialogical architecture—an orchestra of multiple, coordinated action zones. The "syntax" of action is not a linear string of commands but a dynamic, unfolding choreography, like a dance.

Consider the act of speaking. It is not a simple motor program but a complex interplay of the buccal cavity, the tongue, the larynx, and a haltable pulmonary zone supported by the intercostal muscles and diaphragm. This is a system of nested, parallel, and serially coordinated Haltable Action Patterns (HAPs). The failure of purely cognitivist models to account for this generativity stems from their neuro-centric focus; they search for a "language organ" in the brain, when the competence is distributed across the entire bodily network. In our account, the lungs, the saccadic muscles of the eye, the facial muscles used for gesturing, and the exapted forelimbs are all integral cognitive components.

This perspective requires us to see generative syntax not as a phenomenon exclusive to language, but as a general principle of combinatorial action. The ability to play a musical instrument, type on a keyboard, mime, dance, or sing are all potent examples of the same underlying capacity: the sequencing and combination of action-tokens from multiple zones to create a near-infinite variety of meaningful patterns. Language is not a peculiar, isolated faculty but is one manifestation—albeit a highly sophisticated one—of this general dexterity.

This leads to a crucial conclusion about the human place in the evolutionary spectrum. The "explosion of tokens" that characterizes human culture is not the result of a unique biological peculiarity. Rather, it is a matter of *predominance*. The underlying architectural principles of the SMN are shared with other animals. Humans are distinctive only in the degree to which they have developed the capacity for fine-grained modulation and combinatorial sequencing of HAPs. Our cognitive abilities, including language, lie on the same continuum as other forms of animal cognition, grounded in the shared logic of an embodied, acting network.

\subsubsection{From Habits to Syntactic Representations}
\label{ssubsec:habits}
While individual Haltable Action Patterns (HAPs) function as discrete tokens, their true power is realized when they are streamed together into fluent, repeatable sequences, or habits. A habit is a form of embodied, procedural syntax. However, for this syntax to become a shareable, stable representation, it must be punctuated—it must leave a trace.

Crucially, HAPs are not confined to the agent's body; they physically impact the external world. A gesture leaves a visible trace in the air, a step leaves a tactile trace in the sand, and a vocalization leaves an auditory trace in the form of a sound wave. These external traces—visible, tactile, and auditory—are the bridge from the individual to the inter-subjective. They are the punctuated, stable artifacts of the agent's fluid actions.

These traces are the foundation of Transactional Action Patterns (TAPs). An external trace, such as a carved notch in a piece of wood, is an affordance that invites another agent to engage with it, perhaps by making a similar mark or by using it as a counter. Because the agent producing the trace and the agent perceiving it share the same fundamental bodily architecture, the trace can be understood through the same action-based repertoire. This shared source of action and perception is the key: we do not need another, mysterious mechanism to explain how humanity became *Homo symbolicus* or *Homo semioticus*. The symbol is born the moment an action's trace is used as a token for a subsequent, inter-subjective action.

This transactional space, built upon external traces, is the arena where we use and shape objects and surfaces for shared meaning. The deliberate creation of traces for others to interpret is the very essence of encoding, and the interpretation of those traces is decoding. This is the bedrock of reading, writing, and all forms of external, symbolic culture. It is not a different kind of cognition, but a sophisticated, scaffolded application of the same fundamental SMN dynamics that govern how an agent grasps a stone.

\subsection{Constructing the Categories of Time and Number}
\label{subsec:time_number}
Just as space is a construction of the motor system, the fundamental categories of time and number also emerge from the body's intrinsic dynamics. The agent is a symphony of nested clocks, each operating at a different frequency. At the deepest level are the slow, stable rhythms of metabolic cycles and the high-frequency oscillations of atomic and quantum phenomena. Layered on top of these are the involuntary, medium-frequency Fixed Action Patterns (FAPs) like heartbeat and respiration. Finally, at the most accessible layer, are the Haltable Action Patterns (HAPs). This multi-layered, pre-formatted temporal canvas is the raw material of temporal experience.

A single action zone, such as a finger tapping, has a "serialization limit"—a maximum frequency of around 10 Hz at which it can reliably produce discrete actions. However, the agent can achieve much higher frequencies of token generation by alternating between different action zones, as a pianist does with ten fingers or a typist with both hands. This ability to serialize discrete actions against the body's continuous rhythmic background is the origin of number sense. The act of counting is the act of producing a sequence of punctuated HAPs and mapping them to objects or events. "One, two, three" is a motor pattern before it is an abstract concept. Time, therefore, is not a universal river in which the agent is immersed, but the felt experience of the body's own nested rhythms, and number is the emergent property of punctuating that experience with discrete, serial actions.

\subsection{Reconciling Cognitive Divides}
\label{subsec:reconciling}
The SMN framework offers a bridge between cognitivism and enactivism. It is profoundly enactive, as it equates cognition with the modulation of action. However, it does not reject representation; it redefines it. USHAPs are representations: they are internal, stand-ins for external objects, and can be manipulated in simulations. But unlike classical cognitivist symbols, they are never amodal or arbitrarily related to their referents. They are grounded in the very action patterns used to interact with those referents, thus providing a natural solution to the symbol grounding problem.

This also allows for a reinterpretation of dual-process theories. System 1 can be seen as the fast, parallel, and largely unconscious operation of the SMN's FAPs and highly fluent, automatized HAPs. System 2 emerges from the slow, serial, and effortful process of consciously halting, sequencing, and modulating HAPs, particularly in novel situations or during explicit communication via TAPs. Development and expertise represent the process by which effortful System 2 operations (newly learned TAPs) become fluent, embodied System 1 skills (automatized HAPs).

\subsubsection{USHAPs, Simulation, and Imagination}
\label{ssubsec:ushaps}
The conceptual power of the SMN model stems from the distinction between saturated and unsaturated Haltable Action Patterns (HAPs). An action is **saturated** when it is performed with its corresponding object: grasping a physical cup, swallowing actual water, or running on solid ground. The action is constrained and informed by the object in real-time. Conversely, an action is **unsaturated** when it is performed without the object: a mime of grasping, a gesture of drinking, or a "moonwalk" that suspends actual displacement. These Unsaturated HAPs (USHAPs) are the raw material of representation, simulation, and imagination.

This creates a spectrum of representation. A non-arbitrary USHAP, like a mime, bears a direct, iconic relationship to its saturated counterpart and can be understood without a pre-learned rule. However, USHAPs can also be **arbitrary**, as in a "thumbs-up" gesture. We propose that natural language is built upon a vast lexicon of such arbitrary USHAPs (vocal gestures). The evolution towards arbitrary symbols is not accidental; it serves a crucial socio-political function. Arbitrary symbols are opaque to outsiders, creating "private languages" that help forge social identity and can be used for strategic communication, for instance, in warring conditions. The diversity of human scripts and grammars is a testament to this drive for social differentiation.

The ultimate expression of the USHAP is its full internalization as the basis of thought. "Thinking about running" involves the execution of subtle, inexpensive motor patterns—micro-movements in the larynx, slight shifts in breathing, faint saccades of the eye—that are delinked from the costly, saturated action of actually running. This "economy of execution" is what makes abstract thought metabolically feasible. Crucially, what is internalized and tokenized is not the specific muscular action, but its abstract *pattern* or *form*. An agent can trace a circle with a hand, a foot, or their entire body; the neural network identifies the shared, context-independent form. This is how concepts, as USHAPs, are "emancipated" from their context of discovery, allowing them to be applied to new situations.

This framework allows us to reconcile the cognitivist claim of detached representations with the enactivist demand for grounding. A concept, as an USHAP, *is* detached from its original context, which is why we can think about cups without a cup being present. However, it is never un-grounded; it is always physically and biologically grounded in the action patterns of the agent's body. The rules for decoding the *arbitrary* USHAPs of language are then socially and culturally grounded, requiring institutions like schools for their posterity. Semiotics is thus grounded in a nested hierarchy: in the body, and in the culture that the body builds.

\subsubsection{A Ground for Neural and Cognitive Plasticity}
\label{ssubsec:plasticity}
The Sensation-Modulating Network (SMN) is not a static architecture; it is inherently plastic, capable of adapting and reorganizing in response to experience and constraint. This plasticity manifests in two primary ways.

First, the segmented, multi-zone nature of the body provides a crucial form of **redundancy**. The abstract *pattern* of an action can be implemented by multiple, different action zones. An agent can gesture with its hands, but if they are occupied, it can achieve a similar communicative act with a nod of the head or a facial expression. This interchangeability is a powerful source of flexibility and creativity. It is vividly demonstrated in cases where individuals who have lost the use of their limbs learn to write, paint, or perform other complex tasks with their feet and toes. The underlying conceptual schema—the Unsaturated HAP (USHAP)—is not tied to a specific effector, allowing the system to explore and discover alternative ways of achieving its goals.

Second, the ability to **re-enact** delinked patterns (USHAPs) is the engine of neural plasticity. Every time an agent internally simulates an action, the corresponding neural pathways are activated. This process of re-enactment strengthens synaptic connections, carving out and reinforcing the "data structures" or conceptual schemes we discussed earlier. This is the mechanism of learning and memory: a memory is not a stored file but the capacity to re-enact a specific pattern of action, which in turn re-creates a phenomenological experience.

Together, these two aspects of plasticity explain how action schemas and conceptual schemas become **emancipated** from their original context. Because a pattern can be implemented by different parts of the body and can be re-enacted internally without its original object, it takes on a life of its own. This emancipation is what creates the powerful phenomenological appearance of an autonomous, internal "world of representations." The concepts feel detached and abstract precisely because the underlying action patterns are so flexible and portable. Plasticity is thus not just a feature of the brain; it is a fundamental property of an embodied agent that learns and adapts by continuously exploring and re-enacting its possibilities for action.