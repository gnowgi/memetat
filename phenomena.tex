\section{Explaining Cognitive Phenomena and Reinterpreting Evidence}
The explanatory power of the Sensation-Modulating Network (SMN) lies in its capacity to reframe fundamental cognitive questions in terms of action dynamics. By grounding cognition in a specific, yet generalizable, bodily architecture, the model offers a unified account of phenomenology, representation, and syntax, while reinterpreting existing empirical evidence.

\subsection*{From Action Modulation to Phenomenology}
The SMN model addresses the "hard problem" of consciousness by proposing that subjective experience is the agent's perception of its own modulated actions. The constant, rhythmic hum of the Fixed Action Patterns (FAPs) creates a stable, non-conscious background—a "cognitive canvas." A phenomenological event occurs when a Haltable Action Pattern (HAP) is initiated, modulated, or, most critically, paused. This change against the stable background *is* the experience. The "what it is like" to see red is the specific pattern of oculomotor and neural HAPs enacted to foveate on a red object; the feeling of thirst is the modulation of interoceptive patterns against the homeostatic background. Consciousness is not a substance or a passive "theater" but an active process of self-differentiation through action.

\subsubsection*{Action as the Origin of Phenomenal Experience}
In the Sensation-Modulating Network (SMN) framework, a phenomenological experience is not a passive event, such as the mere reception of sensory stimuli. Instead, it is an active construct, a direct consequence of the agent's own self-initiated actions. The process begins when the agent deploys a Haltable Action Pattern (HAP)—a voluntary, modulated action directed towards an affordance in the environment. This action, be it a saccade of the eye, a turn of the head, or the extension of a hand, is not a reaction to a stimulus but is itself the stimulus that generates the experience.

When the agent acts, it necessarily alters the flow of sensory information across its entire body. The HAP actively modulates this sensory stream, creating a specific, transient pattern of differentiation against the stable, homeostatic background provided by the Fixed Action Patterns (FAPs). The "phenomenological response" is precisely this internally-generated, action-driven pattern. For example, the experience of touching a rough surface is not caused by the surface's texture alone; it is constructed by the specific HAP of moving one's fingers across it, which generates a unique pattern of vibrations and pressures. The action and the sensation are inextricable.

Therefore, the agent's world is not something that is revealed to it, but something it brings forth through its own activity. Each self-initiated action is a question posed to the environment, and the resulting modulation of the agent's own sensory state is the answer. This continuous loop of action-based self-stimulation is the very engine of conscious experience, grounding phenomenology not in a mysterious inner theater, but in the concrete, dynamic process of a body actively engaging with its world.

\subsection*{Constructing a Geometric and Semiotic World}
The agent's world is not a pre-given, objective space that is passively perceived, but a "memetat"—a geometric and semiotic habitat constructed through action. The SMN computes this geometry through the coordination of its multiple action zones. For instance, the distance to an object is not calculated from retinal size alone, but is enacted through a fusion of HAPs: the proprioceptive feel of reaching a hand, the muscular strain of focusing the eyes, and the time delay between a sound and its echo. The world's geometry is mapped onto the agent's own bodily geometry and action capabilities.

Meaning arises as this geometric world becomes populated with signs. An object's affordances are the saturated HAPs it invites. A rock affords throwing, sitting, or striking. These saturated HAPs are the object's initial meaning for the agent. When these actions are tokenized as unsaturated HAPs (USHAPs), they become concepts that can be manipulated internally. When they are shared and imitated as Transactional Action Patterns (TAPs), they become external, public symbols. This provides a direct, embodied route from perception to semiotics, grounding the Peircean triad of object, representamen (the action pattern), and interpretant (the resulting experience or subsequent action) in the dynamics of the SMN.

\subsubsection*{Tokenization through Action}
For any symbolic or computational system to emerge, there must be a process for creating discrete, stable units—tokens—that can be reliably identified and manipulated. In a purely dynamic, continuous system, this is a non-trivial problem. The Sensation-Modulating Network (SMN) solves this by leveraging the core faculty of haltability. A Haltable Action Pattern (HAP) is not merely a continuous movement; it is a bounded, repeatable, and recursive event that functions as a proto-symbolic token.

The creation of a token begins with the agent's ability to initiate and terminate an action at will. This act of "framing" an action—giving it a clear beginning and end—is what carves it out from the undifferentiated stream of activity. A grasp, a step, or a vocalization becomes a discrete unit precisely because it can be started, stopped, and, crucially, repeated. This repeatability gives the action-token a stable identity. The agent can perform the "grasp" action now, and then perform the "same" action again later, treating it as an instance of a type.

Furthermore, these action-tokens are inherently recursive and modulatable. A HAP can be embedded within another, as the HAP of wiggling a finger is part of the larger HAP of grasping an object. This nested, part-whole structure is a foundational precursor to syntactic recursion. The segmented architecture of the SMN provides a natural "lexicon" of potential tokens (the set of actions afforded by hands, limbs, mouth, etc.), while the agent's fine motor control allows for subtle modulations of these tokens—a gentle grasp versus a firm grasp—which function like inflections of a core verb.

Through this process, the agent transforms its continuous, dynamic engagement with the world into a set of discrete, manipulable, and meaningful units. These are not abstract, amodal symbols that stand *for* the world; they are the very patterns of the agent's meaningful interaction *with* the world, packaged into a form that can be used for the higher-order cognitive processes that follow.

\subsubsection*{Grounding Tokens in the Graspable Situation}
% TODO: Elaborate on how the ecological situation grounds tokens and creates the subject-object link.

\subsection*{The Embodied Origins of Generative Syntax}
A longstanding challenge in cognitive science is to explain the origin of generative syntax. The SMN model posits that syntax is not a unique, brain-based module for language, but is exapted from the inherent combinatorial structure of the body's action zones. The segmented nature of the SMN provides a finite set of action "lexemes" (the HAPs of different zones). The ability to halt and serially chain these actions provides a natural syntax. 

A complex goal, like eating a fruit, is achieved by a syntactic sequence of HAPs: `[see fruit] + [reach for fruit] + [grasp fruit] + [bring to mouth]`. Each element is a discrete action pattern, and the pauses between them act as syntactic boundaries, allowing for substitution (e.g., grasp a different fruit) or recursion (e.g., grasp another fruit). This action-syntax is the scaffold upon which spoken language is built. The rules of grammar are not abstract and amodal, but are deeply homologous with the rules of combining bodily actions. This reinterprets Chomsky's "universal grammar" not as an innate linguistic module, but as a reflection of the universal architecture of the vertebrate body plan.

\subsubsection*{Combinatorial Action and the Explosion of Tokens}
% TODO: Detail how the choreography of multiple action zones leads to combinatorial possibilities.

\subsubsection*{From Habits to Syntactic Representations}
% TODO: Explain how streaming patterns of HAPs (habits) create punctuated, syntactic representations.

\subsection*{Reconciling Cognitive Divides}
The SMN framework offers a bridge between cognitivism and enactivism. It is profoundly enactive, as it equates cognition with the modulation of action. However, it does not reject representation; it redefines it. USHAPs are representations: they are internal, stand-ins for external objects, and can be manipulated in simulations. But unlike classical cognitivist symbols, they are never amodal or arbitrarily related to their referents. They are grounded in the very action patterns used to interact with those referents, thus providing a natural solution to the symbol grounding problem.

This also allows for a reinterpretation of dual-process theories. System 1 can be seen as the fast, parallel, and largely unconscious operation of the SMN's FAPs and highly fluent, automatized HAPs. System 2 emerges from the slow, serial, and effortful process of consciously halting, sequencing, and modulating HAPs, particularly in novel situations or during explicit communication via TAPs. Development and expertise represent the process by which effortful System 2 operations (newly learned TAPs) become fluent, embodied System 1 skills (automatized HAPs).

\subsubsection*{USHAPs, Simulation, and Imagination}
% TODO: Elaborate on how delinked USHAPs enable simulation and "at will" association.

\subsubsection*{A Ground for Neural and Cognitive Plasticity}
% TODO: Explain how the ability to recreate delinked patterns enables plasticity.
