\subsection{Layered SMN}
\label{subsec:layered}

All living agents generate rhythmic action patterns—oscillatory movements such as beating,  swimming, walking, and breathing—that anchor them in time. Cognition, we argue, arises from  the \emph{capacity to interrupt and reconfigure} such rhythmic patterns, and in that process the organism decodes the space by modulating sensations. This interruption  creates a hierarchy of action patterns:  \emph{Fixed Action Patterns} (FAPs), \emph{Haltable Action Patterns} (HAPs), \emph{Negotiable Action Patterns} (NAPs) and  \emph{Transactional Action Patterns} (TAPs).  This hierarchy makes possible both stability and flexibility in behavior, uniting automatized responses with adaptive control.   
%=== Clarifying the action-pattern ladder with examples === 
\paragraph{From FAP to TAP: operational ladder with examples.} 
\begin{description} 
\item[FAP (Fixed Action Pattern).]  Stereotyped, self-sustaining routine ($\Gamma_{\text{fast}}$) minimally parameterized by context; e.g., the ballistic reach component at movement onset, swallowing reflex, or a canonical gait phase.  
\item[HAP (Haltable Action Pattern).]  As FAP but endowed with a reliable interrupt port $\mathcal{H}_\alpha$ that yields safe arrest or phase reset without loss of task; e.g., saccade cancellation, reach correction on target jump \citep{Aron2007}.  

\paragraph{The idea of a haltability operator.} A core innovation of the SMN framework is the notion of a \emph{haltability operator}.  Biological agents do not simply run motor routines to completion; they often need to interrupt them, pause them, or redirect them in mid-course without losing stability or posture.  The haltability operator is a conceptual tool to capture this capacity.  It represents the minimal act of cognition: the ability to \emph{modulate ongoing rhythmic patterns}—to stop, reset, or reconfigure them—in response to changing circumstances.  For example, halting a reach when the target suddenly moves, stopping a step to avoid an obstacle, or pausing speech to accommodate a listener are all instances of haltability at work.  In this sense, the operator formalizes the agent’s capacity for safe and flexible interruption, which transforms fixed routines into negotiable and ultimately transactional patterns.  The formal sketch given is not meant as a full mathematical model, but as a compact way to express how such interruptions can be represented and studied within a dynamical-systems perspective.  

\paragraph{Haltability as dynamic control.} In classical inhibition models, a muscle halts when an inhibitory signal is sent—an input that drives the system to a stop state.  The SMN picture is subtler. Here, a \emph{coordinate pair of action zones} (e.g., left and right effectors) mutually regulate one another.  The haltability signal still arrives through neural channels, but its source is not a central command structure; it originates in the partner zone of the pair.  Because motor zones are not in direct contact, neural messaging mediates this coordination.  Crucially, haltability is not a zero state of inactivity: when one zone is active, its partner is in a state of heightened alert, poised to take over or intervene.  Haltability thus denotes an \emph{alert equilibrium}—an actively sustained readiness to interrupt, rather than a passive shutdown.  This richer notion requires a formalism beyond binary inhibition: the haltability operator should capture both the active trajectory and the latent, watchful potential of its coordinated partner.  

\paragraph{Formal intuition.} Let $Z_L(t), Z_R(t)$ be state variables for a left–right action pair.  Classical inhibition treats halt as setting $Z_L(t)=0$.  By contrast, SMN models haltability as a pairwise relation: \[ \mathcal{H}(Z_L,Z_R):\quad  
\begin{cases} Z_L(t) \;\text{active trajectory}, \\ Z_R(t) \;\text{alert state: } \dot{Z}_R \approx 0,\; \partial_t \mathbb{E}[Z_R]>0, \end{cases} \] 
meaning the inactive partner is dynamically stabilized but metabolically and informationally alert, prepared to intervene.  Here, $\partial_t \mathbb{E}[Z_R]>0$ expresses that readiness is actively maintained, not passively absent.  The operator $\mathcal{H}$ thus encodes a \emph{co-regulated balance} between zones, rather than a one-way inhibitory command.    
\item[NAP (Negotiable Action Pattern).]  Assemblies of oscillators and synergies that support \emph{on-the-fly adjustment} through assimilation and accommodation.  Unlike HAPs, which are only stoppable or redirectable, NAPs are \emph{modifiable primitives}: they can be reweighted, rephased, and recombined to fit current affordances.  This negotiability underlies habits and skills while preserving schematic stability; e.g., gait adaptation to terrain irregularities, bimanual task coordination, or speech articulation adjustments \citep{HakenKelsoBunz1985,Kelso1995,BizziCheung2013}.  Stabilized NAPs thus provide the modular units that ground schematic aspects of knowledge and cognition within the SMN.  
\item[TAP (Transactional Action Pattern).]  Negotiable Action Patterns that are externalized into the public domain by leaving perceptible traces (gestural, vocal, material) interpretable by other agents.  TAPs thus serve as the \emph{interface of exchange}: they are NAPs made transactional, supporting conversational turn-taking, collaborative coordination, and semiotic expression.  In this sense, TAPs constitute the space where private negotiable patterns become shared structures, grounding communication and generative syntax in embodied action.  \end{description} 

The SMN hypothesis is that cognition emerges as the control of \emph{interruptibility and composition} across this ladder: intelligent agents regulate when and how to destabilize, rephase, and recombine ongoing patterns to transact with affordances.  

\paragraph{Negotiable Action Patterns (NAPs).} Fixed Action Patterns (FAPs) furnish the biological ground of agency: stereotyped routines that run to completion once released.  When such routines acquire reliable interrupt ports, they yield Haltable Action Patterns (HAPs), which permit safe arrest and redirection.  Habitualized HAPs can, over time, sediment into new FAP-like routines, enlarging the agent’s repertoire of automatized behaviors.  Yet when the same HAPs are adjusted \emph{on the fly}—modulated by current affordances, assimilated into familiar schemas, or accommodated to novel contingencies—they give rise to \emph{Negotiable Action Patterns} (NAPs).  NAPs mark a crucial inflection in the hierarchy: unlike FAPs and HAPs, which are tied to stereotypy and stoppability, NAPs are essentially \emph{modifiable mid-level primitives}.  They are typically implemented by assemblies of oscillators and synergies, but what makes them distinctive is their negotiability: they can be reweighted, rephased, and recombined to accommodate new demands while still preserving schematic stability.  In this sense, NAPs are the natural substrate of habits, skills, and schemata—organized and modularized patterns of action that can be flexibly redeployed.   When such NAPs leave perceptible traces—gestural, vocal, or material—that are interpretable by other agents, they cross into the public domain as \emph{Transactional Action Patterns} (TAPs).  TAPs are thus not only action sequences but also entries into a conversational space: they are NAPs externalized for negotiation between agents.  In this layered picture, NAPs provide the private interface of the agent’s SMN, while TAPs constitute the intersubjective surface where semiotic exchange and generative syntax emerge.  This progression grounds semiotics in the phenomenological being of the agent, rather than in an abstract machine, by showing how rhythmic action patterns, once rendered negotiable, become the very medium of communication and meaning.

