\subsection{Bilateral Appendages: From Locomotion to Multi-Zonal Manipulation}
\label{subsec:bilateral_appendages}

After the consolidation of bilateral symmetry, many lineages elaborate \emph{paired appendages} that scaffold new control degrees of freedom in the SMN. In arthropods, serially homologous, jointed limbs (and cranial mouthparts) arise by redeploying a conserved appendage GRN (e.g., \textit{Dll/dac/exd/hth}) under Hox control; segmental joints form via Notch-mediated boundary specification \citep{Panganiban1997OriginEvolutionAppendages,Prpic2009NotchSegmentation,Cordoba2020NotchLegReview,Chen2023NotchInsectReview}. In vertebrates, paired fins/limbs are positioned and patterned by interacting organizers—the AER and ZPA—and signaling families (FGF, SHH, WNT, BMP) integrated with T-box and Hox programs; Tbx4/5 drive outgrowth and field identity while limb \emph{morphotypes} depend on additional inputs (e.g., \textit{Pitx1}) rather than T-box genes alone \citep{Capdevila2001LimbPatterning,RodriguezEsteban1999Tbx4Tbx5,Minguillon2005Tbx5Tbx4NotSufficient,Duboc2021Tbx4FunctionHindlimb}.

The fin–limb macroevolution exemplifies ``deep homology'': tetrapod wrists/ankles/digits emerge by elaborating and partitioning endoskeletal and distal modules already present in sarcopterygian fins \citep{Shubin1997FossilsGenesLimbs,Shubin2006TiktaalikPectoralFin,Onimaru2020FinRayPNAS,Shubin2009DeepHomology}. Current work reconciles classical hypotheses on paired appendage origins (lateral fin-fold vs.\ gill-arch derivation) by tracing cell-source and GRN redeployments in median and paired fins \citep{Tzung2023MedianFinLPM,Diogo2020DualOriginPectoral}. Dorsoventral limb polarity integrates Wnt7a/En1 with \textit{Lmx1b}, establishing muscular and tendinous antagonisms across joints—a template reused in regeneration contexts \citep{Riddle2002Lmx1bDV,Gerber2022DVRegeneration}.

\paragraph{Functional expansion and cognitive payoffs.}
Initially selected for aquatic/terrestrial/arboreal locomotion, paired appendages underwent extensive exaptation—grasping, holding, digging, grooming, social display, and tool-related actions—multiplying available \emph{action zones}. With rigid levers and antagonistic flexor–extensor pairs arranged about synovial pivots, appendages support both \emph{FAPs} (canalized, stereotyped acts) and \emph{HAPs} (haltable, re-routable microprograms). Critically, many taxa recruit appendages \emph{onto the body surface itself}: scratching, pruning, cleaning and palpation produce self-directed SMAPs that differentiate interoceptive/proprioceptive from exteroceptive contingencies (see \S\ref{subsec:smap}). This added anatomical layer thus expands combinatorial \emph{eCAZ} repertoires and underwrites the transition from purely locomotor control to dexterous, exploratory, and ultimately symbolic manipulations.
