\section{Conclusion}
\label{sec:conclusion}
This paper has argued that the persistent divisions in cognitive science can be traced to a foundational oversight: the failure to appreciate the cognitive implications of the agent's biological architecture. In response, we have proposed the Sensation-Modulating Network (SMN), a model that reframes the agent not as a brain in a vat, but as a structured, dynamic body whose primary cognitive act is the modulation of action. By shifting the focus from neural computation to the dynamics of halting and sequencing action patterns, we have outlined a unified framework that accounts for phenomenology, the construction of a meaningful world, and the emergence of generative syntax from embodied activity.

The SMN model represents a foundational departure from several mainstream approaches. It moves beyond neuro-centrism by treating the nervous system not as the seat of cognition, but as a crucial tool for modulating the body's intrinsic dynamics. It rejects the cognitivist notion of amodal symbols, instead grounding representation in the concrete, yet abstractable, form of unsaturated action patterns. Finally, it provides a concrete, architectural basis for enactivist claims, moving from philosophical stance to a testable set of principles.

The implications of this framework open several avenues for future research. Empirically, the model generates specific predictions that can be tested through cognitive robotics and developmental psychology. For instance, experiments could be designed to trace the link between the mastery of complex, sequential motor tasks (action syntax) and the subsequent development of linguistic syntax in children. Theoretically, the model can be extended to other domains of cognition, such as emotion, which can be framed as the phenomenological experience of modulating deep-seated FAPs, or social cognition, understood as the complex interplay of mutually-scaffolding TAPs.

We anticipate that a potential criticism of this framework may be that it seems to leave little room for highly abstract, non-motoric thought, such as mathematical or philosophical reasoning. From this perspective, our model might appear to reduce all cognition to mere bodily movement. However, this would be a misinterpretation of the role of Unsaturated HAPs (USHAPs). The process of emancipation—whereby an action *pattern* is abstracted from any specific motor execution—is precisely the mechanism that grounds these higher forms of thought. The content of abstract reasoning is not the gross bodily action itself, but the subtle, inexpensive, internalized re-enactment of action schemas. Our claim is not that thinking *is* overt movement, but that it is founded upon the same generative, syntactic principles that govern movement, providing a fully grounded yet powerfully abstract cognitive engine.

Which specific substrate in the body is primarily responsible for running these USHAPs is an open empirical question. Though our account is informed by biology, as a theoretical contribution it remains, at best, a grounded speculation. The model's primary contribution, therefore, is to demonstrate that there are alternative, and perhaps more fruitful, ways we can think about thinking.

Ultimately, the SMN model calls for a return to a more integrated view of the cognitive agent. It suggests that to understand the mind, we must first understand the body—not as a mere input-output device for a central brain, but as the very medium of cognition itself. The path to explaining consciousness, language, and reason does not lie in abstracting away from our biological form, but in recognizing that it is the source of our unique cognitive life.
