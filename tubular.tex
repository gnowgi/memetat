
\subsection{Tubularity and the Genesis of Haltable Action Patterns}
\label{subsec:tubular}

\paragraph{From aquatic beginnings to tubes that manage flow}
Life’s first challenges were aquatic. In low-Reynolds aquatic worlds,\marginpar{\footnotesize Low–Re (Re$\ll$1) aquatic world: viscous forces dominate inertia, so reciprocal strokes yield no net transport (Purcell’s scallop theorem), and one-way flow demands nonreciprocal ciliary metachrony or peristaltic waves with valves/sphincters to block backflow \citep{Purcell1977LowRe,Shapiro1969Peristalsis,Vogel1994LifeMovingFluids}. Expanded version in Appendix \ref{reynolds-number}} unicellular eukaryotes solved feeding by \emph{making} flow: beating cilia generate and steer local currents to trap particles and microbes, with coordination handled by planar polarity and basal-body geometry \citep{Brooks2014Multiciliated,Shekhar2023CooperativeHydrodynamics}. Sessile poriferans scaled this idea up into an \emph{aquiferous system}: simple to complex branched tubes where choanocyte pumps create a persistent, largely unidirectional stream from ostia to osculum; critically, sponges also \emph{shut} this flow through whole-body contractions—even without nerves or muscles—to gate filtration when conditions demand it \citep{Goldstein2020SpongeContractions,Morganti2019SpongePumping}. Cnidarians elaborated a gastrovascular sac that doubles ingestion and egestion through a single orifice, sometimes housing endosymbionts in a protected internal micro-habitat. Later, bilaterians evolved a through-gut with separate mouth and anus—an innovation repeatedly analyzed in comparative embryology and evo-devo as a key step toward efficient, directional processing \citep{Hejnol2015AnalEvolution,Nielsen2018MouthAnus,Presnell2016ThroughGut}. In many lineages, tubular logic spread internally: peristaltic vessels, valves, and eventually hearts moved fluids at scale \citep{MonahanEarley2013BVS}. Neural tubes, by contrast, appear with chordates and serve to coordinate bilateral action systems rather than to transport fluid \citep{Holland2015ChordateNervous}. 

\paragraph{The one-way-flow principle and the cost of letting things run}
Across these body plans, a single control principle recurs: keep bulk flow \emph{one-way}. Biophysical solutions include ciliary metachrony, peristaltic waves, check-valves and sphincters, and pulsatile pumps. Recent work in echinoderm larvae shows explicit \emph{gating} at the pylorus and anus to prevent simultaneous entry/exit—an elegant neural solution to enforce unidirectionality in a simple through-gut \citep{Yaguchi2024Sphincter}. But one-way flow is not always optimal to maintain continuously: pumps cost energy, and stereotyped run-to-completion routines can be maladaptive when the world changes. The evolutionary response we emphasize is the emergence of mechanisms that \emph{can stop or pause} ongoing flows. 

\paragraph{From FAPs to HAPs: making stereotyped patterns stoppable}
Classical ethology described many stereotyped, energetically committed routines as \emph{fixed action patterns} (FAPs) \citep{Ronacher2019FAP}. We propose the complementary notion of \emph{haltable action patterns} (HAPs): peripheral motor programs that remain \emph{competent} to run (often via local myogenic/enteric patterning), but are equipped with neural (and sometimes humoral) control layers that can \emph{gate, suspend, or terminate} the pattern quickly when context flips. The gastrointestinal tract is a template case: peristalsis is generated and modulated by enteric circuits and interstitial cells, but is routinely gated by sphincters and by central inputs \citep{Sharkey2022ENS}. In worms and mammals, swallowing launches a peristaltic wave, yet breathing is temporarily suppressed (\emph{deglutition apnea}); recent work identifies a brainstem “postinspiratory complex” (PiCo) that interfaces the swallow CPG with the respiratory Central Pattern Generators (CPG)\marginpar{\footnotesize A central pattern generator (CPG) is a neural circuit producing rhythmic motor output without rhythmic sensory input; in vivo it is gated by descending commands and feedback so the rhythm can start, pause, or stop—an architectural substrate for HAPs \citep{MarderCalabrese1996PR,Kiehn2016NRN,DelNegro2018BreathingMatters}. See more detailsin \ref{CPG}}  to orchestrate this on-the-fly halting and sequencing \citep{Matsuo2009Coordination,Barlow2009CPGOralResp,Moore2014BrainstemOrofacial,Huff2023PiCo}. 

\paragraph{Early HAP exemplars: buccal capture, swallowing, breathing}
Some of the earliest cross-taxa HAPs are in the buccal–pharyngeal complex: \emph{selectively} capturing a bit of the world, holding it, and then either admitting or ejecting it. In nematodes, the pharynx is a tubular neuromuscular pump with intrinsic rhythmicity and a compact “enteric” circuit; pumping rates and particle transport are rapidly up- or down-modulated by neuromodulators and sensory state—i.e., the pattern is powerful but \emph{haltable} \citep{Avery2012CelegansFeeding,Trojanowski2016PharyngealPumping}. In vertebrates, orofacial CPGs (lick, chew, swallow, breathe, vocalize) are coupled but separable; sensory and cortical inputs can gate transitions and suspend one rhythm in favor of another (e.g., pause breathing to swallow; abort swallow to cough) \citep{Moore2014BrainstemOrofacial,Matsuo2009Coordination,Huff2023PiCo}. These mechanisms make selection possible at the entry point: to \emph{withhold}, \emph{sort}, and \emph{sample}—a clear epistemic move from “let it pass” to “hold and judge.” 

\paragraph{Cognitive moral: selectivity rides on stoppability}
The tubular architecture first solved \emph{transport}, but its success hinged on \emph{control of stoppage}: sphincters as gates, valves as decision points, and CPGs under inhibitory and neuromodulatory oversight. The move from FAP-like “always on” pumping to HAP-like “as needed” control economizes energy, reduces wear, and—crucially—enables \emph{discrete choices}: palatable vs.\ not, safe vs.\ unsafe, admit vs.\ reject. In that sense, categorization at the mouth is already cognition—realized as \emph{haltable} motor programs riding a tubular body plan.
