
\subsection{Tubularity and Flow-Guided Control}
\label{subsec:tubular}

\marginpar{Tubes turn space into \emph{lines of action}.}
Across phyla, development repeatedly builds \emph{tubes}: tracheae and lungs, vasculature, gut, renal and exocrine ducts, neural projections.
This convergence is not incidental; tubularity transforms 3D uncertainty into \emph{1D manifolds} along which signals and matter move predictably.
In the SMN, tubes function as \emph{computational waveguides}: they stabilize rhythmic action (peristalsis, pulsation), furnish interrupt points (sphincters, valves), and project vectors for navigation and exchange.

\paragraph{Evo--devo foundations.}
Tube morphogenesis is a conserved solution to transport, exchange, and signaling; it integrates epithelial polarity, lumen formation, and branching programs \cite{LubarskyKrasnow2003TubeMorphogenesis,AffolterZellerCaussinus2009Branching,VarnerNelson2014Branching}.
Branching morphogenesis leverages local growth rules and mechanical constraints to generate efficient, deeply nested transport trees---a substrate ready-made for rhythmic control and interruptibility.

\paragraph{Electrical and physiological polarity.}
Tubulogenesis depends on \emph{apico--basal polarity}, tight junctions, and ion-handling machinery that generate transepithelial potentials and osmotic flows.
Endogenous electrical cues can \emph{bias tubule orientation and cell migration}, while bioelectric patterning supports stable lumen identities and multi-stable organ-scale states \cite{McCaigRajnicekSongZhao2005ElecControl,Levin2014MolecularBioelectricity}.
These mechanisms provide the electrical degrees of freedom by which the SMN modulates tube-based actions.

\paragraph{Habitat coupling: fluid and gravity.}
Fluid forces are not mere consequences; they are \emph{causal} inputs into tube development and function.
During cardiogenesis, intracardiac flow and shear stresses are essential for proper morphogenesis \cite{HoveKoster2003CardioFlow}.
In respiratory and mucociliary tissues, hydrodynamic coupling aligns ciliary beats and creates metachronal waves that coordinate transport \cite{GuiraoJoanny2007CiliaFlow}.
Gravity shapes hydrostatic pressure gradients and affects perfusion and drainage; in aquatic settings, viscosity and buoyancy set Reynolds-number regimes that favor oscillatory pumping and smooth reversals \cite{Vogel1994LifeInMovingFluids}.
\marginpar{Medium physics co-implements control.}

\paragraph{Control affordances of tubes.}
Tubes supply three computational assets to the SMN:
\begin{enumerate}
  \item \textbf{Rhythmic carriers:} Peristaltic and pulsatile waves act as carriers for \emph{tokens} (bolus segments, pressure pulses) that can be gated, counted, or sequenced.
  \item \textbf{Interrupt points:} Valves and sphincters implement \emph{haltability} and re-routing with minimal overhead.
  \item \textbf{Vectors and gradients:} Axial geometry defines preferred directions for flow, chemo--electrical gradients, and information traffic, reducing planning complexity to one dimension.
\end{enumerate}
\marginpar{Tubes = built-in counters, gates, and guides.}

\paragraph{Thermodynamic economy and habitat-assisted computation.}
By confining control to low-dimensional tube axes and exploiting stable fluid/gravity regularities, organisms reduce the need for costly internal writes and resets.
Tokens are carried by the medium, \emph{not stored}; they dissipate naturally unless captured by gates.
This is Landauer-consistent economy: less logical irreversibility internally, more computation via coupling to lawful environmental dynamics \cite{Landauer1961Irreversibility,Bennett2003LandauerNotes,StillEtAl2012ThermoPrediction}.

\paragraph{Formalization sketch.}
Let $u(s,t)$ denote axial flow or wall activation along arc-length $s$ of a tube; peristaltic waves are traveling-wave solutions $u(s,t)=A\cos(ks-\omega t)$.
\emph{Tokens} are localized pulses or thresholded mass/charge packets advected with speed $v$; \emph{interrupts} are boundary conditions (valves) switching between transmitting/reflection states.
Branching yields a directed acyclic graph whose edges are 1D manifolds; \OAP{}s correspond to sequences of valve states routing tokens to task-defined sinks.
\todo{Figure: (i) branching tube network, (ii) token pulses, (iii) valve state machine, (iv) coupling fields (shear, pressure, voltage) as control knobs.}

\paragraph{SMN vs Habitat (capsule).}
\textbf{SMN:} tube-aligned controllers use gates/valves to sequence \OAP{}s on 1D manifolds.
\textbf{Habitat:} fluid mechanics (shear, pressure, viscosity) carries tokens for free and shapes feasible timing, so tubes compute by coupling to the medium rather than by rewriting internal state.
