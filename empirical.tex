\section{Empirical Grounding and Future Directions}
A theoretical model, particularly one that proposes a foundational shift, must do more than present a coherent internal logic. It must demonstrate its value by engaging with the existing empirical landscape and by generating novel, falsifiable predictions. This section addresses both requirements. First, we reinterpret several key experimental findings from cognitive science and neuroscience, arguing that the Sensation-Modulating Network (SMN) model offers a more parsimonious and powerful explanation. Second, we propose a set of new experimental designs derived directly from the core tenets of our model.

\subsection*{Reinterpreting Key Experimental Findings}
Many reproducible experimental results in cognitive science are not in dispute, but their interpretation is heavily dependent on the theoretical framework through which they are viewed. Here, we re-examine several landmark findings through the lens of the SMN.

\subsubsection*{Mirror Neurons}
The discovery of "mirror neurons"—neurons that fire both when an agent performs an action and when it observes another agent performing the same action—has been widely interpreted as the neural basis for understanding the intentions of others, or "mind-reading." The SMN model offers a more direct and less mysterious interpretation. We propose that observing an action triggers a low-amplitude, unsaturated simulation of that same action in the observer's own motor system. The firing of mirror neurons is not the "understanding" of an abstract intention; it is the physical, embodied re-enactment of the observed Haltable Action Pattern (HAP). The feeling of understanding arises as a phenomenological consequence of the observer's own body simulating the action. This view accounts for the data without positing a separate, abstract "intention-reading" module, grounding social cognition in the shared, resonant dynamics of embodied agents.

\subsection*{Novel Experimental Predictions of the SMN Model}
% TODO: Add novel experimental designs here.
