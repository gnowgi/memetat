\section{Empirical Grounding and Future Directions}
\label{sec:empirical}
A theoretical model, particularly one that proposes a foundational shift, must do more than present a coherent internal logic. It must demonstrate its value by engaging with the existing empirical landscape and by generating novel, falsifiable predictions. This section addresses both requirements. First, we reinterpret several key experimental findings from cognitive science and neuroscience, arguing that the Sensation-Modulating Network (SMN) model offers a more parsimonious and powerful explanation. Second, we propose a set of new experimental designs derived directly from the core tenets of our model.

\subsection{Reinterpreting Key Experimental Findings}
\label{subsec:reinterpreting}
Many reproducible experimental results in cognitive science are not in dispute, but their interpretation is heavily dependent on the theoretical framework through which they are viewed. Here, we re-examine several landmark findings through the lens of the SMN.

\subsubsection{Mirror Neurons}
\label{ssubsec:mirror_neurons}
The discovery of "mirror neurons"—neurons that fire both when an agent performs an action and when it observes another agent performing the same action—has been widely interpreted as the neural basis for understanding the intentions of others, or "mind-reading." The SMN model offers a more direct and less mysterious interpretation. We propose that observing an action triggers a low-amplitude, unsaturated simulation of that same action in the observer's own motor system. The firing of mirror neurons is not the "understanding" of an abstract intention; it is the physical, embodied re-enactment of the observed Haltable Action Pattern (HAP). The feeling of understanding arises as a phenomenological consequence of the observer's own body simulating the action. This view accounts for the data without positing a separate, abstract "intention-reading" module, grounding social cognition in the shared, resonant dynamics of embodied agents.

\subsubsection{Phantom Limbs}
\label{ssubsec:phantom_limbs}
The phenomenon of phantom limbs, where an amputee continues to feel the presence of a missing limb, is often explained as the persistence of a purely neural "body map" in the brain. The SMN model provides a more dynamic and embodied explanation. We argue that the deeply ingrained Fixed and Haltable Action Patterns associated with the limb do not simply vanish with the limb itself. These action schemas continue to run on the body's integrated network, generating a powerful and coherent phenomenological experience of the limb because, in our model, the modulated action pattern *is* the experience. The sensation is not a ghostly memory from a map, but the ongoing, active (though now unsaturated) presence of the limb's action repertoire in the global bodily circuit.

\subsubsection{The McGurk Effect}
\label{ssubsec:mcgurk}
The McGurk effect, where observing a speaker's lips form one phoneme (e.g., "ga") while hearing another (e.g., "ba") causes the observer to perceive a third phoneme (e.g., "da"), is a classic case of multi-modal integration. The SMN interprets this not as a simple blending of conflicting sensory data, but as a conflict between a sensory input (the sound) and an internally generated USHAP (the simulated action of producing the sound seen on the lips). The brain resolves the conflict by defaulting to the most coherent, embodied action pattern that could plausibly unite the two streams of information. This demonstrates the primacy of action-schemas in perception; what we perceive is not what we sense, but what we would have to *do* to make sense of our sensations.

\subsection{Novel Experimental Predictions of the SMN Model}
\label{subsec:predictions}
The true test of a scientific model is its ability to generate novel, falsifiable hypotheses. The SMN framework leads to several specific predictions that differ from those of traditional models.

\subsubsection{Testing the "Muscles as GPS" Hypothesis}
\label{ssubsec:gps_test}
As argued in Section \ref{ssubsec:muscles_space}, we claim that spatial computation is not a purely central or visual process but is fundamentally grounded in the musculature. This leads to a clear prediction: disrupting proprioceptive feedback from relevant muscle groups should significantly impair performance on spatial reasoning tasks, even when visual information is perfectly intact. This could be tested using non-invasive methods like targeted Transcranial Magnetic Stimulation (TMS) to temporarily disrupt the motor cortex regions corresponding to specific muscles, or by using techniques like tendon vibration to introduce noisy proprioceptive signals (see Figure \ref{fig:gps_experiment}). A finding that such manipulations degrade spatial, but not, for example, simple color-matching or arithmetic performance, would provide strong evidence for the role of the motor system as a primary organ of spatial computation.

\begin{figure}[ht]
    \centering
    % Placeholder for the actual graphic.
    \fbox{\parbox[c][10cm][c]{12cm}{\centering \Large \textbf{Figure 3: Proposed Experimental Setup} \\ \vspace{1cm} \large (A) Control: Subject performs a spatial reasoning task (e.g., judging distance). \\ \vspace{1cm} (B) Test Condition: Proprioceptive feedback from a key muscle group (e.g., neck or arm) is temporarily disrupted via non-invasive stimulation. \\ \vspace{1cm} \textbf{Prediction:} Performance on the spatial task will be significantly impaired in the test condition, even with full visual input.}}
    \caption{\textbf{Experimental Design to Test the "Muscles as GPS" Hypothesis.} The SMN model predicts that spatial reasoning is not a purely visual or central process but is computed via the motor system. This experiment aims to test that claim by measuring the impact of proprioceptive disruption on spatial judgment tasks. A significant performance drop would support the hypothesis that the musculature is a primary organ of spatial computation.}
    \label{fig:gps_experiment}
\end{figure}

\subsubsection{Correlating Action Syntax and Linguistic Syntax}
\label{ssubsec:syntax_test}
The SMN model posits that linguistic syntax is exapted from the more fundamental capacity for sequencing complex motor actions (action-syntax), as discussed in Section \ref{subsec:syntax}. This predicts a strong, causal, and developmental link between the two. A longitudinal study could track the emergence of complex, sequential, and hierarchical motor skills in infants—such as the ability to use one object as a tool to manipulate another—and correlate it with the emergence of syntactic complexity in their vocalizations and, later, language. The model predicts not just a correlation, but that the mastery of specific motor-syntactic structures will precede the mastery of the homologous linguistic-syntactic structures. This would support the claim that language is not an isolated module but is scaffolded upon the pre-existing combinatorial capacities of the motor system.