\subsection{Embodiment, Attunement, and the Geometry of Life}
One of the striking facts of biology is that animal morphology is not arbitrary — bodies appear to be exquisitely tuned to the physical milieu they inhabit. Whether swimming in water or crawling on land, animals confront forces of flow, drag, and gravity; in turn, the body must exploit, counteract, and encode those forces. The geometry of the environment — of currents, pressure gradients, gravity, surfaces, and obstacles — becomes internalized in the body’s structure.

But this alignment is not achieved by passive receptivity alone. To “understand” one’s surroundings is not simply to have sensors; it is to modulate, filter, and amplify sensations so as to disambiguate spatial structure. A fading, noisy signal must be sharpened. The better an organism’s sensorimotor architecture can differentiate spatial features — front vs. back, up vs. down, left vs. right — the more richly it can parse and act within its world. In that sense, anatomical differentiation is epistemic: a rudimentary, undifferentiated structure would yield only a blurred, undifferentiated world.

Thus, the very problem of decoding external space demands corresponding internal differentiation. As animals transitioned from aquatic to terrestrial habitats, their bodies needed to remap the geometry of hydrostatic equilibria into the geometry of solid supports, gravity, and substrate friction. In doing so, animals did not simply “add limbs” — they elaborated polarities, symmetries, segmentations, tubular networks, antagonistic muscle–skeletal systems, and sensory arrays calibrated to geometry. In short, the body is a “bot” released into the world, whose task is to discover the geometry of its surroundings and itself. If its morphological “toolbox” cannot resolve distinctions, it will lack cognitive access to those distinctions — the world remains opaque.

To situate this in a developmental-evolutionary (evo-devo) framework is crucial. Morphology evolves not by ad hoc appendage addition, but through modulating developmental regulatory networks (gene cascades, morphogen fields, patterning modules) under constraints of plasticity, robustness, and historical contingency (i.e. developmental constraints). The evolution of a cognitive geometry — a spatial understanding — is thus coextensive with the evolution of morphology and development. In what follows, we will show how polarity, segmentation, antagonism and an internal sensorimotor (SMN) body plan all arise as the “geometric scaffolding” by which the organism gradually carves a space of meaningful actable structures.

Animal body plans bear the imprint of a world first encountered as fluid and then as weight: streamlines and vortices in water, and later the shear and torque of terrestrial gravity, set the boundary conditions within which organisms must persist and act. Morphology thus reads like a record of negotiated constraints---fins and fusiform hulls that reduce drag, limb girdles and spines that resist bending and buckling, vestibular organs that encode linear acceleration including the pull of gravity \citep{lighthill1975mathematical,vogel1996life,angelaki2008vestibular}. The vertebrate water-to-land transition makes this concrete: the shift from buoyant propulsion to weight-bearing gait required wholesale re-engineering of levers, joints, and sensory reference frames \citep{clack2012gaining}.

On the cognitive side, perceiving such a world is not a passive uptake of “inputs” but a \emph{skilled} exploration governed by sensorimotor regularities. On ecological and enactive accounts, organisms become \emph{attuned} to environmental invariants and fields of affordances through active movement and practice; perception is constituted by mastery of sensorimotor contingencies rather than by internal snapshots \citep{gibson1979ecological,oregan2001sensorimotor,noe2004action,chemero2009radical,rietveld2014rich}. In 4E cognition (embodied, embedded, enactive, extended), this attunement is not post hoc “interpretation” of a given world but an ongoing achievement of organism–niche dynamics.

Evo–devo clarifies why attunement has a morphological backbone. Development canalizes flows of growth and differentiation into axes, segments, and tubes; modular architectures and conserved patterning systems bias variation toward functionally coherent novelties, thereby making bodies not only viable but \emph{evolvable} \citep{maynardsmith1985developmental,west-eberhard2003developmental,wagner1996evolvability,carroll2008evodevo}. In this picture, decoding space requires corresponding anatomical differentiation: undifferentiated tissues yield undifferentiated distinctions. By contrast, bilateral polarity, antagonistic actuators, articulated linkages, and distributed sensors carve the world into actionable dimensions—front/back, up/down, left/right—and into more abstract geodesics of approach, support, and constraint. Even at the level of movement primitives, classical neurophysiology showed that \emph{antagonistic} organization (excitation in one channel, inhibition in its opponent) furnishes the controllable degrees of freedom that make oriented action possible \citep{sherrington1906integrative,sherrington1905reciprocal}.

From an “engineering” standpoint, much of morphology is a set of computationally-helpful priors: compliant tissues, lever arms, sensory baselines, and canal geometries offload control and stabilize perception–action loops \citep{pfeifer2007body}. In this sense the body is a deployed \emph{bot} whose toolbox—axes, symmetries, segments, tubes, and antagonisms—discovers and exploits the geometry of the world. Where the toolbox cannot differentiate relevant differences, cognition has no grip; where it can, perceptual attunement and skilled performance become possible. The evo–devo story explains why such toolboxes are the way they are, and the 4E story explains how agents learn to use them.

Thus the evolved body plan is also an economic plan, because the organism doesn't have to do entire `computation' or memorize information in the body, it is off-loaded in the habitat.  The coupling between the body and the habitat makes this `bot' not merely distributed and decentralized, but an extended computer. 

We will now turn to some an important distinction between action and interaction, necessary for a cognitive `break'. 
