\subsection{Stochastic and Schematic Cognition (SMN Perspective)}
\label{subsec:stochastic_schematic}

\textit{Two complementary modes shape human cognition.} 
\textbf{Stochastic cognition} is adaptive, probabilistic, and context-sensitive; it exploits embodied dynamics and environmental regularities to guide action selection in real time. 
\textbf{Schematic cognition} is rule-governed and convention-mediated; it operates over explicit representations stabilized by social practices (language, mathematics, logic, code). 
On the SMN view, both modes are realized through \emph{haltable action patterns} (HAPs): the former tunes HAPs to ambient affordances; the latter trains HAPs for producing/decoding \emph{external traces} (texts, diagrams, symbols) that live outside the body even when transient (e.g., speech) \citep{PezzuloCisek2016Affordance,Donald1991Origins,ClarkChalmers1998Extended}.

\paragraph{Arbitrariness, convention, and literacy.}
Linguistic symbols are largely \emph{arbitrary}, stabilized by \emph{conventions} that enable efficient within-group coordination and partial opacity to outsiders \citep{Hockett1960OriginSpeech,Lewis1969Convention}. 
While spoken language can piggyback on stochastic learning and interactional scaffolding, \emph{literacy} is paradigmatically \emph{schematic}: it depends on explicit instruction and culturally recent \emph{neuronal recycling} of visual circuits \citep{Dehaene2009Reading,DehaeneCohen2007CulturalRecycling}. 
The shift from orality to literacy amplified schematic cognition by outsourcing memory to public media and tightening definitions to control ambiguity---from Euclid to Aristotle to Archimedes \citep{Donald1991Origins,Gardenfors2004ConceptualSpaces,Gardenfors2014Geometry}.

\paragraph{A braided view across STEAM.}
Most cultural practices are \emph{braids} of both modes:
\begin{itemize}
  \item \textbf{Engineering \& craft} lean stochastic (tacit skills, iteration, tuning) yet rely on schematic specifications and standards.
  \item \textbf{Science, technology, mathematics} lean schematic (definitions, models, proofs, code) yet depend on stochastic exploration, instrument handling, and intuition.
  \item \textbf{Arts} span a spectrum (folk $\rightarrow$ classical), often leveraging \emph{productive ambiguity} rather than eliminating it.
\end{itemize}
Recognizing the braid helps resolve stale debates: subcultures weigh the modes differently, but \emph{every} agent mixes both, task by task.

\paragraph{AI as mirror and foil.}
Classical AI and programmed computation approximate \emph{pure schematic} processing (explicit rules, symbols) \citep{NewellSimon1976Symbols}. 
Modern ML is dominantly \emph{stochastic} (pattern-based learning, emergent internal structure) \citep{Rumelhart1986Backprop,Lake2017ThinkLikePeople}. 
Unlike natural cognition, current AI lacks \emph{phenomenal} experience \citep{Chalmers1995FacingUp,Block1995OnConfusion}. 
From an SMN standpoint, embodied robots that exploit body--world \emph{dynamics} may achieve competence more cheaply than symbol-heavy pipelines, because they harness emergent control patterns rather than costly pre-specified representations.

\begin{table}[t]
  \centering
  \small
  \caption{Contrasting the two modes and their SMN links.}
  \label{tab:stochastic_vs_schematic}
  \begin{tabular}{p{0.23\textwidth} p{0.33\textwidth} p{0.33\textwidth}}
    \toprule
    \textbf{Aspect} & \textbf{Stochastic cognition} & \textbf{Schematic cognition} \\
    \midrule
    Control & Probabilistic, feedback-rich, embodied dynamics; online settling among affordances \citep{PezzuloCisek2016Affordance} & Rule-governed manipulation of explicit representations; conventions and norms \citep{Lewis1969Convention} \\
    Learning & Trial-and-error, tacit skills, statistical generalization \citep{Schmidt1975Schema,Tenenbaum2011GrowMind} & Instruction, definitions, formalization; deliberate practice \citep{Dehaene2009Reading} \\
    Traces & Often implicit/ephemeral; sensorimotor & Explicit, conventional symbols (texts, diagrams, code) \citep{Donald1991Origins} \\
    Error use & Variability and noise as resources for exploration & Error minimization via standards, proofs, verification \\
    Typical domains & Craft, improvisation, navigation, motor skills & Mathematics, logic, programming, formal modeling \citep{NewellSimon1976Symbols} \\
    SMN link & HAPs tuned to affordances; continuous control & HAPs for producing/decoding public traces; offloading to artifacts \citep{ClarkChalmers1998Extended} \\
    \bottomrule
  \end{tabular}
\end{table}

\paragraph{Implication for STEAM education.}
A curriculum aligned with SMN should: 
\begin{itemize}
  \item \textbf{Integrate modes}: couple stochastic \emph{making/doing} studios with schematic \emph{formalization} workshops.
  \item \textbf{Alternate artifacts}: move between embodied experiments, sketches, and formal notations to train the \emph{production/decoding} HAPs.
  \item \textbf{Leverage ambiguity}: in arts/humanities, treat multiple interpretations as a feature; in STEM, teach when and how to \emph{tighten} definitions.
\end{itemize}
This approach clarifies why humans, even when expert in formal domains, remain irreducibly mixed-mode agents: \emph{a live body enacting a symbolic ecology}.
