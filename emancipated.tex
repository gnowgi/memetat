\subsection{Predominently Emancipated Coordinated Zones and the Evolution of Human Cognition}\label{emancipated}

A core claim of this model is that the generativity of human syntax is not an abstract add-on to sensorimotor life but an exploitation of a specifically human expansion in \emph{coordinated action zones} that are both (i) \emph{haltable} and (ii) \emph{emancipated} from fixed pairings of antagonistic partners. Haltable action patterns (HAPs) create “gaps” in ongoing dynamics—temporal places where an unfolding movement can be suspended, redirected, nested, or recomposed \citep{Aron2014InhibitionKelso1995DynamicPatterns}. Yet halting alone does not yield productivity. Generative syntax also requires \emph{non-monotonous} recomposition—departures from a single, fixed continuation—so that partial sequences can be re-paired with alternative partners and re-paced across zones \citep{Lashley1951SerialOrder,Chomsky1957SyntacticStructures}.

We hypothesize that emancipated coordinated action zones (eCAZ) support \emph{switching} and \emph{alternative pairing} across antagonistic effectors and their neural controllers, enabling combinatorial “motor syntax” that is recursive, hierarchical, and context-sensitive \citep{Bizzi2005ModularControl,Koechlin2007PrefrontalHierarchy,Schieber2004HandSynergies}. This capacity piggybacks on conserved affordance-competition dynamics—multiple candidate actions concurrently specified and biased by task and context \citep{Cisek2010AffordanceCompetition,Pezzulo2016AffordanceLandscape}—but becomes distinctively human as cortical expansion and cortico-subcortical re-wiring increase the ease of pausing, re-targeting, and re-coupling action fragments \citep{PascualLeone2005CorticalPlasticity,Grafton2007MotorCognition}. Mandatory pulsations (e.g., cardiac rhythms) are paradigmatic non-HAPs and thus cannot participate in such recomposition.

On this view, the long-standing cognitivist critique—that enactive accounts cannot explain the systematicity and compositionality of symbols \citep{FodorPylyshyn1988Systematicity,Harnad1990SymbolGrounding}—is addressed by locating generativity in the embodied control architecture: eCAZ provide the physical substrate for non-monotonous, recursively recombinable HAP sequences. Linguistic syntax is then a culturally stabilized specialization of the same eCAZ machinery that underwrites human dexterity, tool-making, dance, and other art forms; genetic endowment furnishes the potential, while practice and culture actualize particular repertoires \citep{Laland2017DarwinCulture,HuttoMyin2013RadicalizingEnactivism}. The detailed mechanism is developed in Section~\ref{emancipated}.

Though this modification can be seen in other animals, in human beings there is a predominance of switching and alternative pairing. This not only explains why our bodies exhibit greater dexterity, playful actions, fooling actions, creative dance movements, generative and combinatorial action patterns.  

All the cultural practices human beings celebrate depend on this new found competence. Though the potential is genetic, its actualization is context dependent. Both engineering (tool use, fabrication, reshaping the world etc) and cultural practices depend on this feature. Thus, same mechanism explains not only generative syntax of languages, but also other art forms that human beings are known for. 

