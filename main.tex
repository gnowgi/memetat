\documentclass[10pt,letterpaper]{article}
\usepackage[top=0.85in,left=2.75in,footskip=0.75in,marginparwidth=2in]{geometry}

% use Unicode characters - try changing the option if you run into troubles with special characters (e.g. umlauts)
\usepackage[utf8]{inputenc}
\usepackage{comment}
% clean citations
\usepackage{cite}

% hyperref makes references clicky. use \url{www.example.com} or \href{www.example.com}{description} to add a clicky url
\usepackage{nameref,hyperref}

% line numbers
% \usepackage[right]{lineno}

% improves typesetting in LaTeX
\usepackage{microtype}
\DisableLigatures[f]{encoding = *, family = * }

% text layout - change as needed
\raggedright
\setlength{\parindent}{0.5cm}
\textwidth 5.25in 
\textheight 8.75in

% Remove % for double line spacing
%\usepackage{setspace} 
%\doublespacing

% use adjustwidth environment to exceed text width (see examples in text)
\usepackage{changepage}

% adjust caption style
\usepackage[aboveskip=1pt,labelfont=bf,labelsep=period,singlelinecheck=off]{caption}

% remove brackets from references
\makeatletter
\renewcommand{\@biblabel}[1]{\quad#1.}
\makeatother

% headrule, footrule and page numbers
\usepackage{lastpage,fancyhdr,graphicx}
\usepackage{epstopdf}
\pagestyle{myheadings}
\pagestyle{fancy}
\fancyhf{}
\rfoot{\thepage/\pageref{LastPage}}
\renewcommand{\footrule}{\hrule height 2pt \vspace{2mm}}
\fancyheadoffset[L]{2.25in}
\fancyfootoffset[L]{2.25in}

% use \textcolor{color}{text} for colored text (e.g. highlight to-do areas)
\usepackage{color}

% define custom colors (this one is for figure captions)
\definecolor{Gray}{gray}{.25}

% this is required to include graphics
\usepackage{graphicx}

% use if you want to put caption to the side of the figure - see example in text
\usepackage{sidecap}

% use for have text wrap around figures
\usepackage{wrapfig}
\usepackage[pscoord]{eso-pic}
\usepackage[fulladjust]{marginnote}
\reversemarginpar

% document begins here
\begin{document}
\vspace*{0.35in}

% title goes here:
\begin{flushleft}
{\Large
\textbf\newline{A Dynamic Architecture of a Cognitive Agent}
}
\newline
% authors go here:
\\
G.~Nagarjuna\textsuperscript{1},
Durga prasad Karnam\textsuperscript{2}
\\
\bigskip
\bf{1} Indian Institute of Science Education and Research, Pune
\\
\bf{2} Indian Institute of Technology, Mumbai
\\
\bigskip
* nagarjuna@iiserpune.ac.in

\end{flushleft}
\noindent \textbf{Long Title:} \textit{A Segmented, Polarized,
  Bilaterally Symmetrical, Antagonistically Organized Multizonal
  Coordinated Pairs in the form of a Layered Sensation Modulating
  Network Implementing Fixed, Haltable and Transactional Action
  Patterns of Conscious Semiotic Agents Accouncing Subjective
  Experience and Generative Syntax}

\subsection*{Abstract}
This paper addresses the long-standing impasse between cognitivist and enactive approaches by proposing a new foundational model for cognitive science. We argue that this division stems from an incomplete understanding of the agent's biological architecture. To resolve this, we introduce the Sensation-Modulating Network (SMN), a model of the whole agent grounded in a specific set of general biological principles: a segmented, polarized, bilaterally symmetrical, and antagonistically organized body plan. Our central thesis is that cognition is not computation but action-modulation. We demonstrate how the capacity to halt ongoing, rhythmic action patterns—giving rise to a hierarchy of Fixed, Haltable, and Transactional Action Patterns (FAPs, HAPs, and TAPs)—is the fundamental cognitive act. This framework explains the emergence of phenomenological experience, the grounding of symbols via "unsaturated" actions, and the origins of generative syntax from the body's own combinatorial structure. By redefining cognition in this way, the SMN provides a unified, biologically-grounded account of consciousness, semiotics, and language.

\bigskip
\noindent \textbf{Keywords:} cognitive architecture, enactivism, symbol grounding, generative syntax, phenomenology, semiotics, action patterns, Sensation-Modulating Network (SMN)
\section{Introduction}
\label{sec:introduction}
The study of cognition is marked by a fundamental tension between seemingly irreconcilable research programs. On one side, cognitivism, rooted in the information processing paradigm, seeks to explain phenomena like language and reasoning through amodal symbols and computational rules, often located within the brain \cite{chomsky1965aspects, fodor_modularity_1983}. On the other, 4E (embodied, embedded, enactive, and extended) approaches ground cognition in the dynamic, situated actions of an agent's entire body within its environment \cite{varela1991embodied,  maturana1991autopoiesis, noe_action_2004}. While both camps agree on a evolutionary and biological basis for cognition, their foundational assumptions about the nature of the body and its role in mental life have led to decades of disagreement. This paper argues that this impasse stems from an incomplete understanding of the biological architecture that underpins cognition. We proceed from the assumption that a correct model of this architecture is the necessary foundation upon which any successful theory of cognition must be built.

We propose that the core challenge for cognitive science is to answer a set of interrelated questions: (1) What structures and dynamics are necessary for phenomenological experience—the so-called "hard problem"—to arise? (2) How does a cognitive agent construct a stable, geometric picture of its internal and external worlds? (3) How can this generated picture be shared meaningfully with other agents? To address these, we depart from neuro-centric models and introduce an alternative framework centered on a specific, yet generalizable, model of a cognitive agent's body plan.

Our central proposal is a dynamic architecture we term the Sensation-Modulating Network (SMN). We model the agent as a layered network of action zones, organized according to fundamental biological principles: segmented, polarized, bilaterally symmetrical, and antagonistically organized. Within this architecture, cognition emerges not from a central controller initiating action, but from the capacity to \textit{halt} and modulate ongoing, rhythmic action patterns. To describe these dynamics, we will introduce a specific terminology: a hierarchy of patterns ranging from deep, immutable \textbf{Fixed Action Patterns (FAPs)}, to consciously accessible \textbf{Haltable Action Patterns (HAPs)}, which, when shared, become \textbf{Transactional Action Patterns (TAPs)}. It is this capacity for modulation, we argue, that grounds semiotics\cite{peirce1992essential}, enables the creation of symbols, and gives rise to generative syntax\cite{chomsky1965aspects}.

This paper unfolds in five parts. First, we will elaborate on the proposed SMN model, detailing its architectural principles and its dynamic properties. Second, we will demonstrate how this model can explain a wide range of cognitive phenomena, from subjective experience to the emergence of abstract concepts. Third, we will situate our model within the broader theoretical landscape, comparing its core tenets and explanatory power to other leading theories of cognition. Fourth, we will reinterpret existing experimental evidence and propose novel, falsifiable predictions derived from our framework. Finally, we will conclude by discussing the broader implications of this model, arguing that it offers a foundational departure from previous approaches and provides a path toward reconciling the long-standing divisions in the field.

Among the implications of this proposal include: a re-characterization of being human. We shall demonstrate that the ``mechanics'' of cogntition are not \textit{peculiar} to human beings, while certain aspects of them are \textit{predominant}. The model also provides an explanation of why machine learning is so \textit{expensive}, data and memory driven, while humans can perform those operations at room temperature. Unlike Turing's model of computation that depends on reading and writing symbols on almost \textit{infinite} memory, we provide an alternate approach to computation that does not store tokens, but generates them, in the body, and interprets them on the fly using the schema in the form of action patterns. The schema are not fixed, when formed, they get revised or destroyed, based on how well they support \textit{predictive processing}. Thus these are not \textit{rational} schema, but \textit{stochastic} in nature. Though human reason \textit{appears} to be akin to bounded rationality, it is so only when we play the rule based game of formal semantics. It is not the body that is bound to rationality but the transactional cutlure of academics. Recent success stories of modern machine learning support this insight that rule bound rationality emerges after several trials and errors, while the stochastic behavior is the foundation. 
\section{The Proposed Model: A Dynamic Architecture}
\label{sec:model}
At the heart of our proposal is the Sensation-Modulating Network (SMN), a framework for understanding the cognitive agent as a whole, rather than as a brain-centric processing unit. The SMN is not a specific organ but the entire functional architecture of the agent, defined by a set of core biological design principles. These principles, while ubiquitous in biology, have been largely overlooked in their cognitive implications.

\subsection{The Nature of Action: A Thermodynamic Distinction}
\label{subsec:action_nature}
Before detailing the architecture of the Sensation-Modulating Network (SMN), we must make a foundational ontological distinction between *action* and *interaction*. While all actions are a form of interaction, not all interactions are actions. In the context of this model, an **interaction** is a conserved process governed by symmetrical physical laws. An **action**, by contrast, is a local, symmetry-breaking perturbation performed by a far-from-equilibrium system \cite{prigogine2018order}. It is a thermodynamic process: the agent must expend energy to initiate, sustain, and modulate its actions. This distinction is crucial for an enactive model, as it defines the agent as a being that actively creates asymmetries in the world, rather than as a passive object merely being pushed and pulled by external forces. Cognition, in this view, is the process of managing these energy-dependent, symmetry-breaking events.

\subsection{Architectural Principles of the SMN}
\label{subsec:architectural_principles}
We model the agent's body as a topologically tubular structure possessing four key properties:
\begin{enumerate}
    \item \textbf{Polarity:} The body has a defined axis, typically from anterior to posterior, which establishes a fundamental directionality for movement and interaction with the environment.
    \item \textbf{Metameric Segmentation:} The body is composed of repeating segments or action zones (e.g., limbs, digits, vocal apparatus). Each zone is a locus of potential action.
    \item \textbf{Bilateral Symmetry:} The body is organized symmetrically around a central axis, creating pairs of coordinated structures.
    \item \textbf{Antagonistic Organization:} Action within and between zones is governed by antagonistic pairs (e.g., flexion and extension). This push-pull dynamic is the fundamental basis of control and modulation.
\end{enumerate}

\begin{figure}[ht]
    \centering
    % Placeholder for the actual graphic.
    % We can use a simple \fbox or \includegraphics later.
    \fbox{\parbox[c][10cm][c]{12cm}{\centering \Large \textbf{Figure 1: Diagram of the SMN Architecture} \ \vspace{1cm} \large (A) Polarity (Anterior-Posterior Axis) \ \vspace{0.5cm} (B) Metameric Segmentation (Action Zones) \ \vspace{0.5cm} (C) Bilateral Symmetry \ \vspace{0.5cm} (D) Antagonistic Pairs}}
    \caption{\textbf{The Architectural Principles of the Sensation-Modulating Network (SMN).} The agent's body is modeled as a topological structure with four key properties. \textbf{(A)} A defined polarity provides directionality. \textbf{(B)} The body is composed of repeating action zones (e.g., limbs, digits). \textbf{(C)} These zones are organized symmetrically around a central axis. \textbf{(D)} Action within each zone is governed by antagonistic pairs. This architecture provides the foundation for all higher-order cognitive dynamics.}
    \label{fig:smn_architecture}
\end{figure}

This architectural plan provides the agent with a multitude of "action zones," each a dynamical system capable of generating rhythmic patterns. The core cognitive faculty, we argue, arises from the agent's ability to manage these patterns.


\subsubsection{The SMN in a Gravitational Field}
\label{ssubsec:gravity}
A foundational oversight in many cognitive models, particularly those that are heavily neuro-centric, is the treatment of the environment as a passive problem-space that places a computational burden on the brain. Even ecological theories, which rightly situate the agent in its environment, have not fully accounted for the constitutive role that fundamental physical forces play in cognition. Our framework begins by asserting that the agent’s body is not merely *in* an environment but is dynamically shaped *by* it. The Sensation-Modulating Network (SMN) is therefore defined, first and foremost, as a system that has evolved to actively and continuously counteract the planet's gravitational field.

This is not a trivial point. Gravity is not a bug to be fixed or a variable to be solved for; it is a constant, predictable, and non-negotiable partner in every action. The entire architecture of the agent—its antagonistic muscle pairs, its skeletal structure, its vestibular system—is a testament to this partnership. This allows for a radical offloading of computation. The agent does not need to store vast amounts of "data" about the world. Instead, it develops and refines "data structures" in the form of action schemas. The stability and predictability of the gravitational field provide a constant, reliable feedback mechanism against which these schemas are calibrated.

This leads to a crucial distinction between biological cognition and the detached, symbolic computation of the machines we have built. An artificial system must be fed data, store it, and run explicit procedures on it—an inefficient process that requires immense energy. The SMN, by contrast, operates primarily in a "saturated mode." When an agent walks, the ground pushes back with every step; when it swims, the fluid resists and supports every movement. The rich, real-time feedback from the world is an ineliminable part of the computational loop. Because the "data" remains external, the agent's internal work is lean, efficient, and possible at room temperature.

Therefore, actions like walking and swimming are not merely locomotion; they are profound epistemic acts. They are how the agent constructs a geometric model of its world, using its own body and the constant of gravity as its measuring instruments. As these actions modulate the agent's sensory subsystems in response to the affordances of the environment, the agent "grasps" the world—not by representing it internally, but by continuously testing and refining its possibilities for action within it. The gravitational field is thus not an incidental feature of our world, but a fundamental and active component of our cognitive architecture.

\subsection{The Primacy of Halting}
\label{subsec:halting}
Contra the received view that the nervous system's primary role is to initiate action, we propose its crucial function for cognition is to \textit{alter} and, most importantly, \textit{halt} ongoing action patterns. We proceed from the foundational assumption that rhythmic, patterned activity is the default, baseline state of biological tissue, a principle observed from the ciliary action in prokaryotes to the emergent synchrony of cardiac cells \cite{landecker2007culturing}. The challenge for a complex agent is not to start moving, but to control, pause, and modulate its intrinsic dynamics. This capacity for halting is what provides the freedom to deviate from fixed trajectories, enabling exploration, deliberation, and the generation of phenomenological experience. A pause in an action sequence is not a lack of activity, but a cognitive act itself—one that creates a space for awareness and choice.

\subsubsection{Haltability and Ecological Affordances}
\label{ssubsec:affordances}
Just as the gravitational field provides a constant, predictable partner in computation, the specific objects and surfaces within the environment provide a rich structure of opportunities for action. Following Gibson, we term these opportunities "affordances" \cite{gibson1977-vh}. A flat, rigid surface affords support for walking; a handle affords grasping; a gap affords leaping across. The concept of haltability is deeply intertwined with these affordances. An agent does not simply execute a fixed motor program; it initiates a Haltable Action Pattern (HAP) in the direction of an affordance, and the specific properties of the environment provide continuous feedback that shapes the action in real-time.

This creates a reciprocal, co-defining relationship. The agent’s capacity to halt and modulate its actions allows it to selectively engage with the world’s affordances, to explore them without being locked into an irreversible action sequence. In turn, the structured nature of these affordances reinforces and refines the agent’s repertoire of HAPs. For example, a child learning to grasp discovers that a soft toy affords a different kind of modulated pressure than a hard wooden block. The environment does not merely trigger an action; it actively teaches the agent *how* to halt and shape that action appropriately. The affordance, therefore, becomes an integral part of the action's control loop.

Haltability is thus not a purely internal capacity but a relational one. It is the agent's contribution to a dynamic dance with the environment. The world offers possibilities for action, and the agent's ability to pause, adjust, and sequence its HAPs is what allows it to navigate these possibilities, transforming a field of potential affordances into a meaningful, enacted world.

\subsubsection{Closed-Loop Actions and Internal Saturation}
\label{ssubsec:closed_loop}
While many Haltable Action Patterns (HAPs) are directed at the external world, a crucial subset of actions involves the body acting upon itself. These reflexive, closed-loop actions—such as licking, sucking, or scratching—are not unsaturated mimes. On the contrary, they are fully **saturated** actions, because the agent's own body serves as the complex, responsive object. In these cases, the distinction between subject and object blurs, and the Sensation-Modulating Network (SMN) operates in a tight, self-referential loop.

This capacity for internal saturation is a foundational attribute of cognition. It allows the agent to generate rich, structured phenomenological experiences without any input from the external world. These actions are often highly gratifying, creating powerful feedback loops that motivate their repetition. This can manifest as fidgeting or other self-stimulatory behaviors, which are not meaningless tics but are instead the SMN actively exploring its own internal affordances and maintaining a state of dynamic equilibrium. These internally-directed, saturated actions are a vital precursor to fully delinked, unsaturated thought, providing a training ground where the agent learns to modulate its own sensory states directly.

\subsection{A Hierarchy of Action Patterns}
\label{subsec:hierarchy}
The SMN's dynamics give rise to a hierarchy of action patterns, distinguished by their degree of modulatability:
\paragraph{Fixed Action Patterns (FAPs)} These are deep, phylogenetically old, and largely involuntary rhythmic patterns that form the foundation of the agent's being (e.g., heartbeat, respiration, peristalsis). They are not directly accessible to conscious modulation and constitute the stable background or "cognitive canvas" upon which experience is drawn.

\paragraph{Haltable Action Patterns (HAPs)} These are actions that can be consciously initiated, paused, and modulated. They are performed by the outer, more flexible layers of the SMN (e.g., reaching, grasping, vocalizing). The ability to halt a HAP is the basis for creating discrete, repeatable, and therefore tokenizable, units of action.

\paragraph{Transactional Action Patterns (TAPs)} When HAPs are performed in a social context, they become TAPs. These are actions that are either directed at another agent or are imitated, forming the basis of communication, shared practices, and cultural learning. TAPs are the building blocks of intersubjective meaning.

\subsection{The Differentiating and Integrating Networks}
\label{subsec:networks}
The SMN is a layered architecture that differentiates and integrates sensations to produce a unified experience. This is achieved through the complementary roles of the body's motor and nervous systems.

We propose that the motor system—the entire musculature—is the primary \textbf{Differentiating and Filtering Network (DFN)}. It is not a mere output device but the very organ of differentiation. Each muscle group, as an action zone, is a mediator between the internal and external worlds. When a muscle contracts or relaxes, it creates a distinction; the specific tension and motion *is* the differentiated sensation. This motor activity informs the rest of the body about its state in relation to both the external world and the internal milieu.

However, differentiation alone is insufficient. For a unified experience to emerge, these local states must be brought together. This is the role of the nervous system as the \textbf{Integrating Network (IN)}. Contra the view of the CNS as a central controller, we posit its primary function is message-passing and broadcasting. It acts as a high-speed conduit, integrating the foreground of attention (driven by HAPs) with the background hum of the body (driven by FAPs). In an antagonistically organized body, the IN ensures that coordinating zones receive the necessary information to manage their push-pull dynamics effectively.

This architecture provides a natural mechanism for grounding concepts. A HAP is initially \textbf{saturated} by a physical object. However, the agent can re-enact the pattern without the object, creating an \textbf{unsaturated HAP (USHAP)}. These USHAPs—delinked from the world but still rooted in the SMN—are the raw material of concepts, simulations, and imagination, thus resolving the classical symbol-grounding problem.

\subsection{Formal Analogies: The Mathematical Grounding of the SMN}
\label{subsec:formal_analogies}
While the SMN is presented as a descriptive biological model, its core principles can be rigorously grounded in the formal language of mathematics. These analogies are not merely illustrative; they demonstrate the model's potential for formalization and computational implementation, showing how mathematics provides the ultimate data structures for describing parsimonious action patterns.

\subsubsection{Structure and Transformation: Topology, Graph, and Category Theory}
\label{ssubsec:formal_structure}
The overall architecture of the agent is fundamentally topological; its properties as a polarized, segmented, tubular structure are invariants that define its basic form. Within this topology, the SMN can be precisely described using **Graph Theory**, where action zones are nodes and the neural and physical connections are edges, allowing for the formal analysis of information flow. More broadly, **Category Theory** provides a powerful language for the entire system, framing the SMN as a category whose 'objects' are the agent's components and whose 'morphisms' are the actions that transform the state of the agent and its relation to the world.

\subsubsection{Dynamics and Control: Dynamical Systems and Control Theory}
\label{ssubsec:formal_dynamics}
Each action zone is a **Dynamical System**, possessing attractors that represent stable, rhythmic action patterns. The coordination between zones, particularly the management of antagonistic pairs, is a classic problem of **Control Theory**, where stability is maintained through nested feedback loops. The agent's ability to modulate its actions can be modeled as a control process that shifts the system between different stable states.

\subsubsection{Action and Information: Group Theory, Petri Nets, and Information Theory}
\label{ssubsec:formal_action}
The structure of action itself can be formalized using **Group Theory**. The set of HAPs forms a group where sequencing is the operation, a sustained halt is the identity element, and reversing an action is the inverse. This captures the deep structure of action schemas. The flow of action and sensation is elegantly modeled by **Petri Nets**, where the motor system (DFN) provides the 'transitions' that change the system's state, and the nervous system (IN) acts as the 'places' that hold the resulting state-information as tokens. Finally, **Information Theory** formalizes the primacy of halting: a continuous dynamic has low information content, and a halt is a symmetry-breaking event that resolves uncertainty, creating a 'bit' of information.

\subsubsection{Phenomenology and Inference: Signal Processing and Bayesian Inference}
\label{ssubsec:formal_phenomenology}
The nature of phenomenological experience finds a powerful analogy in **Signal Processing Theory**. The body's constant FAPs can be seen as a high-frequency 'carrier wave' of being. A conscious experience, driven by a HAP, is a 'modulating signal' that impresses a lower-frequency pattern of information onto this wave. The resulting complex waveform is the experience itself. This aligns with the framework of **Bayesian Inference** and the Free Energy Principle, where the agent is modeled as a system that acts to minimize surprise. The HAPs are precisely the 'active inferences' the agent performs on its world, updating its internal model to reduce uncertainty and maintain its own structural integrity.

\section{Explaining Cognitive Phenomena and Reinterpreting Evidence}
\label{sec:phenomena}
Having established the architectural foundations of the SMN model in Section~\ref{sec:model}, we now demonstrate how this framework provides a unified account of cognitive phenomena that have long puzzled researchers. The explanatory power of the Sensation-Modulating Network (SMN) lies in its capacity to reframe fundamental cognitive questions in terms of action dynamics. By grounding cognition in a specific, yet generalizable, bodily architecture, the model offers a unified account of phenomenology, representation, and syntax, while reinterpreting existing empirical evidence.

This section addresses the three core questions posed in the introduction: (1) how phenomenological experience arises from the SMN architecture, (2) how agents construct geometric pictures of their worlds, and (3) how these pictures can be shared meaningfully with other agents. We will see that the SMN model provides concrete answers to these questions through detailed examples and mechanisms.

Before addressing these core questions, we must first solve what we call the \textit{naming-framing problem}—a fundamental challenge that any cognitive theory must address. This problem combines two classical puzzles in cognitive science: the frame problem and the symbol grounding problem.

\subsection{The Naming-Framing Problem}
\label{subsec:naming_framing}

\subsubsection{The Framing Problem}
A stream of uninterrupted sensations may have a pattern, and it can be processed to identify the pattern. Here we are assuming that the existence of a pattern and its identification is the essence of information processing. It can be done by our modern computers, which are capable of recognizing patterns and establishing correlations as well. But this is an ungrounded process, because, \textit{on its own} the machine cannot detect: Where does the stream come from, from within the machine or outside? Which pattern to attend to? Which pattern is more significant than the other? Which ones to ignore? This is the framing problem. This problem is related to how a cognitive agent understands the context/situation.

Recognition of patterns and relating the patterns to a context is not sufficient for grounding semiotics. We now turn our attention to the formulation and resolution of naming-framing problem.

\subsubsection{The Naming Problem}
Apart from the understanding of context, we need to deal with naming and referencing. Information processing is not only impossible without names but also useless if it cannot name the patterns. One may program the machine to give distinct names to distinct patterns, and can also group them nicely based on some similarity detecting algorithms. It may also develop a complex ontology at the end. But how does the machine establish a reference, of which patterns pertains to which object in the world, or within the machine? And more importantly how could it convey to us the \textit{private} naming convention? Can the machine at least tell them to itself? If so how? These are three philosophical problems at one go: concept formation, symbol grounding problem and that of possibility or impossibility of a private language. Variation in the world and a capacity to detect patterns in a stream of experience is not sufficient to give names to the patterns. We need a naming mechanism in a cognitive agent, and we need a mechanism to communicate with each other, in a community of agents, through names. Let's call this \textit{the naming problem}.

Combining the above two problems, let us call them together a \textit{naming-framing problem}, because we think that the frame problem and symbol grounding problem are intrinsically related. We will now argue that the naming-framing problem can be solved through modulation. In other words, referencing and distinguishing the external world from the internal world will be shown to be possible through the same mechanism.

A world that is homogeneous is no world at all. It offers nothing to know. There are no affordances to offer. Nothing to act on or interact with. For the argument's sake, if we place a cognitive agent in such a homogeneous world, it will not even know if there is a stream at all in her experience, forget about detecting or identifying patterns in them. So, we assume that the world is heterogeneous.

Let's place a community of cognitive agents, with the structure and dynamics specified in the $\{n(SMN)\}$ model presented above, in a heterogeneous world. We shall now present how such an agent can make sense of the world, and what else we need to develop language and culture. As mentioned in the introduction, the framework that is proposed here is based on an ontology of actions, and not interactions. The model of a cognitive agent presented in the above section describes the body as a network of modulators. A modulator in the scheme is the location of the action. We will now account for gaps in actions, and their role in the naming-framing problem.

\subsection{From Action Modulation to Phenomenology}
\label{subsec:phenomenology}
The SMN model addresses the "hard problem" of consciousness by proposing that subjective experience is the agent's perception of its own modulated actions. The constant, rhythmic hum of the Fixed Action Patterns (FAPs) creates a stable, non-conscious background—a "cognitive canvas." A phenomenological event occurs when a Haltable Action Pattern (HAP) is initiated, modulated, or, most critically, paused. This change against the stable background *is* the experience. The "what it is like" to see red is the specific pattern of oculomotor and neural HAPs enacted to foveate on a red object; the feeling of thirst is the modulation of interoceptive patterns against the homeostatic background. Consciousness is not a substance or a passive "theater" but an active process of self-differentiation through action.

\subsubsection{Action as the Origin of Phenomenal Experience}
\label{ssubsec:action_origin}
In the Sensation-Modulating Network (SMN) framework, a phenomenological experience is not a passive event, such as the mere reception of sensory stimuli. Instead, it is an active construct, a direct consequence of the agent's own self-initiated actions. The process begins when the agent deploys a Haltable Action Pattern (HAP)—a voluntary, modulated action directed towards an affordance in the environment. This action, be it a saccade of the eye, a turn of the head, or the extension of a hand, is not a reaction to a stimulus but is itself the stimulus that generates the experience.

When the agent acts, it necessarily alters the flow of sensory information across its entire body. The HAP actively modulates this sensory stream, creating a specific, transient pattern of differentiation against the stable, homeostatic background provided by the Fixed Action Patterns (FAPs). The "phenomenological response" is precisely this internally-generated, action-driven pattern. For example, the experience of touching a rough surface is not caused by the surface's texture alone; it is constructed by the specific HAP of moving one's fingers across it, which generates a unique pattern of vibrations and pressures. The action and the sensation are inextricable.

Therefore, the agent's world is not something that is revealed to it, but something it brings forth through its own activity. Each self-initiated action is a question posed to the environment, and the resulting modulation of the agent's own sensory state is the answer. The senses provide a continuous, undifferentiated stream of differences; it is the motor act that breaks this stream into discrete "snapshots," thereby differentiating the difference in a kind of double derivative. Our mental world is this stream of action-generated snapshots, a process that ceases the moment we stop acting. This continuous loop of action-based self-stimulation is the very engine of conscious experience, grounding phenomenology not in a mysterious inner theater, but in the concrete, dynamic process of a body actively engaging with its world.

\subsection{Constructing a Geometric and Semiotic World}
\label{subsec:semiotic_world}
Having established how phenomenological experience arises from action modulation (Section~\ref{subsec:phenomenology}), we now turn to the second core question: how agents construct geometric pictures of their worlds. The agent's world is not a pre-given, objective space that is passively perceived, but a geometric and semiotic habitat that is constructed through action. The SMN computes this geometry through the coordination of its multiple action zones. For instance, the distance to an object is not calculated from retinal size alone, but is enacted through a fusion of HAPs: the proprioceptive feel of reaching a hand, the muscular strain of focusing the eyes, and the time delay between a sound and its echo. The world's geometry is mapped onto the agent's own bodily geometry and action capabilities.

Meaning arises as this geometric world becomes populated with signs. An object's affordances are the saturated HAPs it invites. A rock affords throwing, sitting, or striking. These saturated HAPs are the object's initial meaning for the agent. When these actions are tokenized as unsaturated HAPs (USHAPs), they become concepts that can be manipulated internally. When they are shared and imitated as Transactional Action Patterns (TAPs), they become external, public symbols. This provides a direct, embodied route from perception to semiotics, grounding the Peircean triad of object, representamen (the action pattern), and interpretant (the resulting experience or subsequent action) in the dynamics of the SMN.

\begin{figure}[ht]
    \centering
    % Placeholder for the actual graphic.
    \fbox{\parbox[c][12cm][c]{12cm}{\centering \Large \textbf{Figure 2: From Saturated Action to Unsaturated Concept} \ \vspace{1cm} \large (A) Saturated HAP: Agent grasps a real cup. \ \textit{Action is constrained by object.} \ \vspace{1cm} (B) Unsaturated HAP (USHAP): Agent mimes grasping. \ \textit{Action pattern is delinked from object.} \ \vspace{1cm} (C) Transactional Action Pattern (TAP): Agent uses gesture to signify "drink" to another agent. \ \textit{USHAP becomes a shared symbol.}}}
    \caption{\textbf{The Grounding of Concepts in Action.} This figure illustrates the progression from concrete action to abstract representation. \textbf{(A)} A **Saturated HAP** is a direct, physical interaction with an object. \textbf{(B)} An **Unsaturated HAP (USHAP)** is the re-enactment of the action's *pattern* without the object, forming the basis of an internal concept. \textbf{(C)} A **Transactional Action Pattern (TAP)** occurs when the USHAP is externalized and used in a social context, becoming a shared symbol.}
    \label{fig:saturation_spectrum}
\end{figure}

\subsubsection{Tokenization through Action}
\label{ssubsec:tokenization}
For any symbolic or computational system to emerge, there must be a process for creating discrete, stable units—tokens—that can be reliably identified and manipulated. In a purely dynamic, continuous system, this is a non-trivial problem. The Sensation-Modulating Network (SMN) solves this by leveraging the core faculty of haltability. A Haltable Action Pattern (HAP) is not merely a continuous movement; it is a bounded, repeatable, and recursive event that functions as a proto-symbolic token.

The creation of a token begins with the agent's ability to initiate and terminate an action at will. This act of "framing" an action—giving it a clear beginning and end—is what carves it out from the undifferentiated stream of activity. A grasp, a step, or a vocalization becomes a discrete unit precisely because it can be started, stopped, and, crucially, repeated. This repeatability gives the action-token a stable identity. The agent can perform the "grasp" action now, and then perform the "same" action again later, treating it as an instance of a type.

Consider the example of talking about a cup by hand-grasping-a-cup action-pattern (gesture). The possibility of grasping action-pattern \textit{without an object} (in this example, a cup) is a significant bifurcation point and an entry into semiotic world. The miming action of grasping an object can become a \textit{name} for objects. The mime for holding a pen, brush, liquid, cup, basket etc. could all be different, based on the affordances these objects offer to the agent. Grasping any object is saturated, while a mime of grasping without the object is unsaturated.

If the action patterns are always \textit{saturated} with the object or event, they can never become names for them. Naming is impossible without breaking this contingent binding. We therefore consider unsaturated action-patterns as a necessary condition for a semiotic life, where naming action-patterns are separated from the object they stand for.

An action-pattern could be bound to a reference in a \textit{hard} or \textit{soft} manner. Harder binding specifically applies largely to gestures (inter-subjectively presented action-patterns). The action-pattern used as a mime, when closely related to the affordances offered by the object or event, are harder. Some mimes transcend the affordances of the object or an event, since they hold no morphological or functional correspondence to them. For example, in a typical Indian classroom, when a student stands-up in the middle of the class and shows his/her little finger, the teacher as well as the rest of the class understand that the student is seeking permission for a bio-break. This may not work in another culture. Because the binding between the little-finger-mime and seeking permission for a bio-break is created \textit{arbitrarily} without any match with the affordances. Whereas using a thumbs up mime to seek permission or a hydration-break is less arbitrary and matches with the affordance of drinking water.

While the possibility of hard-mimes could be a major bifurcation point for communicating agents, the use of soft-mimes for communication is a revolutionary bifurcation point, because this breaks open a world of possibilities. We think that this could be the episode of punctuated equilibrium\cite{gould1977punctuated} in the evolution of hominids. The communities that could use arbitrary mimes (names) had a political and economic advantage over other communities, because arbitrary names gives rise to proprietary/ private (closed group) languages.\cite{corballis2014recursive}

Furthermore, these action-tokens are inherently recursive and modulatable. A HAP can be embedded within another, as the HAP of wiggling a finger is part of the larger HAP of grasping an object. This nested, part-whole structure is a foundational precursor to syntactic recursion. The segmented architecture of the SMN provides a natural "lexicon" of potential tokens (the set of actions afforded by hands, limbs, mouth, etc.), while the agent's fine motor control allows for subtle modulations of these tokens—a gentle grasp versus a firm grasp—which function like inflections of a core verb.

Through this process, the agent transforms its continuous, dynamic engagement with the world into a set of discrete, manipulable, and meaningful units. These are not abstract, amodal symbols that stand *for* the world; they are the very patterns of the agent's meaningful interaction *with* the world, packaged into a form that can be used for the higher-order cognitive processes that follow.

\subsubsection{Grounding Tokens in the Graspable Situation}
\label{ssubsec:grounding}
An action-token, as a discrete and repeatable pattern, does not possess meaning in isolation. Its significance is derived from its context—the "graspable situation" in which it is performed. This grounding is not a mysterious semantic process but a physical one, rooted in how action schemas reconfigure the agent's neural network. Following a Hebbian model \cite{hebb1949organization}, the more an action is repeated in response to a specific ecological affordance, the more the neural pathways involved in that action-perception loop are strengthened. "Neurons that fire together, wire together," creating a durable linkage between the agent, the action, and the object.

These learned, reconfigured neural pathways are the very embodiment of conceptual schemes. In the parlance of computer science, the agent is not storing "data" about the world; it is forging the "data structures" through which the world can be understood. Each of these structures is inherently propositional. The agent (the subject) executes a differentiating action pattern (the predicate or verb phrase) upon an affordance (the object). This interaction naturally constructs a `subject-predicate-object` triple, the fundamental atom of meaning. For instance, the agent (`I`) performs the action (`grasp`) on the object (`cup`).

Crucially, this process does not result in a list of detached, atomic sentences. Because the same agent performs many actions, and the same object affords multiple actions, these triples share nodes. The "cup" node is linked not only to "grasp" but also to "lift," "drink from," and "see." The "I" node is linked to every action the agent can perform. The result is a vast, interconnected, graph-theoretic data model. In this framework, knowledge is not a collection of facts to be processed but *is* the very structure of this network. It is a dynamic, relational, and embodied web of possibilities, continuously shaped by the agent's ongoing engagement with its world.

\subsubsection{Musculature as the Organ of Spatial Computation}
\label{ssubsec:muscles_space}
Just as the motor system is the primary organ of differentiation, it is also the primary organ of spatial computation. Space is not a pre-existing, absolute container that the brain passively represents, but a fundamental category of experience that is actively constructed by the musculature. The body's network of muscles functions as a biological Global Positioning System (GPS), continuously resolving questions of distance, displacement, and orientation through action. This view aligns with the enactive approach to perception, which argues that perception is not something that happens to us, but a skillful activity that we perform \cite{noe_action_2004}.

This framework leads to a strong, falsifiable claim: in the absence of movement, or at least the potential for movement, visuo-spatial construction of the world is impossible. An immobilized agent cannot truly perceive depth or distance because it cannot perform the epistemic acts—the reaching, walking, and turning of the head—that measure the world. The proprioceptive feedback from a stretched arm and the vestibular sensations from a tilted head are not secondary data points that confirm a visual hypothesis; they are the very substance of the spatial calculation. It is the felt effort of muscular action that provides the fundamental metric for space. Therefore, the category of space is not provided by purely neuronal means but is a direct phenomenal consequence of having a body with a specific musculoskeletal architecture acting in the world.

\subsection{The Embodied Origins of Generative Syntax}
\label{subsec:syntax}
Having shown how agents construct geometric and semiotic worlds through action patterns (Section~\ref{subsec:semiotic_world}), we now address the third core question: how can these constructed pictures be shared meaningfully with other agents? This brings us to the question of generative syntax, which is essential for complex communication. A longstanding challenge in cognitive science is to explain the origin of generative syntax. The SMN model posits that syntax is not a unique, brain-based module for language, but is exapted from the inherent combinatorial structure of the body's action zones. The segmented nature of the SMN provides a finite set of action "lexemes" (the HAPs of different zones). The ability to halt and serially chain these actions provides a natural syntax. 

A complex goal, like eating a fruit, is achieved by a syntactic sequence of HAPs: `[see fruit] + [reach for fruit] + [grasp fruit] + [bring to mouth]`. Each element is a discrete action pattern, and the pauses between them act as syntactic boundaries, allowing for substitution (e.g., grasp a different fruit) or recursion (e.g., grasp another fruit). This action-syntax is the scaffold upon which spoken language is built. The rules of grammar are not abstract and amodal, but are deeply homologous with the rules of combining bodily actions. This reinterprets Chomsky's "universal grammar" not as an innate linguistic module, but as a reflection of the universal architecture of the vertebrate body plan.

\subsubsection{Combinatorial Action and the Explosion of Tokens}
\label{ssubsec:combinatorial}
A significant limitation of traditional models of generative syntax, such as those proposed by Chomsky, is the implicit assumption that actions are monotonic or linear sequences controlled by a central authority. This view fails to capture the rich, multi-layered dexterity of a biological agent. The Sensation-Modulating Network (SMN) reframes the body as a dialogical architecture—an orchestra of multiple, coordinated action zones \cite{bernstein2014dexterity}. The "syntax" of action is not a linear string of commands but a dynamic, unfolding choreography, like a dance.

Consider the act of speaking. It is not a simple motor program but a complex interplay of the buccal cavity, the tongue, the larynx, and a haltable pulmonary zone supported by the intercostal muscles and diaphragm. This is a system of nested, parallel, and serially coordinated Haltable Action Patterns (HAPs). The failure of purely cognitivist models to account for this generativity stems from their neuro-centric focus; they search for a "language organ" in the brain, when the competence is distributed across the entire bodily network. In our account, the lungs, the saccadic muscles of the eye, the facial muscles used for gesturing, and the exapted forelimbs are all integral cognitive components.

Clapping can be done, while the body is standing, sitting, walking, talking or running etc. The clapping action zone is disengaged from the other states the body could be in, i.e., it could be performed independent of the rest of the states. The nesting of action patterns can be represented as [sitting(clapping)], [talking(clapping)], [running(clapping)], [walking(clapping)], [singing(clapping)] etc. The nesting becomes complex when we keep walking, while singing and clapping [walking, (clapping (singing))]. The variations in nesting can be seen when the frequency of walking and clapping match, or some modified patterns through skip-clapping, while singing action pattern is going on.

The zones that are in a state of recursive HAPs, which are synchronous, can be considered as the roots of the nested structure. For example, a person clapping a simple rhythmic beat while tapping the feet alternately, left-right, is a very simple nested structure. Here the alternating-tapping is nested in the clap-beat. One can increase the complexity by arbitrarily moving another zone such as turning the neck left to right while also doing the alternating-tapping. The alternating-tapping and the alternate-neck-turning are nested inside the clap-beat. It is up to the creativity of the choreographer, to play with the endless possibilities of modulating haltable zones within the bounds of the body architecture.

This perspective requires us to see generative syntax not as a phenomenon exclusive to language, but as a general principle of combinatorial action. The ability to play a musical instrument, type on a keyboard, mime, dance, or sing are all potent examples of the same underlying capacity: the sequencing and combination of action-tokens from multiple zones to create a near-infinite variety of meaningful patterns. Language is not a peculiar, isolated faculty but is one manifestation—albeit a highly sophisticated one—of this general dexterity.

This leads to a crucial conclusion about the human place in the evolutionary spectrum. The "explosion of tokens" that characterizes human culture is not the result of a unique biological peculiarity. Rather, it is a matter of *predominance*. The underlying architectural principles of the SMN are shared with other animals. Humans are distinctive only in the degree to which they have developed the capacity for fine-grained modulation and combinatorial sequencing of HAPs. Our cognitive abilities, including language, lie on the same continuum as other forms of animal cognition, grounded in the shared logic of an embodied, acting network.

\subsubsection{From Habits to Syntactic Representations}
\label{ssubsec:habits}
While individual Haltable Action Patterns (HAPs) function as discrete tokens, their true power is realized when they are streamed together into fluent, repeatable sequences, or habits. A habit is a form of embodied, procedural syntax. However, for this syntax to become a shareable, stable representation, it must be punctuated—it must leave a trace.

Crucially, HAPs are not confined to the agent's body; they physically impact the external world. A gesture leaves a visible trace in the air, a step leaves a tactile trace in the sand, and a vocalization leaves an auditory trace in the form of a sound wave. These external traces—visible, tactile, and auditory—are the bridge from the individual to the inter-subjective. They are the punctuated, stable artifacts of the agent's fluid actions.

These traces are the foundation of Transactional Action Patterns (TAPs). An external trace, such as a carved notch in a piece of wood, is an affordance that invites another agent to engage with it, perhaps by making a similar mark or by using it as a counter. Because the agent producing the trace and the agent perceiving it share the same fundamental bodily architecture, the trace can be understood through the same action-based repertoire. This shared source of action and perception is the key: we do not need another, mysterious mechanism to explain how humanity became *Homo symbolicus* or *Homo semioticus*. The symbol is born the moment an action's trace is used as a token for a subsequent, inter-subjective action.

This transactional space, built upon external traces, is the arena where we use and shape objects and surfaces for shared meaning. The deliberate creation of traces for others to interpret is the very essence of encoding, and the interpretation of those traces is decoding. This is the bedrock of reading, writing, and all forms of external, symbolic culture. It is not a different kind of cognition, but a sophisticated, scaffolded application of the same fundamental SMN dynamics that govern how an agent grasps a stone.

\subsection{Constructing the Categories of Time and Number}
\label{subsec:time_number}
Just as space is a construction of the motor system, the fundamental categories of time and number also emerge from the body's intrinsic dynamics. The agent is a symphony of nested clocks, each operating at a different frequency. At the deepest level are the slow, stable rhythms of metabolic cycles and the high-frequency oscillations of atomic and quantum phenomena. Layered on top of these are the involuntary, medium-frequency Fixed Action Patterns (FAPs) like heartbeat and respiration. Finally, at the most accessible layer, are the Haltable Action Patterns (HAPs). This multi-layered, pre-formatted temporal canvas is the raw material of temporal experience.

A single action zone, such as a finger tapping, has a "serialization limit"—a maximum frequency of around 10 Hz at which it can reliably produce discrete actions. However, the agent can achieve much higher frequencies of token generation by alternating between different action zones, as a pianist does with ten fingers or a typist with both hands. This ability to serialize discrete actions against the body's continuous rhythmic background is the origin of number sense. The act of counting is the act of producing a sequence of punctuated HAPs and mapping them to objects or events. "One, two, three" is a motor pattern before it is an abstract concept. Time, therefore, is not a universal river in which the agent is immersed, but the felt experience of the body's own nested rhythms, and number is the emergent property of punctuating that experience with discrete, serial actions.

\subsection{Reconciling Cognitive Divides}
\label{subsec:reconciling}
The SMN framework offers a bridge between cognitivism and enactivism. It is profoundly enactive, as it equates cognition with the modulation of action. However, it does not reject representation; it redefines it. USHAPs are representations: they are internal, stand-ins for external objects, and can be manipulated in simulations. But unlike classical cognitivist symbols, they are never amodal or arbitrarily related to their referents. They are grounded in the very action patterns used to interact with those referents, thus providing a natural solution to the symbol grounding problem.

This also allows for a reinterpretation of dual-process theories. System 1 can be seen as the fast, parallel, and largely unconscious operation of the SMN's FAPs and highly fluent, automatized HAPs. System 2 emerges from the slow, serial, and effortful process of consciously halting, sequencing, and modulating HAPs, particularly in novel situations or during explicit communication via TAPs. Development and expertise represent the process by which effortful System 2 operations (newly learned HAPs) become fluent, embodied System 1 skills (automatized HAPs).

\subsubsection{USHAPs, Simulation, and Imagination}
\label{ssubsec:ushaps}
The conceptual power of the SMN model stems from the distinction between saturated and unsaturated Haltable Action Patterns (HAPs). An action is **saturated** when it is performed with its corresponding object: grasping a physical cup, swallowing actual water, or running on solid ground. The action is constrained and informed by the object in real-time. Conversely, an action is **unsaturated** when it is performed without the object: a mime of grasping, a gesture of drinking, or a "moonwalk" that suspends actual displacement. These Unsaturated HAPs (USHAPs) are the raw material of representation, simulation, and imagination.

This creates a spectrum of representation. A non-arbitrary USHAP, like a mime, bears a direct, iconic relationship to its saturated counterpart and can be understood without a pre-learned rule. However, USHAPs can also be **arbitrary**, as in a "thumbs-up" gesture. We propose that natural language is built upon a vast lexicon of such arbitrary USHAPs (vocal gestures). The evolution towards arbitrary symbols is not accidental; it serves a crucial socio-political function. Arbitrary symbols are opaque to outsiders, creating "private languages" that help forge social identity and can be used for strategic communication, for instance, in warring conditions. The diversity of human scripts and grammars is a testament to this drive for social differentiation.

The ultimate expression of the USHAP is its full internalization as the basis of thought. "Thinking about running" involves the execution of subtle, inexpensive motor patterns—micro-movements in the larynx, slight shifts in breathing, faint saccades of the eye—that are delinked from the costly, saturated action of actually running. This "economy of execution" is what makes abstract thought metabolically feasible. Crucially, what is internalized and tokenized is not the specific muscular action, but its abstract *pattern* or *form*. An agent can trace a circle with a hand, a foot, or their entire body; the neural network identifies the shared, context-independent form. This is how concepts, as USHAPs, are "emancipated" from their context of discovery, allowing them to be applied to new situations.

This framework allows us to reconcile the cognitivist claim of detached representations with the enactivist demand for grounding. A concept, as an USHAP, *is* detached from its original context, which is why we can think about cups without a cup being present. However, it is never un-grounded; it is always physically and biologically grounded in the action patterns of the agent's body. The rules for decoding the *arbitrary* USHAPs of language are then socially and culturally grounded, requiring institutions like schools for their posterity. Semiotics is thus grounded in a nested hierarchy: in the body, and in the culture that the body builds.

\subsubsection{A Ground for Neural and Cognitive Plasticity}
\label{ssubsec:plasticity}
The Sensation-Modulating Network (SMN) is not a static architecture; it is inherently plastic, capable of adapting and reorganizing in response to experience and constraint. This plasticity manifests in two primary ways.

First, the segmented, multi-zone nature of the body provides a crucial form of **redundancy**. The abstract *pattern* of an action can be implemented by multiple, different action zones. An agent can gesture with its hands, but if they are occupied, it can achieve a similar communicative act with a nod of the head or a facial expression. This interchangeability is a powerful source of flexibility and creativity. It is vividly demonstrated in cases where individuals who have lost the use of their limbs learn to write, paint, or perform other complex tasks with their feet and toes. The underlying conceptual schema—the Unsaturated HAP (USHAP)—is not tied to a specific effector, allowing the system to explore and discover alternative ways of achieving its goals.

Second, the ability to **re-enact** delinked patterns (USHAPs) is the engine of neural plasticity. Every time an agent internally simulates an action, the corresponding neural pathways are activated. This process of re-enactment strengthens synaptic connections, carving out and reinforcing the "data structures" or conceptual schemes we discussed earlier. This is the mechanism of learning and memory: a memory is not a stored file but the capacity to re-enact a specific pattern of action, which in turn re-creates a phenomenological experience.

Together, these two aspects of plasticity explain how action schemas and conceptual schemas become **emancipated** from their original context. Because a pattern can be implemented by different parts of the body and can be re-enacted internally without its original object, it takes on a life of its own. This emancipation is what creates the powerful phenomenological appearance of an autonomous, internal "world of representations." The concepts feel detached and abstract precisely because the underlying action patterns are so flexible and portable. Plasticity is thus not just a feature of the brain; it is a fundamental property of an embodied agent that learns and adapts by continuously exploring and re-enacting its possibilities for action.
\section{Conclusion}
\label{sec:conclusion}
This paper has argued that the persistent divisions in cognitive science can be traced to a foundational oversight: the failure to appreciate the cognitive implications of the agent's biological architecture. In response, we have proposed the Sensation-Modulating Network (SMN), a model that reframes the agent not as a brain in a vat, but as a structured, dynamic body whose primary cognitive act is the modulation of action. By shifting the focus from neural computation to the dynamics of halting and sequencing action patterns, we have outlined a unified framework that accounts for phenomenology, the construction of a meaningful world, and the emergence of generative syntax from embodied activity.

The SMN model represents a foundational departure from several mainstream approaches. It moves beyond neuro-centrism by treating the nervous system not as the seat of cognition, but as a crucial tool for modulating the body's intrinsic dynamics. It rejects the cognitivist notion of amodal symbols, instead grounding representation in the concrete, yet abstractable, form of unsaturated action patterns. Finally, it provides a concrete, architectural basis for enactivist claims, moving from philosophical stance to a testable set of principles.

The implications of this framework open several avenues for future research. Empirically, the model generates specific predictions that can be tested through cognitive robotics and developmental psychology. For instance, experiments could be designed to trace the link between the mastery of complex, sequential motor tasks (action syntax) and the subsequent development of linguistic syntax in children. Theoretically, the model can be extended to other domains of cognition, such as emotion, which can be framed as the phenomenological experience of modulating deep-seated FAPs, or social cognition, understood as the complex interplay of mutually-scaffolding TAPs.

We anticipate that a potential criticism of this framework may be that it seems to leave little room for highly abstract, non-motoric thought, such as mathematical or philosophical reasoning. From this perspective, our model might appear to reduce all cognition to mere bodily movement. However, this would be a misinterpretation of the role of Unsaturated HAPs (USHAPs). The process of emancipation—whereby an action *pattern* is abstracted from any specific motor execution—is precisely the mechanism that grounds these higher forms of thought. The content of abstract reasoning is not the gross bodily action itself, but the subtle, inexpensive, internalized re-enactment of action schemas. Our claim is not that thinking *is* overt movement, but that it is founded upon the same generative, syntactic principles that govern movement, providing a fully grounded yet powerfully abstract cognitive engine.

Which specific substrate in the body is primarily responsible for running these USHAPs is an open empirical question. Though our account is informed by biology, as a theoretical contribution it remains, at best, a grounded speculation. The model's primary contribution, therefore, is to demonstrate that there are alternative, and perhaps more fruitful, ways we can think about thinking.

Ultimately, the SMN model calls for a return to a more integrated view of the cognitive agent. It suggests that to understand the mind, we must first understand the body—not as a mere input-output device for a central brain, but as the very medium of cognition itself. The path to explaining consciousness, language, and reason does not lie in abstracting away from our biological form, but in recognizing that it is the source of our unique cognitive life.



%This is where your bibliography is generated. Make sure that your .bib file is actually called library.bib
\bibliography{library}

%This defines the bibliographies style. Search online for a list of available styles.
\bibliographystyle{abbrv}

\end{document}
