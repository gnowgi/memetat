\documentclass[10pt,letterpaper]{article}
\usepackage[top=0.85in,left=2.75in,footskip=0.75in,marginparwidth=2in]{geometry}

% use Unicode characters - try changing the option if you run into troubles with special characters (e.g. umlauts)
\usepackage[utf8]{inputenc}
\usepackage{comment}
% clean citations
\usepackage{cite}

% hyperref makes references clicky. use \url{www.example.com} or \href{www.example.com}{description} to add a clicky url
\usepackage{nameref,hyperref}

% line numbers
% \usepackage[right]{lineno}

% improves typesetting in LaTeX
\usepackage{microtype}
\DisableLigatures[f]{encoding = *, family = * }

% text layout - change as needed
\raggedright
\setlength{\parindent}{0.5cm}
\textwidth 5.25in 
\textheight 8.75in

% Remove % for double line spacing
%\usepackage{setspace} 
%\doublespacing

% use adjustwidth environment to exceed text width (see examples in text)
\usepackage{changepage}

% adjust caption style
\usepackage[aboveskip=1pt,labelfont=bf,labelsep=period,singlelinecheck=off]{caption}

% remove brackets from references
\makeatletter
\renewcommand{\@biblabel}[1]{\quad#1.}
\makeatother

% headrule, footrule and page numbers
\usepackage{lastpage,fancyhdr,graphicx}
\usepackage{epstopdf}
\pagestyle{myheadings}
\pagestyle{fancy}
\fancyhf{}
\rfoot{\thepage/\pageref{LastPage}}
\renewcommand{\footrule}{\hrule height 2pt \vspace{2mm}}
\fancyheadoffset[L]{2.25in}
\fancyfootoffset[L]{2.25in}

% use \textcolor{color}{text} for colored text (e.g. highlight to-do areas)
\usepackage{color}

% define custom colors (this one is for figure captions)
\definecolor{Gray}{gray}{.25}

% this is required to include graphics
\usepackage{graphicx}

% use if you want to put caption to the side of the figure - see example in text
\usepackage{sidecap}

% use for have text wrap around figures
\usepackage{wrapfig}
\usepackage[pscoord]{eso-pic}
\usepackage[fulladjust]{marginnote}
\reversemarginpar

% document begins here
\begin{document}
\vspace*{0.35in}

% title goes here:
\begin{flushleft}
{\Large
\textbf\newline{An Architectural Foundation for Cognition: Grounding Semiotics and Syntax in Action Patterns}
}
\newline
% authors go here:
\\
G.~Nagarjuna\textsuperscript{1},
Durga prasad Karnam\textsuperscript{2}
\\
\bigskip
\bf{1} Indian Institute of Science Education and Research, Pune
\\
\bf{2} Indian Institute of Technology, Mumbai
\\
\bigskip
* nagarjuna@iiserpune.ac.in

\end{flushleft}
\noindent \textbf{Long Title:} \textit{A Segmented, Polarized,
  Bilaterally Symmetrical, Antagonistically Organized Multizonal
  Coordinated Pairs in the form of a Layered Sensation Modulating
  Network Implementing Fixed, Haltable and Transactional Action
  Patterns of Conscious Semiotic Agents Accouncing Subjective
  Experience and Generative Syntax}


\subsection*{Abstract}
This monograph addresses the long-standing divide between cognitivist and enactive approaches by proposing a new foundational model for cognitive science. We argue that the impasse stems from an incomplete account of the agent’s biological structure and its cognitive function. To resolve this, we introduce a model of the cognitive agent’s body composed of a \emph{Sensory-Motor Network} (\SMNS), which defines structural organization, and a \emph{Sensation-Modulating Network} (\SMND), which defines its dynamics. Together, these networks express the agent’s evo–devo trajectory in terms of general biological principles: a polarized, tubular, segmented, bilaterally symmetrical, and antagonistically organized body plan. 

Our central thesis is that cognition arises from the ability to halt and negotiate rhythmic action patterns, giving rise to a hierarchy of \emph{Fixed, Haltable, Negotiable, and Transactional Action Patterns} (FAPs, HAPs, NAPs, TAPs). This hierarchy grounds phenomenological experience, explains the emergence of symbols through “unsaturated” yet embodied actions, and accounts for the origin of generative syntax in the body’s combinatorial, multi-zonal structure. By situating cognition in this core biological architecture, the SMN model provides a biologically grounded account of cognition, consciousness, semiotics, language, and culture. 

Through model-based reasoning, we show: (1) how a cognitive agent develops self–world relations, (2) why such a model can support complex cognition at room temperature, (3) how representations, memory, naming, generative syntax, reasoning, and engineering competencies can be understood enactively, and (4) how long-established experimental results in cognitive neuroscience may be reinterpreted in light of this framework. We thus offer a path toward reconciling cognitivist and enactive perspectives.  

\bigskip
\noindent \textbf{Keywords:} cognitive architecture, enactivism, symbol grounding, generative syntax, phenomenology, semiotics, action patterns, Sensation-Modulating Network (SMN)

\section{Introduction}
\label{sec:introduction}
The study of cognition is marked by a fundamental tension between seemingly irreconcilable research programs. On one side, cognitivism, rooted in the information processing paradigm, seeks to explain phenomena like language and reasoning through amodal symbols and computational rules, often located within the brain \cite{chomsky1965aspects, fodor_modularity_1983}. On the other, 4E (embodied, embedded, enactive, and extended) approaches ground cognition in the dynamic, situated actions of an agent's entire body within its environment \cite{varela, noe_action_2004}. While both camps agree on a biological basis for cognition, their foundational assumptions about the nature of the body and its role in mental life have led to decades of disagreement. This paper argues that this impasse stems from an incomplete understanding of the biological architecture that underpins cognition. We proceed from the assumption that a correct model of this architecture is the necessary foundation upon which any successful theory of cognition must be built.

We propose that the core challenge for cognitive science is to answer a set of interrelated questions: (1) What structures and dynamics are necessary for phenomenological experience—the so-called "hard problem"—to arise? (2) How does a cognitive agent construct a stable, geometric picture of its internal and external worlds? (3) How can this generated picture be shared meaningfully with other agents? To address these, we depart from neuro-centric models and introduce an alternative framework centered on a specific, yet generalizable, model of a cognitive agent's body plan.

Our central proposal is a dynamic architecture we term the Sensation-Modulating Network (SMN). We model the agent as a layered network of action zones, organized according to fundamental biological principles: segmented, polarized, bilaterally symmetrical, and antagonistically organized. Within this architecture, cognition emerges not from a central controller initiating action, but from the capacity to \textit{halt} and modulate ongoing, rhythmic action patterns. To describe these dynamics, we will introduce a specific terminology: a hierarchy of patterns ranging from deep, immutable **Fixed Action Patterns (FAPs)**, to consciously accessible **Haltable Action Patterns (HAPs)**, which, when shared, become **Transactional Action Patterns (TAPs)**. It is this capacity for modulation, we argue, that grounds semiotics, enables the creation of symbols, and gives rise to generative syntax.

This paper unfolds in five parts. First, we will elaborate on the proposed SMN model, detailing its architectural principles and its dynamic properties. Second, we will demonstrate how this model can explain a wide range of cognitive phenomena, from subjective experience to the emergence of abstract concepts. Third, we will situate our model within the broader theoretical landscape, comparing its core tenets and explanatory power to other leading theories of cognition. Fourth, we will reinterpret existing experimental evidence and propose novel, falsifiable predictions derived from our framework. Finally, we will conclude by discussing the broader implications of this model, arguing that it offers a foundational departure from previous approaches and provides a path toward reconciling the long-standing divisions in the field.
\section{The Proposed Model: A Dynamic Architecture}
At the heart of our proposal is the Sensation-Modulating Network (SMN), a framework for understanding the cognitive agent as a whole, rather than as a brain-centric processing unit. The SMN is not a specific organ but the entire functional architecture of the agent, defined by a set of core biological design principles. These principles, while ubiquitous in biology, have been largely overlooked in their cognitive implications.

\subsection*{The Nature of Action: A Thermodynamic Distinction}
Before detailing the architecture of the Sensation-Modulating Network (SMN), we must make a foundational ontological distinction between *action* and *interaction*. While all actions are a form of interaction, not all interactions are actions. In the context of this model, an **interaction** is a conserved process governed by symmetrical physical laws. An **action**, by contrast, is a local, symmetry-breaking perturbation performed by a far-from-equilibrium system. It is a thermodynamic process: the agent must expend energy to initiate, sustain, and modulate its actions. This distinction is crucial for an enactive model, as it defines the agent as a being that actively creates asymmetries in the world, rather than as a passive object merely being pushed and pulled by external forces. Cognition, in this view, is the process of managing these energy-dependent, symmetry-breaking events.

\subsection*{Architectural Principles of the SMN}
We model the agent's body as a topologically tubular structure possessing four key properties:
\begin{enumerate}
    \item \textbf{Polarity:} The body has a defined axis, typically from anterior to posterior, which establishes a fundamental directionality for movement and interaction with the environment.
    \item \textbf{Metameric Segmentation:} The body is composed of repeating segments or action zones (e.g., limbs, digits, vocal apparatus). Each zone is a locus of potential action.
    \item \textbf{Bilateral Symmetry:} The body is organized symmetrically around a central axis, creating pairs of coordinated structures.
    \item \textbf{Antagonistic Organization:} Action within and between zones is governed by antagonistic pairs (e.g., flexion and extension). This push-pull dynamic is the fundamental basis of control and modulation.
\end{enumerate}
This architectural plan provides the agent with a multitude of "action zones," each a dynamical system capable of generating rhythmic patterns. The core cognitive faculty, we argue, arises from the agent's ability to manage these patterns.

\subsubsection*{The SMN in a Gravitational Field}
A foundational oversight in many cognitive models, particularly those that are heavily neuro-centric, is the treatment of the environment as a passive problem-space that places a computational burden on the brain. Even ecological theories, which rightly situate the agent in its environment, have not fully accounted for the constitutive role that fundamental physical forces play in cognition. Our framework begins by asserting that the agent’s body is not merely *in* an environment but is dynamically shaped *by* it. The Sensation-Modulating Network (SMN) is therefore defined, first and foremost, as a system that has evolved to actively and continuously counteract the planet's gravitational field.

This is not a trivial point. Gravity is not a bug to be fixed or a variable to be solved for; it is a constant, predictable, and non-negotiable partner in every action. The entire architecture of the agent—its antagonistic muscle pairs, its skeletal structure, its vestibular system—is a testament to this partnership. This allows for a radical offloading of computation. The agent does not need to store vast amounts of "data" about the world. Instead, it develops and refines "data structures" in the form of action schemas. The stability and predictability of the gravitational field provide a constant, reliable feedback mechanism against which these schemas are calibrated.

This leads to a crucial distinction between biological cognition and the detached, symbolic computation of the machines we have built. An artificial system must be fed data, store it, and run explicit procedures on it—an inefficient process that requires immense energy. The SMN, by contrast, operates primarily in a "saturated mode." When an agent walks, the ground pushes back with every step; when it swims, the fluid resists and supports every movement. The rich, real-time feedback from the world is an ineliminable part of the computational loop. Because the "data" remains external, the agent's internal work is lean, efficient, and possible at room temperature.

Therefore, actions like walking and swimming are not merely locomotion; they are profound epistemic acts. They are how the agent constructs a geometric model of its world, using its own body and the constant of gravity as its measuring instruments. As these actions modulate the agent's sensory subsystems in response to the affordances of the environment, the agent "grasps" the world—not by representing it internally, but by continuously testing and refining its possibilities for action within it. The gravitational field is thus not an incidental feature of our world, but a fundamental and active component of our cognitive architecture.

\subsection*{The Primacy of Halting}
Contra the received view that the nervous system's primary role is to initiate action, we propose its crucial function for cognition is to \textit{alter} and, most importantly, \textit{halt} ongoing action patterns. We proceed from the foundational assumption that rhythmic, patterned activity is the default, baseline state of biological tissue, a principle observed from the ciliary action in prokaryotes to the emergent synchrony of cardiac cells. The challenge for a complex agent is not to start moving, but to control, pause, and modulate its intrinsic dynamics. This capacity for halting is what provides the freedom to deviate from fixed trajectories, enabling exploration, deliberation, and the generation of phenomenological experience. A pause in an action sequence is not a lack of activity, but a cognitive act itself—one that creates a space for awareness and choice.

\subsubsection*{Haltability and Ecological Affordances}
Just as the gravitational field provides a constant, predictable partner in computation, the specific objects and surfaces within the environment provide a rich structure of opportunities for action. Following Gibson, we term these opportunities "affordances." A flat, rigid surface affords support for walking; a handle affords grasping; a gap affords leaping across. The concept of haltability is deeply intertwined with these affordances. An agent does not simply execute a fixed motor program; it initiates a Haltable Action Pattern (HAP) in the direction of an affordance, and the specific properties of the environment provide continuous feedback that shapes the action in real-time.

This creates a reciprocal, co-defining relationship. The agent’s capacity to halt and modulate its actions allows it to selectively engage with the world’s affordances, to explore them without being locked into an irreversible action sequence. In turn, the structured nature of these affordances reinforces and refines the agent’s repertoire of HAPs. For example, a child learning to grasp discovers that a soft toy affords a different kind of modulated pressure than a hard wooden block. The environment does not merely trigger an action; it actively teaches the agent *how* to halt and shape that action appropriately. The affordance, therefore, becomes an integral part of the action's control loop.

Haltability is thus not a purely internal capacity but a relational one. It is the agent's contribution to a dynamic dance with the environment. The world offers possibilities for action, and the agent's ability to pause, adjust, and sequence its HAPs is what allows it to navigate these possibilities, transforming a field of potential affordances into a meaningful, enacted world.

\subsubsection*{Closed-Loop Actions and Internal Saturation}
While many Haltable Action Patterns (HAPs) are directed at the external world, a crucial subset of actions involves the body acting upon itself. These reflexive, closed-loop actions—such as licking, sucking, or scratching—are not unsaturated mimes. On the contrary, they are fully **saturated** actions, because the agent's own body serves as the complex, responsive object. In these cases, the distinction between subject and object blurs, and the Sensation-Modulating Network (SMN) operates in a tight, self-referential loop.

This capacity for internal saturation is a foundational attribute of cognition. It allows the agent to generate rich, structured phenomenological experiences without any input from the external world. These actions are often highly gratifying, creating powerful feedback loops that motivate their repetition. This can manifest as fidgeting or other self-stimulatory behaviors, which are not meaningless tics but are instead the SMN actively exploring its own internal affordances and maintaining a state of dynamic equilibrium. These internally-directed, saturated actions are a vital precursor to fully delinked, unsaturated thought, providing a training ground where the agent learns to modulate its own sensory states directly.

\subsection*{A Hierarchy of Action Patterns}
The SMN's dynamics give rise to a hierarchy of action patterns, distinguished by their degree of modulatability:
\begin{itemize}
    \item \textbf{Fixed Action Patterns (FAPs):} These are deep, phylogenetically old, and largely involuntary rhythmic patterns that form the foundation of the agent's being (e.g., heartbeat, respiration, peristalsis). They are not directly accessible to conscious modulation and constitute the stable background or "cognitive canvas" upon which experience is drawn.
    \item \textbf{Haltable Action Patterns (HAPs):} These are actions that can be consciously initiated, paused, and modulated. They are performed by the outer, more flexible layers of the SMN (e.g., reaching, grasping, vocalizing). The ability to halt a HAP is the basis for creating discrete, repeatable, and therefore tokenizable, units of action.
    \item \textbf{Transactional Action Patterns (TAPs):} When HAPs are performed in a social context, they become TAPs. These are actions that are either directed at another agent or are imitated, forming the basis of communication, shared practices, and cultural learning. TAPs are the building blocks of intersubjective meaning.
\end{itemize}

\subsection*{The Differentiating and Integrating Networks}
The SMN is a layered architecture that differentiates and integrates sensations to produce a unified experience. This is achieved through the complementary roles of the body's motor and nervous systems.

We propose that the motor system—the entire musculature—is the primary \textbf{Differentiating and Filtering Network (DFN)}. It is not a mere output device but the very organ of differentiation. Each muscle group, as an action zone, is a mediator between the internal and external worlds. When a muscle contracts or relaxes, it creates a distinction; the specific tension and motion *is* the differentiated sensation. This motor activity informs the rest of the body about its state in relation to both the external world and the internal milieu.

However, differentiation alone is insufficient. For a unified experience to emerge, these local states must be brought together. This is the role of the nervous system as the \textbf{Integrating Network (IN)}. Contra the view of the CNS as a central controller, we posit its primary function is message-passing and broadcasting. It acts as a high-speed conduit, integrating the foreground of attention (driven by HAPs) with the background hum of the body (driven by FAPs). In an antagonistically organized body, the IN ensures that coordinating zones receive the necessary information to manage their push-pull dynamics effectively.

This architecture provides a natural mechanism for grounding concepts. A HAP is initially \textbf{saturated} by a physical object. However, the agent can re-enact the pattern without the object, creating an \textbf{unsaturated HAP (USHAP)}. These USHAPs—delinked from the world but still rooted in the SMN—are the raw material of concepts, simulations, and imagination, thus resolving the classical symbol-grounding problem.
\section{Explaining Cognitive Phenomena and Reinterpreting Evidence}
The explanatory power of the Sensation-Modulating Network (SMN) lies in its capacity to reframe fundamental cognitive questions in terms of action dynamics. By grounding cognition in a specific, yet generalizable, bodily architecture, the model offers a unified account of phenomenology, representation, and syntax, while reinterpreting existing empirical evidence.

\subsection*{From Action Modulation to Phenomenology}
The SMN model addresses the "hard problem" of consciousness by proposing that subjective experience is the agent's perception of its own modulated actions. The constant, rhythmic hum of the Fixed Action Patterns (FAPs) creates a stable, non-conscious background—a "cognitive canvas." A phenomenological event occurs when a Haltable Action Pattern (HAP) is initiated, modulated, or, most critically, paused. This change against the stable background *is* the experience. The "what it is like" to see red is the specific pattern of oculomotor and neural HAPs enacted to foveate on a red object; the feeling of thirst is the modulation of interoceptive patterns against the homeostatic background. Consciousness is not a substance or a passive "theater" but an active process of self-differentiation through action.

\subsubsection*{Action as the Origin of Phenomenal Experience}
In the Sensation-Modulating Network (SMN) framework, a phenomenological experience is not a passive event, such as the mere reception of sensory stimuli. Instead, it is an active construct, a direct consequence of the agent's own self-initiated actions. The process begins when the agent deploys a Haltable Action Pattern (HAP)—a voluntary, modulated action directed towards an affordance in the environment. This action, be it a saccade of the eye, a turn of the head, or the extension of a hand, is not a reaction to a stimulus but is itself the stimulus that generates the experience.

When the agent acts, it necessarily alters the flow of sensory information across its entire body. The HAP actively modulates this sensory stream, creating a specific, transient pattern of differentiation against the stable, homeostatic background provided by the Fixed Action Patterns (FAPs). The "phenomenological response" is precisely this internally-generated, action-driven pattern. For example, the experience of touching a rough surface is not caused by the surface's texture alone; it is constructed by the specific HAP of moving one's fingers across it, which generates a unique pattern of vibrations and pressures. The action and the sensation are inextricable.

Therefore, the agent's world is not something that is revealed to it, but something it brings forth through its own activity. Each self-initiated action is a question posed to the environment, and the resulting modulation of the agent's own sensory state is the answer. This continuous loop of action-based self-stimulation is the very engine of conscious experience, grounding phenomenology not in a mysterious inner theater, but in the concrete, dynamic process of a body actively engaging with its world.

\subsection*{Constructing a Geometric and Semiotic World}
The agent's world is not a pre-given, objective space that is passively perceived, but a "memetat"—a geometric and semiotic habitat constructed through action. The SMN computes this geometry through the coordination of its multiple action zones. For instance, the distance to an object is not calculated from retinal size alone, but is enacted through a fusion of HAPs: the proprioceptive feel of reaching a hand, the muscular strain of focusing the eyes, and the time delay between a sound and its echo. The world's geometry is mapped onto the agent's own bodily geometry and action capabilities.

Meaning arises as this geometric world becomes populated with signs. An object's affordances are the saturated HAPs it invites. A rock affords throwing, sitting, or striking. These saturated HAPs are the object's initial meaning for the agent. When these actions are tokenized as unsaturated HAPs (USHAPs), they become concepts that can be manipulated internally. When they are shared and imitated as Transactional Action Patterns (TAPs), they become external, public symbols. This provides a direct, embodied route from perception to semiotics, grounding the Peircean triad of object, representamen (the action pattern), and interpretant (the resulting experience or subsequent action) in the dynamics of the SMN.

\subsubsection*{Tokenization through Action}
% TODO: Explain how repeatable, recursive action patterns become tokens.

\subsubsection*{Grounding Tokens in the Graspable Situation}
% TODO: Elaborate on how the ecological situation grounds tokens and creates the subject-object link.

\subsection*{The Embodied Origins of Generative Syntax}
A longstanding challenge in cognitive science is to explain the origin of generative syntax. The SMN model posits that syntax is not a unique, brain-based module for language, but is exapted from the inherent combinatorial structure of the body's action zones. The segmented nature of the SMN provides a finite set of action "lexemes" (the HAPs of different zones). The ability to halt and serially chain these actions provides a natural syntax. 

A complex goal, like eating a fruit, is achieved by a syntactic sequence of HAPs: `[see fruit] + [reach for fruit] + [grasp fruit] + [bring to mouth]`. Each element is a discrete action pattern, and the pauses between them act as syntactic boundaries, allowing for substitution (e.g., grasp a different fruit) or recursion (e.g., grasp another fruit). This action-syntax is the scaffold upon which spoken language is built. The rules of grammar are not abstract and amodal, but are deeply homologous with the rules of combining bodily actions. This reinterprets Chomsky's "universal grammar" not as an innate linguistic module, but as a reflection of the universal architecture of the vertebrate body plan.

\subsubsection*{Combinatorial Action and the Explosion of Tokens}
% TODO: Detail how the choreography of multiple action zones leads to combinatorial possibilities.

\subsubsection*{From Habits to Syntactic Representations}
% TODO: Explain how streaming patterns of HAPs (habits) create punctuated, syntactic representations.

\subsection*{Reconciling Cognitive Divides}
The SMN framework offers a bridge between cognitivism and enactivism. It is profoundly enactive, as it equates cognition with the modulation of action. However, it does not reject representation; it redefines it. USHAPs are representations: they are internal, stand-ins for external objects, and can be manipulated in simulations. But unlike classical cognitivist symbols, they are never amodal or arbitrarily related to their referents. They are grounded in the very action patterns used to interact with those referents, thus providing a natural solution to the symbol grounding problem.

This also allows for a reinterpretation of dual-process theories. System 1 can be seen as the fast, parallel, and largely unconscious operation of the SMN's FAPs and highly fluent, automatized HAPs. System 2 emerges from the slow, serial, and effortful process of consciously halting, sequencing, and modulating HAPs, particularly in novel situations or during explicit communication via TAPs. Development and expertise represent the process by which effortful System 2 operations (newly learned TAPs) become fluent, embodied System 1 skills (automatized HAPs).

\subsubsection*{USHAPs, Simulation, and Imagination}
% TODO: Elaborate on how delinked USHAPs enable simulation and "at will" association.

\subsubsection*{A Ground for Neural and Cognitive Plasticity}
% TODO: Explain how the ability to recreate delinked patterns enables plasticity.

\section{Conclusion}
This paper has argued that the persistent divisions in cognitive science can be traced to a foundational oversight: the failure to appreciate the cognitive implications of the agent's biological architecture. In response, we have proposed the Sensation-Modulating Network (SMN), a model that reframes the agent not as a brain in a vat, but as a structured, dynamic body whose primary cognitive act is the modulation of action. By shifting the focus from neural computation to the dynamics of halting and sequencing action patterns, we have outlined a unified framework that accounts for phenomenology, the construction of a meaningful world, and the emergence of generative syntax from embodied activity.

The SMN model represents a foundational departure from several mainstream approaches. It moves beyond neuro-centrism by treating the nervous system not as the seat of cognition, but as a crucial tool for modulating the body's intrinsic dynamics. It rejects the cognitivist notion of amodal symbols, instead grounding representation in the concrete, yet abstractable, form of unsaturated action patterns. Finally, it provides a concrete, architectural basis for enactivist claims, moving from philosophical stance to a testable set of principles.

The implications of this framework open several avenues for future research. Empirically, the model generates specific predictions that can be tested through cognitive robotics and developmental psychology. For instance, experiments could be designed to trace the link between the mastery of complex, sequential motor tasks (action syntax) and the subsequent development of linguistic syntax in children. Theoretically, the model can be extended to other domains of cognition, such as emotion, which can be framed as the phenomenological experience of modulating deep-seated FAPs, or social cognition, understood as the complex interplay of mutually-scaffolding TAPs.

Ultimately, the SMN model calls for a return to a more integrated view of the cognitive agent. It suggests that to understand the mind, we must first understand the body—not as a mere input-output device for a central brain, but as the very medium of cognition itself. The path to explaining consciousness, language, and reason does not lie in abstracting away from our biological form, but in recognizing that it is the source of our unique cognitive life.



%This is where your bibliography is generated. Make sure that your .bib file is actually called library.bib
\bibliography{library}

%This defines the bibliographies style. Search online for a list of available styles.
\bibliographystyle{abbrv}

\end{document}
