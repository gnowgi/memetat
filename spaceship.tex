\subsection{A Thought Experiment: The Alien Spaceship Mission}
To make these abstract questions concrete, consider the following thought experiment. An alien mission has sent a spaceship on an expedition to Earth to construct a picture of the world. The spaceship has various sensors (transducers) on its outer surface and inside—light, sound, contact, temperature, and chemosensors. It has an internal (inbuilt) clock and an internal sense of time, but no pre-built location sensors, though it has accelerometers to detect changing states of motion. Complicating matters, the crew is strictly confined to the ship and cannot physically move out to explore the world they seek to understand.

The engineers who built the ship placed the sensors at various locations, with connectors to make signals from the sensors available to respective processing chips. An integrated snapshot is constructed only on the basis of concurrency (at any given point in time) on their dashboards. Since all signals are encoded similarly, they have no signature of where they are coming from (sources), including whether causes for the signals are generated from within the ship or from outside.

The crew's mission is to locate the source of the input signals coming from the sensors and construct a comprehensive picture of the world. The ship's crew knows only how to count and compute the signals coming from the sensors. The programmers and mathematicians can write programs to compute, and their mechanical/electrical engineers can design and deploy actuators and additional sensors as and when necessary. If \textit{and only if} required, a few more such ships can be deployed for the mission, and they can share results through their communication channels.

What do they need to do within the ship to compute the location of the source of the input signals? How did the crew construct the picture (geometry) of the world? The answer to this puzzle is expected to be in the form of the mechanism—structure and dynamics—of the spaceship.

This thought experiment captures the essence of the cognitive problem: how does an agent, confined to its own body, construct a meaningful picture of both its internal and external worlds? The signals reaching the dashboards appear transiently. At any given time, there is a snapshot of signals, including those coming from various transducers. The snapshot passes from a temporally ordered stack of screens, from one screen to another at each passing instance. Since the tokens disappear instantly, the only way to hold onto a pattern for a while is to pass them to another chip. There is no inbuilt way of retrieving the snapshots. The only way is to recurrently subject the spaceship to similar exposures. Such is the phenomenology available to the crew.

How do they differentiate a set of tokens as coming from a specific kind of transducer, so that each datatype can be distinguished? How do they distinguish signals coming from within the spaceship from outside? The crew is inside the 'cave' and cannot move out. What innovations must they make to construct a structure and dynamics of the world?

This thought experiment illustrates that the core problem of cognition is fundamentally about constructing a geometric picture of the world through action-based differentiation, not passive reception of pre-given information. The spaceship crew's solution—developing actuators to modulate their sensory streams and create differentiated patterns—provides a key insight into how biological agents solve this same problem.
