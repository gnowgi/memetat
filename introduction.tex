\section{Introduction}
\label{sec:introduction}
The study of cognition is marked by a fundamental tension between seemingly irreconcilable research programs. On one side, cognitivism, rooted in the information processing paradigm, seeks to explain phenomena like language and reasoning through amodal symbols and computational rules, often located within the brain \cite{chomsky1965aspects, fodor_modularity_1983}. On the other, 4E (embodied, embedded, enactive, and extended) approaches ground cognition in the dynamic, situated actions of an agent's entire body within its environment \cite{varela1991embodied,  maturana1991autopoiesis, noe_action_2004}. While both camps agree on a evolutionary and biological basis for cognition \cite{chomsky1965aspects, varela1991embodied}, their foundational assumptions about the nature of the body and its role in mental life have led to decades of disagreement. This paper argues that this impasse stems from an incomplete understanding of the biological architecture that underpins cognition. We proceed from the assumption that a correct model of this architecture is the necessary foundation upon which any successful theory of cognition must be built.

We propose that the core challenge for cognitive science is to answer a set of interrelated questions: (1) What structures and dynamics are necessary for phenomenological experience—the so-called "hard problem" \cite{chalmers1995facing}—to arise? (2) How does a cognitive agent construct a stable, geometric picture of its internal and external worlds? (3) How can this generated picture be shared meaningfully with other agents? To address these, we depart from neuro-centric models and introduce an alternative framework centered on a specific, yet generalizable, model of a cognitive agent's body plan.

Our central proposal is a dynamic architecture we term the Sensation-Modulating Network (SMN). We model the agent as a layered network of action zones, organized according to fundamental biological principles: segmented, polarized, bilaterally symmetrical, and antagonistically organized. Within this architecture, cognition emerges not from a central controller initiating action, but from the capacity to \textit{halt} and modulate ongoing, rhythmic action patterns. To describe these dynamics, we will introduce a specific terminology: a hierarchy of patterns ranging from deep, immutable \textbf{Fixed Action Patterns (FAPs)}, to consciously accessible \textbf{Haltable Action Patterns (HAPs)}, which, when shared, become \textbf{Transactional Action Patterns (TAPs)}. It is this capacity for modulation, we argue, that grounds semiotics\cite{peirce1992essential}, enables the creation of symbols, and gives rise to generative syntax\cite{chomsky1965aspects}.

We are often misled by the \textit{apparent} anatomical peculiarities of the human body—the large brain, bipedal locomotion, dexterous hands, and sophisticated vocal apparatus—and the semiotic peculiarities that seem to correlate with them. This has led to a focus on what makes humans unique \cite{hauser2002uniquely}, rather than on the foundational biological principles we share with other animals. This paper argues for a model-driven approach that does not mistake correlation for causation. We contend that much of the biology relevant to cognition has been ignored precisely because it is not unique to humans. Our obsession with our own specialness has prevented us from seeing the general principles at work.

What if movement does not require an initiating message from a control center? Physics teaches us that force is needed not to maintain motion, but to change it \cite{newton1687principia}. Borrowing this insight, we speculate that the nervous system's primary cognitive role is not to initiate action, but to *alter* it. This is not an ungrounded speculation. We know that organisms without nervous systems exhibit movement and that even isolated cardiac tissue can beat rhythmically \citep{llinas2002vortex, landecker2007culturing}. The capacity for action is intrinsic to living tissue; the challenge is to control it.

\subsection{Novel Aspects of the Proposal}
While our Sensation-Modulating Network (SMN) framework shares important affinities with enactive, embodied, ecological, and extended theories of cognition, several key differences and novel contributions set it apart. 
\begin{description}
    \item[Architectural Specificity and Biological Detail] 
    Segmentation, Polarity, Symmetry, Antagonism: We foreground a precise architectural plan—the agent as a segmented, polarized, bilaterally symmetrical, and antagonistically organized network. Many embodied or enactive theories invoke a generic "body" \cite{clark1997being, gallagher2005how}, but our model provides a concrete, biologically plausible template for how cognition emerges from structural features common across animal life.      Rather than treating embodiment as a metaphor, we show how the coordination of action zones forms the substrate for cognitive acts, grounding higher phenomena in actual bodily organization.

    \item[Primacy of Halting and Modulation]
    A central claim is that cognition does not merely initiate actions, but modulates and halts ongoing rhythms, treating the pause itself as a cognitive act. Traditional frameworks focus on action initiation or perception \cite{fodor_modularity_1983, chomsky1965aspects}, but our proposal recognizes halting as the locus of freedom, deliberation, and phenomenological experience.  This provides a new operational focus: how organisms carve out “tokens” and syntactic units from a continuous stream of action via halts.

    \item[Nested Action Patterns] 
    The division into Fixed Action Patterns (FAPs), Haltable Action Patterns (HAPs), and Transactional Action Patterns (TAPs) as dynamically layered and context-sensitive is more granular than most existing accounts. Especially unique is the idea that cognitive flexibility and symbolic capacity arise from the transition between these layers—from deeply ingrained rhythms to consciously modulated and socially shared actions.

    \item[Tokenization and Syntax Grounded in Action Architecture]
    Unlike classic cognitivist accounts, which see tokens as amodal symbols \cite{fodor1975language, pylyshyn1984computation}, our model demonstrates how tokens emerge from bounded, repeatable, haltable actions. Syntactic structure is not imposed from without, but exapted from the combinatorial possibilities inherent in the body's architecture. This is a bottom-up view of syntax, tracing linguistic generativity to motoric organization, rather than positing a dedicated language module.
    \item[Grounding of Semiotics and Symbols in Action Patterns]
    The solution to the “symbol grounding problem” is embodied and dynamic: unsaturated action patterns (USHAPs) become the physical substrate of concepts and internal representations. Transactional Action Patterns (TAPs), as socially shared, trace-making behaviors, show how meaning and symbols are enacted and externalized via body-environment couplings, not arbitrarily assigned. This provides a functional bridge from action to meaning, linking Peircean semiotics with Hebbian \cite{hebb1949organisation} and sensorimotor mechanisms.

    \item[Plasticity and Redundancy in Action Schemas]
    The model’s emphasis on redundancy and interchangeability among motor zones (e.g., hands, feet, facial gestures) allows for an elegant account of neural and cognitive plasticity—concepts and functions are emancipated from their original bodily location, facilitating creative compensation, learning, and adaptation.

    \item[Demonstrating Formalization using Multiple Mathematical Analogies] 
    The proposal integrates topology, graph theory, category theory, group theory, Petri nets, control theory, information theory, and signal processing to bridge dynamic biological architectures with rigorous formal systems. This multi-level analogical approach allows new kinds of computational modeling and theoretical precision.

    \item[Epistemic Role of Environmental Forces]
    Unlike neuro-centric or ecological models that treat the environment as a passive data source, we highlight fields like gravity and aquatic habitat as constitutive partners in cognition—agents actively calibrate “data structures” using predictable physical feedback, reducing internal computational load.

    \item[Unified, Fully Testable Framework]
    We move from philosophical stance to empirical program, offering distinct experimental predictions: e.g., disrupting proprioception should impair spatial reasoning \cite{proprioception_spatial, proprioception_reasoning}, mastery of action syntax should precede linguistic syntax, etc..

    \item[Reconciliation of Representation with Enactivism]
    Rather than rejecting representation or mental simulation, our model redefines representation as re-enactment of unsaturated action patterns (USHAPs), thus bridging the gap between radical enactivism and cognitivist abstraction in a uniquely grounded way.

    \item[Questioning Traditional Dichotomies in Philosophy]
    Objectivity and Subjectivity; Empirical and Rational; Universals and Particulars; Form and Matter; Mind and Body.

\end{description}


Among the implications of this proposal include: a re-characterization of being human. We shall demonstrate that the ``mechanics'' of cogntition are not \textit{peculiar} to human beings, while certain aspects of them are \textit{predominant}. The model also provides an explanation of why machine learning is so \textit{expensive}, data and memory driven, while humans can perform those operations at room temperature. Unlike Turing's model of computation that depends on reading and writing symbols on almost \textit{infinite} memory, we provide an alternate approach to computation that does not store tokens, but generates them, in the body, and interprets them on the fly using the schema in the form of action patterns. The schema are not fixed, when formed, they get revised or destroyed, based on how well they support \textit{predictive processing}. Thus these are not \textit{rational} schema, but \textit{stochastic} in nature. Though human reason \textit{appears} to be akin to bounded rationality, it is so only when we play the rule based game of formal semantics. It is not the body that is bound to rationality but the transactional cutlure of academics. Recent success stories of modern machine learning support this insight that rule bound rationality emerges after several trials and errors, while the stochastic behavior is the foundation. 

This paper unfolds in five parts. First, we will elaborate on the proposed SMN model, detailing its architectural principles and its dynamic properties. Second, we will demonstrate how this model can explain a wide range of cognitive phenomena, from subjective experience to the emergence of abstract concepts. Third, we will situate our model within the broader theoretical landscape, comparing its core tenets and explanatory power to other leading theories of cognition. Fourth, we will reinterpret existing experimental evidence and propose novel, falsifiable predictions derived from our framework. Finally, we will conclude by discussing the broader implications of this model, arguing that it offers a foundational departure from previous approaches and provides a path toward reconciling the long-standing divisions in the field.