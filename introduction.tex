\section{Introduction}
\label{sec:introduction}

The study of cognition has seen major shifts over the last century. Early work was dominated by behaviorism, with rigorous experimental models of conditioning, reinforcement, and associative learning in both animals and humans, united by a commitment to externally observable behavior \citep{pavlov1927, skinner1938, skinner1953science, thorndike1898, watson1913psychology}. These models were later criticized for their inability to account for human language, leading to the rise of cognitivism. Cognitivist accounts posited an innate language faculty, modular brain regions, and a distinctive generative syntax \citep{chomsky1965aspects, chomsky1986knowledge, fodor1975language, fodor_modularity_1983, pinker1994language, pinker1997mind}. Despite agreement that cognition has biological and evolutionary roots, the two traditions diverged on whether human–animal differences are of degree or of kind, and whether they arise from nurture or nature \citep{chomsky1975reflections, pinker2002blankslate, watson1924behaviorism, skinner1971beyond}. A parallel line of thought, drawing on neo-Kantian and dialectical traditions, emphasized the interplay of internal and external conditions in shaping cognition \citep{piaget-biology-knowledge, piaget1970genetic, ponty1969phenomenology, Merleau-Ponty2013-vs, Vygotsky1978-bk}.

Today the field is marked by a fundamental tension between two dominant programs. On one side, cognitivism—rooted in a nativist, information-processing paradigm—explains language and reasoning through amodal representations and computational rules, often assumed to reside in the brain \citep{chomsky1965aspects, fodor_modularity_1983}. On the other, 4E approaches (embodied, embedded, enactive, extended) ground cognition in the dynamic, situated action of the whole body within its environment \citep{varela1991embodied, maturana1991autopoiesis, noe_action_2004, gallagher2023-book, hutto2012radicalizing}. While both perspectives recognize the evolutionary and biological basis of cognition \citep{chomsky1965aspects, varela1991embodied}, they part ways on the role of the body: whether it is merely an implement for brain-based processes or a constitutive ground of mental life. This disagreement has persisted for decades. We return to it in detail in Section~\ref{sec:comparison}.
 
This monograph argues that this impasse stems from an incomplete and inaccurate understanding of the biological architecture, and the role played by each of the organs that are part of the architecture in various aspects of cognition. We proceed from the assumption that a comprehensive architecture is the necessary foundation upon which any successful theory of cognition must be built. Most of cognitive phenomena fall in place if assume the proposed architecture and dynamics of the body situated in the world. 

What if movement and behavior do not require initiation from a central control message? The Galilean–Newtonian revolution taught us that force is not needed to maintain motion, but only to change it. By analogy, we propose that the nervous system’s primary cognitive role is not to generate action, but to participate in its modulation by \emph{routing signals that alter ongoing dynamics}. Organisms without nervous systems already exhibit movement, and even isolated cardiac tissue continues to beat rhythmically on its own \citep{llinas2002vortex, landecker2007culturing}. The capacity for action is thus intrinsic to living cells. Unless redirected by internal or external affordances, action patterns persist—a \emph{principle of inertia for cognitive science}. 

This counter-intuitive insight—that action precedes control rather than being controlled—forms the foundation of our SMN model. As we will elaborate in Section~\ref{sec:model}, this leads to a fundamentally different understanding of how cognition emerges from biological architecture.

This paper argues for a model-driven approach that does not mistake correlation for causation. We contend that much of the biology relevant to cognition has been ignored precisely because it is not unique to humans. Our obsession with our own specialness has prevented us from seeing the general principles at work.

Our argument depends on adopting a model-driven approach. Following Nersessian’s account of model-based reasoning \citep{Nersessian1999_ModelBasedReasoning}, we treat models not as mirrors of reality but as tools for thinking, simulating, and reorganizing concepts. What we present here is not a single model but a family of models—an architecture—that captures the structural and dynamic conditions under which cognition emerges. The architecture itself is neither true nor false, but semantically coherent in Stegmüller’s sense \citep{Stegmueller1976_StructureDynamics}. We contend that many established facts fall into place if we \emph{grant} this architecture, even though the model is not derived from them. In this respect, our work is guided by alternative assumptions rather than conventional derivation from data.

%An alternative way to introduce the problem is through a thought experiment. 

\subsection{A Thought Experiemnt}

An alien mission sent a spaceship on an expedition to the earth to construct a picture of the world.

The spaceship has various sensors (transducers) on its outer surface and inside. It has light, sound, contact, temperature, and chemosensors. The ship has an internal (inbuilt) clock and an internal sense of time. But it has no pre-built location sensors, however it has accelerometers to detect changing state of motion.  Complicating matters, the crew is strictly confined to the ship and cannot physically move out of the spaceship to explore the world they seek to understand.

The engineers who built the ship placed the sensors at various places, and there were connectors to make the signals from the sensors available to the respective chip. An integrated snapshot is constructed only on the basis of concurrency (at any given point in time) on their dashboards. Since all the signals are encoded similarly, they have no signature of where they are coming from (sources), including whether causes for the signals are generated from within the ship or from outside. 

The crew's mission is to locate the source of the input signals coming from the sensors and construct a comprehensive picture of the world.

The ship's crew knew only how to count and compute the signals coming from the sensors. The programmers and mathematicians can write programs to compute, and their mechanical/electrical engineers can design and deploy actuators and additional sensors as and when necessary. If *and only if* required, a few more such ships can be deployed for the mission, and they can share the results through their communication channels.

What do they need to do within the ship to compute the location of the source of the input signals? How did the crew construct the picture (geometry) of the world?

The answer to this puzzle is expected to be in the form of the mechanism --  structure and dynamics -- of the spaceship.

\subsection{A Modern Version of Plato's Allegory of a Cave}

The signals reaching the dashboards of the crew of the spaceship appear transiently. At any given time, there is a snapshot of signals, including the signals coming from the various transducers. The snapshot passes from a temporally ordered stack of screens, from one screen to another at each passing instance. 
Since the tokens disappear instantly the only way to hold on to a pattern for a while is to pass them to another chip. There is no inbuilt way of retrieving the snapshots. The only way is to recurrently subject the spaceship to similar exposures. Such is the phenomenology available to the crew. 

How do they differentiate a set of tokens as coming from a specific kind of transducer, so that each datatype can be distinguished?  How do they distinguish the signals coming from within the spaceship from outside?

The crew is inside the 'cave' and cannot move out.  What innovations they have to make to construct a structure and dynamics of the world?




% Two major breakthroughs in science and technology helped us to understand so many things, while at the same time, they are also impeding cognitive science. The discovery of DNA, genetic code, genetic basis of variations leading to the modern synthetic theory of evolution, and the enormous and disruptive impact of information processing models in modern society have both illuminated and obscured our understanding of cognition.
 
\subsection{Novel Aspects of the Proposal}

While our Sensation-Modulating Network (SMN) framework shares affinities with enactive, embodied, ecological, and extended theories of cognition, it advances several distinctive contributions. For clarity, we group these novel aspects into five clusters: architectural principles, dynamics of action, semiotics and syntax, formalization and methodology, and philosophical implications. 

\begin{figure}[t]
\begin{adjustwidth}{-2in}{0in}
\centering
\begin{tikzpicture}[
  font=\small,
  >=Latex,
  node distance=10mm and 12mm,
  % --- Styles ---
  core/.style={rounded corners=6pt, draw=black, line width=0.7pt, fill=gray!10, inner sep=6pt, text width=36mm, align=center},
  group/.style={rounded corners=6pt, draw=black, line width=0.6pt, fill=gray!4, inner sep=6pt, text width=36mm, align=left},
  tag/.style   ={draw=black, fill=white, line width=0.5pt, inner sep=2pt, font=\footnotesize\bfseries, rectangle, rounded corners=3pt},
  arrow/.style ={-{Latex[length=2mm]}, line width=0.7pt},
  rel/.style   ={-{Latex[length=2.8mm]}, line width=0.6pt, dashed},
  bullet/.style={circle, fill=black, minimum size=2pt, inner sep=0pt},
  title/.style ={font=\bfseries},
]

% --- Core node ---
\node[core] (smn) {\textbf{SMN Architecture}\\[1mm]
\footnotesize Sensation-Modulating Network\\[-0.2mm]
\footnotesize (biological body plan + routing/modulation)};

% --- Five groups placed around in a star layout ---
\node[group, above=20mm of smn] (A) {%
\textbf{A. Structure}\\[1mm]
\scriptsize Polarity, topology, segmentation,\\
\scriptsize bilateral symmetry, antagonism.\\
\scriptsize Plasticity \& redundancy across zones.\\
\scriptsize Environmental forces as constitutive.};

\node[group, right=30mm of smn, yshift=8mm] (B) {%
\textbf{B. Dynamics}\\[1mm]
\scriptsize Primacy of halting \& modulation.\\
\scriptsize Layered action patterns:\\
\scriptsize FAP $\rightarrow$ HAP $\rightarrow$ NAP $\rightarrow$ TAP.\\
\scriptsize Memory as degeneracy of action patterns\\
\scriptsize Streams and snapshots of tokens as phenomenology};

\node[group, left=30mm of smn, yshift=8mm] (C) {%
\textbf{C. Syntax}\\[1mm]
\scriptsize Tokens from haltable actions.\\
\scriptsize Syntax exapted from body combinatorics.\\
\scriptsize USHAPs, TAPs, trace-making.\\
\scriptsize Representation as re-enactment.};

\node[group, below left=18mm and 18mm of smn] (D) {%
\textbf{D. Formal Sketch}\\[1mm]
\scriptsize Topology, graphs, categories, groups.\\
\scriptsize Petri nets, control \& information theory.\\
\scriptsize Testable predictions.};

\node[group, below right=18mm and 18mm of smn] (E) {%
\textbf{E. Philosophy}\\[1mm]
\scriptsize Recasts mind/body, objective/subjective,\\
\scriptsize universals/particulars, form/matter.};\\

% --- Arrows from core to groups ---
\draw[arrow] (smn.north) -- (A.south);
\draw[arrow] (smn.east) -- (B.west);
\draw[arrow] (smn.west) -- (C.east);
\draw[arrow] (smn.south west) -- (D.north east);
\draw[arrow] (smn.south east) -- (E.north west);

%\draw[rel] (A.south) -- ($(A.south)!0.55!(smn.north)$);
%\draw[rel] (B.west) -- ($(B.west)!0.55!(smn.east)$);
%\draw[rel] (C.east) -- ($(C.east)!0.55!(smn.west)$);
%\draw[rel] (D.north east) -- ($(D.north east)!0.5!(smn.south west)$);
%\draw[rel] (E.north west) -- ($(E.north west)!0.5!(smn.south east)$);

% --- Compact legend ---
%\node[anchor=north west, align=left] at ($(smn.south west)+(-28mm,-22mm)$) {%
%\begin{minipage}{0.48\linewidth}
%\footnotesize
%\textbf{Legend}\\[-0.25em]
%\begin{tabular}{@{}ll@{}}
%\raisebox{0.6ex}{\tikz\draw[arrow] (0,0)--(0.8,0);} & feedforward specification from SMN \\
%\raisebox{0.6ex}{\tikz\draw[rel] (0,0)--(0.8,0);}   & reciprocal constraint/feedback \\
%\end{tabular}
%\end{minipage}
%};


\end{tikzpicture}
\caption{Schematic overview: the SMN architecture (center) and five core proposal clusters (A–E). }
\label{fig:smn_overview}
\end{adjustwidth}
\end{figure}



\paragraph{A. Architectural Principles}
At the foundation, we foreground a biologically specific plan: the cognitive agent as a polarized, tubular, segmented, bilaterally symmetrical, and antagonistically organized network. Unlike many embodied or enactive theories that invoke a generic ``body'' \citep{clark1997being, gallagher2005how}, our model specifies the concrete architectural features from which cognition emerges. Within this architecture:
\begin{itemize}
    \item Polarity, topology, segmentation, bilateral symmetry, and antagonism serve as structural invariants grounding cognitive function.
    \item Plasticity and degeneracy across motor zones (hands, feet, facial gestures) account for neural and cognitive flexibility, enabling compensation, creativity, and adaptation.
    \item Environmental forces such as gravity or aquatic dynamics are treated as constitutive partners in cognition, reducing internal computational load through predictable feedback.
\end{itemize}

\paragraph{B. Dynamics of Action}
Cognition, in our account, does not originate in initiation but in modulation. The nervous system routes signals to alter ongoing dynamics:
\begin{itemize}
    \item \emph{Primacy of halting and modulation}: pauses and interruptions are cognitive acts, loci of freedom and deliberation, not mere absences of motion.
    \item \emph{Nested action patterns}: Fixed (FAPs), Haltable (HAPs), Negotiable (NAPs), and Transactional (TAPs) Action Patterns form a layered continuum, with cognitive flexibility arising from transitions across these strata.
    \item \emph{Memory}: body holds degenerate action schema as memory that have both ontogenetic and phylogenetic origins.
    \item \emph{Phenomenology}: the streams and snapshots of tokens filled in the place nodes of the Petri Net provide subjective experience
\end{itemize}

\paragraph{C. Semiotics and Syntax}
From these dynamics emerge semiotic and linguistic capacities, grounded in bodily action:
\begin{itemize}
    \item Tokens arise not as amodal symbols \citep{fodor1975language, pylyshyn1984computation}, but as repeatable, haltable actions generated stochastically in context.
    \item Body holds dynamic and degenerate action patterns (FAPs, HAPs, NAPs and TAPs) as memory
    \item Syntactic structure is exapted from the combinatorial organization of the body, yielding a bottom-up account of generativity.
    \item Unsaturated action patterns (USHAPs) provide substrates for concepts and representations; TAPs externalize meaning in shared traces, linking Peircean semiotics with Hebbian mechanisms \citep{hebb1949organisation}.
    \item Representation is redefined as re-enactment, reconciling enactivist critiques with cognitivist appeals to mental simulation.
\end{itemize}

\paragraph{D. Formalization and Methodology}
The framework is designed for precision and testability:
\begin{itemize}
    \item We integrate topology, graph theory, category theory, group theory, Petri nets, control theory, information theory, and signal processing to formalize biological architectures.
    \item The result is a unified, empirically testable framework capable of generating novel predictions, e.g., that disrupting proprioception impairs spatial reasoning \citep{proprioception_spatial, proprioception_reasoning}, or that mastery of action syntax precedes linguistic syntax.
\end{itemize}

\paragraph{E. Philosophical Implications}
Finally, the SMN model challenges several entrenched dichotomies in philosophy and cognitive science. By grounding cognition in a layered biological architecture, it reopens the terms of debate around objectivity and subjectivity, empirical and rational knowledge, universals and particulars, form and matter, and the relation of mind to body. We also question cognitivists aversion to actions, and radical enactivists aversion to representations. 

\paragraph{The Three Problems for Cognitive Science}
We propose that the core challenge for cognitive science is to answer a set of interrelated questions: (1) What structures and dynamics are necessary for phenomenological experience---the so-called ``hard problem'' \cite{chalmers1995facing}---to arise? (2) How does a cognitive agent construct a stable, geometric picture of its internal and external worlds? (3) How can this generated picture be shared meaningfully with other agents?  

Among the broader implications of this proposal is a re-characterization of what it means to be human. The mechanics of cognition, we argue, are not peculiar to humans, though certain aspects are predominant in our species. Our model also explains why machine learning is so computationally expensive—data- and memory-driven—while humans can operate effectively “at room temperature.” Unlike Turing’s model of computation, which depends on storing and manipulating tokens on an effectively infinite tape, our framework treats tokens as generated dynamically in the body and interpreted on the fly. These tokens are transient: once their immediate role is fulfilled, they are destroyed, leaving only parsimonious stochastic schemas of action patterns. 

Such schemas are not fixed or rational in the classical sense; they are continuously revised or discarded depending on how well they support predictive processing. Human reasoning appears rational only when enacted within rule-based games of formal semantics. What underlies this apparent rationality, however, is stochastic bodily processing that grounds memory and phenomenological experience. Rationality, in this view, is not intrinsic to the body but emerges in the transactional culture of symbolic practices, such as science and academia. Recent advances in machine learning corroborate this insight: rule-based rationality surfaces only after extensive trial and error, while stochastic behavior provides the true foundation.

This paper unfolds in five parts. Section~\ref{sec:model} develops the SMN architecture, detailing its biological principles and dynamic properties, along with sketch formalizations. Section~\ref{sec:phenomena} shows how the model explains a range of cognitive phenomena, from subjective experience to abstract concepts, by tracing the progression from saturated to unsaturated but grounded action patterns. Section~\ref{sec:comparison} situates the model among major theoretical frameworks, showing how it bridges the divide between cognitivist and 4E approaches. Section~\ref{sec:empirical} reinterprets existing experimental findings and derives novel, falsifiable predictions. Finally, the conclusion reflects on broader implications, arguing that this framework provides a biologically grounded path toward reconciling long-standing divisions in cognitive science.

We begin, then, with a detailed examination of the SMN architecture, showing how the biological principles create the conditions for cognitive emergence.






