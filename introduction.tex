\section{Introduction}
\label{sec:introduction}
The study of cognition is marked by a fundamental tension between seemingly irreconcilable research programs. On one side, cognitivism, rooted in the information processing paradigm, seeks to explain phenomena like language and reasoning through amodal symbols and computational rules, often located within the brain \cite{chomsky1965aspects, fodor_modularity_1983}. On the other, 4E (embodied, embedded, enactive, and extended) approaches ground cognition in the dynamic, situated actions of an agent's entire body within its environment \cite{varela1991embodied,  maturana1991autopoiesis, noe_action_2004}. While both camps agree on a biological basis for cognition, their foundational assumptions about the nature of the body and its role in mental life have led to decades of disagreement. This paper argues that this impasse stems from an incomplete understanding of the biological architecture that underpins cognition. We proceed from the assumption that a correct model of this architecture is the necessary foundation upon which any successful theory of cognition must be built.

We propose that the core challenge for cognitive science is to answer a set of interrelated questions: (1) What structures and dynamics are necessary for phenomenological experience—the so-called "hard problem"—to arise? (2) How does a cognitive agent construct a stable, geometric picture of its internal and external worlds? (3) How can this generated picture be shared meaningfully with other agents? To address these, we depart from neuro-centric models and introduce an alternative framework centered on a specific, yet generalizable, model of a cognitive agent's body plan.

Our central proposal is a dynamic architecture we term the Sensation-Modulating Network (SMN). We model the agent as a layered network of action zones, organized according to fundamental biological principles: segmented, polarized, bilaterally symmetrical, and antagonistically organized. Within this architecture, cognition emerges not from a central controller initiating action, but from the capacity to \textit{halt} and modulate ongoing, rhythmic action patterns. To describe these dynamics, we will introduce a specific terminology: a hierarchy of patterns ranging from deep, immutable **Fixed Action Patterns (FAPs)**, to consciously accessible **Haltable Action Patterns (HAPs)**, which, when shared, become **Transactional Action Patterns (TAPs)**. It is this capacity for modulation, we argue, that grounds semiotics, enables the creation of symbols, and gives rise to generative syntax.

This paper unfolds in five parts. First, we will elaborate on the proposed SMN model, detailing its architectural principles and its dynamic properties. Second, we will demonstrate how this model can explain a wide range of cognitive phenomena, from subjective experience to the emergence of abstract concepts. Third, we will situate our model within the broader theoretical landscape, comparing its core tenets and explanatory power to other leading theories of cognition. Fourth, we will reinterpret existing experimental evidence and propose novel, falsifiable predictions derived from our framework. Finally, we will conclude by discussing the broader implications of this model, arguing that it offers a foundational departure from previous approaches and provides a path toward reconciling the long-standing divisions in the field.