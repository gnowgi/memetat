
\subsection{Evolutionary Biology of Anatomical Disengagement Leading to Cognitive Disengagement}
\label{subsec:evolutionary_disengagement}

The capacity for haltability that underlies cognition has deep evolutionary roots. We can trace the emergence of cognitive capabilities through what we call the \textit{evolutionary biology of anatomical disengagement}—the progressive decoupling of body functions that enabled increasingly sophisticated action patterns.

\subsubsection{From Harder-Actions to Softer-Actions}
Let us call the tightly-coupled actions as \textit{harder-actions} (e.g. the coupling between locomotion and feeding in the earthworms), and the decoupled actions as \textit{softer-actions}. During the course of evolution more anatomical disengagements may have given rise to the availability of more such softer-actions.

These examples indicate how one could speculatively weave a story of the evolution as a story of decoupling the body into multiple zones, where each zone can act partially independent from another, and exhibit a distinguishable action pattern. The development of tongue, lips, jaws, pharynx, larynx, gills, lungs, fins, tails, ears, eyes, limbs, toes, fingers, neck, shoulder, hip and so on are interpreted in this story as anatomical disengagement (or decoupling). One can draw a tree of anatomical disengagement representing the epi-physiological bifurcation over and above the phylogenetic tree of evolution.

For example, in early vertebrates as in cephalopods, the feeding and breathing action patterns (filter feeding habit) are not decoupled. In Gnathostoms, we see the stoma (mouth) differentiated through the evolution of jaws enabling decoupled breathing habits from feeding habits. Episodes of decoupling could be reconstructed for the evolution of simpler buccal cavity differentiated into complex cavity, developing teeth, tongue, lips, pharynx, larynx etc. In parallel, an undulating body (as in lampreys) gets decoupled into localised and bilaterally symmetrical fins, decoupling locomotion as a entire-body function. Similarly, one could consider the decoupling of the undulating alimentary canal from the entire undulation of the body.

These episodes of differentiation might have been naturally selected because of the economic value of decoupling, as localised movement is inexpensive than whole body movement. Each episode of such differentiation is an episode of decoupling leading to the evolution of independent action patterns (habits). Thus the gradual polarisation and bilateral symmetry of the body-plan through evolution leading to differentiation of action zones can be woven into a phylogenetic story of decoupled and localised zones of action patterns (habits).

\subsubsection{Economic Dimensions of Disengagement}
This disengagement has an economic dimension, without which it is difficult to understand how it could have played a role in natural selection. The agent can do more work with less effort (spending less energy) because of disengagement. A body plan of an organism that has a coupled movement for both ingestion and locomotion is expensive, than when they are decoupled. Moving when not eating, or eating when not moving is a new found possibility.

Once we have multiple softer-action zones, it is possible to rest some while the others are active. This is the context for the genesis of \textit{haltability}. Haltable variations, one can speculate, could be sexually/culturally selected. The principle of economy also enabled the organism to perform one action while halting another. Isn't this how we describe modulation? The aspect of control we ascribe to modulation arises only when we hold one variable while modifying another. Can we use this insight to ground the regulatory actions required for cognitive processing in haltable action patterns? We demonstrate how this can be the butterfly effect in cognition.

\subsubsection{The Principle of Layering}
\label{subsec:principle_layering}

Since modulating certain beats such as heartbeats is not affordable, we may consider situating actions over and above the core physiological mechanisms. In order for the actions to be affordable, the interactions of the sustaining layer must continue, and they should generate sufficient surplus. When we say cognition is enactive, it implies that the emancipation from sustaining mechanisms is expensive. Though autopoietic mechanisms may include actions, they are uninterruptible, hence no liberty to introduce gaps here. Hence autopoiesis as a mechanism to compensate the lost energy and matter takes care of the sustaining layer and provides the necessary surplus in the system making actions possible \cite{maturana1991autopoiesis}. This is also an action, but the uninterrupted pace at which this action takes place has no liberty for introducing \textit{gaps} in this layer. In other words, the system can't physiologically afford to halt. However, it is this state that could enable ephemeral actions on the periphery of an autopoietic system whenever and wherever possible. This is made possible by a differentiated body plan that enables a division of labour. Some layers are busy in not only replenishing the loss of energy and matter but also generating surplus energy and matter, such that other layers in the body can \textit{halt}. This design now has room for free action. In this perspective, it is an uninterrupted action of some layer that grants freedom to some other layers. It is this partial break from uninterrupted work, that gives rise to the freedom to enter into the cognitive domain. It is in this subtle sense, that our model differs from Maturana and Varela's account of the connections between biology and cognition. The subtlety we introduce is haltability.

This differentiation of layers, as against the uniform distribution of work, in a system facilitating deviation from the normal course of actions, gave rise to the roots of cognitive state. We shall call this \textit{the principle of layering}, which is over and above the design principles of polarization and asymmetry we discussed earlier. The sense of being over and above can be characterised by naming metaphorically, this principle as epi-physiological or epi-biological.

\subsubsection{Cognitive Space as Memetat: The Geometric Semiotic Habitat}
\label{subsec:memetat}

Building on the principle of layering, we now introduce the concept of \textit{memetat}—the cognitive space that emerges from the SMN architecture. Cognitive space is a transient geometrical space, \textit{memetat}, constructed through multiple, recursive, recurrent, and haltable serial action patterns, called \textit{memets}, which can be retained by reenacting and creating interpretable traces. The agent's body is modeled as a layered sensation-modulating network (SMN), which is functionally distinguished into a differentiating and filtering network (DFN) and an integrating network (IN).

The SMN is modeled as a polarized and bilaterally symmetrical PetriNet (PN), a bipartite graph of transitions and places. A set of place nodes of the PN enter SMN from the environment where the cognitive agent is situated via the interfacing nodes of transitions. The transition nodes of the PN are sensory transducers and actuators, constituting the sensory-motor part of the architecture. The resulting tokens from the sensory-motor part of the network form as the other set of the place nodes of the PN in the form of a nested stack of neural connections holding transient tokens for a while. The tokens pass through one stack of network into another until they vanish completely. The phenomenological space is constructed by the tokens in this transient nested stack of networks through multiple, recursive, recurrent and haltable serial action patterns by the sensation-modulating network.

The geometry is computed by calculating (1) the differentiation of differences in the phenomena through self-modulation, (2) the relative location of the phenomena following the principle of \textit{mapping the delay with distance}, and (3) the principle of the concomitance of phenomena granted by a dynamic \textit{fire-together-wire-together} architecture of the SMN. The spectrum of perception to conception (spectrum of abstraction) is explained through a model of a spectrum of saturation of multiple sensory modalities and the corresponding dynamic spectrum of engagement and disengagement of action patterns. Meaning and evaluation are grounded in the deeper layers of the layered architecture of the SMN. Transient action patterns can be memorized and recalled only by re-enacting them, while the traces of action patterns can be persistent and hence can be recalled and gamified. Human culture is modeled as a confluence of multiple microworlds, where each microworld is constructed by mutually stimulating rules following actions that define the boundaries of the transactional playground.


\subsubsection{Habitat as Co-Architect: Fluid and Gravity}
\label{subsec:habitat}
The SMN co-evolves with its habitat. Two ambient fields shape its design:
\begin{enumerate}
    \item \textbf{Aquatic fluid}: viscosity and buoyancy enable efficient oscillatory control, 
    stabilizing limit-cycle behaviors useful for locomotion and sensing.
    \item \textbf{Gravity}: a constant vector field that grounds posture, balance, and orientation.
\end{enumerate}
These fields provide \emph{external computation}: the environment carries lawful structure that the agent exploits.
\marginpar{Distributed computation in the habitat complements decentralized control in the SMN.}
\todo{Cite Gibson (1979), O'Regan \& Noë (2001) on sensorimotor contingencies; cite work on gravity as a strong prior.}

\subsubsection{Thermodynamic Economy: Action Over Storage}
Cognition at biological ``room temperature'' is possible because organisms offload computation to stable environmental regularities. 
Rather than encoding exhaustive state spaces, agents store and refine \emph{action repertoires} that couple to habitat affordances. 
This design is energetically parsimonious: it minimizes memory maintenance and leverages the free structure present in fluid and gravity.
\marginpar{Action patterns as energy-efficient ``indices'' into the world.}
\todo{Add citations: Friston (Free Energy Principle), embodied thermodynamics; historical precursors (e.g., Moravec).}

\subsubsection{Formalizing FAP--HAP--TAP}
Let $x(t)$ denote a low-dimensional controller capturing a segment's antagonistic pair. 
\textbf{FAPs} correspond to limit cycles $\Gamma$ in the phase portrait (stable rhythmic behaviors). 
\textbf{HAPs} introduce control parameters that create bifurcation surfaces $\mathcal{B}$ allowing halts and reversals. 
\textbf{TAPs} are sequences of controlled entries/exits from neighborhoods of $\Gamma$ across segments, 
coordinated by inter-segment couplings and habitat feedback. 
\todo{Insert a figure: nested cycles (FAP) with interrupt surfaces (HAP) and transactional arcs (TAP); add references to attractors/bifurcations.}





