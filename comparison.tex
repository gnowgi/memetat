\section{Reconciliation and Comparison with Other Theories}
\label{sec:comparison}

It is ironic that the both the camps, cognitivist and enactivist, despite their ground in evolutionary biology had to disagree so strongly. Can we reconcile them? Is it action or representation? 

Though embodied cognitive models do not consider brain as the sole cognitive organ\cite{noe_action_2004,OReganNoe2001Sensorimotor,varela1991embodied,clark1997being} the idea that brain is still considered a the \textit{site} of cognition, relegating the rest of the body non-processing roles. On this count, we depart from both cognitivists and proponents of 4E cognition, and propose that central nervous system doesn't process information, but acts like a \textit{motherboard}, selective routing and broadcasting messages between the various action zones of the body. The actual processing happens in the \textit{whole-body}, which we model as \textit{senation-modulating network}.  Cognition is not a function of any one organ of the body, but as a whole.  Just as a cell is a structural and functional unit of life, and not any one organelle of a cell, sensation-modulating network is the structural and functional unit of cognition, not any one organ of the body. 

The claim that the brain is the seat of cognitive processes has a deep lineage, stretching from classical antiquity through early modern medicine to contemporary neuroscience. This is intuitive, because we have strong correlations. We are driven by the \textit{apparent} anatomical peculiarities of the human body—the proportionally large brain, bipedal locomotion, dexterous hands, and sophisticated vocal apparatus, speach, thought, reason, rich socio-cultural life— they all must go hand-in-hand. There are several cognitive scientists, biologists and philosphsers who consider brain as the site of cognition\cite{Gazzaniga2018Consciousness,Churchland2013Touching,Dehaene2014Consciousness,Jackendoff2002Foundations,AdamsAizawa2010Cogs}.  The most compelling correlation drawn is human language, rule driven representational competencies. The founders of cognitivism\cite{chomsky1965aspects,fodor_modularity_1983} gave the philosophical support leading to the establishment of cognitive neuroscience. The experimental methods used by them were not only ingenious but scientific, that is beyond question. What is questionable is to look for an organ in the body that is responsible for cognition. 

 In the recent past the most notable champion to look elsewhere is Alva Noe. After all we knew from evolution that single or multi-cellular organisms exhibited movements as well as variations in patterns without a nervous system \citep{llinas2002vortex}. Even a lump of cardiac tissue is known to beat rhythmically, not only a detached beating heart from the body, leading to myogenic hypothesis \citep{landecker2007culturing}. Almost all animal cells have a cytoskeleton, to generate sufficiently sophisticated movements without a nervous system. The cells do navigate around their space even without depending on special sense organs, that is to say that every cell responds to stimulus, and have endogenous means of acting on the environment. More and more experimental evidence is catching up to show that non-neuronal cognitive phenomena, intelligence, memory and basal cognition are all over biological space \citep{Levin2023, biomimetics-Levin2023}. And almost all known material in the universe is electrically active \citep{Peratt1996plasma}, not merely the sense organs, muscles and the neurons in the body. Therefore to single out the nervous system for being a central cognitive organ based on phrenological studies is unfounded \citep{anderson2014phrenology}.


Having demonstrated how the SMN model provides unified explanations for cognitive phenomena in Section~\ref{sec:phenomena}, we now situate this framework within the broader theoretical landscape of cognitive science. The Sensation-Modulating Network (SMN) is not proposed in a vacuum. It enters into a rich and complex dialogue with the major theories of cognition and consciousness, both classical and modern. To clarify the unique contribution of our model, it is essential to explicitly map its points of affinity and, more importantly, its fundamental differences with these established frameworks. This section situates the SMN in the theoretical landscape, demonstrating how it selectively incorporates insights from previous work while offering a novel synthesis grounded in a more detailed and biologically plausible architectural foundation.

As we will see, the SMN model's greatest contribution lies not in rejecting existing approaches, but in providing a framework that reconciles them. This reconciliation is particularly important given the current state of cognitive science, where the success of large language models has created new opportunities for theoretical synthesis.

\subsection{The Reconciliation Challenge: Bridging Cognitivism and 4E Cognition}
The context of recent advances in the GPT model can be interpreted as an irony: a behaviorist reinforcement learning model successfully implements cognitivist's representational model of generative grammar. In other words, the cognitive aspects of rules and representations, i.e., of the language world, seem to be learnable and usable in a transactional setting through a behaviorist model. If we assume that the GPT model does `truthfully' capture how the learning of symbolic patterns, generative transformations of patterns, and using them in a social, transactional setup of our cognition can be implemented in an artificial machine, what puzzles about human cognition remain if we adhere to 4E cognitive framework? 

The core challenge posed to the 4E framework by cognitivists is to account for how a generative (combinatorial) representational learning model be grounded in an enactive, embodied, embedded, and extended cognitive agent. This proposal is an attempt to address this challenge, thus exploring a possibility of a reconciliation.

We employ an abductive approach by beginning with a description of our model as a hypothesis and then proceeding to reason using it, and testing its explainability. In the process, we demonstrate how reconciliation of the existing cognitive frameworks is possible.

\subsection{Grounding Syntax in the Context of Large Language Models}
Considering that pattern recognition, recombination and responding are the core of the GPT model and syntax is responsible for generative grammar, \textbf{\textit{the problem is to ground/ locate syntax in an artificial/ biological agent capable of generating variable atomic action patterns.}} Identification of atomic action patterns in an enactive model is a requirement for generative potential. We assume that a pattern is a necessary condition to encode or hold information.

The success of large language models presents a fascinating paradox for cognitive science. These models, built on behaviorist reinforcement learning principles, have achieved remarkable success in implementing what appears to be cognitivist representational models of generative grammar. This suggests that the apparent opposition between behaviorist learning and cognitivist representation may be more superficial than fundamental. The SMN model provides a framework for understanding how this reconciliation is possible: both learning and representation can be grounded in the same underlying mechanism of action pattern modulation. 

\subsection{Haltability as a Requirement for Generating Syntax}
A linear invariant action pattern cannot generate syntax. \textbf{\textit{How does an agent generate a punctuation in the action pattern without halting?}} Therefore, a 4E model must base symbol grounding among atomic reproducible haltable action patterns (HAPs). Haltability is an affordance, only when life-sustaining action patterns (e.g., heartbeats) continue. The bottom-most metabolic layer provides a canvas or a phenomenological background composed of invariant (inter)action patterns.

This insight is crucial for understanding how syntax emerges from action. Traditional models of syntax focus on the combinatorial rules for combining symbols, but they often fail to explain how the symbols themselves are created. The SMN model addresses this by showing that symbols (tokens) emerge from the agent's ability to halt ongoing action patterns, creating discrete, repeatable units that can then be combined according to syntactic rules. The pause itself becomes a cognitive act—the creation of a boundary that enables symbolic manipulation.

\subsection{How Can an Enactive Model Be Generational?}
\textbf{\textit{How can an enactive model be generational?}} In a body with multiple haltable action zones, syntactical variations and nested compositions can be implemented. For example, multiple haltable action zones in fingers, wrist, arm, hip, and leg movements while dancing; multiple haltable action zones in the buccal cavity, lungs, and vocal apparatus together can generate numerous syntactical variations leading to a complex symbolic action space (language). These actions are affordable precisely because they can be disengaged from life-sustaining actions. This enhances their symbolic potentiality.

The key insight here is that generativity emerges from the combinatorial possibilities inherent in the body's segmented architecture. Each action zone provides a finite set of possible actions (a "lexicon" of motor tokens), and the ability to halt and sequence these actions across multiple zones creates the syntactic structure necessary for generative systems. This is not unique to language—it applies to any complex, coordinated behavior that involves multiple body parts working together in structured sequences.

\subsection{An Embodied Account for Symbol Ungrounding}
\textbf{\textit{How do we map HAPs (symbols) to experiences? Where are the rules — the mapping between HAPs and experiences — emerging from?}} When the action and experience are tightly coupled, the requirement for a mapping does not arise. This tight coupling can be punctuated by haltable action patterns, which can be arbitrary, giving us the apparent feeling of being disembodied. Meaning as experience and phenomenological.

This addresses one of the central puzzles in cognitive science: how do symbols become "ungrounded" from their original context while maintaining their meaning? The SMN model shows that this ungrounding is not a mysterious semantic process but a physical one. When an action pattern is performed without its original object (creating an Unsaturated HAP), it becomes a token that can be manipulated independently. The "rules" for mapping these tokens to experiences emerge from the agent's repeated interactions with the world, creating stable associations through the Hebbian principle of "fire together, wire together."

\subsection{The Necessity of HAPs for TAPs and Culture}
\textbf{\textit{How do different agents engage each other?}} Since haltable action patterns are imitable, without compromising on life-sustaining actions, agents can afford to dance, sing and play together, celebrating the social/ cultural games (rule-based practices) and creating micro-worlds (memetats). This way a 4E framework can be shown to be generative in a representationally rich world. The potential of HAPs becoming transactional action patterns (TAPs) such as signaling, commanding, instructing, talking, etc. makes cultural practices implementable within a 4E framework.

This is crucial for understanding how culture emerges from individual cognition. The key insight is that HAPs are inherently imitable because they are haltable—they can be performed without disrupting essential life functions. This makes them ideal candidates for social transmission. When multiple agents perform similar HAPs in coordination, they create Transactional Action Patterns (TAPs) that form the foundation of cultural practices. The shared nature of the body plan across agents makes this coordination possible, creating the inter-subjective space necessary for culture.

\subsection{Connecting Internal and External Representations}
\textbf{\textit{What constitutes shared memories?}} Since action patterns can generate traces in the material world, with variable life in the form of sounds, inscriptions, etc., they become available for other agents as external representations/ memories. This enables a possibility for a rich inter-subjective space, creating grounded and mediated knowledge space within the 4E framework.

This addresses the puzzle of how knowledge can be shared between agents. The SMN model shows that shared memories are not mysterious mental entities but physical traces of action patterns that persist in the environment. When an agent performs a HAP, it leaves traces (sounds, marks, movements) that other agents can perceive and interpret through their own action-based understanding. This creates a bridge between internal representations (USHAPs) and external representations (traces), enabling the transmission of knowledge across agents and generations.

\subsection{Toward a Unified Framework}
Having grounded mediated cognition within the world of HAPs and TAPs, the so-called disembodied rules and representations can be shown to be actually embodied. Thus we see the possibility of representations within the enactive models of cognition and see no reason to reject representations as unreal. The data-centric machine learning models such as GPT can be interpreted as extreme extensions of externalized traces of human cultural actions. Similarly, aversion towards behaviorism by cognitivism is unfounded since GPT demonstrated generative capabilities through a reinforcement learning model. And the behaviorist aversion to internal memories and processes is equally unfounded because the learning model is not a black box, but a statistical neural network. 

The SMN model provides a framework for reconciling the major theoretical divides in cognitive science. By grounding both learning and representation in the same underlying mechanism of action pattern modulation, it shows that the apparent opposition between behaviorism, cognitivism, and 4E cognition is more superficial than fundamental. All three approaches capture different aspects of the same underlying process: the agent's dynamic engagement with its world through modulated action patterns. The SMN model provides the architectural foundation that makes this reconciliation possible, offering a unified framework that preserves the insights of each approach while transcending their limitations.

\subsection{Classical Computational and Representational Theories}
\label{subsec:comparison_classical}

\subsubsection{Classical Cognitivism (Fodor, Chomsky)}
\label{ssubsec:cognitivism}
The "mind-as-computer" metaphor of classical cognitivism, which posits that cognition is the rule-governed manipulation of amodal symbols, represents the SMN's primary point of departure. However, the two frameworks share a common interest in the nature of discrete, combinatorial systems.

\paragraph{Affinities:} The SMN model concurs with cognitivism on a crucial point: any system capable of supporting complex thought and language must have a way to create discrete, repeatable tokens and a set of rules for combining them (a syntax). The cognitivist focus on generative syntax correctly identifies a central problem that any viable theory must solve.

\paragraph{Differences:} The divergence is foundational. First, concerning the \textbf{nature of symbols}, cognitivism's symbols are abstract, amodal, and arbitrarily related to their referents, leading to the intractable symbol grounding problem. The SMN's tokens (USHAPs), by contrast, are modal, non-arbitrary, and intrinsically grounded in the very action patterns the agent uses to interact with the world. Second, regarding the \textbf{locus of cognition}, cognitivism places it squarely in the brain. The SMN model distributes it across the entire body-environment system, with the brain (as the Integrating Network) playing a crucial but not exclusive role. Finally, concerning the \textbf{origin of syntax}, Chomskian linguistics posits an innate, brain-based, and uniquely linguistic module. The SMN argues that syntax is exapted from the universal, pre-existing combinatorial logic of the body's segmented motor architecture. Language, in our view, inherits its syntactic structure from action; it does not invent it.
\subsection{Theories of Embodiment and Situatedness}\label{subsec:comparison_embodiment}
The SMN model shares a great deal of philosophical ground with theories that fall under the banner of 4E (Embodied, Embedded, Enactive, and Extended) cognition. However, it differs in its level of architectural specificity and its treatment of representation.  \subsubsection{Ecological Psychology (Gibson)}\label{ssubsec:ecological}\paragraph{Affinities:} The SMN is deeply indebted to Gibson's pioneering work. We fully embrace the concept of "affordances" as the primary way the environment reveals itself to an agent. The core idea of a tightly coupled agent-environment system, where the world is understood in terms of its possibilities for action, is central to our model.
\paragraph{Differences:} The primary divergence lies in the emphasis on the agent's contribution to the dynamic. Gibson's theory focused primarily on the rich structure of the environment and the agent's ability to "directly perceive" it. The SMN provides a detailed account of the \textit{bodily architecture} that makes this perception possible, arguing that the world is not just perceived but actively \textit{constructed} through action. Our model details the specific mechanisms—the segmented, antagonistic body plan, the primacy of halting, and the motoric origin of space and time—that an agent uses to carve a meaningful world out of the field of affordances.  \subsubsection{Enactivism (Varela, Maturana, Thompson)} \label{ssubsec:enactivism} \paragraph{Affinities:} The SMN is a fundamentally enactive model. It shares the core tenets that cognition is "sense-making" performed by an autonomous agent, that action is inseparable from perception, and that cognition is a process of bringing forth a world.  \paragraph{Differences:} The key difference is in the level of mechanistic detail and the stance on representation. While enactivism provides a powerful philosophical and biological foundation, it often remains at a high level of abstraction. The SMN proposes a \textit{specific and concrete set of architectural and dynamic principles} (the body plan, the FAP/HAP/TAP hierarchy, the DFN/IN distinction) to explain \textit{how} enaction is physically realized. Furthermore, where some radical forms of enactivism reject representation entirely, the SMN offers a middle path. It redefines representation as the re-enactment of "unsaturated" action patterns (USHAPs), thereby preserving the power of abstract thought, simulation, and imagination within a fully embodied and grounded framework. This allows the SMN to bridge the gap with cognitivist insights in a way that other enactive models cannot.
\subsection{Modern Neuroscientific and Information-Theoretic Models}\label{subsec:comparison_modern}The SMN also enters into a crucial dialogue with contemporary theories that model cognition and consciousness in terms of information, prediction, and neural dynamics.
\subsubsection{Predictive Processing and The Free Energy Principle (Friston)}\label{ssubsec:fep}
\paragraph{Affinities:} The SMN is highly compatible with the core idea of the Free Energy Principle (FEP), which posits that organisms act to minimize prediction error or "surprise." The epistemic actions of the SMN—the probing and testing of the world through HAPs—can be seen as the physical implementation of the FEP's central mechanism of "active inference." Both models view the agent as a system fundamentally driven to reduce its uncertainty about the world.
\paragraph{Differences:} The two models operate at different levels of explanation. The FEP is a high-level, abstract computational principle that describes \textit{what} a cognitive system must do to remain viable (minimize free energy). The SMN, by contrast, is a lower-level architectural model that proposes \textit{how} a biological agent physically accomplishes this. It provides the concrete bodily machinery—the antagonistic action zones, the DFN/IN structure, the hierarchy of action patterns—that could implement the abstract mandate of the FEP.\subsubsection{Theories of Consciousness (IIT and GWT)} \label{ssubsec:consciousness_theories}
\paragraph{Affinities:} The SMN shares with Integrated Information Theory (IIT) the commitment to a physically grounded explanation of consciousness. With Global Workspace Theory (GWT), it shares the core concept of a "broadcasting" mechanism that makes information globally available to the system.  \paragraph{Differences:} The divergence is significant. IIT defines consciousness as a measure of a system's intrinsic causal power and integrated information (Phi); it is a property of a system's structure. The SMN, however, defines consciousness as a dynamic \textit{process} that arises from an agent's \textit{actions} in its environment. For the SMN, consciousness is something an agent \textit{does}, not something a structure \textit{has}. Similarly, while GWT posits a global workspace, the \textit{content} of that workspace is typically understood as sensory or cognitive information. In the SMN, what is broadcast by the Integrating Network are the states of the body's various action zones. The content of consciousness is not disembodied information, but the globally available state of the acting body itself.
\begin{table}[ht]
\centering
\caption{A Comparative Summary of Cognitive Theories}
\label{tab:theory_comparison}
\resizebox{\textwidth}{!}{
\begin{tabular}{|l|l|l|l|l|}
\hline
\textbf{Theory} & \textbf{Locus of Cognition} & \textbf{Nature of Representation} & \textbf{Role of Body} & \textbf{Key Contribution} \\
\hline
\textbf{SMN Model} & \textbf{Entire Body-Environment System} & \textbf{Grounded, Unsaturated Action Patterns (USHAPs)} & \textbf{The very medium of cognition} & \textbf{Provides a specific, unified architectural mechanism} \\
\hline
Classical Cognitivism & Brain & Amodal, Arbitrary Symbols & Input/Output Device & Formalized computation and syntax \\
Ecological Psychology & Agent-Environment System & Rejects Internal Representations & Perceptual System & Concept of affordances \\
Enactivism & Agent-Environment System & Rejects Internal Representations & Autonomous, Sense-Making System & Principle of autopoiesis and sense-making \\
Predictive Processing & Brain (as a statistical model) & Probabilistic, Generative Models & Instrument for Active Inference & Formalizes cognition as prediction error minimization \\
IIT / GWT & Brain (specific neural structures) & Information Patterns & Input/Source of Data & Models the structure/process of consciousness \\
\hline
\end{tabular}}
\end{table}