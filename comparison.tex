\section{Affinities and Differences with Other Theories}
\label{sec:comparison}
The Sensation-Modulating Network (SMN) is not proposed in a vacuum. It enters into a rich and complex dialogue with the major theories of cognition and consciousness, both classical and modern. To clarify the unique contribution of our model, it is essential to explicitly map its points of affinity and, more importantly, its fundamental differences with these established frameworks. This section situates the SMN in the theoretical landscape, demonstrating how it selectively incorporates insights from previous work while offering a novel synthesis grounded in a more detailed and biologically plausible architectural foundation.

\subsection{Classical Computational and Representational Theories}
\label{subsec:comparison_classical}

\subsubsection{Classical Cognitivism (Fodor, Chomsky)}
\label{ssubsec:cognitivism}
The "mind-as-computer" metaphor of classical cognitivism, which posits that cognition is the rule-governed manipulation of amodal symbols, represents the SMN's primary point of departure. However, the two frameworks share a common interest in the nature of discrete, combinatorial systems.

\paragraph{Affinities:} The SMN model concurs with cognitivism on a crucial point: any system capable of supporting complex thought and language must have a way to create discrete, repeatable tokens and a set of rules for combining them (a syntax). The cognitivist focus on generative syntax correctly identifies a central problem that any viable theory must solve.

\paragraph{Differences:} The divergence is foundational. First, concerning the **nature of symbols**, cognitivism's symbols are abstract, amodal, and arbitrarily related to their referents, leading to the intractable symbol grounding problem. The SMN's tokens (USHAPs), by contrast, are modal, non-arbitrary, and intrinsically grounded in the very action patterns the agent uses to interact with the world. Second, regarding the **locus of cognition**, cognitivism places it squarely in the brain. The SMN model distributes it across the entire body-environment system, with the brain (as the Integrating Network) playing a crucial but not exclusive role. Finally, concerning the **origin of syntax**, Chomskian linguistics posits an innate, brain-based, and uniquely linguistic module. The SMN argues that syntax is exapted from the universal, pre-existing combinatorial logic of the body's segmented motor architecture. Language, in our view, inherits its syntactic structure from action; it does not invent it.
