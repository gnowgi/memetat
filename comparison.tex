\section{Affinities and Differences with Other Theories}
\label{sec:comparison}
The Sensation-Modulating Network (SMN) is not proposed in a vacuum. It enters into a rich and complex dialogue with the major theories of cognition and consciousness, both classical and modern. To clarify the unique contribution of our model, it is essential to explicitly map its points of affinity and, more importantly, its fundamental differences with these established frameworks. This section situates the SMN in the theoretical landscape, demonstrating how it selectively incorporates insights from previous work while offering a novel synthesis grounded in a more detailed and biologically plausible architectural foundation.

\subsection{Classical Computational and Representational Theories}
\label{subsec:comparison_classical}

\subsubsection{Classical Cognitivism (Fodor, Chomsky)}
\label{ssubsec:cognitivism}
The "mind-as-computer" metaphor of classical cognitivism, which posits that cognition is the rule-governed manipulation of amodal symbols, represents the SMN's primary point of departure. However, the two frameworks share a common interest in the nature of discrete, combinatorial systems.

\paragraph{Affinities:} The SMN model concurs with cognitivism on a crucial point: any system capable of supporting complex thought and language must have a way to create discrete, repeatable tokens and a set of rules for combining them (a syntax). The cognitivist focus on generative syntax correctly identifies a central problem that any viable theory must solve.

\paragraph{Differences:} The divergence is foundational. First, concerning the **nature of symbols**, cognitivism's symbols are abstract, amodal, and arbitrarily related to their referents, leading to the intractable symbol grounding problem. The SMN's tokens (USHAPs), by contrast, are modal, non-arbitrary, and intrinsically grounded in the very action patterns the agent uses to interact with the world. Second, regarding the **locus of cognition**, cognitivism places it squarely in the brain. The SMN model distributes it across the entire body-environment system, with the brain (as the Integrating Network) playing a crucial but not exclusive role. Finally, concerning the **origin of syntax**, Chomskian linguistics posits an innate, brain-based, and uniquely linguistic module. The SMN argues that syntax is exapted from the universal, pre-existing combinatorial logic of the body's segmented motor architecture. Language, in our view, inherits its syntactic structure from action; it does not invent it.
\n\subsection{Theories of Embodiment and Situatedness}\n\label{subsec:comparison_embodiment}\nThe SMN model shares a great deal of philosophical ground with theories that fall under the banner of 4E (Embodied, Embedded, Enactive, and Extended) cognition. However, it differs in its level of architectural specificity and its treatment of representation.\n\n\subsubsection{Ecological Psychology (Gibson)}\n\label{ssubsec:ecological}\n\paragraph{Affinities:} The SMN is deeply indebted to Gibson's pioneering work. We fully embrace the concept of "affordances" as the primary way the environment reveals itself to an agent. The core idea of a tightly coupled agent-environment system, where the world is understood in terms of its possibilities for action, is central to our model.\n\n\paragraph{Differences:} The primary divergence lies in the emphasis on the agent's contribution to the dynamic. Gibson's theory focused primarily on the rich structure of the environment and the agent's ability to "directly perceive" it. The SMN provides a detailed account of the *bodily architecture* that makes this perception possible, arguing that the world is not just perceived but actively *constructed* through action. Our model details the specific mechanisms—the segmented, antagonistic body plan, the primacy of halting, and the motoric origin of space and time—that an agent uses to carve a meaningful world out of the field of affordances.\n\n\subsubsection{Enactivism (Varela, Maturana, Thompson)}\n\label{ssubsec:enactivism}\n\paragraph{Affinities:} The SMN is a fundamentally enactive model. It shares the core tenets that cognition is "sense-making" performed by an autonomous agent, that action is inseparable from perception, and that cognition is a process of bringing forth a world.\n\n\paragraph{Differences:} The key difference is in the level of mechanistic detail and the stance on representation. While enactivism provides a powerful philosophical and biological foundation, it often remains at a high level of abstraction. The SMN proposes a *specific and concrete set of architectural and dynamic principles* (the body plan, the FAP/HAP/TAP hierarchy, the DFN/IN distinction) to explain *how* enaction is physically realized. Furthermore, where some radical forms of enactivism reject representation entirely, the SMN offers a middle path. It redefines representation as the re-enactment of "unsaturated" action patterns (USHAPs), thereby preserving the power of abstract thought, simulation, and imagination within a fully embodied and grounded framework. This allows the SMN to bridge the gap with cognitivist insights in a way that other enactive models cannot.
\n\subsection{Modern Neuroscientific and Information-Theoretic Models}\n\label{subsec:comparison_modern}\nThe SMN also enters into a crucial dialogue with contemporary theories that model cognition and consciousness in terms of information, prediction, and neural dynamics.\n\n\subsubsection{Predictive Processing and The Free Energy Principle (Friston)}\n\label{ssubsec:fep}\n\paragraph{Affinities:} The SMN is highly compatible with the core idea of the Free Energy Principle (FEP), which posits that organisms act to minimize prediction error or "surprise." The epistemic actions of the SMN—the probing and testing of the world through HAPs—can be seen as the physical implementation of the FEP's central mechanism of "active inference." Both models view the agent as a system fundamentally driven to reduce its uncertainty about the world.\n\n\paragraph{Differences:} The two models operate at different levels of explanation. The FEP is a high-level, abstract computational principle that describes *what* a cognitive system must do to remain viable (minimize free energy). The SMN, by contrast, is a lower-level architectural model that proposes *how* a biological agent physically accomplishes this. It provides the concrete bodily machinery—the antagonistic action zones, the DFN/IN structure, the hierarchy of action patterns—that could implement the abstract mandate of the FEP.\n\n\subsubsection{Theories of Consciousness (IIT and GWT)}\n\label{ssubsec:consciousness_theories}\n\paragraph{Affinities:} The SMN shares with Integrated Information Theory (IIT) the commitment to a physically grounded explanation of consciousness. With Global Workspace Theory (GWT), it shares the core concept of a "broadcasting" mechanism that makes information globally available to the system.\n\n\paragraph{Differences:} The divergence is significant. IIT defines consciousness as a measure of a system's intrinsic causal power and integrated information (Phi); it is a property of a system's structure. The SMN, however, defines consciousness as a dynamic *process* that arises from an agent's *actions* in its environment. For the SMN, consciousness is something an agent *does*, not something a structure *has*. Similarly, while GWT posits a global workspace, the *content* of that workspace is typically understood as sensory or cognitive information. In the SMN, what is broadcast by the Integrating Network are the states of the body's various action zones. The content of consciousness is not disembodied information, but the globally available state of the acting body itself.
