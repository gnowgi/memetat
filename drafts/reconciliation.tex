\section{Reconcialation Argument}

Can behaviorist, cognitivist, and 4E frameworks be reconciled?
The context of recent advances in the GPT model can be interpreted as an irony: a behaviorist reinforcement learning model successfully implements cognitivist's representational model of generative grammar. In other words, the cognitive aspects of rules and representations, i.e., of the language world, seem to be learnable and usable in a transactional setting through a behaviorist model. If we assume that the GPT model does `truthfully' capture how the learning of symbolic patterns, generative transformations of patterns, and using them in a social, transactional setup of our cognition can be implemented in an artificial machine, what puzzles about human cognition remain if we adhere to 4E cognitive framework? The core challenge posed to the 4E framework by cognitivists is to account for how a generative (combinatorial) representational learning model be grounded in an enactive, embodied, embedded, and extended cognitive agent. This proposal is an attempt to address this challenge, thus exploring a possibility of a reconciliation.

We in this paper employ an abductive approach by beginning with a description of our model as a hypothesis and then proceeding to reason using it, and testing its explainability. In the process, we demonstrate how reconciliation of the existing cognitive frameworks is possible.

\subsection{Grounding Syntax in the context of LLM} Considering that pattern recognition, recombination and responding are the core of the GPT model and syntax is responsible for generative grammar, \textbf{\textit{the problem is to ground/ locate syntax in an artificial/ biological agent capable of generating variable atomic action patterns. }}Identification of atomic action patterns in an enactive model is a requirement for generative potential. We assume that a pattern is a necessary condition to encode or hold information. 
{\bf Cite some evidence for LLM in deep-learning, neural networks, generative grammar and reinforcement learning. Being a language model, it uses computational linguistic based on Chomskian ideas.}

\subsection{Haltability as a requirement for generating syntax} A linear invariant action pattern cannot generate syntax. \textbf{\textit{How does an agent generate a punctuation in the action pattern without halting?}} Therefore, a 4E model must base symbol grounding among atomic reproducible haltable action patterns (HAPs). Haltability is an affordance, only when life-sustaining action patterns (e.g., heartbeats) continue. The bottom-most metabolic layer provides a canvas or a phenomenological background composed of invariant (inter)action patterns.

\subsection{Generational enactive model} \textbf{\textit{How can an enactive model be generational?}} In a body with multiple haltable action zones, syntactical variations and nested compositions can be implemented. For example, multiple haltable action zones in fingers, wrist, arm, hip, and leg movements while dancing; multiple haltable action zones in the buccal cavity, lungs, and vocal apparatus together can generate numerous syntactical variations leading to a complex symbolic action space (language). These actions are affordable precisely because they can be disengaged from life-sustaining actions. This enhances their symbolic potentiality.

\subsection{Embodied account for Symbol Ungrounding} \textbf{\textit{How do we map HAPs (symbols) to experiences? Where are the rules —  the mapping between HAPs and experiences — emerging from?}} When the action and experience are tightly coupled, the requirement for a mapping does not arise. This tight coupling can be punctuated by haltable action patterns, which can be arbitrary, giving us the apparent feeling of being disembodied. Meaning as experience and phenomenological.

\subsection{Necessity of HAPs for the possibility of TAPs (and in turn culture)} \textbf{\textit{How do different agents engage each other?}} Since haltable action patterns are imitable, without compromising on life-sustaining actions, agents can afford to dance, sing and play together, celebrating the social/ cultural games (rule-based practices) and creating micro-worlds (memetats). This way a 4E framework can be shown to be generative in a representationally rich world. The potential of HAPs becoming transactional action patterns (TAPs) such as signaling, commanding, instructing, talking, etc. makes cultural practices implementable within a 4E framework.

\subsection{Connecting the internal and external representations} \textbf{\textit{What constitutes shared memories?}} Since action patterns can generate traces in the material world, with variable life in the form of sounds, inscriptions, etc., they become available for other agents as external representations/ memories. This enables a possibility for a rich inter-subjective space, creating grounded and mediated knowledge space within the 4E framework.

Having grounded mediated cognition within the world of HAPs and TAPs the so-called disembodied rules and representations can be shown to be actually embodied. Thus we see the possibility of representations within the enactive models of cognition and see no reason to reject representations as unreal. The data-centric machine learning models such as GPT can be interpreted as extreme extensions of externalized traces of human cultural actions. Similarly, aversion towards behaviorism by cognitivism is unfounded since GPT demonstrated generative capabilities through a reinforcement learning model. And the behaviorist aversion to internal memories and processes is equally unfounded because the learning model is not a black box, but a statistical neural network. Thus, we see that the existing philosophical differences among behaviorism, cognitivism, and 4E cognition can be reconciled.
