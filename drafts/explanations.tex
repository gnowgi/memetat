\section{The Functional Model of the Cognitive Agent}


In this section we present an argument that the human cognitive life, for that matter of any other cognitive agent, cannot be explained without an asymmetrical ontology and a gap-creating epistemology.  

\emph{The asymmetry condition is satisfied by polarized body plan and a bilaterally symmetrical localization of sensory and motor boards.} The asymmetry condition is satisfied by polarized body plan and a bilaterally symmetrical localization of sensory and motor boards. 
The structural body plan provides an embodied action schema for grasping the world. It is more natural for most of the animals to walk, run, swim or fly forward than backward because of the body plan. The architecture of the limbs of all the terrestrial organisms is asymmetrical, based on a limited degrees of freedom at each joint. Polarized orientation of the body to the ground with ventral and dorsal body differentiated, facilitates dealing with the field of gravitation. For example, an organism with sphere like body with feet all around can not differentiate the orientations (front-back, side-ways, up-down). All movement must deal with the given field, which itself is polarized. Legs in such a field cannot move, if the body plan is not antagonistic to the field. The anatomy of aquatic creatures also have to deal with the gravitational field, but it is not as pronounced as it is with terrestrial beings due to other aquatic forces. Anterior to posterior polarization is more pronounced in aquatic beings than dorsal and ventral. Since animal locomotion is an attempt to resist or oppose it's inertial state of motion, the architecture of the body plan must support it by allowing it to apply force. The asymmetry manifested in the arrangement of bones in the feet of any animal enables the organism to exert this force thereby constructing the orientation (front-back) with respect to the body and in relation to the environment. A symmetrical body plan can not provide a frame of reference.

The bilateral symmetry facilitates in alternating action patterns (e.g. movements alternating in legs while walking, running etc.) and a possibility of antagonistic action patterns. The antagonism in the body plan is so intrinsic that every joint functions through opposing pairs of muscles. This is possibly the only way how the inertia can be resisted, because one needs to apply force only for contraction. Since contraction is the only way of getting work done antagonism in the anatomical design is the only way to balance.\emph{Embodied experimental design as a cognitive foundation} Concentration of sensory boards in association with opposing motor architecture facilitates modulation. This concentration is a requirement for the experimental analysis of the world, since the environment impacts the body in a diffused way. The diffused environment requires a concentrated action to resolve the impact. Just as the experimental design in a scientific lab provides a means of regulating one variable while allowing the impact of another, the body architecture of animals is that of a living laboratory. The engineering design employed in scientific experiments is an extended body plan.

\emph{A formatted body and its action patterns determines the structure of the phenomena.} Specialized sensory boards are also organized to suit this body plan. As a result, the possible action schemes can resolve the structure of the world. This resolution is logically impossible without movement, and most importantly the patterns of movement facilitated by the body plan. The problems of cognition cannot be resolved without addressing this basal mechanism. Mere sense organs and the nervous system are not sufficient to build a complete framework. The nervous system cannot process the data stream coming from sensory organs without a mediating and modulatory system. Data can be stored only if we have a formatted body. The format of the body is not only the structure, but also the format of actions. The action patterns depend on the body, modeled as a sensory-motor network.  We shall elaborate on how this model could ground cognition in this section.

\emph{Framing problem} A stream of uninterrupted sensations may have a pattern, and it can be processed to identify the pattern. Here we are assuming that the existence of a pattern and its identification is the essence of information processing. It can be done by our modern computers, which are capable of recognizing patterns and establishing correlations as well. But this is an ungrounded process, because, \textit{on its own} the machine cannot detect: Where does the stream come from, from within the machine or outside? Which pattern to attend to? Which pattern is more significant than the other? Which ones to ignore? This is the framing problem. This problem is related to how a cognitive agent understands the context/situation.

Recognition of patterns and relating the patterns to a context is not sufficient for grounding semiotics. We now turn our attention to the formulation and resolution of naming-framing problem.

\emph{Naming problem} Apart from the understanding of context, we need to deal with naming and referencing. Information processing is not only impossible without names but also useless if it cannot name the patterns. One may program the machine to give distinct names to distinct patterns, and can also group them nicely based on some similarity detecting algorithms. It may also develop a complex ontology at the end. But how does the machine establish a reference, of which patterns pertains to which object in the world, or within the machine? And more importantly how could it convey to us the \textit{private} naming convention?  Can the machine at least tell them to itself?  If so how? These are three philosophical problems at one go: concept formation, symbol grounding problem and that of possibility or impossibility of a private language. Variation in the world and a capacity to detect patterns in a stream of experience is not sufficient to give names to the patterns. We need a naming mechanism in a cognitive agent, and we need a mechanism to communicate with each other, in a community of agents, through names. Let's call this \textit{the naming problem.} 

Combining the above two problems, let us call them together a \textit{naming-framing problem}, because we think that the frame problem and symbol grounding problem are intrinsically related.
We will now argue that the naming-framing problem can be solved through modulation. In other words, referencing and distinguishing the external world from the internal world will be shown to be possible through the same mechanism.

A world that is homogeneous is no world at all. It offers nothing to know. There are no affordances to offer. Nothing to act on or interact with. For the argument's sake, if we place a cognitive agent in such a homogeneous world, it will not even know if there is a stream at all in her experience, forget about detecting or identifying patterns in them. So, we assume that the world is heterogeneous.   

Let's place a community of cognitive agents, with the structure and dynamics specified in the $\{n(SMN)\}$ model presented above, in a heterogeneous world. We shall now present how such an agent can make sense of the world, and what else we need to develop language and culture. As mentioned in the introduction, the framework that is proposed here is based on an ontology of actions, and not interactions. The model of a cognitive agent presented in the above section describes the body as a network of modulators. A modulator in the scheme is the location of the action. We will now account for gaps in actions, and their role in the naming-framing problem.

What is needed in such a modulating unit in order to do its job, namely to regulate, to modulate? For argument's sake, suppose there is an action that happens relentlessly at an invariable frequency, what kind of differentiation can we expect from such an actuator? However, such relentlessly recurring actions are commonly found among animals. Even in higher animals like humans, beating patterns like heartbeat are uninterruptible. There could be a range of variations in the frequency and amplitude of these beats, which can be modulated by modifying physiological conditions. But such beats, when halted/ paused, may prove fatal to the organism. Halting such action patterns is physiologically unaffordable.

Let us recollect the architectural plan of the body. First, the modulators are distributed across the body in a polarised and bilaterally symmetrical manner. They are concentrated more to one side than the other. This body plan is to defy the forces, flows and fields in the world whenever and wherever possible. Yielding or total submission to the external environment cannot generate knowledge about the environment. Developing this ability to resist is \textit{action}. When we say the model is \textit{enactive}, it implies this condition.

\marginnote{The principle of layering} Since modulating certain beats such as heartbeats is not affordable, we may consider situating actions over and above the core physiological mechanisms. In order for the actions to be affordable, the interactions of the sustaining layer  must continue, and they should generate sufficient surplus. When we say cognition is enactive, it implies that the emancipation from sustaining mechanisms is expensive. Though autopoietic mechanisms may include actions, they are uninterruptible, hence no liberty to introduce gaps here. Hence autopoiesis as a mechanism to compensate the lost energy and matter takes care of the sustaining layer and provides the necessary surplus in the system making actions possible \cite{maturana1991autopoiesis}. This is also an action, but the uninterrupted pace at which this action takes place has no liberty for introducing \textit{gaps} in this layer. In other words, the system can't physiologically afford to halt. However, it is this state that could enable ephemeral actions on the periphery of an autopoietic system whenever and wherever possible. This is made possible by a differentiated body plan that enables a division of labour. Some layers are busy in not only replenishing the loss of energy and matter but also generating surplus energy and matter, such that other layers in the body can \textit{halt}. This design now has room for free action. In this perspective, it is an uninterrupted action of some layer that grants freedom to some other layers. It is this partial break from uninterrupted work, that gives rise to the freedom to enter into the cognitive domain. It is in this subtle sense, that our model differs from Maturana and Varela's account of the connections between biology and cognition. The subtlety we introduce is haltability. 

This differentiation of layers, as against the uniform distribution of work, in a system facilitating deviation from the normal course of actions, gave rise to the roots of cognitive state. We shall call this \textit{the principle of layering}, which is over and above the design principles of polarization and asymmetry we discussed earlier. The sense of being over and above can be characterised by naming metaphorically, this principle as epi-physiological or epi-biological.

\emph{Biological roots of disengagement} Beats, as uninterrupted action patterns, support the agents to move. If an entire body is engaged in movement, as in the case of some worms, the degrees of freedom are limited. In an earthworm, the ingestion beats and locomotion beats are coupled. The worm cannot move (locomotion) without eating (when in a burrowing habitat --- soil), and cannot eat without locomotion. In another body plan, say of nematodes, the ingestion movement in the buccal zone is independent of the locomotion. This sort of decoupling of functions emancipates one zone from the other. The worm can feed without moving the entire body, and move without feeding.

\emph{Homologous roots of anatomical disengagement, modulation and haltability} These examples indicate how one could speculatively weave a story of the evolution as a story of decoupling the body into multiple zones, where each zone can act partially independent from another, and exhibit a distinguishable action pattern. The development of tongue, lips, jaws, pharynx, larynx, gills, lungs, fins, tails, ears, eyes, limbs, toes, fingers, neck, shoulder, hip and so on are interpreted in this story as anatomical disengagement (or decoupling). One can draw a tree of anatomical disengagement representing the epi-physiological bifurcation over and above the phylogenetic tree of evolution. For example, in early vertebrates as in cephalopods, the feeding and breathing action patterns (filter feeding habit) are not decoupled. In Gnathostoms, we see the stoma (mouth) differentiated through the evolution of jaws enabling decoupled breathing habits from feeding habits. Episodes of decoupling could be reconstructed for the evolution of simpler buccal cavity differentiated into complex cavity, developing teeth, tongue, lips, pharynx, larynx etc. In parallel, an undulating body (as in lampreys) gets decoupled into localised and bilaterally symmetrical fins, decoupling locomotion as a entire-body function. Similarly, one could consider the decoupling of the undulating alimentary canal from the entire undulation of the body. These episodes of differentiation might have been naturally selected because of the economic value of decoupling, as localised movement is inexpensive than whole body movement. Each episode of such differentiation is an episode of decoupling leading to the evolution of independent action patterns (habits). Thus the gradual polarisation and bilateral symmetry of the body-plan through evolution leading to differentiation of action zones can be woven into a phylogenetic story of decoupled and localised zones of action patterns (habits).

Lets call the tightly-coupled actions as \textit{harder-actions} (e.g. the coupling between locomotion and feeding in the earthworms), and the decoupled actions as \textit{softer-actions} \cite{nagarjuna_muscularity_2005}. During the course of evolution more anatomical disengagements may have given rise to the availability of more such softer-actions.

We need to cut the story short to revert back to how the transformation of a tightly coupled body plan into a loosely-coupled body plan, from harder-actions to softer-actions, is relevant as a context for cognitive science. The cognitive hypothesis we propose, given this transformation of the body plan, is: the greater the disengagement of differentiated action zones, the greater is the agent's capacity to modulate the incoming stream of experiences. This disengagement itself is a function of anatomical polarization and layering.

This disengagement has an economic dimension, without which it is difficult to understand how it could have played a role in natural selection. The agent can do more work with less effort (spending less energy) because of disengagement. A body plan of an organism that has a coupled movement for both ingestion and locomotion is expensive, than when they are decoupled. Moving when not eating, or eating when not moving is a new found possibility.

Once we have multiple softer-action zones, it is possible to rest some while the others are active. This is the context for the genesis of \textit{haltability}. Haltable variations, one can speculate, could be sexually/culturally selected. The principle of economy also enabled the organism to perform one action while halting another. Isn't this how we describe modulation? The aspect of control we ascribe to modulation arises only when we hold one variable while modifying another. Can we use this insight to ground the regulatory actions required for cognitive processing in haltable action patterns?  We demonstrate how this can be the butterfly effect in cognition. We now move to discuss how haltable-action-patterns (HAPS) can become units of analysis for cognitive \textit{behavior}.

\emph{Homology of FAPs and HAPs} This story has affinities with the view that dexterity and movement of the body contributes substantially to cognition \cite{bernstein2014dexterity}. And in cognitive neuro-science the emphasis has always been on how CNS or brain modulates motor actions. In the context of the current model, it is important to mention the concept of fixed action patterns (FAPs), which are no different from the multiple softer-action zones mentioned above. As Llinas argues the synergistic coordinated action of a cluster of muscles take part in \textit{fixed action patterns} (FAPs), which play an important role in his narrative of how to build mind from body \cite{llinas2002vortex}. In fact the choice of the expression haltable action patterns (HAPs) is inspired from Llinas, which suggests the contrasting feature of our model with that of Llinas. \footnote{ additional citations of FAPs coinage}

What is the role of halting in cognition?
Recalling the structure and dynamics of the SMNs explained in the above section (see figure~\ref{zone}), we presented a view where the incoming stream of sensations go almost unnoticed without modulation. In an architecture where the sense organs are mounted on the available multiple modulators, the stream of sensations change in accordance with the action performed. This correlation binds the sensation with actions giving rise to perception. This is in line with the positions of Merleau Ponty \cite{ponty1969phenomenology} and Alva Noe \cite{noe_action_2004}, who argued for an embodied and enactive view of perception and cognition. 

In the synergistic motor assembly of FAPs, which is essentially a sensori-motor assembly, the foundations for modulating sensation may have been laid providing a mechanism for DFN.

The fundamental question that we can ask is: What makes modulation possible? In the current views in cognitive sciences, modulation is typically managed by the brain; therefore researchers seek to locate the zones/ regions of control within the brain. By contrast, in our model, by suggesting that there is a strong link between haltability and modulation, we locate the mechanism in the zones of disengagement. In this interpretation, the anatomical disengagement, the ability to halt, and the ability to control are homologous. The terms used to describe higher order cognition --- `modulation', `regulation' and `control' --- are gross descriptions of phenomena, whereas haltability rooted in anatomical disengagement is a more nuanced and observable description of the phenomenon. Further, it avoids the need to identify a part or the system as an organ of control, instead ascribes this to the entire SMN, as a systemic ability. Therefore resolving modulation in terms of haltability adds greater rigor by providing an observable criterion for modulation.

\emph{Rooting Epistemology in the gaps} As detailed above, the disengaged motor assemblies are capable of acting independent of others. The effective capacity emerging out of this is the potential to remain transiently inactive. The pattern of halting provides the structure for the rest of the story.

\emph{Rooting syntax in a sequence of HAPs} As we understand from the theories of information, the logical conditions required for variations can be provided by gaps. For example, in a minimalist possible code, such as Morse code, the patterns of dots are created with interruptions (gaps). The various possible patterns of truth and falsity or 0 and 1, used as a foundation of encoding and processing in computer science, demonstrates the potential of gaps in generating variations. The case of binary code tells us what is required, and the former case of Morse code tells us how it can be implemented, though it may not be the only way to implement. Introduction of \textit{gaps} makes syntax possible, which in turn enables the generation of as many patterns as required. Since the practical needs are a small subset of the logically infinite number of possible patterns, this is sufficient for encoding knowledge. In sequential patterns, syntax and pattern are identical i.e., they are two sides of the same coin. Syntax is a feature of sequential patterns, which can be grounded in the patterns of halting.

\emph{Arbitrary mapping} However, understanding how to generate numerous action-patterns is apparently a simpler issue than decoding the action-patterns and their reference (what they stand for). Because, action-patterns can be generated arbitrarily as well. It is also possible to map any arbitrary action-pattern to an arbitrary reference. However, in a given context, the mapping is required to be \textit{conserved} for semantic coherence over time. This fixing the map between an action-pattern and what it could stand for does not make sense without reproducibility and continued conservation of the mapping of the action-pattern to the reference.

\emph{Memories are reproducible action-patterns} Reproduction of an action-pattern is possible without binding it to a reference. For example, a melody generated either by an instrument or orally, does not have to stand for any reference, or they could stand for multiple references, making them ambiguous. It is a feature of artistic creations to escape from a stable reference. A general feature of melodies is that it is hard to forget. Given the fact that melodies are generated by action-patterns, we could consider what we remember are like melodies. We agree with the view that memory is \textit{for} action\cite{glenberg1997memory}, but we also argue, as a framework based on action ontology, that reproducible action-patterns and/or their traces \textit{are} memories.

Enacting an action-pattern and decoding what it stands for can be distinguished. 
The capacity to decode an action pattern requires holding on or recollecting the mapping with a reference. 
The entry of reference in our discussion is necessitated by a separation between an object and its \textit{name}.
Let's use the example of talking about a cup by hand-grasping-a-cup action-pattern (gesture). 
\emph{Entry into Semiotic world} The possibility of grasping action-pattern \textit{without an object} (in this example, a cup) is a significant bifurcation point and an entry into semiotic world. 
The miming action of grasping an object can become a \textit{name} for objects.
The mime for holding a pen, brush, liquid, cup, basket etc. could all be different, based on the affordances these objects offer to the agent. Grasping any object is saturated, while a mime of grasping without the object is unsaturated. 
Mimes can stand for not only objects but also verbs. For example, we can sign someone ``to get in'' or ``to get out'', as well as ``please come in'' or ``you may go now''\footnote{ add citations --- Roth, Lakoff, Goldin-Meadow, Brazilian math-educator, Piaget and Vygotsky ... }

If the action patterns are always \textit{saturated} with the object or event, they can never become names for them. Naming is impossible without breaking this contingent binding. We therefore consider unsaturated action-patterns as a necessary condition for a semiotic life, where naming action-patterns are separated from the object they stand for.

An action-pattern could be bound to a reference in a \textit{hard} or \textit{soft} manner. Harder binding specifically applies largely to gestures (inter-subjectively presented action-patterns). 
The action-pattern used as a mime, when closely related to the affordances offered by the object or event, are harder. 
Some mimes transcend the affordances of the object or an event, since they hold no morphological or functional correspondence to them. \emph{Possible relation to modal and amodal concepts} For example, in a typical Indian classroom, when a student stands-up in the middle of the class and shows his/her little finger, the teacher as well as the rest of the class understand that the student is seeking permission for a bio-break. This may not work in another culture. Because the binding between the little-finger-mime and seeking permission for a bio-break is created \textit{arbitrarily} without any match with the affordances. Whereas using a thumbs up mime to seek permission or a hydration-break is less arbitratry and matches with the affordance of drinking water.

While the possibility of hard-mimes could be a major bifurcation point for communicating agents, the use of soft-mimes for communication is a revolutionary bifurcation point, because this breaks open a world of possibilities. 
We think that this could be the episode of punctuated equilibrium\cite{gould1977punctuated} in the evolution of homonids. The communities that could use arbitrary mimes (names) had a political and economic advantage over other communities, because arbitrary names gives rise to proprietary/ private (closed group) languages.\cite{corballis2014recursive}

Once we move from gestures as mimes to the traces of action-patterns, such as the sounds or inscriptions standing for an object or event, the separation between them is so deep that it is difficult to decode them by simple correlations without training or lived experiences. 
Affordances of objects and events, in the world of traces of action-patterns, hardly help to decode. Some traces of action-patterns, such as inscriptions that bear a similar morphology to an object or an action, e.g., smilies, are harder (closer to the reference). But enter the world of alphabets used to create names we enter into the ``software'' world of representations.

Enter the ``software'' world, we enter the world of rule following games. Thus the mapping between the patterns and their references provides the rules. 
This jump from action-patterns to rules is not a step taken but a major leap. 
Rule following action-patterns provide a spring-board to another world.
So, we need to halt by asking the question: what makes rule following games possible?
This is a highly involved problem, since we have suddenly entered into a context of a community of agents, and not merely an individual agent. 
This cannot be resolved unless we demonstrate how in the proposed model, we can account for shared memory and shared experiences in a community of agents.
We shall indicate an approach of resolving this problem, if not actually entirely solving, here.

\subsection{Co-construction of memetat and self-identity}

\emph{Recurrent self-modulated action-patterns become habits. Every habit has an inherent syntax. These HAPs are \textit{action schemes}\cite{piaget1970genetic}.}

One of the first outcomes that an SMN needs to make in the model presented above is to differentiate the experiences into what is in the body and what is outside the body. 
The modulations of perceptions of one zone affecting the other zone happen at the same time, and therefore these zones get connected through FTWT principle. 
There exists a reinforcing loop, because of the fact that the two zones obtain stimulation at the same time. For example, thumb/toe sucking action observed in the fetus prior to birth, has simultaneous stimulation from two zones viz., the thumb/toe zone and the buccal zone. [see if you have any reference for these regions wiring together]. 
On the other hand, the fetus kicking action-pattern on the walls of the womb of a mother, are those action patterns where one SMN is stimulating another SMN. These are two independent networks. 
Though the stimulation happens at the same time, the reinforcing loops within the network do not exist. 

After birth when a child kicks the crib, an SMN is acting on a non-SMN. This action does not have a reinforcing loop in the SMN. So the crib is outside the body. So are the rattles and teddy bears in the vicinity of the child. 
As far as the active SMN is concerned the boundary between the body and the world is clearly drawn. No scope of solipsism here. The phenomenology of objects within the internal network and the external world are different.

An embrace between two bodies (SMNs), say between a mother and a baby, is another case. Here two networks are mutually self-stimulating by both the SMNs enacting similar HAPs. Multiple zones are stimulated on each body at the same time, resulting in an inter-subjective feedback that reinforces the synchronous synergistic bonding. Here, even though the SMNs do not have direct connections at the network level, they are connected by synergy developed through synchronous self-stimulation and their effects. Such HAPs are repeated for the mutual positive reinforcement leading to conservation of an inter-subjective space. This is absent when a child kicks the crib. When the fetus is kicking the womb, it may appear as if two networks are acting on each other, but the only actor there is the fetus, for the mother's womb has no self-modulating capacity. Therefore, mother's body is practically an external environment to the fetus. In the above 3 cases, we have 3 different situations of the modulation --- in the crib kicking case, the baby modulates and experiences, but the crib does not; in the womb-kicking case, the fetus modulates and experiences, but the mother is only at the receiving end; whereas in the case of embrace, SMNs of both the baby and the mother mutually modulate and experience, and hence we can ascribe an inter-subjective space here.  In such an inter-subjective space the agents are not merely acting, but \textit{transacting}. This transition from action space to transaction space is the entry into \textit{memetat}.

An infant kicking a rattle is yet another case. The generation of sound at the time of kicking the rattle is an example of self-production of sound. This also generates interest (attentional anchor), since this is in effect a zone (kicking zone) stimulating other zones (hearing zone, touching zone, visual zone etc). This exploration is motivating due to one's ability to control the production of changes in multiple sensory modalities. Similar example is when the child kicks a ball: when they kick the ball, though the ball is not part of the network, the other zones --- audio-visual-haptic zones --- provide the temporally correlated events. This action when repeated binds the experiences resulting from action. This is also a loop nevertheless. But the internal feedback loop is absent. Correlation is not a sufficient indication of causation, similarly firing together is not a sufficient indication of causal connection. In the thumb-toe sucking case on the other hand, the loop gets closed with a greater degree of certainty. Even in the case of kicking a rattle or playing drums, though there is no direct closure of the loop, there exists experiential closure leading to incorporation of the object into the agent's action-experience space. Often we enter into this recreational zone when we dissolve into a \textit{flow} \cite{Mihaly}.

When a child holds a bat and explores the world outside though the bat is not part of the network, it provides haptic experiences extending the SMN's action space becoming part of its peri-personal space.\footnote{ citation Maravita yee ...}

The possibility of holding zones such as hands, beak, mouth, etc. is yet another major bifurcation point. The external object that the agent can hold extends the explorations and experiences, beyond what was affordable through the body itself. For example, objects that are hotter or sharper or heavier etc. can also become part of the explorable and experientiable spaces. Thus, a combination of tools with holding action zones, extends the action space leading to extended the habitat as well as perceptual space.

The action patterns as mentioned above, become habits, and the habits become names (initially as gestures) the corresponding external objects become part of the habitat (explorable and experientiable spaces) which turn into memetat (transactable spaces). 

One may take several such examples to understand how the SMN can demarcate internal world from external world. This abstraction of internal and external is operational, and participates at the root of constructing \textit{self} and concept formation at the same time. Interestingly, the formation of self is not independent of the formation of what is not self. The context of operations/actions that are germane to the separation of the internal and external and the context of operations that are germane to the separation of the self and the-other (non-self) are not different. They are co-constructed, as a thematic pair \footnote{ cite Gerald Holten, G Nagarjuna}. This is how a sensation modulating network or sensory motor network as an SMN acquires an entry into a cognitive ground.

\subsection{Nested HAPs}

Having alluded to the possibility of naming through haltable-action-patterns, and the memet-memetat differentiation, we shall now address how nested-action-patterns can be constructed through self-modulation in an SMN with multiple zones of HAPs.

\emph{Representing the complexity of nested HAPs} Clapping can be done, while the body is standing, sitting, walking, talking or running etc. The clapping action zone is disengaged from the other states the body could be in, i.e., it could be performed independent of the rest of the states. The nesting of action patterns can be represented as [sitting(clapping)],  [talking(clapping)],  [running(clapping)],  [walking(clapping)],  [singing(clapping)] etc. The nesting becomes complex when we keep walking, while singing and clapping [walking, (clapping (singing))]. The variations in nesting can be seen when the frequency of walking and clapping match, or some modified patterns through skip-clapping, while singing action pattern is going on.

\emph{Iterative, recursive and alternative sequences of HAPs} In order to generate a sequence of HAPs performed at different zones, it is necessary for the zones to have the capacity to act independently. We shall use the term `iterating sequence' referring to repeated actions in the same zone. We shall use `recursive sequence' when actions involve multiple zones, and when actions can happen together. In the case of alternating action patterns involving multiple zones there is no recursion. We shall use the term `alternating sequence' for this case. Thus there can be 3 types of sequences that are distinguished:  iterative sequence, recursive sequence and alternating sequence. Recursive and alternating sequences can also be iterated. These form the syntactical aspects of HAPs. The core logical requirements of richer syntactical representation are satisfied by HAPs that can be iterative, recursive and alternative.

Though, iteration implies a sequence of repeated actions, distinguishing an iterating sequence and a recursive sequence is significant for understanding semiotics. Alternating the iterative and recursive action patterns is the basis for the creation of rich symbolic forms. Logically, HAPs are required for creating such variations. It may appear logically sufficient to create variations in a sequence involving a single zone to generate syntax, e.g. Morse code.  However, it is cognitively insufficient, because the gaps can not be recognised without a reference clock (another iterating action sequence).

The zones that are in a state of recursive HAPs, which are synchronous, can be considered as the roots of the nested structure. For example, a person clapping a simple rhythmic beat while tapping the feet alternately, left-right, is a very simple nested structure. Here the alternating-tapping is nested in the clap-beat. One can increase the complexity by arbitrarily moving another zone such as turning the neck left to right while also doing the alternating-tapping. The alternating-tapping and the alternate-neck-turning are nested inside the clap-beat. It is up to the creativity of the choreographer, to play with the endless possibilities of modulating haltable zones within the bounds of the body architecture.

Let us take another example of clapping while walking forward, backward, sideways while also turning the neck, hips, shoulders etc. All these actions may happen in a sequence or may happen while holding one zone in a fixed or a stable recursive action-state as a beat. The beats need not be performed from within one body. When more than one agent is participating, then the sequences and the beats can be performed either sequentially or synchronously. The permutations and combinations of these possibilities are endless, even in this simple example. When the speech zone enters, then we add to the complexity, further.

However, the number of such zones that are available for creating alternating nested sequence, in a context, determines how complex the symbolic life of that agent can be. Can't this be one of the comparative parameter of cognition among agents, both within and across species? These abilities can not be taken for granted in a body architecture, for they may not be realised without situating in suitable social and natural contexts.
 
\emph{Creating and recreating internal world} Considering the speech as a peculiar ability of human-body, it is important to understand that it is not a single zone of action, but involves multiple zones. Apart from employing multiple zones in speech, all the actions are self-stimulating/ self-modulating as well. For example, the left and right and the top and bottom parts of the complex vocal apparatus touch each other: lips, tongue, pharynx, larynx stimulate each other in a variable sequence while modulating the inhalation and exhalation halting patterns. This is a paradigm example of generating a complex sequence of self-stimulating perceptions, because the body is not acting on an external object, but on itself. This introduces a complex phenomenology that can be generated by the body as and when intended and possible. In this example, we are not moving in the external space, but with-in. Another special feature of this speech complex is to generate traces of the possible HAPs as audible sounds and visible movements. However, it is possible to halt the generation of sound, while keeping the actions sustained. So, within this speech complex, the various ways of sequencing of HAPs, nesting some sequences or repeating a sequence of nested-sequences are possible.

A similar account can be given for sign-language where the complex facial-zone and the pair of hands and fingers move. Thus complex and infinite nested sequences are possible within the body, without bringing in the scripting complex. The latter (scripting-complex) is the set of traces of the former (speech-complex and gesture-complex), and therefore not possible without them.
 
\begin{figure}[ht] 
%\renewcommand{\thefigure}{2}
\includegraphics[width=\textwidth]{nesting-rHAPS.pdf}
\caption{\color{Gray} \textbf{A schematic representing how nested action sequences are possible in a body with independently haltable multiple zones. The schema shows the two zones A and B representing self-stimulation through the left and the right hands involved in a clapping action pattern. While the clapping is sustained, the zone-C is singing, say ``Happy Birthday'' song. The nested structure can be represented by a simple formula \{(AB)C\}. }}
\label{nesting} % \label works only AFTER \caption within figure environment
\end{figure}

\emph{Articulating-architecture of an SMN} Prior to the development of generating sequence of visible actions and traces the SMN is also capable of using HAPs for engineering the space and objects in the environment. Let's recall that the mechanical structure of the body is a bilaterally symmetrical nested structure, which can be shown as a tree based on where the joints are located. This mechanical structure also exhibits polarity by asymmetrical joints. The available degrees of freedom provide the body with a clear demarcation of actions possible as ventral, dorsal, anterior and posterior. The zones that we discussed above are mounted on this mechanical skeleton. This structure already defines the possible embodied abstractions such as front and back, forward and backwards, up and down, top and bottom, etc. as the thematic pairs. These thematic pairs provide a conceptual scaffolding emerging out of the asymmetrical and polarised structure of the SMN itself. Beyond this foundation, the mechanical structure offers our ability to extend the possible explorations and experiences by facilitating the tool-use. The informed readers can see a number of simple and complex machines in the architecture of the body itself, which facilitates the extended explorations.

Let's take the example of throwing a stone while hunting. Holding a stone in one's hand is a complex task which is unavailable to many animals. This ability is a function of the multiple haltable-action-zones. The complexity and dexterity of human hand is well-known. The hand, first of all, should be disengaged from walking/standing/supporting states. One hand has to be freed, while at the same time, reach out to the stone, grasp it, extend the entire arm, while holding the grasping state independent as is, move the arm opposite to the direction of throw, and release the grasp while moving the arm towards the direction of throw at the right time. We ignored several other zones of the body that participate actively in this articulated action, calibrating their action states with each other to achieve the intended goal by dynamic compensation. Holding a posture requires to halt several zones, such as differential bending of the legs without loosing balance, bending at the hips, shoulders, neck muscles, holding the mounted senses of the head steady so that the gaze of the target is not lost. Apart from the above mechanical compensations towards accomplishing the action, there are numerous other compensations from the lower layers in the form of altered breathing and heart-beats during the run and at the point of throwing, to generate sufficient propulsion. We often see a reactive gush of breath or yelling just about the time of releasing the stone. All the above zones have to be kept stable, while some others dynamically are changing their states, and achieve the action of throwing, by a synergistic coordination of one zone after the other. No wonder, even our closest relatives could not achieve to do an apparently normal task that all humans can do with little difficulty. Though, this is not a task that can be done without learning and practice, as is the case with all the HAPs,  children take years to learn this. Human achievements of throwing a stone can happen while the body is walking or running as well. This capacity made our ancestors better hunters, because we could do action at a distance, without a physical proximity to the prey. These are discussed in detail in several stories on the evolution of homonids. This achievement cannot be explained without granting HAPs supported by an articulating-architecture of the complex SMN.   

These coordinated HAPs are nested both anatomically and functionally. Though at each zone the actions appear sequential, they are played out by holding some zones at a constant state, while some other zones are maintaining a pattern of action, yet another zone exhibits another pattern. As explained above, this is possible because there are many zones where HAPs are possible.

While demonstrating the syntactical and nested structure of HAPs, we ignored cognitive effects within the SMN. The HAPs are grounded in the SMN, because each action generates a specific sensation (and associated emotion) and an integrated feeling as a result of the networked body. We analyse below the example of hunting as a function of the action on one hand and the corresponding experiences on the other within the SMN.

\emph{Description of a hunting sapien}
There are at least three different action zones or $\Delta$ networks --- $\Delta$(running), $\Delta$(shouting), $\Delta$(throwing) networks --- functioning in a coordinated manner. We represent these coordinated action sequences as the nested HAPs and the corresponding differentiation of differences. Following the delta notation as described in the equations \ref{delta_notation} and \ref{delta_eg}, we can describe the state of SMN in the form of $\Delta$ networks as follows:
\begin{equation}\label{running-eq}
SMN(hunting)= [\Delta(running) + \{\Delta(shouting) + (\Delta(throwing))\}]
\end{equation}

$\Delta$(running) is a FAP involving the sensory motor boards essential for running. These include vision, proprioception, pressure sensing from the feet, auditory, location sensors in the muscles and bones of the legs etc.
\begin{equation}\label{running-eq}
\Delta(running) = \Delta^{vision, proprioception, pressure-sensing, auditory, location-sensors...}_{feet, legs, arms,...}    
\end{equation}

Similarly, $\Delta$(shouting) is also a HAP involving a complex set of sensory and motor boards for talking at the zone of vocalisation apparatus. These include the voice box, larynx, the tongue, lips, diaphragm and rib-cage, lungs, etc. The action patterns in this zone are haltable and they can be further elaborated as a nested set of serial $\Delta$ networks.
\begin{equation}\label{shouting-eq}
\Delta(shouting) = \Delta^{proprioception, auditory, tactile ...}_{tongue, lips, larynx, diaphragm, rib cage, lungs,...}    
\end{equation}
This above representation is still very simplistic, since these are not yet resolved into the HAPs of the larynx, the tongue, lips, diaphragm, lungs etc, and the corresponding $\Delta$ networks, thus making it a HAP-complex nested in another HAP complex. Another set of detailed nested action sequences is involved in the process. If the shouting action was more impulsive, then one may regard it as a FAP, as against more specific articulated shouting action (e.g. shouting to a fellow hunter), where more complex serial action patterns may be resolved.

Further, $\Delta$(throwing) is HAP-complex involving the sensory and motor boards for throwing. The hunter is holding something while other actions of running and shouting happen, and at a specific instance release the stone in a specific direction. These include the vision, proprioception, pressure-sensing, location sensors in the muscles and bones of the arms, elbows, fingers etc.
\begin{equation}\label{throwing-eq}
\Delta(throwing) = \Delta^{vision, proprioception, pressure-sensing, auditory, location-sensors...}_{eyes, feet, legs, arms, fingers, wrist...},    
\end{equation}
where vision in turn can be further resolved into another HAP-complex involving highly sophisticated sensory motor zones.
Thus our analysis differs from the behavioral approach which ignores the internal cognitive network behind the visible actions. The shouting, running and throwing have a specific nested relation. The corresponding perceptual effects are determined by the possibilities allowed by the articulatable body-plan of the SMN.

Given this perspective, a question that arises is, how can we explain the radical differences between human and other closer ancestors that share very similar anatomy. Why aren't they as cognitively capable as humans are? This question can be addressed by the predominance of a multitude of the following: 
\begin{enumerate}
    \item the possible HAPs
    \item self-stimulating HAPs
    \item the nested HAP sequences
    \item the unsaturated HAPs
    \item the transactional HAPs
\end{enumerate}

The multiplicity of the zones that are capable of HAPs, of all varieties, could create a butterfly effect. As a result despite a common base the emergent phenomena appear as if they are peculiar. The patterns that arise can be understood in terms of cymatic patterns within the SMN due to recurring nature of the actions. The body plan, we propose, plays an important role in triggering and regulating the emergence of patterns, of which several of them are recurrent. This could be the basis of the apparent widened gap, a sort of punctuated equilibrium, that we see between human beings and other animals. This is similar kind of mechanism that we see in pleiotropic phenomena. The branch of features that one factor could trigger could all collapse together when the pleiotropic factor is absent. We consider HAPs as that pleiotropic factor of being human. 

In terms of multitude of independent zones, certainly, the differences are not many. But, multitude of self-stimulating and nested zones gives the apparent qualitative difference.

The number of independent zones increases significantly from fish to qudraped, then from quadraped to biped. The latter phenomenon is considered to have played a major role in the evolution of homonids, often referred in the literature as emancipation of forelimbs. While every action is self-modulated, the only difference is about how many zones from within the body can act on or modulate each other. This reflects in many apparently simple actions that humans perform like opening a bottle, etc. 
Only those animals which have this kind of disengagement are capable of `sophisticated' problem-solving behaviour predominantly seen among birds and mammals. The differentiation and disengagement in various degrees of problem solving capabilities can be seen among mammals such as rodents, ungulates, proboscids, canines, monkeys and apes etc., and among birds such as corvid group, weaver birds, parrots etc. Since our body is based on body plan and not necessarily on the group of animals, we also find some exceptions of problem-solving behaviour even among other groups in invertebrates, arthropods, and cephalopods. Wherever we see elaborate problem-solving behaviour, we expect to see elaborate joints with more possibilities of disengaged movements in the body plan. Even among these isolated cases, we could notice multiple FAPs and HAPs. Based on the model, we could explain the problem solving behaviour among the animal kingdom in terms of the variety of HAPs.
These examples clearly indicate that there is only a difference of degree and not of kind. The predominant role of body architecture in making certain actions possible and repeatable with less effort is not a minor difference, and hence, the organised (engineered) environment the body plan offers, contributes substantially.

Therefore the problem-solving behaviour is a function of the HAPs of various kinds. Greater details of this kind of comparative analysis of kinesiology can be a research project that can bring greater support to the thesis argued in this paper.

Nested mounting of action zones can be studied by understanding the anatomy and kinesiology. For example, eye movements are mounted on the orbit of the skull, while the skull is mounted on the vertebral column (neck). Similarly fingers are mounted on the palm, which in turn is mounted on wrist, which in turn is mounted on elbow, and then the shoulder. The complexity of modulation depends on the degrees of freedom as shown in the figure\emph{ image of kinesiology from wiki}.

 What does an increased degree of disengagement contribute to cognition? Based on the model, the differentiation and filtering networks (DFNs) support to distinguish one modality from another, through a controlled series of action pattern: modulating one modality, while holding the rest of them. Consequently, the nature of experience transforms from an integrated visuo-spatial snapshot into serially articulated experience: from synthesis to analysis. On this count, human beings are markedly different from most other creatures, in understanding the world in serially articulated modality.  \emph{Predominance of Articulating architecture in a human SMN}

The phenemenologist's description of as 'being-there' in the lived world is more akin to an integrated experience devoid of articulation. We as human beings re-create a this integrated image experience into a sequential (a linear form of) articulation by employing the syntactical aspects described in the model. Articulating a picture in a linear form is a typical human cultural feature. The movement from 'being-there' to 'articulating-there' is an effortful transition. We employ, through our ontogeny, various techniques to inculcate these abilities from childhood to adulthood. Songs, poems and rhymes, comics, stories etc are, instruments of enculturation helping us to move from being-there to articulating-there. Artists creatively explore this space by helping us to transition from one form to the another.

Therefore, any peculiar human condition virtually boils down to the predominance of serially articulating architecture of human SMN. 

\subsection{HAPs and Cognition}
How does multi-zonal disengagement and self-modulation contribute to cognition? In other words, how does the serially articulating architecture of human SMN contributes to complex human cognition?

\emph{action schemas and conceptual schemes} The examples that we gave above are in creating representations and traces as well as in tool-use. As described in the section above [], the sophistication of each zone directly contributes to differentiation of features of the world, based on the differentiating and filtering mechanism. Thus, greater the detailed differences in modulation, the greater are the variations in the action schema. As already well established by Piaget, action schemas are Kantian conceptual schemes differentiating experience. The work by Lakoff and Johnson on embodied nature of knowledge enriches this perspective. By introducing the cognitive agent as a articulating architecture in terms of an SMN, we also explored the way the agent can distinguish what is within and what is not. This thus provides an embodied and enactive way of resolving not only subjective and objective demarcation, but also a clear basis for spatio-temporal cognition.

\emph{traces of action schemas and representations} Apart from embodied action schemata, the actions leave traces, both transient and persistent, giving rise to external representations. For example, the HAPs of speech-action generates transient sound as a trace. Whereas walking on a beach leaves the traces of foot prints, which persists longer than sound. Other actions like writing the speech on a paper or a black board leaves traces which persist even longer. All these traces are potential external representations, available in an inter-subjective space for perception and interpretation. All the symbolic activities in our cultural lives fall in the domain of generating, perceiving and interpreting traces, which boils down to encoding and decoding actions.

\cite{dove-ungrounding} identified three challenges in the context of abstraction as an indicator of evaluating a framework of cognitive science: generalization, flexibility and disembodiment.  

Based on this division, \cite{harnad1990symbol} formulated the symbol grounding problem. It is a feature of representations that they are independent or disembodied from the body, which is one of the core arguments for the traditional mind and body dichotomy.  The 4E camp is in search of grounding cognitive phenomena in an embodied, extended and situated actions of the agent. Broadly, the foundation is sought in either representations or actions.   Cognitive scientists of both the camps are largely committed to find a biological and evolutionary account of the phenomena. Therefore, cognitivist's core challenge is to find how disembodied representations are implemented in the brain or body, some kind of symbol \textit{ungrounding} problem \}.

4E emphasizes the role of social, ecological, affective and sensory-motor systems in concept formation. 
Cognitivists emphasize the role of non-embodied explanations implicating amodal systems (computational or information procesing models, e.g.)



\emph{symbol grounding problem} How arbitrary the action schemata are, determine how \textit{ungrounded} the representations could be. The HAPs generating the sound of `mama' referring to mother are less arbitrary than those generating the sound of `mother' itself. Similarly the HAPs generating the sound of `meow' or `bow-wow' referring to a cat or a dog is less arbitrary than the sound of the words `cat' and `dog' respectively. The arbitrary HAPs, as explained above, are possible because of the freedom acquired due to disengagement from the context. The references of the traces like `mother', `cat' and `dog' are conventional. Even among conventions, ostensive definitions are more grounded than defining a `cat' as a four-legged mammal. Some conventions or rules are more grounded in the situation than others. But let's keep in mind that the so-called ungrounded and arbitrary HAPs are still actions, because they indeed are the actions performed by the body. In this sense, however arbitrary the symbols / traces are, they all emerge from HAPs which are always grounded. At the time of reading/interpreting the traces, they get regrounded. This is the way the current model could address the symbol grounding problem.

Therefore, it is not the HAPs that are ungrounded nor the traces of them. But the arbitrary mapping, between the HAPs/traces and the reference, can be ungrounded. Thus the greater the use of mapping rules, the greater do they move away from the situation. The human natural and artificial languages exist in a space of abundance of such rules, which we refer to as the memetat making the corresponding HAPs the memets. The memetat is constructed by mapping-habits of humans. Having characterized memetat as a transactional space, we have now identified that this inter-subjective space is made possible due to the mapping rules. 

The turning point or the transition of habit-habitat into memet-memetat happens once rule following actions, mapping HAPs, get into the life. This transition also marks the unambiguous tight coupling of action and phenomenological experience of habit-habitat, while this coupling is ambiguous and loose in memet-memetat. This zone of transition is a place of philosophical reconciliation between the cognitivism and 4E models of cognition. The emergence of memet-memetat can be marked by an existence of dictionaries or mapping rule-books. This is also the zone where reinforcement learning models begin to operate, since the dictionaries may not be available as explicitly encoded scriptures. The abundance of these rule-books is a marker, often considered as a peculiarity of human condition.

Based on this perspective, it is clear that the predominant use of memets generating arbitrary traces (representations, symbols) is apparently peculiar to being human. In this context, let's take the example of the territory marking by cats and dogs. Normally, when the bladder is full the dogs and cats release plenty of urine, which is not a HAP. When the pee is used as a mark of territory the halted urination is employed, because it creates a pattern. The mapping between peeing and the territory is arbitrary. Because the territory could also be marked by foot-prints or digging the soil etc. The mapping is more grounded to the situation in the animal world than in the human world. Therefore, we can not deny memetat to animals. The early dictionaries are in the making.

In the current cognitive science, the research groups are divided as situated and cognitivist camps on the basis of grounded and ungrounded representations. In our perspective, both embodied and the so called disembodied representations are grounded in any of the 5 kinds of HAPs.

In addition to explaining the symbolic world (the world of representations), the engineering capacities can also be explained using the 5 kinds of HAPs (1. the possible HAPs, 2. self-stimulating HAPs, 3.the nested HAP sequences, 4. the unsaturated HAPs, 5. the transactional HAPs). Most engineering activities can be accounted by the use of 1,2,3 and 4 in various degrees. Simple tool use can be accounted for by 1,2,3 --- HAPs and nested HAPs, whereas modern day engineering involves a significant component of symbol use and manipulations, which also involve unsaturated HAPs (4). With the networked machines, even transactional HAPs maybe enabled or offloaded to the engineered actions space in the form of socially interacting computers or machines or robots. 

Earlier, we saw the morphological conditions of multiple zones acting independent of one another and modulating one another, as a basis for tool-use and mechanical sophistication. Much of the engineering innovations are externalised tokens of bio-mimicry. The case of various machines built by the Hero of Alexandria and explained by Archimedes, using geometry in terms of mechanical advantage, is a good case to understand the engineering context. Here, mimicking the animal architecture they made multiple machines. The initial simple machines may have been more grounded to animal architecture, however as the engineering explorations discovered soon, the pulleys and wheels which have no correlation to animal architecture, have been introduced. We argue that simple machines like wheels and pulleys are like arbitrary ungrounded innovations in engineering. They are ungrounded only because the living body does not have pulleys and wheels. We argue that simple machines based on levers transformed engineering artefacts, when wheels and pulleys were introduced. Just as by introducing rule-based languages we transcended the limits of communicating through gesturing, with the introduction of wheels and pulleys, we transcended the limits of possible movements in machines using pivoted joints. This enabled extending the abilities of the body in exploring the world, but also externalised body parts, which can be employed independent of the body.

The disjointed moving artefacts like a wheel --- contrasting with a stick, hammer or knife --- controlled by an axle in a pulley or wheel in a vehicle, catapulted engineering. Metaphorically, just as unsaturated HAPs enabled storing meaning in the symbols, the disjointed artefacts enabled storing of energy in the machines. These two moments in the human history must have played a revolutionary impact on human life. Depending on interpretations stored external to the body and and depending on artefacts with stored energy undoubtedly marks branching points in human history making humans apparently "diverge" from biological evolution.

Therefore memet-memetat pair is not only about the increased arbitrary symbolic rule spaces, but also about extended externalised constructions (both buildings and machines). In the history of humankind, the bifurcation point of introduction of arbitrary symbols and externalising human actions through engineering innovations must have co-evolved, since the ontological roots for both these two `peculiar' helical strands are complementary. Often these two are construed as branches and therefore our use of the term `strand' to describe them needs to be considered as a contrasting metaphor. They serve and reinforce each other through their metaphorical strength. While one strand is about nested communicative constructions, the other strand is about nested engineering constructions. Yet they continued as complementary strands with inter-twined new-found affordances and scientific language games.

Now, where does the artful expressions and creative material fabrications fall in this picture? In other words, where does the imaginative explorations lie? As briefly indicated above, we have drawn a distinction between saturated and unsaturated HAPs. We argued that the bifurcation caused by the disengaged HAPs generates a creative field. This field includes not only new combinations of names (expressions) but also new combinations of materials. The new-found arbitrary arena, which supports the various permutations and combinations, opens up the possibilities of new semantics as well as new material fabrications, which are situated in the memetat. This field can afford to create `meaningless' expressions as well as `useless' material fabrications, arbitrary traces. Therefore the exploratory freedom granted by HAPs becomes a context of discovery as against the context of justification \cite{reichenbach1938experience, popper2005logic}. The redundancy of operations in the context of discovery is well-known [cite], because there are multiple ways in which one can create something or solve a problem. Some may be elegant, some may be less so. Some may be economical, while the others are not. Some may be reproduced by others with ease, and some may not be. These explorations in the newly formed action patterns, spaces, traces, artefacts are considered meaningful/useful/affordable or not depends on multiple evaluation norms: reproducible (veracity, falsifiable), reference-able, rigor, validity, fitness (consistency, coherence), sustainable (survival, conservable, stable, balance, health, reparable), affordable, storable and retrievable (containment), usable, computable, probable, economical (efficient, simple), aesthetics (symphony, harmony, symmetry, melodic, beautiful), sensitive, conventional (conformable), transactable (tradable, exchangeable, communicable), secure (harmfulness), moral etc. All of these evaluations provide a richer ground for semantics. All of these provide the value space, a rich ground for semantics. We, therefore, argue that semantics can't be narrowly couched in terms of truth and falsity as done by logical positivists. Similarly, other attempts to couch semantics glorifying any one of the values as primary, are incomplete theories of semantics. In this picture, the semantic and artistic exploration can't be seen as independent explorations, since they are both grounded in the HAPs.

The framework has no way of distinguishing between humanities and natural sciences. The pursuit of seeking rigor in expression and reproducibility through experimentation can place multiple human endeavours on a scale, and not necessarily create a categorical distinction. 
Though rigor is favored in STEM (Science, Technology, Engineering, Mathematics), ambiguity is favoured in poetry, music and arts in general. Arts as explained above is a creative exploration space and therefore least judgmental. Similarly, ecological pursuit favours conservation as against excessive consumption and exploration. Thus though HAPs provide a ground for freedom of action, the value space acts as a buffer in tolerating these novel tendencies.  The richer the affordable HAPs space is, the richer is the value space, and therefore an arena for negotiation space gets enriched. 

Thus the creative domain of art is the medium of the two strands (expressions and material fabrications). The complementarity of the two strands is mediated by the aforementioned rich value space as a glue between them. The fabric of cultural life is built around the two strands glued together by multiple values.

\subsection{From HAPs to TAPs: Co-construction of Inter-subjectivity}
In the previous section, we discussed how an agent can differentiate between the external and internal objects, based on the criterion of feedback loop. Something similar happens in the construction of inter-subjective spaces between the agents.

There are two factors involved in the construction of the subject and the construction of inter-subjective space: the asymmetry in the body plan and the variations in the external world. The available actions (FAPs) are determined by the body plan, therefore genetically determined. The variations (HAPs) in the available actions (FAPs) are determined by the affordances in the external world, which includes both physical and other agents.

Let's recall that HAPs are possible due to the body plan of the subject that is endowed with multiple zones of FAPs. Similarity in the body plan of a community of agents in a common pool of the world will automatically lead to similar HAPs.

The distinction drawn between interactions and actions is vital in helping us in constructing the inter-subjective space. When an agent acts on an inanimate object the response in the form of reaction we get is instantaneous. When we kick the wall, a HAP, the wall `kicks' you back instantaneously. This is the character of physical interactions.  But, what if the wall doesn't? What if the leg passes through it? What if no sound, no visual difference?  This amounts to an attempt to stimulate oneself, but couldn't. One expects a reaction, but its absence causes surprise, therefore a cognitive dissonance, and in turn a trigger for withdrawal or further exploration and learning.

What if the wall reacts after a while. If we hear no sound instantaneously but after a while, we are not only surprised but will be motivated to enter into a transactional relation with the wall. What if the wall reverberates like a drum? Though there exists an instantaneous reaction, it is followed by an echo.  What if the echo is louder than the sound at the kicking instance? All these are unexpected. This may cause fear with demotivating effect, or may motivate to repeat the action once again. What if the wall moves away when we kick? What if the wall echos back only two times and stops? What if the wall begins to cry after the kick? What if the wall disappears soon after the kick? What if the reaction is a flash of light instead of sound? What if I kick one wall, and another wall kicks back? What if the reaction arrives from another wall after a while? What if an agent is situated in a world where no other agent exists? What if only one species of agents exist in a world? And what if only one agent of each species exists in a world? 

All these are thought experiments that will help us to understand the possibilities of variety of reactions from the objects in the world. Objects offer variations of reactions in time and space when we act on them. They help us differentiate the differences among them. Along the way we find other agents, who can act and perform HAPs.

Inter-subjective space develops through transactional actions where the gaps and delays between action and reaction begin to have a pattern. We can call these patterns as transactional action patterns (TAPs). And as we have shown, the TAPs develop through the HAPs. The TAPs as we can see can happen with animate agents and inanimate objects. The transactional space is not only between agents of the same species. In human ontogeny, this includes toys, people, pets, plants, etc. The expectations and motivations surface in the TAP-space. Thus, the sense-making is rooted in the affective grounds of expectations and motivations. So, the meaning associated to the TAP-space arises through the transactions.

This is the space that the artists exploit to create novel experiences. The stabilised TAPs become part of traditions and culture. The syntactical and generational features of HAPs become part of the TAPs as well. The syntax along with semantics becomes accessible inter-subjectively, though with a grain of uncertainty and inscrutability all through. The possibilities of mismatch between the expectations and motivations makes this space a permanent space of negotiations. Thus inter-subjective space becomes a space of negotiations between the agents.

To explain the peculiar cognitive condition, the dynamic model of the body is layered with a deeper interactional space, followed by a layer of life-sustaining actions called beats, followed by a layer of FAPs, followed by a layer of HAPs, and finally a layer of TAPs. Thus the layered architecture refers to both structure and function. From the core to the periphery of this model, the degrees of regulated freedom increase with the onset of each layer.

It is not surprising that our ontogeny involves exposure to multiple plays and games which help us enter into the inter-subjective space starting from the various toys we are offered to play with: repeat-after-me type of games (halt-after-me), repeat-along-with-me (halt-with-me) type of games, question-answering games (halt differently from me) and then imitation games (halt-like-me) etc. Since this exposure predominantly happens in human life by other subjects (by social members of the community), it appears that inter-subjectivity develops due to social means alone. The actions or their traces encode expectations and motivations which enable decoding by other participating agents. These `nursery' games are an invitation to the encoding decoding actions of the memet-memetat space --- which therefore prepares us for a world full of interpretation ahead of life.

It is left to the imagination of the reader, to see how languages, tool-use and various creative construction games we play with and without using technology can be part of the memet-memetat space. 

It is important to note that the mapping relations between the HAPs and TAPs with intension and extension (sense and reference) are introduced as rules. The following and breaking of the rules in a memetat is a never ending space of negotiation. Those language and other cultural games that we engage in are rule-following spaces. The precursor to the rule-following space is a space for exploring novel HAPs and TAPs, which we can call as a play in contrast to a game. Games have rules to follow, whereas a play may break or form new rules or form HAPs and TAPs without any rules. Thus play is closer to artistic exploration, which exposes us to novel experiences, finding crevices within traditional sedimented edifices.

When we enter into conventional games of rigorous kind, where the mapping is strictly adhered to by definitions, the element of play is reduced for maximising reproducibility. The STEM games fall in this category. Rigour is a pursuit of eliminating ambiguities within the negotiation space, which is reflected in the practices of engineering, science and mathematics. However, this difference is not a categorical, but in degree where all cultural activities can be located on a spectrum.

The social and natural selection of the games we play in the cultural space provides which HAPs and FAPs are successful in meeting the expectations and motivations. This is a continuous co-evolving space of engaging in negotiation and interpretation. This transactional spectrum is situated between two extremes with intimate transaction such as an embrace between two subjects at one end and spatio-temporally separated transaction when we read and respond to a historical document or an archaeological trace at the other. This wide spectrum of transactional space of TAPs is apparently ungrounded due to the gaps and halts. However, the grounding comes to the surface whenever encoding and decoding actions happen. Therefore the memetat can be built based on an ontology of actions and their traces.

\begin{figure}[ht] 
%\renewcommand{\thefigure}{2}
\includegraphics[width=\textwidth]{side_view LSMN.pdf}
\caption{\color{Gray} \textbf{A schematic representing layered architecture of SMN.}}
\label{side_view_lSMN} % \label works only AFTER \caption within figure environment
\end{figure}

\begin{figure}[ht] 
%\renewcommand{\thefigure}{2}
\includegraphics[width=\textwidth]{top_view LSMN.pdf}
\caption{\color{Gray} \textbf{A schematic representing layered architecture from a top view.}}
\label{top_view_lSMN} % \label works only AFTER \caption within figure environment
\end{figure}

\subsection{Learning and Memory Grounded in Affordability}
As repeatedly mentioned here, the units of cognition are action-patterns. Patterns are available both as constellations, i.e. as pictures or snapshots, as well as melodies, i.e. sequences or streams. The former are atemporal. The so-called atemporality of constellations is due to the stable relations over time. They are differentiated based on geometrical relations. The latter, like melodies, are differentiated on the basis of how the constellations vary over time. Relations are the discernibles which form patterns and the variations are the changes in relations. The core aspect of what we can discern are relations but not things per se. What stays as memory is nothing but stability of the pattern --- the relations in the constellations and melodies. Memory is not for action; action patterns \textit{are} memories. Therefore, this model suggests that memory is not a substance that can be stored but a reproducible pattern of relations. The apparent storage property of memory arises from the potential to reproduce.
This potential is a function of re-forming and de-forming the action patterns as FAPs and HAPs. This is a dynamic negotiation space and not something that can be described in terms of the binaries of innate and learnt [Beyond Modularity: Karmiloff-Smith].

Affordability of holding an action pattern determines the ability of storing memory. Holding action patterns is expensive than holding the traces. The traces give us the apparent character of a storable memory. Though action patterns (both HAPs and FAPs) are transient, they can be reproduced. FAPs are easily reproduced and are triggered by the immediate environment, while reproducing HAPs requires additional triggers and practice. HAPs, as discussed earlier, can be saturated or unsaturated. As HAPs evolve into TAPs, the requirements of context and practice increase, while reducing the cost of storing. The spectrum from beats to TAPs is very broad. We shall expand on the various kind of things we can memorise. 

In the proposed model, we assume that the polarised and asymmetric architecture and the sensory motor network of the body is evolutionarily granted. Most of these aspects can be considered as evolutionarily learnt and genetically memorised, in a broader sense of the term "memory". However since the body plan and SMN aspects are crucial for cognitive development we can not ignore how dependent the phenomenon is on what is biologically granted. As we distinguish between interactions (molecular and cellular level) and actions (organ level), let us focus on the learning and memory related to the latter. 

The bridge between interactions and actions that have an important role in cognition is the layer of beats. Though variations in the beating patterns (heart beat and breathing pattern) are limited, these are triggered by both internal as well as external environment. Their potential in generating cognitive experiences can not be ignored. Often we might consider such experiences only within the bracket of emotions and feelings, in our framework they play very vital role in evaluatory interpretation of other actions, and this forms the site for grounding semantics. While we perform FAPs, HAPs and TAPs, we can not afford to cross the tolerance limits of variations possible by the spectrum of beats. There for this is a significant bridge between interactions and actions. The entire spectrum of learning and memory with respect to FAPs, HAPs, and TAPs are negotiable only within this affordability space. 

The body plan (through genetic memory) and the habitat (provided by the living context) determine the FAPs such as flipping of fins and tails, swallowing and undulation, walking, running, self-pruning etc. These FAPs form the habit-habitat coupling. Based on the affordances offered by the internal and external environments, a wide range of variations are possible in these action patterns. Though largely these are determined by genetic factors and the habitat, since these actions are performed by the SMN, they play the role of differentiating the difference in the habitat.Thus the extent of variations in action patterns increases by several folds once the FAPs layer enters the scene. The reproducibility of these variations in action patterns constitutes the memory in this layer. The motor action is almost inevitable in a live SMN, as evident in pre-natal FAPs. The initial trigger for FAPs is largely internal therefore genetic. Apart from the emotions and feelings, the FAPs are the action schemas, in Piagetian terms, using which the SMN explores the world.

Over and above the genetically determined action schemas the affordances of the habitat trigger variations introducing gaps in the FAPs. As more bifurcations become possible enabling disengagement, the spectrum of variations possible within the FAPs enhances manifold which catapults the SMN into the layer HAPs. The flexible (dexterous) motor anatomy has the potential to transform (re-form) the network following the FTWT principle, which makes the SMN plastic. Without variations in the action patterns enabled by the motor anatomy (plasticity of the action zones), the new connections in the network are not possible (plasticity of network). We tend to propose the primacy of motor plasticity resulting into neuronal plasticity. This speculation is grounded in the evolutionary and embryological history (phylogeny and ontogeny), that all cells are capable of movement and communication with each other without neurons. However once neuronal connections are available in an organic structure, it is untenable to ascribe primacy of cognitive function to any one part of the SMN --- action-over-network or network-over-action. The newfound connections and newfound HAPs are indiscernible in a network architecture. Therefore it is untenable to say the memory is exclusively in the connecting units or in the motor units. Locating memory, is hence a untenable research project, except ascribing it to the entire network. However, as we move from HAPs to TAPs the locatable, storable, and retrievable nature of the traces of HAPs, representations, become possible. Insofar as the representations are available to the cognitive agent, they are manifested organically, as a whole, and therefore cannot be ascribed to any one part of the SMN. In the later section, we discuss the experimental interpretations for the putative evidence that neurons are sites of memory.

Memorisation or learning of HAPs is apparently contradictory in an interesting way. The emergence of a HAP is at the breaking of a pre-formed pattern. For HAPs require gaps, halts, breakages and new variations --- a creative force. Now, when we talk about memorisation and learning of HAPs, it is about reproducing the action pattern again and again. The orientation here is to sediment the pattern to FAPs to an extreme of establishing hard connections between the zones --- a conservative force. Once you learn a HAP very well, it ceases to be a HAP. An explanation for this could be that the FAPs are the attractors ---  the conserving and reproducing tendency --- of multiple zones of SMN as dynamical systems. It is therefore a matter of habit for a biological system, to repeat almost all happenstances. It is a default self-reproducing embodied force, as vindicated by autopoetic model of life, a kind of inertia in action. 

The biological order maintains a landscape of the SMN, where the topos of the landscape is determined by the body plan (polarised or not, symmetrical or not). The dynamical nature of the FAPs generate the attractors in the landscape. The available articulatory freedom, provided by the multiplicity of the action zones, gives rise to the soundscape, so to speak of the HAPs and TAPs. The soundscape is a transient state, and exists as long as the FAPs, HAPs and TAPs persist, which is the wakeful state. When the action zones are in a restful state, e.g. in a sleep, the soundscape is like a calm surface of a lake. This analogy with the active cognitive state as a soundscape and the biological body as the landscape can be mapped to the layered architecture of the SMN where the interactional space of the SMN forms the landscape, while the action space forms the soundscape.

integrating and differentiating: The spatio-temporal modalities of cognition \textit{appear} complementary. In our model, they can be mapped to the DFN and IN respectively. An important implication of this is that spatial cognition is evolutionarily recent, and depends on prior temporal backdrop. The differentiating and filtering network provides geometrical understanding (because we consider motor sub-units as location sensors), while the integrating network of the SMN provides time which is grounded in the layer of beats. The IN part of the network connects on one hand with the sensory motor subsystem and the physiological (beats) subsystem on the other. The connection with IN with rest of the metabolic body is likely regulated through the messaging endocrine subsystem. Thus IN is like an intermediary between the physiological beats on the lower end and the cognitive subsystem comprising of the action-zones of FAPs, HAPs and TAPs on the other.

The inter-subjective medium is created by various art-forms, which are realised by action patterns. These art-forms can be broadly presented on a spectrum with the sequential patterns (e.g. music) on one end and synchronous patterns (e.g. painting) on the other. 
Literature presented in oral form exploits sequential auditory snapshots (words) presenting pictures and narrating stories triggering both imagery as well as a narrative. Thus literature in oral form is predominantly sequential, making the pictorial form of words almost invisible. The complementary artform of drama is predominantly pictorial. The literature in the written form makes the picture of words (as inscriptions) clearer than in the oral form.  The most recent artforms like cinematography exploit a sequence of snapshots along with multiple tracks of audio presenting richer spatio-temporal experience.  The streaming and picturing artforms get into our culture as mixed offerings of varied predominance. Thus the phenomenological experiences presented by the art-forms are composed by streaming qualities and snapshots.

This model of culture is analogous to the wavelength and frequency representation of periodic phenomena, where the wavelength as a picture emerges only after lapsing of multiple beats. The predominance of low frequency of repeated patterns of pictures has led to the quintessential human cultural behaviour. The widening of the gaps between the patterns and insertion of other patterns in between them makes it difficult to see the recurring/periodic nature of this `waveform'. As already noted by the linguists that the repeated use the same words but in different combinations making it highly generative.
A community of embodied beings through combinatorial mechanics, can remember and transmit memets through HAPs and TAPs. The information is encoded in the pattern of gaps. The order and disorder of the gaps is the analogical connection between information and entropy as proposed by Shannon [cite].

Though human life is dominated by visuo spatial artforms, we should not lose sight of the culinary artforms, that aren't given sufficient space in academics. However these are given the deserved priority in commercial TAPs in the form of culinary and perfumery industries. The culinary art-forms evoke a richer experience invoking gustatory and olfactory modalities. Though not as predominant in every culture, the art-form of perfumery cannot be ignored. The experiential aspects of gustatory and olfactory modalities are predominantly integratory than differentiating in nature. However the experts who modulate these environments perform the HAPs to present the experience, just as a painter presents the picture at one go. The modalities of the SMN that are involved in different art-forms are responsible for the differentiating and integrating our complex cognitive/cultural forms-of-life.

Space is constructed through action, time is given through interactions. This is consistent with the layered architecture of the model. Since actions originated later in evolution than interactions, time is so to speak \textit{a priori} in relation to spatial cognition. However, this does not mean that we understand time prior to space. Explicit understanding of time is posterior to explicit understanding of space. This conceptual space corresponds to the occurrants and continuants in the ontology of events and objects [Barry Smith]. Therefore the `data-structure' of experience and that of memory correspond to each other if not identical.

An artist, who presents an integrated experience in the form of a painting, a dish, a sculpture, a perfume, etc. performs HAPs over prolonged period of time predominantly employing multiple DFN zones. Musicians and dramatists who present the experience in the form of a song or a play, orchestra etc. synchronise the action patterns and experiences together at the time of performance. However developing the skills and composing require prolonged periods of time predominantly employing multiple DFN zones. The modern recording technology masks the simultaneity of performances and perception/ experience by presenting only the traces. They include multiple ways of recording (imprinting and encoding) traces of HAPs creating a \textit{gap} between performance and perception/ experience. This gap creation is the root of creating memory. Among the story-tellers --- from oral ones to a novelist --- we find the same model, where the utterance of the story by the story-teller and experience of the listening is simultaneous, while in a novel there is a gap between the writing and reading (performance and experience). 

The relationship between memory and the gap between performance and experience can be further elaborated based on the gap-creating epistemology. The spectrum and the character of gaps between performing action and perception/experience creates the quintessential human cultural-life through various art-forms involving traces. The entry to one end of this spectrum happens by participating in the forms of life such as acquisition of first language. Since perception happens only through action, the gap (however small) between performance and experience is inevitable, as an implication of action based perception. A pattern is inconceivable without a gap, because it is literally composed of gaps. This makes the patterns carry information.

FAPs are the memories inherited directly from the body plan of the organism (agent/ being). Prior to the FAPs, in our model, there is no cognitive memory. The variations in FAPs are created because of the haltability. Halting is a gap-creating action. Gap-creation generates new patterns, and the ability to reproduce these gaps is the new-found memory. As described in the previous sections, halting is also the basis of differentiation and filtering of experience. 

In this context, it is important to note that the patterns can be \textit{redundantly} generated by multiple action zones. E.g. as elaborated in earlier sections, a pattern generated by feet, can be reproduced using fingers or a tongue, lips etc. In other words, a pattern generated by feet is \textit{translated} by the finger, tongue lips etc. This is the mechanism of creating one to many relationships, a logical necessity for an abstraction. Though, in principle, it is possible to reproduce a pattern by a single action zone, our model prescribes that no memory or abstraction is possible without other action zones and/or involving other modalities performing similar action patterns (falsifiable criterion). This implies that without \textit{translation} between action zones no abstraction or memory is possible. The so called distillation that is ascribed to abstraction is nothing but the possibility of redundant performances by multiple action zones. This, according to our model, is the root of concept-formation. Considering the importance of this link between concept-formation and memory, we shall elaborate on this through the following examples.

Let's consider a stage performance of a group of artists with dancers and musicians. Some of the artists were performing in the orchestra while some were acting and dancing, while some others were singing. The effect of this performance is experienced by the spectator through multiple modalities, because it involves both visual and auditory experiences. We shall first consider the case of experiencing a live performance. Here, the gap between performance and experience is \textit{minimal}, because it is live and \textit{least} involvement of memory. That the gap is minimal and not nil and the involvement of memory is least and not absent, must be taken note of. This is an implication of action-based perception/ experience, which requires action (either covert or overt) by the spectators while experiencing. One way of appreciating this can be by understanding the fact that everyone can not be a spectator for a given performance. This is because, to become a spectator, prior exposure to the cultural practices is required. Appreciating the specific patterns in the performance needs to be cultivated. The body is not a passive receptor, but a trainable (programmable) receptor. 

The live performance is richer in the sense that it involves several performers with each performer employing multiple action zones involving multiple modalities. In the case of experiencing an audio recording of the event, the visual and ambient experiences of the auditorium would be missing. The gap between experience and performance increases in this case. In the case of experiencing a video (audio-visual) recording of the event, gap is reduced, but still exists. The video record gives a richer experience than the audio record though it is not as rich as experiencing the live performance as a spectator. So, the richness of experience has to do with the involvement of multiple modalities -- e.g. visual, auditory, tactile and proprioceptive (body swings) modalities. By elimination of each of the modalities, we are progressively reducing the richness. This is analogous to the how a gesture is less rich than the actual saturated action. The richness --- involvement of multiple modalities --- is a function of level of saturation of action patterns. The redundant generatability of patterns involving multiple action zones and modalities is the anchor for continuing to have similar experience even at the expense of richness of experience. 

For a person who was in the performance, and experiencing the recording at a later time, the experience is not the same as that of the person who was not in the performance. So, what is the difference? The multiple zones that were active while the performance was happening had their contribution in abstracting the action patterns, e.g., the body swings, tapping of feet, singing in chorus and in anticipation, applauding actions in excitement, watching fellow spectators swinging, effects of lights and sounds, etc. Listening to the recording for the person who was in the live performance helps n reinforcing and recollecting all these richer experiences. While the person, who was only listening to the concert record, can relate to the music based on their past exposure. Our model therefore predicts the multiple activation zones if one were to perform an experiment on the contrasting subjects. The effortless synchronization behavior in a larger crowd anchored to a rhythmic cues is well known whether in a military parade drill or a sports stadium. This multi-modal synchronization across multiple zones in the SMN adds to the richer experience as well as means of memorising. Similar technique is used in the kindergarten for learning language and social skills. In such an environment, it takes effort not to clap along. Thus the greater the richness of involving multiple modalities and zones of the SMN, the greater is our ability in re-enacting which in our model is same as recollection.  [Add references for the swarm behaviour and the involvement of multiple action modalities and zones in memory formation]. This also implies that the highly saturated experiences involving multiple modalities and zones help in abstracting the patterns. Since we claim that it is the patterns that are remembered and not action  per se, storing and retrieval is distributed across the network. While the DFN is distributed, the IN creates hubs. Therefore the model predicts that the damaging the IN hubs could affect reenacting despite distributed character of the network.

Action-zones are distributed in the body plan in a polarised and asymmetric, as detailed earlier sections. Specificity and directedness is a character of  experience that results due to DFNs, while generality and wholesomeness is a character of experience that results in due to INs. Therefore it is very important to delineate these two complementary effects of experience, differentiating and integrating. Integrating aspects of experiences do not need training because of the wired nature of the SMN. One can't stop the flow of information/ signals in a connected network. However differentiating and filtering actions need effort and training. Therefore seeking attention is effortful than an ambient experience. Not employing specific zones, therefore is relaxing than being in a zone.

The distinction drawn between emotion and feeling (cite Damasio et al.) can be linked to differentiating and integrating aspects of experience respectively. The about-ness of propositional attitudes can be attributed to the DFN. While the non-localisable attributes can be attributed to IN.

\textbf{Relationship between our model and predictive models:} The model described above is coherent with predictive models of cognition and memory. AS already mentioned in the previous sections, memories are like melodies. The action patterns that our SMNs are attuned to have reproducible tendency giving rise to the predictive behaviour. And modifications to existing attunements are effortful, which involve new cycles of accommodation and assimilation. Cultural differences that we find in both language, music, dance, etc. are difference in such attunements. Once tuned to a particular action patterns, it is effortful to deviate from them. And this effort again must come from our ability to halt.

The character of processes that we can identify in the cognitive domain can be named as saturating, desaturating and resaturating. Richer involvement of multiple modalities and zones is a saturated state, as in a spectator or a actor in a concert or auditorium/ stadium. Elimination of some of the modalities and action zones amounts to desaturation, as in experiencing through recorded performances. Desaturation is a wide spectrum that may or may not involve traces that are external to the SMN, such as scripted language or music notations. Resaturation is nothing but an interpretative action where we begin to add other modalities and zones. Interpretation is therefore a restaurating action. The resauturation, interpretation and remembering are ontologically indifferent.

as a receptor is prepared 
Saturation, desaturation and resaturation.  
Transduction and translation.
The richer experience 

Since every action has a corresponding experience in a networked body, the experiences can be different for different zones. Thus despite the differences between experiences, there is an apparent identity. This apparent identity is the content of memory.
Similarity is of the pattern, which can be achieved from multiple zones of the SMN.


or through explicit training. The reason why the first language  The history of human history is also a history gap-widening technologies between performance and experience. The modern digital multi-media enhances the gap between the performance and experience to extreme levels, but increases the fidelity of recreation of experiences. Thus the memory cannot be located within the SMN or within the socio-technical space. The attempt to localise memory ignores the past performance. Therefore just as life is an organic outcome of a complex cellular network, memory is an organic outcome of a complex network of SMNs.

Spatio-temporal resolution is more recent than olfaction which is more recent than gustatory. Specialised organs for olfaction evolved later than gustatory organs. Thus evolution of specialised organs for each modality is an indicator of the polarisation required for differentiation. Therefore the architectural plan of the body is a recapitulation of the evolutionary origins as stated in the biogenic principle. Thus, what is sedimented through evolution is a memory of repeated action-patterns that worked for the being in sustaining their lives.

The various zones that participate in DFN directly are a result of evolved architecture of the body, which as presented above already provides the schema for assimilation and accommodation of action patterns. Metaphorically, one can say the stratification of the SMN is an indicator of cognitive evolution just as geological stratification is an indicator of organic evolution. There is a microcosm in the body that can't be ignored, nor can we ignore the interactions between the multitude of microcosms.




- memory to be linked to memets, and not to habits; generational vs copying/imitation; SMN's learning is computational/mathematical/procedural: SMN after learning has a function to produce the result; result is not stored, it is computed. 

- footprint as a paradigm of traces; traces  are storable/retrievable.   

- mathematical function is a paradigm of internal memory, functions are stored not the results. learnt HAPS are like mathematical functions.

- memory is returning to an unsaturated experience (state of SMN), returning to a fractal identity (pattern) 

- DoD captures patterns. 

The nature of poetry and music reflect the 

The spectrum of variations that are possible on one hand are more conservative as in classical music and on the other side is the speech with being poetic in between. The extreme end of the model is that of classical music which is structured in a layered manner, which is an idealised version of the model. Various cultural actions that we do can be interpreted as instances of this idealised model, with possible variations within its scope. The patterns that are closer to the idealised model are easiest to remember/ reproduce. A formal model on the other extreme written in disembodied vocabulary such as mathematics is rigorous and difficult to remember without a rule book.

HAPS closely linked to FAPS and env: variations in FAPS; hopping, sequencing; some animal/bird calls, 

imitation: action based perception comes with imitation as a by product. listening is an action; anticipatory action, predictive processing 

saturated HAPS; grasping tools; 

unsaturated HAPS; gestures, speech, abstration  

nested HAPS; hunting 

Rules, TAPS

Rule following implicitly 

Rule following explicitly (dictionary)

performative 

declarative 

evaluative 




The inertia in the beats forms part of the predictive model in cognitive science. The action patterns continue to remain in a given state, unless halted by any perturbing factor (internal and external). The memory continues to remain in a given state unless disturbed by a perturbing factor (internal or external). However, since the cognitive action is based on halting resulting from perturbation from an environment (internal or external), the cognitive state is to be located at the interface of beats, FAPs and HAPs. As the lower layers of beats provide the backdrop (as already seen in section xy), the newly formed and de-formed FAPs/HAPs are `encoded' on that backdrop/canvas.

The action patterns can be monotonous due to habituation. One pattern can be discerned from the others, because of the possible variety of the action patterns. The dynamical systems modelling can account for variety of such action patterns by provisioning for multiple attractors in the system, becoming multiple storable units of information. The problem of cognition is not about the availability of multiple patterns, but about the formation/generation of sequential patterns of them. The problem is to account how the variations in the patterns are possible, and not merely that the patterns are possible. The haltability must enter the picture to explain the generation of new patterns --- the new patterns set in the SMN through the perturbations triggered by the external/internal affordances. As long as, pre-existing stable HAPs (FAPs) are unperturbed, they are predictable. Our life is largely unpredictable, because the sequences that we generate could be novel. Novelty is subjected to cultural selection, with a possibility of another sedimented action pattern.

By invoking the principle of identity of indiscernibles [cite Leibniz] we can formulate identity of indiscernible action patterns. If for every property $F$, object $x$ has $F$ if and only if object $y$ has $F$, then $x$ is identical to $y$. The property $F$ is nothing but the indiscernible action pattern. An object in our case, is a discernible constellation, which means the relationship between the nodes are constant. The properties of the constellation are cognisable attributes such as locatable, audible, visible, tastable, smellable, tactile. Applying the principle of indiscernibles if there is no change in location, sound, sight etc., the constellation is indiscernible. This is consistent with the principle of differentiation of difference as cognition. Thus discernibility is cognisability.
