
\section{Introduction}

The \textit{apparent} anatomical peculiarities of the human body, semiotic (sociocultural, linguistic, computational, and representational) peculiarities, and sufficiently strong phenomenological correlations between the two peculiarities are
masking us to see the underlying truth.  The anatomical peculiarities that we refer to are the large size of the brain, bipedal locomotion emancipating the forelimbs for other tasks, dexterous body, prehensile hand, and sophisticated vocal apparatus. The semiotic functional peculiarities are representational, rational, formal, linguistic, computing, and information processing faculties.  In our attempt to provide a scientific foundation it is natural to look for correlations between the anatomical peculiarities with the functional peculiarities.  

This debate is not unrelated to whether human cognition is continuous or discontinuous with other animals.   How could generic biology explain the peculiar and qualitatively different human condition?  Since there are several qualitative differences between us and other mammals, there must have been biological differences as well.    



On the other hand, so much of known biology that is relevant for cognition is being ignored because all animals have them.  

We argue through this monograph that we need a model-driven approach to explain the correlations, and not deploy the correlations as a part of the model.  
We attempt to convince the reader to give up neuro-centric models. 





It is a multitude of predominant action possibilities,  

Our obsession to describe and explain the peculiarities interfering to see through the predominant architectural aspects. 



We like to characterize ourselves, humans, from other living creatures by identifying some peculiar capacities.  The most prominent among them are our ability to speak, reason, and the use of tools.  The various achievements in the advancement of arts, humanities, or STEM (science, technology, engineering, and mathematics) would not have been possible without language, reason, and technology.  The recent breakthrough in artificial intelligence, employing large language models, reinforces such a characterization. 

One of the points that we want to bring home in this work is to demonstrate how these success stories helped us to understand special features of ourselves, but they have also `successfully' conered up a special feature that is at the root of the so called peculiar capacities we harbor.  
Several major breakthroughs in science and technology are responsible for this covering up: the obsession to locate our special cognitive and cultural practices to language; tremendous impact of genetics in shaping modern biology; enormous and disruptive impact of information processing models in modern society; recent breakthrough in artificial intelligence, employing large language models; 

Two major breakthroughs in science and technology helped us to understand so many things, while at the same time, they are also impeding the cognitive science. One discovery and another invention 
The discovery of DNA, genetic code, genetic basis of variations leading to the modern synthetic theory of evolution 

After all we knew from evolution that single or multi-cellular organisms exhibited movements as well as variations in patterns without a nervous system \citep{llinas2002vortex}. Even a lump of cardiac tissue is known to beat rhythmically, not only a detached beating heart from the body, leading to myogenic hypothesis  \citep{landecker2007culturing}.  Almost all animal cells have a cytoskeleton, to generate sufficiently sophisticated movements without a nervous system.  The cells do navigate around their space even without depending on special sense organs, that is to say that every cell responds to stimulus, and have endogenous means of acting on the environment.  More and more experimental evidence is catching up to show that non-neuronal cognitive phenomena, intelligence, memory and basal cognition are all over biological space \citep{Levin2023, biomimetics-Levin2023}. And almost all known material in the universe is electrically active \citep{Peratt1996plasma}, not merely the sense organs, muscles and the neurons in the body. Therefore to single out the nervous system for being a central cognitive organ based on phrenological studies is unfounded \citep{anderson2014phrenology}.    

Noem Chomsky argued that sophisticated hierarchical structures cannot be created by monotonic patterns \citep{chomsky1957syntactic}. 

What if movement does not need an initiating message from some control center? We know from physics that to maintain inertia we do not need a force, unless we need a change of moving state. If we have a lesson to borrow from this context and apply it to cognition, could we consider the idea that to alter an action pattern we may need an external influence/message, but not to maintain it. Could it be that the central nervous system, where we look for all answers pertaining to cognition, is required to alter a action patterns and not to initiate them? This speculation is not ungrounded at all! 


This essay introduces an alternative framework for a scientific study of conscious cognition and cultural practices that are predominant, if not peculiar, to being human.  We take an architectural approach to explore an answer to the question: What kind of biological architecture of a cognitive agent could implement phenomenological and semiotic features? 

We have had several fascinating advances in cognitive science during the last century. 
Initially dominated by rigorous experimental behaviorist models of learning among both animals and human beings covering conditioning, learning by associations, positive and negative reinforcement with a common methodological commitment to externally visible behavioral parameters \citep{pavlov1927, skinner1938, skinner1953science, thorndike1898, watson1913psychology}.  These studies were criticized by another research program, widely known as cognitivism, for their inability to account for language acquisition among human beings, arguing for a peculiar innate language faculty attributable to specific and modular regions in the brain, while also underlining the peculiar generative syntax of human language \citep{chomsky1965aspects, chomsky1986knowledge, fodor1975language, fodor_modularity_1983, pinker1994language, pinker1997mind}. Despite their common ground of biological and evolutionary origins of cognitive abilities, they disagreed on the question of whether the differences between human and animal cognition is a matter of predominance or peculiarity and whether the sources of these differences are due to nurture or nature \citep{chomsky1975reflections, pinker2002blankslate, watson1924behaviorism, skinner1971beyond}. Among European scholarship, on the other hand, we notice a greater influence of neo-Kantian and dialectic approaches where both internal and external conditions affect human cognition \citep{piaget-biology-knowledge, piaget1970genetic, ponty1969phenomenology, Merleau-Ponty2013-vs, Vygotsky1978-bk}.    


some of the ignored biological design principles 



We propose that the core problem of cognitive science is (1) to solve how an agent constructs a geometric picture of the world around it and (2) how to communicate that picture to other agents meaningfully.  To solve these two core problems, the agent must have a phenomenological subjective experience of the world on the one hand and share it with other agents with similar abilities.  What structures and dynamics do we need \textit{in} the agent such that subjective phenomenology is possible, and how can that be shared inter-subjectively and meaningfully with other agents? The innate ability, we assume, the agent is born with is an internal in-built clock.  This clock, we assume, is granted through 1.3 billion years of evolutionary history.  Evolutionary history also grants a mechanism, structure, and dynamics to construct a geometrical picture of the world.  The geometrical construction of the world is enabled by \textit{halting} the dynamic mechanism without deviating from the pattern, a well, resulting an awareness of space and time --- cognition.  The space is created by a set of nested action patterns that are fixed, haltable, and transactionable. 

Cognitive space is a transient geometrical space, \textit{memetat}, constructed through multiple, recursive, recurrent, and haltable serial action patterns, called \textit{memets}, which can be retained by reenacting and creating interpretable traces. The agent's body is modeled as a layered sensation-modulating network (SMN), which is functionally distinguished into a differentiating and filtering network (DFN) and an integrating network (IN).  

The SMN is modeled as a polarized and bilaterally symmetrical PetriNet (PN), a bipartite graph of transitions and places.  A set of place nodes of the PN enter SMN from the environment where the cognitive agent is situated via the interfacing nodes of transitions. The transition nodes of the PN are sensory transducers and actuators, constituting the sensory-motor part of the architecture. The resulting tokens from the sensory-motor part of the network form as the other set of the place nodes of the PN in the form of a nested stack of neural connections holding transient tokens for a while.   The tokens pass through one stack of network into another until they vanish completely.  The phenomenological space is constructed by the tokens in this transient nested stack of networks through multiple, recursive, recurrent and haltable serial action patterns by the sensation-modulating network.   

The geometry is computed by calculating (1) the differentiation of differences in the phenomena through self-modulation, (2) the relative location of the phenomena following the principle of \textit{mapping the delay with distance}, and (3) the principle of the concomitance of phenomena granted by a dynamic \textit{fire-together-wire-together} architecture of the SMN. The spectrum of perception to conception (spectrum of abstraction) is explained through a model of a spectrum of saturation of multiple sensory modalities and the corresponding dynamic spectrum of engagement and disengagement of action patterns. Meaning and evaluation are grounded in the deeper layers of the layered architecture of the SMN.  Transient action patterns can be memorized and recalled only by re-enacting them, while the traces of action patterns can be persistent and hence can be recalled and gamified. Human culture is modeled as a confluence of multiple microworlds, where each microworld is constructed by mutually stimulating rules following actions that define the boundaries of the transactional playground. We end the article with a re-interpretation of the well-known empirical and experimental results that are well established, followed by concluding remarks on how the proposed model could account for human nature.


By doing so, we propose that this mechanism is part of a common root system that resolves the problems of consciousness and the ability to communicate through a language characterized by generative syntax, both of which are considered essential traits of being human.  




\subsection{A Possible Solution}

We propose a dynamic architecture of a cognitive agent inspired by a generic anatomical plan of an animal body endowed with conscious phenomenology and grounded semiotics capable of participating in extended mimetics.  The perspective developed is informed by both information processing models rich of representational content of cognitivism on the one hand as well as action-based embodied, situated and extended models of cognition on the other.  Thus offering a dynamic architecture enabling a reconciliatory foundation for cognitive science. We offer regiorously formulated hypotheses that integrate both the endogenous and exogenous conditions that make conscious cognition possible as well as the indicators of the phenomena. 

The core proposals include:
\begin{enumerate}
    \item Cognitive world is a mediated construct of a geometric semiotic habitat, called \textit{memetat}. 
    \item A cognitive agent is modeled as a sensation modulating network (SMN), whose function is to compute the location of sensations by serially modulating action patterns called \textit{memets}.  
    \item Perception of the world is an outcome of placing the sensations relative to each other, using the principle of differentiation of difference (change in action patterns, not action patterns per se).  
    \item The agent, an SMN, is a multilayered process network of antagonistic coordinated pairs oscillating at relatively high frequencies at the layer beneath---as fixed action patterns (FAP).  The lower layers are mandatory and can be modulated with the least degrees of freedom. In contrast, the top layers can afford to halt, without deviating from the general oscillatory pattern---as haltable action patterns.  Both FAPs and HAPs are dynamical systems.
    \item The bottom layers of the SMN provide a multi-dimensional cognitive stage in the form of an integrating network (IN) provided by the FAPs. In contrast, the top layers form a differentiating and filtering network (DFN) provided by the HAPs. 
    \item 
    Imitability of action patterns, encoding by copying, TAPs
    \item Traces of action patterns, semiotic space 
    \item saturated and unsaturated HAPs, abstract space
    \item speculation of the mechanism that enables halting: limited communication routes between the CPs. 
    \item deeper layers contributing to the meaning and value to actions, cf. Damasio. 
    \item rules and representations, grounding generative syntax TAPs 
    \item haltability as a condition for syntax 
    \item games and microworlds
    \item mimetics 
\end{enumerate}
The paper is situated within the debates between the two major \textit{apparently} opposing research programs in cognitive science – cognitivism (\cite{chomsky}~ \cite{fodor_modularity_1983}~\cite{pinker}) and 4E cognition (\cite{gibson1991ecological}~\cite{Merleau-Ponty2013-vs}~\cite{varela}~\cite{maturana1991autopoiesis}~\cite{noe_action_2004}). The cognitivists would have an empirical basis if we can find amodal representations and rules within the brain of the cognitive agent. And the empirical basis for enactivists emerges if we can ground cognitive phenomena in the action space of the body and its environment (including both physical and social). Both the research programs, however, do agree on a biological basis. While the cognitivists try to locate the symbolic patterns in the firings of neurons and their connections, the enactivists look at dynamical systems, including neuro-sensory-motor systems, in the body and outside. Neither of them ignores the body; as a result, cognitive neuroscience is practiced by both research programs. In terms of executing the research programs in computer science, the information processing model favors cognitivism since computers use amodal bits as representations. In contrast, the connectionist and reinforcement machine learning models (neural networks) and cognitive robotics favor the enactivists.
Another research program based on predictive processing goes into deeper physics layers in favor of enactivism. Almost all the research programs have a common ground in biology, organic evolution, and the principles of scientific investigation. One may wonder why serious disagreements still exist! 

Did they get their biology wrong? Or did they neglect some well known biological principles because they thought they are not related to cognition. 

The research programs do not appear to be reconcilable because both camps depend on an \textit{incorrect} model of the body of a cognitive agent in terms of its structure and what its components do.  Often, the assumptions on which their research programs are based are insufficiently grounded. Therefore, we propose an alternative dynamic architecture of the cognitive agent. This model revisits and redefines the biological roots of cognition, where we depart radically from the received views of cognitive neuroscience. 

Our core argument is presented in the form of inter-disciplinary model-based reasoning\cite{Nersessian_2002} \cite{nancy2022}. The argument holds if we are granted the proposed dynamic architecture of a cognitive agent for its ability to explain semiotics.  The proposed model is to find an answer to the question: \textit{What makes a cognitive agent's participation in semiotic form-of-life possible?} The justification for the model is presented as an inference to the best explanation, a form of abduction.\cite{lipton2017inference}  At the same time, we persuade by making explicit the assumptions and the grounds for assuming them. 

What phenomena do we include seeking explanation in the semiotic space? The core identifier of semiotics is the use of \textit{signs}, what they signify \textit{the object}, and the agent  (the interpreter) which/who interprets the sign. These three elements broadly define the scope of the phenomena that fall in the space as carved out by C.S. Pierce.\cite{peirce1992essential} Semiotics is not merely about syntax and semantics of the various media that are part of our lives. Still, it includes us, the agents who interpret, making it the ideal field for cognitive science. In this context, we aim to explain what dynamics a cognitive agent employs to construct a subject and an object from its phenomenology. How do we differentiate ourselves from other objects? What are the embodied dynamics that make this partition possible? And then, among those other objects, some of them begin to stand for some others, the signs. We will also address whether the signs are internal, within the agent's body, or external, outside the body. Thus, the context of generating the elements of semiotics falls within the scope of this monograph, specifically addressing what biological dynamics enable semiotics.\footnote{We prefer to use the term `dynamics' instead of `mechanics'.  This is to underline that the proposed model is not mechanical and can be implemented independently of agents but by involving agents and their actions.}  Further, this separation also creates an inter-subjective space. Therefore, explaining this space also falls in the explanatory scope of this model.  Arrival of the common signs and objects enables transactional actions and communications.  Thus, a spectrum of space starting on one end of an unmediated existence of being-in-the-world to the other end of an entirely mediated space of being-in-the-mediated-world is the cognitive space, from an extremely implicit to extremely explicit way of life. Between these two logical extremes, we situate the actual cognitive space.

This space opens up several problems in cognitive science; therefore, it is not set as an objective to address all of them in this monograph.  But, to address the set objective, the problem of reconciling the major camps in cognitive science, cognitivism, and enactivism, we shall attend to the following problems: symbol-grounding problem\cite{harnad1990symbol}, symbol-ungrounding problem\cite{dove-ungrounding}, and explaining how to generate rules and representations that make-up to generative grammar through agent's actions. Other core and classical problems in cognitive science, such as memory, consciousness, learning, various other mimetic sub-cultures, etc., will be alluded to by pointing to the direction of resolving them.

We attempt to ground the representations (symbols) in the body as action patterns and in the world as traces and outcomes of these action patterns. By grounding representations in an enactive agent, we demonstrate that one can be an enactivist without subscribing to radical antirepresentationalism. Simultaneously, we demonstrate that one can be cognitivist without subscribing to anti-behaviorism.  
Thus, this model attempts to reconcile the two camps in cognitive science through a collection of hypotheses and reasoning based on them. 

Since this argument depends on a dynamic architecture of cognitive agents, we present the structure and dynamics in the next section.

