
\subsection{Multi-Zonal Specialization: Routing Tasks Across Heterogeneous Substrates}
\label{subsec:multizonal}

\marginpar{Zones = different physics, different priors.}
Beyond segmentation, animals exhibit \emph{multi-zonal specialization}: regions (neural and non-neural) with distinct morphologies, materials, timescales, and couplings.
In the SMN, zones constitute \emph{heterogeneous control substrates}---some excitable and fast, others viscoelastic and slow; some tightly gap-junction coupled, others diffusively or mechanically coupled.
Zonation enables \emph{routing}: \OAP{}s recruit different zones as the situation demands, while \TAP{}s emerge when actions enlist zones that write \emph{public traces} (marks, deposits, shared states).

\paragraph{Evo--devo foundations: regionalization across taxa.}
Regional identities arise by conserved patterning programs that parcel the organism into \emph{fields} with distinct control opportunities.
Hindbrain rhombomeres provide a canonical example of serial regionalization with sharp lineage boundaries and specific projections \cite{LumsdenKrumlauf1996Hindbrain}.
Forebrain arealization follows the prosomeric model of nested longitudinal and transverse domains \cite{PuellesRubenstein2003Prosomeric}.
At the cortical level, early patterning sets up area biases that later experience can refine \cite{SurRubenstein2005CortexPatternPlasticity}.
Outside the CNS, arthropod \emph{tagmosis} (head, thorax, abdomen) illustrates regional specialization of repeated elements into functionally distinct zones \cite{Scholtz2010Tagmosis}.

\paragraph{Zonation as control architecture.}
Zones differ in transfer functions and priors: elastic vs rigid, high- vs low-pass, excitable vs passive.
The SMN exploits these differences by \emph{task matching}: predictive smoothing in viscoelastic zones; precise timing in excitable zones; persistent bias in bioelectricly polarized zones \cite{Levin2014MolecularBioelectricity}.
Cerebellar microzones and their olivo-cerebellar loops exemplify fine-grained error-driven control layered on top of slower postural zones \cite{AppsHawkes2009CerebellarZones}; basal ganglia–thalamocortical loops instantiate selection/routing among competing controllers \cite{AlexanderDeLongStrick1986ParallelLoops}.

\paragraph{Habitat coupling.}
Zones are differentially exposed to the medium: distal appendage zones interact with boundary layers and wakes; axial postural zones bear gravitational loading; epithelial exchange zones sense and shape flow/chemistry.
The habitat thus supplies \emph{zone-specific statistics} that the SMN learns to rely on for \OAP{} routing and, when externalized, for \TAP{} formation (e.g., building nests, laying chemical trails).

\paragraph{Thermodynamic economy.}
Specialization reduces internal erasures: rather than rewriting a monolithic controller, the SMN composes \OAP{}s by switching among zones with \emph{pre-tuned priors}, minimizing logically irreversible policy overwrites \cite{Landauer1961Irreversibility,Bennett2003LandauerNotes}.
Where durable public traces are beneficial, \TAP{}s exploit stable substrates (mud, fiber, pheromonal fields), offloading memory to the habitat.

\paragraph{Formalization sketch.}
Let zones be nodes $Z_k$ in a directed graph with edge weights $W_{ij}$ encoding feasible couplings (mechanical, electrical, chemical).
An \OAP{} is a path in this graph with control inputs that gate edges (enable/disable) and adjust local gains.
A \TAP{} is an \OAP{} whose execution writes to an external state $E$ (a shared field or artifact) such that $E_{t+1}\neq E_t$ and is readable by other agents.
\marginpar{\TAP{} = \OAP{} $+$ public write.}
\todo{Figure: (i) zone graph with distinct transfer functions, (ii) \OAP{} path, (iii) \TAP{} writing to an external field, e.g., pheromone or constructed niche.}

\paragraph{SMN vs Habitat (capsule).}
\textbf{SMN:} heterogeneous zones with distinct transfer functions enable task routing as \OAP{}s; \TAP{}s arise when actions write public traces.
\textbf{Habitat:} zone-specific statistics (loads, flows, chemistries) provide priors that reduce costly internal policy rewrites.


\begin{table}[t]
\begin{adjustwidth}{-\notecolumn}{0pt} % borrow the left notes strip
\centering
\setlength{\tabcolsep}{6pt}
\renewcommand{\arraystretch}{1.15}
\begin{tabularx}{\dimexpr\linewidth+\notecolumn\relax}{l c c Y}
\toprule
\textbf{Zone (substrate)} & \textbf{Dominant coupling} & \textbf{Time scale} & \textbf{Prior / role}\\
\midrule
Excitable (neural)              & synaptic / gap junction & ms–s             & precise timing, prediction \\
Viscoelastic (muscle/ECM)       & mechanical              & 10\,ms–10\,s     & damping, posture, smoothing \\
Epithelial / bioelectric        & voltage / gap junction  & s–hours          & set-points, pattern memory \\
Fluid interface (cilia/tubes)   & hydrodynamic            & 10\,ms–s         & carriers, token transport \\
Gravito-postural                & vestibular / proprio    & 10\,ms–s         & leveling, orientation priors \\
\bottomrule
\end{tabularx}
\caption{Zone couplings, time scales, and priors.}
\label{tab:zones}
\end{adjustwidth}
\end{table}

\paragraph{Adaptation.}
Zones calibrate their transfer functions (excitability, stiffness, polarization) to environmental statistics; routing policies then favor pre-tuned zones, minimizing policy overwrites, and promoting \TAP{}s when durable public traces (e.g., constructed niches) pay off.
