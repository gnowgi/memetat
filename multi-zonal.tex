
\subsection{Multi-Zonal Specialization: Routing Tasks Across Heterogeneous Substrates}
\label{subsec:multizonal}

\marginpar{Zones = different physics, different priors.}
Beyond segmentation, animals exhibit \emph{multi-zonal specialization}: regions (neural and non-neural) with distinct morphologies, materials, timescales, and couplings.
In the SMN, zones constitute \emph{heterogeneous control substrates}---some excitable and fast, others viscoelastic and slow; some tightly gap-junction coupled, others diffusively or mechanically coupled.
Zonation enables \emph{routing}: \NAP{}s flexibly recruit different zones as the situation demands, while \TAP{}s arise only when negotiable patterns leave \emph{public traces} (marks, deposits, shared states) that become accessible to other agents.

\paragraph{Evo--devo foundations: regionalization across taxa.}
Regional identities arise by conserved patterning programs that parcel the organism into \emph{fields} with distinct control opportunities.
Hindbrain rhombomeres provide a canonical example of serial regionalization with sharp lineage boundaries and specific projections \cite{LumsdenKrumlauf1996Hindbrain}.
Forebrain arealization follows the prosomeric model of nested longitudinal and transverse domains \cite{PuellesRubenstein2003Prosomeric}.
At the cortical level, early patterning sets up area biases that later experience can refine \cite{SurRubenstein2005CortexPatternPlasticity}.
Outside the CNS, arthropod \emph{tagmosis} (head, thorax, abdomen) illustrates regional specialization of repeated elements into functionally distinct zones \cite{Scholtz2010Tagmosis}.

\paragraph{Zonation as control architecture.}
Zones differ in transfer functions and priors: elastic vs rigid, high- vs low-pass, excitable vs passive.
The SMN exploits these differences by \emph{task matching}: predictive smoothing in viscoelastic zones; precise timing in excitable zones; persistent bias in bioelectricly polarized zones \cite{Levin2014MolecularBioelectricity}.
Cerebellar microzones and their olivo-cerebellar loops exemplify fine-grained error-driven control layered on top of slower postural zones \cite{AppsHawkes2009CerebellarZones}; basal ganglia–thalamocortical loops instantiate selection/routing among competing controllers \cite{AlexanderDeLongStrick1986ParallelLoops}.
\NAP{}s provide the negotiable substrate that allows such heterogeneous zones to be flexibly recombined.

\paragraph{Habitat coupling.}
Zones are differentially exposed to the medium: distal appendage zones interact with boundary layers and wakes; axial postural zones bear gravitational loading; epithelial exchange zones sense and shape flow/chemistry.
The habitat thus supplies \emph{zone-specific statistics} that the SMN learns to rely on for \NAP{} routing.
When such patterns are externalized into stable environmental modifications, they become \TAP{}s---for example, building nests, laying chemical trails, or leaving marks that other agents can read.

\paragraph{Thermodynamic economy.}
Specialization reduces internal erasures: rather than rewriting a monolithic controller, the SMN composes \NAP{}s by switching among zones with \emph{pre-tuned priors}, minimizing logically irreversible policy overwrites \cite{Landauer1961Irreversibility,Bennett2003LandauerNotes}.
Where durable public traces are beneficial, \TAP{}s exploit stable substrates (mud, fiber, pheromonal fields), offloading memory to the habitat and anchoring intersubjective coordination.
\paragraph{Formalization sketch.}
Let $x_i(t)$ be the state of the $i$th segmental controller (e.g., a central pattern generator modeled as a phase oscillator).
Nearest-neighbor coupling $K_{ij}$ yields traveling waves or standing patterns depending on phase-lag targets.
In the simplest case, a lattice of oscillators (Kuramoto-type) produces smooth waves; \emph{interrupts} are transient detunings or phase resets at select nodes.
\begin{align}
\dot{\theta}_i &= \omega_i + \sum_{j \in \mathcal{N}(i)} K_{ij} \sin(\theta_j - \theta_i - \phi_{ij}) + u_i(t),
\end{align}
where $u_i(t)$ captures modulatory inputs from the SMN:
\begin{itemize}
  \item \textbf{FAPs} are stereotyped oscillatory routines (the baseline $\omega_i$ dynamics).
  \item \textbf{HAPs} introduce haltability: $u_i(t)$ imposes temporary pauses or resets.
  \item \textbf{NAPs} implement negotiable coordination: rephasing or reweighting of couplings $K_{ij}$ under perceptual predicates (affordance tests).
  \item \textbf{TAPs} arise only when such negotiable patterns leave external traces (gesture, vocalization, artifact), becoming available for intersubjective transaction.
\end{itemize}
\marginpar{Kuramoto lattice as a \emph{control surface}.}
\todo{Figure: (i) segmented chain with CPGs, (ii) phase-lag targets for swimming/walking, (iii) local phase reset producing a step or turn, (iv) habitat arrows (gravity/fluid) shaping feasible lags.}

\paragraph{SMN vs Habitat (capsule).}
\textbf{SMN:} segmental CPGs coordinate via local phase rules; interrupts compose into \NAP{}s, which provide the substrate for adaptive negotiation.  
\textbf{Habitat:} environmental structure (e.g., gravity, buoyancy, fluid drag) shapes feasible phase-lags, constraining and simplifying coordination.  

\paragraph{Adaptation.}
Segmental CPGs adapt phase and gain through local plasticity while surrounding tissues update bioelectric set-points, stabilizing useful NAPs (e.g., common gaits) and enabling rapid local interrupts without global rewrites.  
Segmentation thus discretizes control into re-usable modules: \emph{FAPs become haltable, HAPs become negotiable, and NAPs can be externalized as TAPs}.  
The SMN leverages these synergies to retime and recombine actions with low wiring cost \citep{BizziCheung2013}.
