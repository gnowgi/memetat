\subsection{Segmentation \& Bilateral Symmetry: Co-evolved Architecture for Comparison, Control, and Coordination}
\label{subsec:segm_bilateral}

\marginpar{Co-evolution: segments $\leftrightarrow$ nerve cord; segments $\leftrightarrow$ bilaterality.}
Metameric \emph{segmentation} and \emph{bilateral symmetry} are not merely adjacent chapters in animal evo–devo; they are tightly co-evolved design moves of Bilateria.\footnote{On definitions and historical usage of “segment,” see \citep{HannibalPatel2013WhatIsASegment}. On arthropod segmentation mechanisms and their evolution, see \citep{ChipmanArias2019ArthropodSegmentation}.} 
Bilateral body layouts provide two laterally arranged sensorimotor arrays that enable \emph{counter-variation} and on-line \emph{comparison} (differences/ratios) across left–right channels—crucial for localization, error signals, and stabilization. Segmentation then multiplies these comparators along the anterior–posterior (A–P) axis into a chain of quasi-repeated modules, yielding a powerful \emph{distributed} controller: many small, locally coupled comparators that can synchronize or decouple as task demands shift.

\paragraph{Decentralized from the start: “CNS” as DNS.}
The textbook label \emph{central} nervous system is historically misleading for early bilaterians. What evolves with metamerism is an \emph{elongate nerve cord} with iterated ganglia and longitudinal tracts—an inherently \emph{decentralized} architecture whose local circuits (reciprocal inhibition, segmental pattern generators) coordinate across segments.\marginpar{DNS $>$ CNS.\quad Iterated ganglia + longitudinal tracts.} 
Comparative molecular mapping of annelid nerve cords reveals conserved mediolateral patterning domains aligned to segments—evidence for an ancient blueprint of \emph{nervous system centralization} that is nonetheless organized segment-wise and distributed along the axis \citep{DenesEtAl2007CellAnnelidCord,Telford2007SingleOriginCNS}. 
Viewed functionally, control is \emph{centralized-by-coupling}, not centralized-by-locus: a DNS whose “center” is the traveling coordination pattern.

\paragraph{Why segmentation fits bilaterality.}
Given a bilateral scaffold, iterating A–P modules (segments) yields: (i) lateral comparators at each station (left–right sensing/actuation), (ii) axial phase-relations for traveling waves (locomotion, peristalsis), and (iii) redundancy for robustness and regeneration.\marginpar{Segments = repeated L–R comparators + axial waveguides.} 
Evo–devo evidence suggests multiple routes to metamerism across lineages while drawing on a shared molecular toolkit; debates on homology vs convergence persist, but the functional synergies with bilateral layouts are clear \citep{HannibalPatel2013WhatIsASegment,ChipmanArias2019ArthropodSegmentation}.

\paragraph{Wnt/FGF/Notch–Hox: conserved A–P patterning and the segmentation clock.}
Posterior growth zones use Wnt/FGF signaling and a Notch-linked \emph{segmentation clock} to periodically specify somites/segments; phase relations between Wnt and Notch oscillations are causal for proper boundary formation \citep{AulehlaPourquie2010SignalingGradients,WahlEtAl2007FGFUpstream,SonnenEtAl2018PhaseShift}. 
\emph{Hox} genes then regionalize the repeated units (neck, thorax, abdomen; cervical–thoracic–lumbar, etc.), turning repetition into \emph{differentiated repetition}—a stratagem that couples modularity with specialization.\marginpar{Clock $\rightarrow$ segments; Hox $\rightarrow$ regional identities.}

\paragraph{Metamerism $\leftrightarrow$ nerve cord co-evolution.}
Segmentally patterned ganglia/neuromeres align with the metameric body plan in many taxa; molecular domain maps in annelids indicate that a mediolateral nervous-system architecture likely predates major bilaterian splits, supporting the view that nerve-cord organization and segmental patterning evolved in step rather than in isolation \citep{DenesEtAl2007CellAnnelidCord}.

\paragraph{Bilateral axes and the chordate–protostome inversion.}
Cross-phyla comparisons of dorsoventral (D–V) patterning propose an inversion of the D–V axis between protostomes (ventral cords) and chordates (dorsal cord), highlighting conserved axial logic beneath surface differences \citep{ArendtNueblerJung1994NatureInversion,Gerhart2000PNASInversion}. 
Either way, what matters for control is the bilateral duplication of sensors/effectors organized by conserved A–P/D–V signals—fertile ground for distributed comparators and DNS-style coordination.

\paragraph{Epistemological note.}
In Piaget’s \emph{genetic epistemology}, organismal structure (e.g., repeated segments, bilateral comparators) and sensorimotor coordination co-construct the very conditions of knowing. \emph{Biology and Knowledge} likewise frames cognition as emerging from, and constrained by, organic regulation—precisely what a metameric–bilateral, distributed nervous system (DNS) provides for stabilizing invariants in a changing world \citep{Piaget1970GeneticEpistemology,Piaget1971BiologyAndKnowledge}. The morphological plans stabilized by natural selection are, at base, selections by the media that act on bodies—internal fluids, external flow fields, and gravity. Given that organisms cannot opt out of this immediate physical surround, the post-\emph{tubularity} layer (segmentation + bilaterality) equips them to counter these forces more effectively. The whole body—not just “sense organs” or a privileged neural locus—participates via the Sensation–Modulating Network (SMN), whose action patterns attune the organism to these constraints. Beyond biochemical self-maintenance through autopoiesis \citep{MaturanaVarela1980}, these patterned actions furnish the raw material for autonomy that becomes explicit at later organizational layers. Crucially, segmentation and bilateral symmetry enlarge the \emph{repertoire} of action-variation beyond what polarity and tubularity alone afford.
