\subsection{The Sensation-Modulating Network: A Framework for Structure and Dynamics of a Conscious Cognitive Agent}

\textbf{Structure} Our central proposal is a dynamic architecture, a body plan, we term the Sensation-Modulating Network (SMN). We model the agent as a layered network of action zones, organized according to fundamental biological principles: polarized, tubular, segmented, bilaterally symmetrical, and antagonistically organized. Within this architecture, cognition emerges not from a central controller initiating action, but from the capacity to \textit{halt} and modulate ongoing, rhythmic action patterns. 

To describe these dynamics, we introduce a specific terminology that will be developed throughout this paper (see Section~\ref{sec:model}): a hierarchy of patterns ranging from deep, immutable \textbf{Fixed Action Patterns (FAPs)}, to consciously accessible \textbf{Haltable Action Patterns (HAPs)}, which, when shared, become \textbf{Transactional Action Patterns (TAPs)}. It is this capacity for modulation, we argue, that grounds semiotics\cite{peirce1992essential}, enables the creation of symbols, and gives rise to generative syntax\cite{chomsky1965aspects}. As we will demonstrate in Section~\ref{sec:phenomena}, this framework provides a unified account of phenomenology, representation, and syntax.

\subsection{Core Proposals of the SMN Framework}
The SMN model is built on a set of rigorously formulated hypotheses that integrate both endogenous and exogenous conditions that make conscious cognition possible. Our core proposals include:

\begin{enumerate}
    \item \textbf{Cognitive world as mediated construct}: The cognitive world is a mediated construct of a geometric semiotic habitat, called \textit{memetat}, constructed through multiple, recursive, recurrent, and haltable serial action patterns, called \textit{memets}.
    
    \item \textbf{SMN as location solver}: A cognitive agent is modeled as a sensation modulating network (SMN), whose function is to compute the location of sensations by serially modulating action patterns called \textit{memets}.
    
    \item \textbf{Perception as differentiation}: Perception of the world is an outcome of placing the sensations relative to each other, using the principle of differentiation of difference (change in action patterns, not action patterns per se).
    
    \item \textbf{Layered antagonistic architecture}: The agent, an SMN, is a multilayered process network of antagonistic coordinated pairs oscillating at relatively high frequencies at the layer beneath---as fixed action patterns (FAP). The lower layers are mandatory and can be modulated with the least degrees of freedom. In contrast, the top layers can afford to halt, without deviating from the general oscillatory pattern---as haltable action patterns. Both FAPs and HAPs are dynamical systems.
    
    \item \textbf{DFN/IN functional distinction}: The bottom layers of the SMN provide a multi-dimensional cognitive stage in the form of an integrating network (IN) provided by the FAPs. In contrast, the top layers form a differentiating and filtering network (DFN) provided by the HAPs.
    
    \item \textbf{Imitability and encoding}: Imitability of action patterns, encoding by copying, enables the creation of Transactional Action Patterns (TAPs).
    
    \item \textbf{Traces and semiotic space}: Traces of action patterns create semiotic space, enabling external representations and shared memory.
    
    \item \textbf{Saturation spectrum}: Saturated and unsaturated HAPs create abstract space, where saturated HAPs are bound to objects while unsaturated HAPs become concepts.
    
    \item \textbf{Haltability mechanism}: The speculation of the mechanism that enables halting involves limited communication routes between coordinated pairs (CPs).
    
    \item \textbf{Meaning and value}: Deeper layers contribute to the meaning and value of actions, connecting to Damasio's work on emotion and cognition.
    
    \item \textbf{Rules and representations}: Rules and representations are grounded in generative syntax through TAPs, providing the foundation for symbolic thought.
    
    \item \textbf{Haltability as syntax condition}: Haltability is a necessary condition for syntax, as gaps in action patterns create the punctuation necessary for combinatorial structure.
    
    \item \textbf{Games and microworlds}: Cultural practices emerge as games and microworlds, constructed through rule-following actions that define transactional playgrounds.
    
    \item \textbf{Mimetics}: The framework supports extended mimetics, enabling the transmission and evolution of cultural practices.
\end{enumerate}

