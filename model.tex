
\section{The Proposed Model: A Dynamic Architecture}
\label{sec:model}

At the heart of our proposal is the \emph{Sensation-Modulating Network} (SMN), 
a framework for understanding the cognitive agent as an integrated whole rather than 
as a brain-centered processor. The SMN addresses central questions of cognitive science 
by specifying how agents construct geometric pictures of their world through 
\emph{action-based differentiation}. It is not conceived as a single organ but as the 
functional organization of the agent, defined by a set of core biological design principles. 
These principles—polarity, tubular topology, segmentation, bilateral symmetry, multi-zonal and antagonistic organization—
are ubiquitous across animal bodies, yet their implications 
for cognition have not been fully explored, though they are well recognized within biology.
\marginpar{The body-plan is dynamically coupled with the habitat and cannot be studied independently to describe or explain cognitive phenomena.}

All living agents generate rhythmic action patterns—oscillatory movements such as beating, 
swimming, walking, and breathing—that anchor them in time. Cognition, we argue, arises from 
the \emph{capacity to interrupt and reconfigure} such rhythmic patterns, and in that process the organism decodes the space by modulating sensations. This interruption 
creates a hierarchy of action patterns: 
\emph{Fixed Action Patterns} (FAPs), \emph{Haltable Action Patterns} (HAPs), and 
\emph{Transactional Action Patterns} (TAPs). 
This hierarchy makes possible both stability and flexibility in behavior, uniting automatized responses with adaptive control. 
\todo{Add references on rhythmic motor control and dynamical systems: Kelso \& Schöner, Buzsáki, and Haken-Keslo-Bunz model.}
\marginpar{Interruptibility as a minimal cognitive act.}

\paragraph{Interruptibility operator.}\label{interrupt-operator}
Let $\Gamma$ be a \FAP{} limit cycle with phase $\theta(t)$.
Define the halt operator $\mathcal{H}_{\alpha}$ which, when applied at time $t_0$, transiently perturbs control parameters so that
\[
\mathcal{H}_{\alpha}[\Gamma](t_0\!:\!t_0{+}\alpha):\quad \dot\theta(t)\approx 0
\]
moving the trajectory to a non-oscillatory set for duration $\alpha$. 
A \HAP{} is $(\Gamma,\mathcal{H}_\alpha)$; an \OAP{} is a finite sequence $\big(\mathcal{H}_{\alpha_k}\big)$ across modules/zones; a \TAP{} is any \HAP{}/\OAP{} that also writes a public trace readable by other agents.

The SMN is realized through the evolved architecture of the body plan: segmented, polarized, 
bilaterally symmetrical, and antagonistically organized. We shall first sketch the body plan in the following sections, followed by their cognitive role. 


\subsection{Evolutionary Biology of Anatomical Disengagement Leading to Cognitive Disengagement}
\label{subsec:evolutionary_disengagement}

The capacity for haltability that underlies cognition has deep evolutionary roots. We can trace the emergence of cognitive capabilities through what we call the \textit{evolutionary biology of anatomical disengagement}—the progressive decoupling of body functions that enabled increasingly sophisticated action patterns.

\subsubsection{From Harder-Actions to Softer-Actions}
Let us call the tightly-coupled actions as \textit{harder-actions} (e.g. the coupling between locomotion and feeding in the earthworms), and the decoupled actions as \textit{softer-actions}. During the course of evolution more anatomical disengagements may have given rise to the availability of more such softer-actions.

These examples indicate how one could speculatively weave a story of the evolution as a story of decoupling the body into multiple zones, where each zone can act partially independent from another, and exhibit a distinguishable action pattern. The development of tongue, lips, jaws, pharynx, larynx, gills, lungs, fins, tails, ears, eyes, limbs, toes, fingers, neck, shoulder, hip and so on are interpreted in this story as anatomical disengagement (or decoupling). One can draw a tree of anatomical disengagement representing the epi-physiological bifurcation over and above the phylogenetic tree of evolution.

For example, in early vertebrates as in cephalopods, the feeding and breathing action patterns (filter feeding habit) are not decoupled. In Gnathostoms, we see the stoma (mouth) differentiated through the evolution of jaws enabling decoupled breathing habits from feeding habits. Episodes of decoupling could be reconstructed for the evolution of simpler buccal cavity differentiated into complex cavity, developing teeth, tongue, lips, pharynx, larynx etc. In parallel, an undulating body (as in lampreys) gets decoupled into localised and bilaterally symmetrical fins, decoupling locomotion as a entire-body function. Similarly, one could consider the decoupling of the undulating alimentary canal from the entire undulation of the body.

These episodes of differentiation might have been naturally selected because of the economic value of decoupling, as localised movement is inexpensive than whole body movement. Each episode of such differentiation is an episode of decoupling leading to the evolution of independent action patterns (habits). Thus the gradual polarisation and bilateral symmetry of the body-plan through evolution leading to differentiation of action zones can be woven into a phylogenetic story of decoupled and localised zones of action patterns (habits).

\subsubsection{Economic Dimensions of Disengagement}
This disengagement has an economic dimension, without which it is difficult to understand how it could have played a role in natural selection. The agent can do more work with less effort (spending less energy) because of disengagement. A body plan of an organism that has a coupled movement for both ingestion and locomotion is expensive, than when they are decoupled. Moving when not eating, or eating when not moving is a new found possibility.

Once we have multiple softer-action zones, it is possible to rest some while the others are active. This is the context for the genesis of \textit{haltability}. Haltable variations, one can speculate, could be sexually/culturally selected. The principle of economy also enabled the organism to perform one action while halting another. Isn't this how we describe modulation? The aspect of control we ascribe to modulation arises only when we hold one variable while modifying another. Can we use this insight to ground the regulatory actions required for cognitive processing in haltable action patterns? We demonstrate how this can be the butterfly effect in cognition.

\subsubsection{The Principle of Layering}
\label{subsec:principle_layering}

Since modulating certain beats such as heartbeats is not affordable, we may consider situating actions over and above the core physiological mechanisms. In order for the actions to be affordable, the interactions of the sustaining layer must continue, and they should generate sufficient surplus. When we say cognition is enactive, it implies that the emancipation from sustaining mechanisms is expensive. Though autopoietic mechanisms may include actions, they are uninterruptible, hence no liberty to introduce gaps here. Hence autopoiesis as a mechanism to compensate the lost energy and matter takes care of the sustaining layer and provides the necessary surplus in the system making actions possible \cite{maturana1991autopoiesis}. This is also an action, but the uninterrupted pace at which this action takes place has no liberty for introducing \textit{gaps} in this layer. In other words, the system can't physiologically afford to halt. However, it is this state that could enable ephemeral actions on the periphery of an autopoietic system whenever and wherever possible. This is made possible by a differentiated body plan that enables a division of labour. Some layers are busy in not only replenishing the loss of energy and matter but also generating surplus energy and matter, such that other layers in the body can \textit{halt}. This design now has room for free action. In this perspective, it is an uninterrupted action of some layer that grants freedom to some other layers. It is this partial break from uninterrupted work, that gives rise to the freedom to enter into the cognitive domain. It is in this subtle sense, that our model differs from Maturana and Varela's account of the connections between biology and cognition. The subtlety we introduce is haltability.

This differentiation of layers, as against the uniform distribution of work, in a system facilitating deviation from the normal course of actions, gave rise to the roots of cognitive state. We shall call this \textit{the principle of layering}, which is over and above the design principles of polarization and asymmetry we discussed earlier. The sense of being over and above can be characterised by naming metaphorically, this principle as epi-physiological or epi-biological.

\subsubsection{Cognitive Space as Memetat: The Geometric Semiotic Habitat}
\label{subsec:memetat}

Building on the principle of layering, we now introduce the concept of \textit{memetat}—the cognitive space that emerges from the SMN architecture. Cognitive space is a transient geometrical space, \textit{memetat}, constructed through multiple, recursive, recurrent, and haltable serial action patterns, called \textit{memets}, which can be retained by reenacting and creating interpretable traces. The agent's body is modeled as a layered sensation-modulating network (SMN), which is functionally distinguished into a differentiating and filtering network (DFN) and an integrating network (IN).

The SMN is modeled as a polarized and bilaterally symmetrical PetriNet (PN), a bipartite graph of transitions and places. A set of place nodes of the PN enter SMN from the environment where the cognitive agent is situated via the interfacing nodes of transitions. The transition nodes of the PN are sensory transducers and actuators, constituting the sensory-motor part of the architecture. The resulting tokens from the sensory-motor part of the network form as the other set of the place nodes of the PN in the form of a nested stack of neural connections holding transient tokens for a while. The tokens pass through one stack of network into another until they vanish completely. The phenomenological space is constructed by the tokens in this transient nested stack of networks through multiple, recursive, recurrent and haltable serial action patterns by the sensation-modulating network.

The geometry is computed by calculating (1) the differentiation of differences in the phenomena through self-modulation, (2) the relative location of the phenomena following the principle of \textit{mapping the delay with distance}, and (3) the principle of the concomitance of phenomena granted by a dynamic \textit{fire-together-wire-together} architecture of the SMN. The spectrum of perception to conception (spectrum of abstraction) is explained through a model of a spectrum of saturation of multiple sensory modalities and the corresponding dynamic spectrum of engagement and disengagement of action patterns. Meaning and evaluation are grounded in the deeper layers of the layered architecture of the SMN. Transient action patterns can be memorized and recalled only by re-enacting them, while the traces of action patterns can be persistent and hence can be recalled and gamified. Human culture is modeled as a confluence of multiple microworlds, where each microworld is constructed by mutually stimulating rules following actions that define the boundaries of the transactional playground.


\subsubsection{Habitat as Co-Architect: Fluid and Gravity}
\label{subsec:habitat}
The SMN co-evolves with its habitat. Two ambient fields shape its design:
\begin{enumerate}
    \item \textbf{Aquatic fluid}: viscosity and buoyancy enable efficient oscillatory control, 
    stabilizing limit-cycle behaviors useful for locomotion and sensing.
    \item \textbf{Gravity}: a constant vector field that grounds posture, balance, and orientation.
\end{enumerate}
These fields provide \emph{external computation}: the environment carries lawful structure that the agent exploits.
\marginpar{Distributed computation in the habitat complements decentralized control in the SMN.}
\todo{Cite Gibson (1979), O'Regan \& Noë (2001) on sensorimotor contingencies; cite work on gravity as a strong prior.}

\subsubsection{Thermodynamic Economy: Action Over Storage}
Cognition at biological ``room temperature'' is possible because organisms offload computation to stable environmental regularities. 
Rather than encoding exhaustive state spaces, agents store and refine \emph{action repertoires} that couple to habitat affordances. 
This design is energetically parsimonious: it minimizes memory maintenance and leverages the free structure present in fluid and gravity.
\marginpar{Action patterns as energy-efficient ``indices'' into the world.}
\todo{Add citations: Friston (Free Energy Principle), embodied thermodynamics; historical precursors (e.g., Moravec).}

\subsubsection{Formalizing FAP--HAP--TAP}
Let $x(t)$ denote a low-dimensional controller capturing a segment's antagonistic pair. 
\textbf{FAPs} correspond to limit cycles $\Gamma$ in the phase portrait (stable rhythmic behaviors). 
\textbf{HAPs} introduce control parameters that create bifurcation surfaces $\mathcal{B}$ allowing halts and reversals. 
\textbf{TAPs} are sequences of controlled entries/exits from neighborhoods of $\Gamma$ across segments, 
coordinated by inter-segment couplings and habitat feedback. 
\todo{Insert a figure: nested cycles (FAP) with interrupt surfaces (HAP) and transactional arcs (TAP); add references to attractors/bifurcations.}






\subsection{The Nature of Action: A Thermodynamic Distinction}
\label{subsec:action_nature}

Before detailing the architecture of the Sensation-Modulating Network (SMN), we must make a foundational ontological distinction between \textit{action} and \textit{interaction}. While all actions are a form of interaction, not all interactions qualify as actions. This difference has often been blurred in cognitive science, but it is central to an enactive model of cognition.

\paragraph{Interaction as Symmetry.}  
In the physical sciences, \textbf{interaction} refers to conserved processes governed by the symmetrical laws of physics. Two bodies that collide, two charges that repel, or two waves that interfere are all instances of interaction: they unfold according to conservation principles of momentum, charge, and energy. Importantly, no agent needs to initiate these processes; they are lawful consequences of system dynamics. In this sense, interaction is reciprocal and symmetrical: if body A exerts a force on body B, body B exerts an equal and opposite force on body A. To distinguish them we may refer to them as \textit{p-interactions}.

\paragraph{Action as Asymmetry.}  
By contrast, \textbf{action} is a thermodynamic process in which a far-from-equilibrium system actively breaks symmetry by expending energy to perturb its environment \cite{prigogine2018order}. An organism or agent acts when it recruits metabolic resources to move, signal, or otherwise produce a directed change. Unlike passive interaction, action is not merely the unfolding of conservation laws; it involves initiating, sustaining, and regulating asymmetries. To act is to resist being carried along by the default flow of physical interactions and instead to impose new gradients, forces, or patterns. This asymmetry is precisely what individuates agents as distinct from the surrounding physical world. In this work we will use the term ``action'' to this sense of the term, and further argue that cognitive actions fall in this category.

\paragraph{Agency and Normativity.}  
Philosophical treatments of action highlight further dimensions of asymmetry. Actions are individuated by their relation to an agent, their embedding in temporal structures, and their susceptibility to normative assessment \cite{barandiaran2009defining}. In acting, the agent can succeed or fail, be correct or incorrect, act responsibly or irresponsibly. Interactions in the physical sense lack this evaluative dimension: we do not describe  or judge the gravitational pull of the Earth on the Moon as mistaken or appropriate. Action thus introduces normativity and intentional orientation into what otherwise remain symmetrical exchanges.

\paragraph{Social Transactions as Actions.}  
At the level of intersubjective space, social \textbf{interactions} are better understood as coordinated networks of actions. A conversation, a gesture, or a shared task involves each agent producing asymmetrical perturbations that are reciprocally taken up by others. These are not interactions in the purely physical sense but transactions mediated by meaning, normativity, and temporality. Social interaction therefore presupposes action: it is the transactional layering of many local, energy-expending, symmetry-breaking events enacted by individuals. The SMN model treats this layering explicitly, distinguishing between (i) p-interactions: the physical interaction domain where symmetry prevails, and (ii) s-interactions: the agentive/social domain where asymmetries accumulate into transactions.

\paragraph{Cognition as Energy-Dependent Asymmetry Management.}  
From this perspective, cognition is not the passive registration of physical interactions but the active management of actions. The agent must marshal resources, regulate asymmetries, and coordinate transactions at multiple levels—from muscular contractions to linguistic exchanges. By distinguishing action from interaction, we frame the SMN as a model of how energy-dependent asymmetry at the bodily level scaffolds intersubjective asymmetry at the social level.

% --- put in your preamble if not already there ---
% \usepackage{tikz}
% \usetikzlibrary{arrows.meta, positioning, fit, backgrounds}

% --- adjustable width to accommodate custom margins ---
% \newcommand{\layerwidth}{0.8\linewidth} % change to 1.05\linewidth if you want to extend into a left margin

\begin{figure}[t]
\centering
\begin{tikzpicture}[
  font=\small,
  >=Latex,
  box/.style={
    draw=black, rounded corners=6pt, line width=0.6pt,
    inner sep=6pt, fill=gray!4, align=left, text width=\layerwidth
  },
  title/.style={font=\bfseries},
  bullet/.style={inner sep=0pt, outer sep=0pt},
  arrowup/.style={-{Latex[length=3.5mm]}, line width=0.6pt},
  arrowdown/.style={-{Latex[length=3.5mm]}, line width=0.6pt, dashed},
]

% --- Layers (bottom to top) ---
\node[box] (phys) {%
  \parbox{\layerwidth}{%
    \textbf{Physical Interaction}\\[-0.3em]
    \begin{itemize}[left=1.2em]
      \item symmetric laws; reciprocity (e.g., $F_{AB}=-F_{BA}$)
      \item conservation (energy, momentum, charge)
      \item no agent required; lawful dynamics
    \end{itemize}
  }%
};

\node[box, above=10mm of phys] (action) {%
  \parbox{\layerwidth}{%
    \textbf{Action (Agentive / Biological)}\\[-0.3em]
    \begin{itemize}[left=1.2em]
      \item local symmetry-breaking; imposed gradients
      \item energy expenditure; far-from-equilibrium
      \item agency \& normativity (success/failure)
    \end{itemize}
  }%
};

\node[box, above=10mm of action] (social) {%
  \parbox{\layerwidth}{%
    \textbf{Social Transactions (Intersubjective)}\\[-0.3em]
    \begin{itemize}[left=1.2em]
      \item coordination of actions; shared norms \& meanings
      \item dialog, joint attention, conventions
      \item evaluable in social/ethical space
    \end{itemize}
  }%
};

% --- Upward emergence arrows ---
\draw[arrowup] (phys.north) -- node[right=2pt, align=left] {emerges from /\\ grounded in} (action.south);
\draw[arrowup] (action.north) -- node[right=2pt, align=left] {emerges from /\\ coordinated by} (social.south);

% --- Downward constraint arrows (dashed) ---
\draw[arrowdown] (social.south east)++(-8mm,0) -- ++(0,-8mm)
  node[midway, right=2pt, align=left] {normative\\ constraints};
\draw[arrowdown] ([xshift=8mm]action.south east) -- ++(0,-8mm)
  node[midway, right=2pt, align=left] {task/goal\\ constraints};

% --- Background emphasis boxes (optional) ---
\begin{scope}[on background layer]
  \node[fit=(phys),   fill=gray!2, draw=gray!40, rounded corners=6pt, inner sep=10pt] {};
  \node[fit=(action), fill=gray!2, draw=gray!40, rounded corners=6pt, inner sep=10pt] {};
  \node[fit=(social), fill=gray!2, draw=gray!40, rounded corners=6pt, inner sep=10pt] {};
\end{scope}

% --- Caption label inside figure (optional) ---
\node[below=6mm of phys, align=center] (cap) {%
  \emph{Layered ontology:} actions are energy-expending, symmetry-breaking\\
  perturbations enacted by agents; social transactions are coordinated networks of actions.%
};

\end{tikzpicture}

\caption{Vertical layered representation distinguishing symmetric \emph{physical interactions}, agentive \emph{actions} (asymmetry, energy use, normativity), and \emph{social transactions} as coordinated networks of actions with intersubjective norms. Upward arrows denote emergence/grounding; dashed downward arrows indicate top-down constraints.}
\label{fig:action-interaction-layers}
\end{figure}




\subsubsection{A Principle of Inertia of Cognitive Phenomena}
\textbf{Action Patterns: }The unit of analysis for cognition is \textit{action patterns}, and not actions.  This requires us to shift our attention from actions \textit{per se} to \textit{change in actions}. Actions are identified or distinguished on the basis of a change in actions or simply action patterns. When we refer to the pattern, we are referring to the pattern of change, a temporal feature of action and not a static structure. 

We could then define an idealized inertial state as a model system:  \textit{A cognitive agent remains in its state of pre-existing action-patterns, until halted by either internal or external affordances.} 

This postulate is justified by the insight that \textit{differentiation of difference} is a valid condition for cognitive awareness.\cite{bateson2000steps} Deviation from a pattern has more information than an invariant pattern. The term `fixed-action-patterns' (FAPs) originated in ethology to describe several overt but innate (inborn) behavioral patterns. We use the term to describe not only the overt behavioral action patterns, but also action patterns inside the body: heart-beat, ingesting, swallowing, peristalsis, along with the overt patterns such as walking, running, scratching, pruning, digging, swimming etc.  Considering that these are genetic, they become the \textit{action schema} available as potential \textit{conceptual schema} for grasping the world around them.  Here, we follow the path adopted by Jean Piaget's neo-Kantian genetic epistemology.\cite{piaget-biology-knowledge}  We will use these principles and ground mental representations in ``muscular activity patterns'' within the SMN\cite{land-schack-2013frontiers}.

\subsubsection{Actions Precede Coordination of Actions}

We make another grounded assumption that movement and action appear in living organisms earlier than their regulatory mechanisms, based on evolutionary history \cite{Levin2014}.  Organisms without neurons exhibit movement. Therefore, we use this as an important premise in our argument that to initiate action, centralized or distributed controlling sub-systems are not required.  

We see a pattern in the evolutionary history \cite{Levin2014}.  Organisms that arrived early in evolutionary history show more uninterrupted action patterns than those that arrived late. The greater coordination of actions we see in the recent organisms exhibits a greater variety of interrupted action patterns. This indicates an insightful connection between coordinated action and haltability of action. Single-celled organisms are more dynamic but exhibit fewer action patterns. Multicellular organization introduces constraints in uninterrupted movement but enables new \textit{syntax} in the possible action patterns by introducing \textit{gaps}.  


\subsection{Polarity as the First Cognitive Coordinate System}
\label{subsec:polarity}
Movement demands an anatomical asymmetry, even in single celled organisms, though the polarity is not as conspicuous as in cephalization we see among bilaterals.  Before the advent of true bilateral symmetry, concentration of action or sensory organization at an “anterior” end was not well established in protozoa or the earliest multicellular organisms. Instead, most protozoa and early animals either exhibit no clear anterior-posterior axis or only display weak, functionally driven polarization.   Flagellates and ciliates, display a form of polarity related to their mode of movement and feeding. Among early metazoans like sponges and cnidarians generally lack clear anterior-posterior asymmetry. Their bodies are often radially symmetrical, and nerve nets (when present) are diffuse rather than concentrated at a single end.\cite{MANUEL2009polarity} An amoeaba which defies any form of symmetry is known to show some dorso-vental differentiation in the cytoskeleton\cite{Taniguchi2023AmoebaMembrane}.  The hghly conserved Hox genes in bilaterians and Wnt signaling pathway in determining the axis amd symmetry in morphogenesis of animals\cite{DiMaio2015WntSymmetry} indicates a deeper `toolbox' to tackle spacial coordiante system for navigation.  The timing of Wnt and Hox gene expression during embryogenesis shows a distinct yet interrelated sequence where Wnt signaling generally precedes and initiates Hox gene activation.  

It may appear extremely distant connection between biology and cognition, but the differentiation of the body axis into anterior and posterior, specilization of cephalization would play an important role in the \SMND, specifically in the archestration of multizonal network of action patterns in our model. 

\marginpar{Polarity furnishes a \emph{coordinate frame} for control.}
We take \emph{polarity} to mean oriented asymmetries at multiple scales: cellular (apico--basal and planar cell polarity), tissue/organ polarity, and whole-body axes (anterior--posterior, dorsal--ventral, and left--right).
On the evo--devo path, polarity is among the earliest and most conserved specifications of the body plan, providing the geometric scaffolding upon which segmentation, antagonistic actuation, and bilateral comparison later operate.
In the SMN, polarity is not merely anatomical: it is the \emph{computational resource} that gives actions a direction for prediction, comparison, and interruption.

\paragraph{Developmental grounding.}
Classically, polarity is formalized through \emph{positional information}: graded cues and boundary conditions that endow cells with interpretable coordinates \cite{Wolpert1969PositionalInformation}.
Axial patterning (e.g., anterior--posterior via Hox codes) further refines these coordinates into modular programs \cite{McGinnisKrumlauf1992Hox}.
Dorsal--ventral polarity is implemented by conserved BMP/Chordin antagonism across bilaterians \cite{DeRobertisSasai1996CommonPlan}, while \emph{planar cell polarity} (PCP) aligns subcellular structures across epithelia to produce coherent tissue-scale orientation \cite{GoodrichStrutt2011PCP}.
These mechanisms show that polarity is multi-level, redundant, and robust---properties that the SMN exploits for resilient control.

\paragraph{Bioelectric implementations and pattern memory.}
Endogenous bioelectric networks provide a \emph{physiological encoding} of polarity and target morphology.
Voltage gradients and gap-junction networks can bias axis specification and organ identity, and---crucially---they support \emph{multi-stable pattern memories} that persist across morphological change \cite{Levin2012MorphogeneticFields,Levin2014MolecularBioelectricity}.
This non-neural, distributed substrate directly grounds the SMN's claim that control states need not be stored as data in a central location; they can be \emph{maintained as attractors} in tissue-level dynamics.

\paragraph{From polarity to control.}
For the SMN, polarity anchors three control advantages:
\begin{enumerate}
    \item \textbf{Predictive orientation:} An oriented axis makes future sensory flow and mechanical contingencies more predictable along canonical directions (forward/back, dorsal/ventral).
    \item \textbf{Antagonistic haltability:} Opposed effectors aligned to an axis (flexor/extensor; ad-/abductor) implement \emph{interruptibility} by design; halting is a switch along the same axis.
    \item \textbf{Comparative inference:} Left--right polarity combined with bilateral symmetry enables \emph{counter-variation} and differencing---the computational basis of comparison and error signals.
\end{enumerate}
%\marginpar{Interruptibility is \emph{easier} when opposition is built-in.}

\paragraph{Coupling to habitat: gravity and fluid.}
Polarity is co-constructed with the medium.
Gravity supplies a constant vector field that organisms internalize via vestibular and proprioceptive systems \cite{AngelakiCullen2008VestibularMultisensory}, turning dorsal--ventral and head--tail distinctions into control priors for posture, balance, and locomotion.
In aquatic habitats, viscosity and buoyancy make oscillatory control stable and efficient; oriented bodies leverage wake interactions and boundary layers to \emph{predict} flow and economize effort \cite{Vogel1994LifeInMovingFluids,Alexander2003PrinciplesLocomotion}.
Thus, the habitat directly participates in `computation' by reducing uncertainty along polarized axes.\marginpar{Action patterns index lawful structure in the world.}


%\paragraph{Thermodynamic economy.}
%Because polarity supplies reliable directions and comparisons, the SMN can store \emph{action patterns} rather than data.
%Under Landauer's bound, logically irreversible updates (policy resets, overwrites) dissipate heat; by coupling to stable axes (gravity) and medium regularities (fluid flow), organisms \emph{lower the frequency of costly resets} while retaining flexible control \cite{Landauer1961Irreversibility,Bennett2003LandauerNotes,StillEtAl2012ThermoPrediction}.

\todo{Figure suggestion: a tri-axial schematic (A--P, D--V, L--R) overlaid with (i) antagonistic pairs, (ii) vestibular gravity vector, (iii) fluid streamlines indicating predictable sensorimotor contingencies. Label links to \cref{subsec:habitat} and \cref{subsec:bio-arch}.}

\paragraph{Cognitive function of polarity (tokens and vectors).}
Polarity converts undifferentiated space into a usable \emph{code}: oriented axes act as \emph{vectors} for prediction and control, while antagonistic actuation and gated halts yield discrete \emph{tokens}—countable, sequenceable units of action. In this way, polarity provides the SMN with the minimal representational kit for spatial decoding: vectors to project, tokens to compose. Subsequent sections (tubes, segmentation, bilateral comparison) elaborate how this kit scales into navigable, interruptible programs in complex habitats.


\paragraph{SMN vs Habitat}
\textbf{SMN:} polarity orients local controllers and antagonistic pairs, enabling interruptible actions along canonical axes.
\textbf{Habitat:} gravity provides a constant vector; fluids provide directionally predictable flows—together supplying priors/vectors that the SMN exploits to decode space with minimal internal storage.
\todo{redraw them later, for the time being this is a place holder picture. add cross references to figures as they keep floating around in the document.}

\begin{figure}[h]
\centering
\begin{tikzpicture}[
  x=1cm,y=1cm, line join=round, line cap=round, thick,
  >=Latex, % for nicer arrows
  every node/.style={font=\small}
]

% Layout parameters
\def\X{0}            % center x for all stages
\def\gap{3.2}        % vertical spacing between stages
\def\W{1.6}          % cylinder radius in x (ellipse a)
\def\H{0.45}         % cylinder radius in y (ellipse b)
\def\wall{0.18}      % wall thickness
\def\segN{6}         % number of segments for stages 3–5
\def\leglen{0.9}     % appendage length (stage 5)

%%%%%%%%%%%%%%%%%%%%%%%%%%%%
% Stage 1: Polarized body (upright ellipse; anterior/posterior)
%%%%%%%%%%%%%%%%%%%%%%%%%%%%
\coordinate (S1) at (\X,0);

% body outline
\draw (S1) ellipse [x radius=0.7, y radius=1.3];

% anterior cap (visually distinct tip)
\draw[fill=black!8] (\X,1.3) ellipse [x radius=0.35, y radius=0.25];

% polarity markers
\draw[->] (\X+1.2,1.0) -- ++(0,0.5) node[above] {anterior};
\draw[->] (\X+1.2,-1.0) -- ++(0,-0.5) node[below] {posterior};

\node[anchor=west] at (\X+2.2,0) {Polarized body (anterior--posterior)};

%%%%%%%%%%%%%%%%%%%%%%%%%%%%
% Arrow to next
%%%%%%%%%%%%%%%%%%%%%%%%%%%%
\draw[->, very thick] (\X, -1.8) -- ++(0,-0.8);

%%%%%%%%%%%%%%%%%%%%%%%%%%%%
% Helper macro to draw a hollow cylinder at center C=(x,y) with width W, height H, wall thickness T
% Includes front inner/outer rims (solid), back inner/outer rims (dashed), and side walls
%%%%%%%%%%%%%%%%%%%%%%%%%%%%
\newcommand{\HollowCylinder}[4]{% Cx,Cy,W,H
  \pgfmathsetmacro{\Cx}{#1}
  \pgfmathsetmacro{\Cy}{#2}
  \pgfmathsetmacro{\a}{#3}
  \pgfmathsetmacro{\b}{#4}
  \pgfmathsetmacro{\ai}{\a-\wall}
  \pgfmathsetmacro{\bi}{\b-\wall}
  % back rims (dashed)
  \draw[dashed] (\Cx,\Cy) ellipse [x radius=\a, y radius=\b];
  \draw[dashed] (\Cx,\Cy) ellipse [x radius=\ai, y radius=\bi];
  % side walls
  \draw (\Cx-\a,\Cy) -- (\Cx-\ai,\Cy);
  \draw (\Cx+\a,\Cy) -- (\Cx+\ai,\Cy);
  % front rims (solid)
  \draw (\Cx,\Cy) ellipse [x radius=\a, y radius=\b];
  \draw (\Cx,\Cy) ellipse [x radius=\ai, y radius=\bi];
}

%%%%%%%%%%%%%%%%%%%%%%%%%%%%
% Stage 2: Hollow cylinder (tube)
%%%%%%%%%%%%%%%%%%%%%%%%%%%%
\coordinate (S2) at (\X,-\gap);
\HollowCylinder{\X}{-\gap}{\W}{\H}
\node[anchor=west] at (\X+2.2,-\gap) {Tubular (hollow cylinder)};

% Arrow to next
\draw[->, very thick] (\X, -\gap-1.0) -- ++(0,-0.8);

%%%%%%%%%%%%%%%%%%%%%%%%%%%%
% Helper macro: segmented hollow cylinder
% draws vertical segment lines along body, leaving lumen continuous
%%%%%%%%%%%%%%%%%%%%%%%%%%%%
\newcommand{\SegmentedHollowCylinder}[5]{% Cx,Cy,W,H,segments
  \pgfmathsetmacro{\Cx}{#1}
  \pgfmathsetmacro{\Cy}{#2}
  \pgfmathsetmacro{\a}{#3}
  \pgfmathsetmacro{\b}{#4}
  \pgfmathsetmacro{\n}{#5}
  % base hollow cylinder
  \HollowCylinder{\Cx}{\Cy}{\a}{\b}
  % segment separators (outer wall only)
  % place \n-1 separators between -a and +a
  \pgfmathsetmacro{\step}{(2*\a)/\n}
  \foreach \k in {1,...,\numexpr\segN-1\relax}{
    \pgfmathsetmacro{\xx}{\Cx - \a + \k*\step}
    % draw short verticals for outer wall separation (avoid crossing the ellipse caps)
    \draw (\xx,\Cy+\b) -- (\xx,\Cy+\b+0.25);
    \draw (\xx,\Cy-\b) -- (\xx,\Cy-\b-0.25);
    % a subtle cue across the wall thickness
    \draw (\xx,\Cy+\b) -- (\xx,\Cy+\b-0.25);
    \draw (\xx,\Cy-\b) -- (\xx,\Cy-\b+0.25);
  }
}

%%%%%%%%%%%%%%%%%%%%%%%%%%%%
% Stage 3: Segmented hollow cylinder (contiguous segments)
%%%%%%%%%%%%%%%%%%%%%%%%%%%%
\coordinate (S3) at (\X,-2*\gap);
\SegmentedHollowCylinder{\X}{-2*\gap}{\W}{\H}{\segN}
\node[anchor=west] at (\X+2.2,-2*\gap) {Segmented hollow cylinder (contiguous lumen)};

% Arrow to next
\draw[->, very thick] (\X, -2*\gap-1.0) -- ++(0,-0.8);

%%%%%%%%%%%%%%%%%%%%%%%%%%%%
% Stage 4: Bilaterally symmetrical hollow segmented cylinder
% add a vertical midline and mirrored lateral wall accents per segment
%%%%%%%%%%%%%%%%%%%%%%%%%%%%
\coordinate (S4) at (\X,-3*\gap);
\SegmentedHollowCylinder{\X}{-3*\gap}{\W}{\H}{\segN}

% midline symmetry cue
\draw[dash dot] (\X,-3*\gap-1.0) -- (\X,-3*\gap+1.0);

% mirrored wall accents (tiny symmetric ticks per segment)
\pgfmathsetmacro{\stepx}{(2*\W)/\segN}
\foreach \k in {0,...,\numexpr\segN\relax}{
  \pgfmathsetmacro{\xx}{\X - \W + \k*\stepx}
  % left & right tiny ticks near the wall (symmetric)
  \draw (\xx,-3*\gap+\H+0.15) -- ++(-0.18,0);
  \draw (\xx,-3*\gap+\H+0.15) -- ++(0.18,0);
  \draw (\xx,-3*\gap-\H-0.15) -- ++(-0.18,0);
  \draw (\xx,-3*\gap-\H-0.15) -- ++(0.18,0);
}

\node[anchor=west,align=left] at (\X+2.2,-3*\gap)
  {Bilaterally symmetrical\\hollow segmented cylinder};

% Arrow to next
\draw[->, very thick] (\X, -3*\gap-1.0) -- ++(0,-0.8);

%%%%%%%%%%%%%%%%%%%%%%%%%%%%
% Stage 5: Same as 4 + bilaterally symmetrical appendages
%%%%%%%%%%%%%%%%%%%%%%%%%%%%
\coordinate (S5) at (\X,-4*\gap);
\SegmentedHollowCylinder{\X}{-4*\gap}{\W}{\H}{\segN}

% midline symmetry cue
\draw[dash dot] (\X,-4*\gap-1.0) -- (\X,-4*\gap+1.0);

% appendages: paired legs from the outer wall, symmetric L/R at select segments
% choose every other segment index for neatness
\pgfmathsetmacro{\stepx}{(2*\W)/\segN}
\foreach \k in {1,3,5}{
  \pgfmathsetmacro{\xx}{\X - \W + \k*\stepx}
  % left legs (upper and lower)
  \draw (\xx,-4*\gap+\H) -- ++(-\leglen, 0.45);
  \draw (\xx,-4*\gap-\H) -- ++(-\leglen,-0.45);
  % right legs (upper and lower)
  \draw (\xx,-4*\gap+\H) -- ++(\leglen, 0.45);
  \draw (\xx,-4*\gap-\H) -- ++(\leglen,-0.45);
}

\node[anchor=west,align=left] at (\X+2.2,-4*\gap)
  {Bilaterally symmetrical\\appendages on segmented body}; 
  
\end{tikzpicture}
\caption{Schematic, vertically aligned progression: (1) polarized body, (2) hollow cylinder, (3) segmented hollow cylinder, (4) bilaterally symmetrical hollow segmented cylinder, (5) same with bilaterally symmetrical appendages.}
\end{figure}



\subsection{Tubularity and the Genesis of Haltable Action Patterns}
\label{subsec:tubular}

\paragraph{From aquatic beginnings to tubes that manage flow}
Life’s first challenges were aquatic. In low-Reynolds aquatic worlds,\marginpar{\footnotesize Low–Re (Re$\ll$1) aquatic world: viscous forces dominate inertia, so reciprocal strokes yield no net transport (Purcell’s scallop theorem), and one-way flow demands nonreciprocal ciliary metachrony or peristaltic waves with valves/sphincters to block backflow \citep{Purcell1977LowRe,Shapiro1969Peristalsis,Vogel1994LifeMovingFluids}. Expanded version in Appendix \ref{reynolds-number}} unicellular eukaryotes solved feeding by \emph{making} flow: beating cilia generate and steer local currents to trap particles and microbes, with coordination handled by planar polarity and basal-body geometry \citep{Brooks2014Multiciliated,Shekhar2023CooperativeHydrodynamics}. Sessile poriferans scaled this idea up into an \emph{aquiferous system}: simple to complex branched tubes where choanocyte pumps create a persistent, largely unidirectional stream from ostia to osculum; critically, sponges also \emph{shut} this flow through whole-body contractions—even without nerves or muscles—to gate filtration when conditions demand it \citep{Goldstein2020SpongeContractions,Morganti2019SpongePumping}. Cnidarians elaborated a gastrovascular sac that doubles ingestion and egestion through a single orifice, sometimes housing endosymbionts in a protected internal micro-habitat. Later, bilaterians evolved a through-gut with separate mouth and anus—an innovation repeatedly analyzed in comparative embryology and evo-devo as a key step toward efficient, directional processing \citep{Hejnol2015AnalEvolution,Nielsen2018MouthAnus,Presnell2016ThroughGut}. In many lineages, tubular logic spread internally: peristaltic vessels, valves, and eventually hearts moved fluids at scale \citep{MonahanEarley2013BVS}. Neural tubes, by contrast, appear with chordates and serve to coordinate bilateral action systems rather than to transport fluid \citep{Holland2015ChordateNervous}. 

\paragraph{The one-way-flow principle and the cost of letting things run}
Across these body plans, a single control principle recurs: keep bulk flow \emph{one-way}. Biophysical solutions include ciliary metachrony, peristaltic waves, check-valves and sphincters, and pulsatile pumps. Recent work in echinoderm larvae shows explicit \emph{gating} at the pylorus and anus to prevent simultaneous entry/exit—an elegant neural solution to enforce unidirectionality in a simple through-gut \citep{Yaguchi2024Sphincter}. But one-way flow is not always optimal to maintain continuously: pumps cost energy, and stereotyped run-to-completion routines can be maladaptive when the world changes. The evolutionary response we emphasize is the emergence of mechanisms that \emph{can stop or pause} ongoing flows. 

\paragraph{From FAPs to HAPs: making stereotyped patterns stoppable}
Classical ethology described many stereotyped, energetically committed routines as \emph{fixed action patterns} (FAPs) \citep{Ronacher2019FAP}. We propose the complementary notion of \emph{haltable action patterns} (HAPs): peripheral motor programs that remain \emph{competent} to run (often via local myogenic/enteric patterning), but are equipped with neural (and sometimes humoral) control layers that can \emph{gate, suspend, or terminate} the pattern quickly when context flips. The gastrointestinal tract is a template case: peristalsis is generated and modulated by enteric circuits and interstitial cells, but is routinely gated by sphincters and by central inputs \citep{Sharkey2022ENS}. In worms and mammals, swallowing launches a peristaltic wave, yet breathing is temporarily suppressed (\emph{deglutition apnea}); recent work identifies a brainstem “postinspiratory complex” (PiCo) that interfaces the swallow CPG with the respiratory Central Pattern Generators (CPG)\marginpar{\footnotesize A central pattern generator (CPG) is a neural circuit producing rhythmic motor output without rhythmic sensory input; in vivo it is gated by descending commands and feedback so the rhythm can start, pause, or stop—an architectural substrate for HAPs \citep{MarderCalabrese1996PR,Kiehn2016NRN,DelNegro2018BreathingMatters}. See more detailsin \ref{CPG}}  to orchestrate this on-the-fly halting and sequencing \citep{Matsuo2009Coordination,Barlow2009CPGOralResp,Moore2014BrainstemOrofacial,Huff2023PiCo}. 

\paragraph{Early HAP exemplars: buccal capture, swallowing, breathing}
Some of the earliest cross-taxa HAPs are in the buccal–pharyngeal complex: \emph{selectively} capturing a bit of the world, holding it, and then either admitting or ejecting it. In nematodes, the pharynx is a tubular neuromuscular pump with intrinsic rhythmicity and a compact “enteric” circuit; pumping rates and particle transport are rapidly up- or down-modulated by neuromodulators and sensory state—i.e., the pattern is powerful but \emph{haltable} \citep{Avery2012CelegansFeeding,Trojanowski2016PharyngealPumping}. In vertebrates, orofacial CPGs (lick, chew, swallow, breathe, vocalize) are coupled but separable; sensory and cortical inputs can gate transitions and suspend one rhythm in favor of another (e.g., pause breathing to swallow; abort swallow to cough) \citep{Moore2014BrainstemOrofacial,Matsuo2009Coordination,Huff2023PiCo}. These mechanisms make selection possible at the entry point: to \emph{withhold}, \emph{sort}, and \emph{sample}—a clear epistemic move from “let it pass” to “hold and judge.” 

\paragraph{Cognitive moral: selectivity rides on stoppability}
The tubular architecture first solved \emph{transport}, but its success hinged on \emph{control of stoppage}: sphincters as gates, valves as decision points, and CPGs under inhibitory and neuromodulatory oversight. The move from FAP-like “always on” pumping to HAP-like “as needed” control economizes energy, reduces wear, and—crucially—enables \emph{discrete choices}: palatable vs.\ not, safe vs.\ unsafe, admit vs.\ reject. In that sense, categorization at the mouth is already cognition—realized as \emph{haltable} motor programs riding a tubular body plan.


\subsection{Segmented Architecture: Modular Controllers and Coordinated Rhythms}
\label{subsec:segmented}

\marginpar{Segmentation = reusable control modules.}
Segmentation is a pervasive evo--devo strategy that decomposes the body into \emph{repeating units} capable of semi-autonomous control. 
For the SMN, segmentation furnishes a library of \emph{modular controllers}---each segment hosting local pattern generators and antagonistic actuators---that can be composed into flexible whole-body behaviors.

\paragraph{Evo--devo foundations.}
Across bilaterians, repeated structures arise through conserved patterning logics.
In vertebrates, somitogenesis couples a tissue-scale \emph{wavefront} to local \emph{oscillators}, producing periodic somites \cite{CookeZeeman1976ClockWavefront,Pourquie2003SegmentationClock,Pourquie2011SegmentationClock}.
In arthropods, diverse mechanisms (simultaneous vs sequential, long-germ vs short-germ) implement segment addition beyond the \emph{Drosophila-only} narrative \cite{PeelChipmanAkam2005ArthropodSegmentation}.
These blueprints instantiate a general computational theme: \emph{clock + wavefront} turns continuous tissue into discrete, reusable units.

\paragraph{Control: CPGs, coordination, and interruptibility.}
A central virtue of segmentation is the distribution of \emph{central pattern generators} (CPGs)---local rhythm factories for movement.
Segmental CPGs can operate independently yet coordinate via inter-segment couplings to form coherent global gaits \cite{MarderBucher2001CPG,Grillner2006Lamprey,Ijspeert2008CPG}.
Interruptibility comes ``for free'': local halts, phase resets, and re-routing are applied per segment, yielding \emph{HAPs} without destabilizing the global pattern (e.g., obstacle negotiation, gait transitions).
\marginpar{Global behavior via local phase relationships.}

\paragraph{Bioelectric coupling of segmental control.}
Although classic accounts emphasize genetic and synaptic mechanisms, bioelectric networks provide complementary, fast \emph{tissue-level} couplings: gap-junction connectivity and resting-potential landscapes can synchronize or bias segmental timing, modulate growth, and stabilize multi-stable states \cite{Levin2014MolecularBioelectricity}.
In the SMN, these voltage dynamics are an additional layer knitting segmental controllers into a coherent organism.

\paragraph{Habitat coupling: gravity and fluid.}
Segment chains translate habitat physics into control constraints.
In gravity, stacked segments distribute load, stabilize posture, and enable local corrective torques; in fluids, serial bending waves exploit boundary-layer and wake interactions for efficient thrust \cite{Alexander2003PrinciplesLocomotion}.
Thus, segmentation not only reduces computational complexity; it directly \emph{matches} the structure of the medium.

\paragraph{Thermodynamic economy.}
Modularity lowers the need for costly global policy rewrites: most adjustments are local phase tweaks and gate flips.
Under Landauer's principle, fewer global resets mean less unavoidable dissipation; segmentation favors \emph{action-pattern reuse} over data storage \cite{Landauer1961Irreversibility,Bennett2003LandauerNotes,StillEtAl2012ThermoPrediction}.
\marginpar{Economy by local updates, not global rewrites.}

\paragraph{Formalization sketch.}
Let $x_i(t)$ be the state of the $i$th segmental controller (e.g., a phase oscillator).
Nearest-neighbor coupling $K_{ij}$ yields traveling waves or standing patterns depending on phase-lag targets.
In the simplest case, a lattice of phase oscillators (Kuramoto-type) produces smooth waves; \emph{interrupts} are transient detunings or phase resets at select nodes.
\begin{align}
\dot{\theta}_i &= \omega_i + \sum_{j \in \mathcal{N}(i)} K_{ij} \sin(\theta_j - \theta_i - \phi_{ij}) + u_i(t),
\end{align}
where $u_i(t)$ encodes halts (HAPs) and negotiable reconfigurations (NAPs) via temporary changes to $\omega_i$ or $K_{ij}$.
\marginpar{Kuramoto lattice as a \emph{control surface}.}
\todo{Figure: (i) segmented chain with CPGs, (ii) phase-lag targets for swimming/walking, (iii) local phase reset producing a step or turn, (iv) habitat arrows (gravity/fluid) shaping feasible lags.}

\paragraph{SMN vs Habitat (capsule).}
\textbf{SMN:} segmental CPGs coordinate via local phase rules; interrupts stay local and compose into \NAP{}s.\footnote{NAPs are often realized by oscillator assemblies (OAPs in earlier drafts), but their distinct feature is negotiability: they can be rephased, reweighted, and recombined rather than running as stereotyped oscillations.}
\textbf{Habitat:} gravity partitions load across segments; fluids reward traveling-wave phase lags—medium structure constrains and simplifies coordination.

\paragraph{Adaptation.}
Segmental CPGs adapt phase/gain through local plasticity while surrounding tissues update bioelectric set-points, stabilizing desired phase-lag patterns for common gaits and enabling quick local interrupts without global rewrites.

Segmentation discretizes control into re-usable modules, making \emph{locally composable} HAPs. SMN leverages segmental synergies to retime and recombine actions with low wiring cost \citep{BizziCheung2013}.


\subsection{Bilateral Symmetry: Counter-Variation, Comparison, and Error Signals}
\label{subsec:bilateral}

\marginpar{Two sides: built-in \emph{comparators}.}
\textbf{Bilateral symmetry} is one of the most pervasive body-plan regularities of the Bilateria.
For control, having two laterally arranged sensorimotor arrays yields a simple but powerful computation: \emph{counter-variation}.
By sampling the world in parallel from two neighboring vantage points and actuating with antagonistic pairings, organisms can compute \emph{differences} and \emph{ratios} on the fly---the basis for localization, leveling, and course correction.

\paragraph{Evo--devo foundations.}
The left--right (L--R) axis is established early and robustly, integrating genetic programs (Nodal/Lefty/Pitx2) with physical mechanisms such as \emph{nodal cilia}-driven flow in vertebrates \cite{Nonaka1998NodalCilia,Hamada2002LRPatterning}.
Planar polarity pathways align subcellular asymmetries across tissues, coordinating mirror halves; bioelectric networks bias and stabilize axial identities and L--R outcomes \cite{AwLevin2009LRPCP,Levin2014MolecularBioelectricity}.
These mechanisms yield laterally paired structures whose small asymmetries can be leveraged by control.

\paragraph{Bioelectric implementations.}
Endogenous voltage gradients and gap-junction networks participate in L--R specification and can reprogram L--R patterning when perturbed, indicating multi-stable, tissue-level attractors \cite{Levin2014MolecularBioelectricity}.
This supports the SMN's view that comparator architectures and their calibration can be maintained without centralized memory, as distributed physiological states.

\paragraph{Habitat coupling and stereo inference.}
Bilateral sampling reduces spatial uncertainty by turning 3D estimation into solvable 2-point problems.
Binocular disparity supports depth estimation and vergence control \cite{CummingDeAngelis2001Stereopsis}; binaural interaural timing and level differences support azimuthal localization \cite{GrothePeckaMcAlpine2010Localization}.
In gravity, bilateral proprioception enables leveling and balance; in fluids, contra-lateral phase relationships optimize thrust and turning.
Thus, the physical medium amplifies the value of paired comparisons.

\paragraph{Control advantages for the SMN.}
Two-sided architectures provide:
\begin{enumerate}
  \item \textbf{Error signals by subtraction:} $\Delta s = s_L - s_R$ implements a local gradient estimate usable for immediate course correction without storing past states.
  \item \textbf{Noise reduction by averaging:} $(s_L+s_R)/2$ attenuates uncorrelated noise while preserving common-mode signals.
  \item \textbf{Antagonistic actuation:} L--R effectors implement fine steering through small differential activations; interruptibility is achieved by switching differential sign.
\end{enumerate}
\marginpar{Finite differences without memory buffers.}

\paragraph{Thermodynamic economy.}
Because bilateral comparison supplies gradients \emph{now}, the SMN avoids costly internal writes and resets associated with constructing explicit spatial maps.
Under Landauer's principle, fewer policy overwrites mean less unavoidable heat; stereo sampling trades memory for \emph{simultaneity} and \emph{structure} \cite{Landauer1961Irreversibility,Bennett2003LandauerNotes,StillEtAl2012ThermoPrediction}.

\paragraph{Formalization sketch.}
Let $s_L(t)$ and $s_R(t)$ denote left/right sensor streams.
A minimal steering controller uses
\begin{align}
e(t) &= s_L(t) - s_R(t), \qquad m(t) = \tfrac{1}{2}\big(s_L(t)+s_R(t)\big), \\
u_{\text{steer}}(t) &= k_e\, e(t) + k_m\, \dot{m}(t),
\end{align}
with $k_e,k_m$ tuned to medium-specific dynamics.
Stereo control thus implements an \OAP{} by sequencing lateral differences through habitat-shaped transfer functions (gravity for leveling; fluid for turning), with halts encoded as $u_{\text{steer}} \to 0$.
\todo{Figure: paired sensors/effectors with (i) disparity/ITD geometry, (ii) contra-lateral phase control, (iii) gravity vector for leveling.}

\noindent\emph{Energy note.} Any logically irreversible update incurs
$\dot Q \;\ge\; k_B T \ln 2 \cdot \dot N_{\text{erase}}$;
by composing actions from stable carriers/priors (gravity, fluids, tissue attractors), the SMN lowers $\dot N_{\text{erase}}$ and heat.


\paragraph{SMN vs Habitat (capsule).}
\textbf{SMN:} bilateral sensors/effectors compute error-by-subtraction and average-by-addition for immediate steering.
\textbf{Habitat:} stereo geometry in gravity and fluids turns paired sampling into robust gradient estimates, avoiding explicit spatial maps.

Bilateral symmetry yields mirror-related control frames; the SMN treats interlimb phase-relations as control variables, enabling fast switches (in-phase $\leftrightarrow$ anti-phase) as task demands vary \citep{HakenKelsoBunz1985,Kelso1995}.



\subsection{Multi-Zonal Specialization: Routing Tasks Across Heterogeneous Substrates}
\label{subsec:multizonal}

\marginpar{Zones = different physics, different priors.}
Beyond segmentation, animals exhibit \emph{multi-zonal specialization}: regions (neural and non-neural) with distinct morphologies, materials, timescales, and couplings.
In the SMN, zones constitute \emph{heterogeneous control substrates}---some excitable and fast, others viscoelastic and slow; some tightly gap-junction coupled, others diffusively or mechanically coupled.
Zonation enables \emph{routing}: \OAP{}s recruit different zones as the situation demands, while \TAP{}s emerge when actions enlist zones that write \emph{public traces} (marks, deposits, shared states).

\paragraph{Evo--devo foundations: regionalization across taxa.}
Regional identities arise by conserved patterning programs that parcel the organism into \emph{fields} with distinct control opportunities.
Hindbrain rhombomeres provide a canonical example of serial regionalization with sharp lineage boundaries and specific projections \cite{LumsdenKrumlauf1996Hindbrain}.
Forebrain arealization follows the prosomeric model of nested longitudinal and transverse domains \cite{PuellesRubenstein2003Prosomeric}.
At the cortical level, early patterning sets up area biases that later experience can refine \cite{SurRubenstein2005CortexPatternPlasticity}.
Outside the CNS, arthropod \emph{tagmosis} (head, thorax, abdomen) illustrates regional specialization of repeated elements into functionally distinct zones \cite{Scholtz2010Tagmosis}.

\paragraph{Zonation as control architecture.}
Zones differ in transfer functions and priors: elastic vs rigid, high- vs low-pass, excitable vs passive.
The SMN exploits these differences by \emph{task matching}: predictive smoothing in viscoelastic zones; precise timing in excitable zones; persistent bias in bioelectricly polarized zones \cite{Levin2014MolecularBioelectricity}.
Cerebellar microzones and their olivo-cerebellar loops exemplify fine-grained error-driven control layered on top of slower postural zones \cite{AppsHawkes2009CerebellarZones}; basal ganglia–thalamocortical loops instantiate selection/routing among competing controllers \cite{AlexanderDeLongStrick1986ParallelLoops}.

\paragraph{Habitat coupling.}
Zones are differentially exposed to the medium: distal appendage zones interact with boundary layers and wakes; axial postural zones bear gravitational loading; epithelial exchange zones sense and shape flow/chemistry.
The habitat thus supplies \emph{zone-specific statistics} that the SMN learns to rely on for \OAP{} routing and, when externalized, for \TAP{} formation (e.g., building nests, laying chemical trails).

\paragraph{Thermodynamic economy.}
Specialization reduces internal erasures: rather than rewriting a monolithic controller, the SMN composes \OAP{}s by switching among zones with \emph{pre-tuned priors}, minimizing logically irreversible policy overwrites \cite{Landauer1961Irreversibility,Bennett2003LandauerNotes}.
Where durable public traces are beneficial, \TAP{}s exploit stable substrates (mud, fiber, pheromonal fields), offloading memory to the habitat.

\paragraph{Formalization sketch.}
Let zones be nodes $Z_k$ in a directed graph with edge weights $W_{ij}$ encoding feasible couplings (mechanical, electrical, chemical).
An \OAP{} is a path in this graph with control inputs that gate edges (enable/disable) and adjust local gains.
A \TAP{} is an \OAP{} whose execution writes to an external state $E$ (a shared field or artifact) such that $E_{t+1}\neq E_t$ and is readable by other agents.
\marginpar{\TAP{} = \OAP{} $+$ public write.}
\todo{Figure: (i) zone graph with distinct transfer functions, (ii) \OAP{} path, (iii) \TAP{} writing to an external field, e.g., pheromone or constructed niche.}

\paragraph{SMN vs Habitat (capsule).}
\textbf{SMN:} heterogeneous zones with distinct transfer functions enable task routing as \OAP{}s; \TAP{}s arise when actions write public traces.
\textbf{Habitat:} zone-specific statistics (loads, flows, chemistries) provide priors that reduce costly internal policy rewrites.


\begin{table}[t]
\begin{adjustwidth}{-\notecolumn}{0pt} % borrow the left notes strip
\centering
\setlength{\tabcolsep}{6pt}
\renewcommand{\arraystretch}{1.15}
\begin{tabularx}{\dimexpr\linewidth+\notecolumn\relax}{l c c Y}
\toprule
\textbf{Zone (substrate)} & \textbf{Dominant coupling} & \textbf{Time scale} & \textbf{Prior / role}\\
\midrule
Excitable (neural)              & synaptic / gap junction & ms–s             & precise timing, prediction \\
Viscoelastic (muscle/ECM)       & mechanical              & 10\,ms–10\,s     & damping, posture, smoothing \\
Epithelial / bioelectric        & voltage / gap junction  & s–hours          & set-points, pattern memory \\
Fluid interface (cilia/tubes)   & hydrodynamic            & 10\,ms–s         & carriers, token transport \\
Gravito-postural                & vestibular / proprio    & 10\,ms–s         & leveling, orientation priors \\
\bottomrule
\end{tabularx}
\caption{Zone couplings, time scales, and priors.}
\label{tab:zones}
\end{adjustwidth}
\end{table}

\paragraph{Adaptation.}
Zones calibrate their transfer functions (excitability, stiffness, polarization) to environmental statistics; routing policies then favor pre-tuned zones, minimizing policy overwrites, and promoting \TAP{}s when durable public traces (e.g., constructed niches) pay off.

\subsection{Antagonism}\label{antagonism}

A fundamental principle in the body plan of bilaterians is the organization of movement through antagonistic pairs of effectors. Muscles, by their biological constitution, can only contract (i.e., pull) but not push. This morphological constraint necessitates the evolution of opposing pairs of muscles—flexors and extensors, abductors and adductors, pronators and supinators—that can act in reciprocal fashion to enable controlled motion \cite{Huxley1932_RelativeGrowth,Schilling2011_AntagonisticEvolution}. The bilaterian body plan, with its segmentation and bilateral symmetry, provides the structural basis for this arrangement.

At the neural level, antagonism is implemented by the canonical circuit motif of reciprocal inhibition. Sherrington’s classic work established that the activation of one muscle in a pair is accompanied by inhibition of its antagonist, ensuring coordinated and efficient movement \cite{Sherrington1906_IntegrativeAction}. This basic inhibitory scheme remains central to vertebrate and invertebrate motor control and underlies the formation of stable motor patterns.

From a biomechanical perspective, antagonistic control confers several advantages. Co-activation of antagonists allows modulation of joint stiffness, enabling both stability and compliance. Such mechanisms are critical for adaptive interaction with the environment and have inspired numerous robotic implementations of antagonistic actuators \cite{Hogan1984_ImpedanceControl,Tondu2012_McKibbenMuscle}. In biological systems, this provides the capacity not only to move but also to regulate impedance and precision.

In the context of the sensation-modulating network (SMN), antagonism is not merely a mechanical solution but also a higher-order organizational principle. Antagonistic structures encode dynamic balances: approach versus avoidance, activation versus inhibition, flexion versus extension. These embodied oppositions structure the repertoire of available action schemas and affordances, contributing to the cognitive architecture of the SMN itself \cite{Bizzi2013_MuscleSynergies}. Thus, antagonism serves as both a morphological constraint and a generative principle for embodied cognition.

\begin{figure}[htbp]
  \centering
  \includegraphics[width=0.8\textwidth]{graphics/smn_antagonism_bilateral_cns.pdf}
  \caption{%
    Antagonism at multiple levels of the body plan. 
    Local antagonism operates between flexor and extensor pairs within each action zone (red and blue arrows). 
    Reciprocal inhibition is mediated through central nervous system pathways (green arrows into the spinal cord). 
    Cross-body coordination (purple arrows) connects bilaterally symmetrical zones, showing how antagonism is both a morphological and neural principle.%
  }
  \label{fig:smn-antagonism}
\end{figure}

\subsubsection{Antagonistic Architecture could Generate Fixed and Haltable Action Patterns}
Antagonism as an explanatory theme in biology is widely recognized. The role of agonistic and antagonistic muscles in motor coordination is well established. Each action zone may have multiple sets of agonist and antagonist \textit{actors}.  A contraction leading to a pull of agonist alternated by an antagonist's pull may generate a simple open-close pattern, one cycle: one fixed action pattern (FAP). 

This can be modeled as a Petri net \cite{peterson1977petri}, a bipartite graphical representation used widely for modeling processes as changing states are affected by transitions. The two kinds of nodes in the graph are places and transitions. A transition has a prior state represented as an input place value (token) and a post-state as a resulting place value. 

For example, a sense organ as a transducer can be represented as a transition of an analog physical perturbation from the environment (an input) into an action potential within the agent. This transduction is the basic source of input tokens to the agent, forming sensations.
\begin{figure}[ht] 
\includegraphics[width=\textwidth]{graphics/PN_Transduction.pdf}
\caption{\textbf{Transduction:
}The transducer is represented as a square node, which will fire when a condition of at least one token as input place value is satisfied, giving rise to the resulting place value.
Places are represented as circles with or without tokens (the values).}
\label{transduction}
\end{figure}

However, a pull action from the complement muscle between the cycle can generate a halting action in between.  Thus two actors can regulate each other's action without involving any other controlling agent or actor. A fine-grain halting may lead to greater coordination.  This is one model of implementing a simple haltable-action-pattern (HAP). 

Haltability, coordination of actions, emerges due to dynamic antagonism among the zones. Having established the fundamental principles of action and haltability, we now turn to the specific biological architecture that implements these principles. The SMN model proposes that cognition emerges from a particular body plan that is shared across animal life but has been largely ignored in cognitive science.

Antagonistic pairs realize physical implementations of reciprocal inhibition, giving SMN a robust substrate for $\mathcal{H}_\alpha$ (halt/phase-reset) without losing posture—key for safe interruptibility.

%=== Negotiable Action Patterns as conditional sequences ===
\paragraph{NAPs as conditional sequences.}
Negotiable Action Patterns (NAPs) can be thought of as \emph{action scripts} that unfold step by step, 
but with built-in checkpoints that test whether conditions are met. 
At each step, an action is performed while the environment is monitored; 
if the condition continues to hold, the action runs, but if it changes, the sequence can be halted, redirected, or switched to another action. 

In computer science, such conditional sequences are known as guarded traces \citep{Dijkstra1975,BaetenWeijland1990}: sequences of actions where each step executes only if its guard condition holds:  a sequence of actions ($a_1,a_2,\ldots,a_k$) paired with conditions ($g_1,g_2,\ldots,g_k$) that guard the transitions. Here we borrow the term not to stress its technical heritage but to capture the simple idea of actions unfolding under conditions, with the possibility of interruption or branching.

The haltability operator $\mathcal{H}_\alpha$ ensures that interruptions are safe: the agent can pause or reconfigure without collapsing its posture or coordination.

%--- Formal sketch (kept minimal) ---
\[
\gamma \;=\; (a_1, g_1)\to (a_2, g_2)\to \cdots \to (a_k, g_k), 
\qquad a_i\in \mathcal{A},\; g_i:\Sigma\to\{\top,\bot\}
\]

%--- Simple diagram ---
\begin{center}
\begin{tikzpicture}[node distance=16mm,>=stealth,auto,scale=0.95, every node/.style={transform shape}]
\node[draw,rounded corners,align=center,inner sep=3pt] (a1) {$a_1$\\(HAP/NAP)};
\node[draw,rounded corners,align=center,inner sep=3pt,right=of a1] (a2) {$a_2$};
\node[draw,rounded corners,align=center,inner sep=3pt,right=of a2] (a3) {$a_3$};
\path[->] (a1) edge node {$g_1$} (a2)
          (a2) edge node {$g_2$} (a3);
\path[->,dashed] (a2) edge[bend left=45] node[above] {$\mathcal{H}_{\alpha}$} (a1);
\path[->,dashed] (a2) edge[bend left=20] node[below] {$\mathcal{H}_{\beta}$} (a3);
\end{tikzpicture}
\end{center}
\vspace{-1ex}

In plain terms, a NAP is like a flexible habit or skill: it has a familiar structure, but the checkpoints (guards) and haltability make it negotiable and adaptable. 
When such NAPs leave traces that others can perceive—through gesture, voice, or external marks—they become \emph{Transactional Action Patterns} (TAPs), which enter the intersubjective domain of conversation and shared meaning \citep{Gibson1979,PezzuloCisek2016}.


%=== Negotiable Action Patterns as guarded traces ===
\paragraph{NAPs as guarded traces.}
Let $\mathcal{A}$ be the set of available HAP/NAP controllers and $\Sigma$ the set of perceptual predicates (affordance tests). 
A NAP can be expressed as a finite guarded trace
\[
\gamma \;=\; (a_1, g_1)\to (a_2, g_2)\to \cdots \to (a_k, g_k), 
\qquad a_i\in \mathcal{A},\; g_i:\Sigma\to\{\top,\bot\},
\]
executed by the policy
\[
\pi_\gamma(x) \;=\; 
\begin{cases}
\mathcal{H}_{\alpha_i}\!\circ a_i(x), & \text{while } g_i(\Sigma)=\top \\[4pt]
\text{branch to } (a_{i+1},g_{i+1}), & \text{on guard switch}.
\end{cases}
\]
Here $\mathcal{H}_{\alpha_i}$ denotes the haltability relation: it manages mid-course interruptions (pause, rephase, or re-route) without collapsing the pattern. 
This formalism captures how the SMN organizes flexible, negotiable action structures in an individual agent.

This schematizes NAPs as \emph{guarded compositions} of interruptible primitives. 
When such negotiable patterns are externalized into perceptible traces—gestural, vocal, or material—they become \emph{Transactional Action Patterns} (TAPs), entering the intersubjective domain where agents coordinate and converse \citep{Gibson1979,PezzuloCisek2016}.

\paragraph{Perceptual predicates (affordance tests).}
In the formal sketch of NAPs, each transition is guarded by a condition $g_i$. 
We call these conditions \emph{perceptual predicates} or \emph{affordance tests}. 
They are not abstract symbols but concrete checks the organism performs on its environment. 
Examples include:
\begin{itemize}
  \item \textbf{Motor–muscle level:} proprioceptors in flexor and extensor muscles signal whether the current joint angle or tension is within safe range; if a threshold is crossed, the ongoing action halts or adjusts. 
  \item \textbf{Sensorimotor level:} tactile receptors in the fingertips detect slippage during a grasp; this predicate triggers a branch from “lift” to “tighten grip.” 
  \item \textbf{Perceptual–ecological level:} vision and balance systems check whether a surface is flat and wide enough to step on; if not, the stepping routine is inhibited. 
  \item \textbf{Cognitive–social level:} auditory cues from a conversational partner signal that a turn has ended; the speaking action pattern is released in response.
\end{itemize}
Each of these can be written in the formalism as a binary test ($g_i:\Sigma \to \{\top,\bot\}$), but their richness lies in being embodied affordance checks, grounded in specific sensory modalities. 
They are the \emph{gates} that make NAPs negotiable: deciding whether to continue, halt, or branch without requiring abstract symbolic representation. 
Thus, perceptual predicates operationalize Gibson’s idea of affordances \citep{Gibson1979}, anchoring them in the agent’s bodily architecture and neural physiology.

\paragraph{Petri Nets as process models of SMN.}
While the idea of guarded traces has its origin in process algebra \citep{Dijkstra1975,BaetenWeijland1990}, 
for our purposes a more accessible and biologically faithful formalism is provided by \emph{Petri Nets} (PNs). 
PNs are bipartite graphs consisting of \emph{places} and \emph{transitions}, with directed arcs between them. 
In the SMN interpretation, the distributed processing units (DPUs)—flexor/extensor or left/right zones—are modeled as \emph{transitions}, 
while the mediated states carried by the central nervous system are modeled as \emph{places}. 
Tokens circulating through the net represent proprioceptive and sensory messages, with haltability implemented as gating conditions on token flow. 
This mapping makes visible how coordination emerges: when one zone is active (a transition firing), its partner zone is in an alert state, awaiting tokens for activation. 
Because Petri Nets have both a rigorous process algebra semantics and an intuitive graphical representation, they serve as an ideal bridge between formal dynamical models and multi-disciplinary accessibility.



% ==== PREAMBLE (if not already included) ====
% \usepackage{tikz}
% \usetikzlibrary{arrows.meta, positioning}
% \usepackage{tikzsymbols} % optional if you want token styles

% ==== PETRI NET DIAGRAM ====
\begin{figure}[t]
\centering
\begin{tikzpicture}[node distance=20mm,>=Stealth, every node/.style={transform shape}]
  % Places
  \node[place,tokens=1] (p1) {};
  \node[place,right=50mm of p1] (p2) {};

  % Transitions
  \node[transition,below=15mm of p1] (t1) {};
  \node[transition,below=15mm of p2] (t2) {};

  % Arcs
  \draw[->] (p1) -- (t1);
  \draw[->] (t1) -- (p2);
  \draw[->] (p2) -- (t2);
  \draw[->] (t2) -- (p1);
\end{tikzpicture}

\caption{Petri Net representation of a flexor--extensor antagonistic pair. 
Circles are \emph{places} (mediated CNS states carrying sensory tokens); 
rectangles are \emph{transitions} (local action zones/DPUs). 
Tokens (black dots) represent proprioceptive and sensory signals flowing through the system. 
Haltability corresponds to gating of token flow, ensuring that when one zone is active the antagonistic partner is in an alert state.}
\label{fig:pn-flexor-extensor}
\end{figure}

% ==== MULTI-ZONE SMN PETRI NET ====
\begin{figure}[t]
\centering
\begin{tikzpicture}[
  node distance=14mm and 20mm,
  >=Stealth,
  every node/.style={transform shape},
  tokenstyle/.style={place,minimum size=6mm},
  trstyle/.style={transition,minimum width=3mm,minimum height=10mm}
]

% ---------- Central broadcast place (CNS integration) ----------
\node[tokenstyle,tokens=2,label=above:{\footnotesize CNS broadcast state}] (CNS) {};

% ---------- ZONE A: Shoulder (Flexor/Extensor) ----------
% local proprioceptive place
\node[tokenstyle,tokens=1, left=36mm of CNS, yshift=18mm] (Aprop) {};
% transitions
\node[trstyle, below=10mm of Aprop,label=left:{\scriptsize A\_F}] (AF) {};
\node[trstyle, right=10mm of AF,label=right:{\scriptsize A\_E}] (AE) {};
% arcs (local)
\draw[->] (Aprop) -- (AF);
\draw[->] (Aprop) -- (AE);
% antagonistic loop via CNS
\draw[->] (AF) -- (CNS);
\draw[->] (AE) -- (CNS);

% ---------- ZONE B: Elbow ----------
\node[tokenstyle,tokens=1, left=36mm of CNS, yshift=-22mm] (Bprop) {};
\node[trstyle, below=10mm of Bprop,label=left:{\scriptsize B\_F}] (BF) {};
\node[trstyle, right=10mm of BF,label=right:{\scriptsize B\_E}] (BE) {};
\draw[->] (Bprop) -- (BF);
\draw[->] (Bprop) -- (BE);
\draw[->] (BF) -- (CNS);
\draw[->] (BE) -- (CNS);

% ---------- ZONE C: Wrist ----------
\node[tokenstyle,tokens=1, below=24mm of CNS, xshift=-10mm] (Cprop) {};
\node[trstyle, below=10mm of Cprop,label=left:{\scriptsize C\_F}] (CF) {};
\node[trstyle, right=10mm of CF,label=right:{\scriptsize C\_E}] (CE) {};
\draw[->] (Cprop) -- (CF);
\draw[->] (Cprop) -- (CE);
\draw[->] (CF) -- (CNS);
\draw[->] (CE) -- (CNS);

% ---------- ZONE D: Jaw ----------
\node[tokenstyle,tokens=1, right=36mm of CNS, yshift=12mm] (Dprop) {};
\node[trstyle, below=10mm of Dprop,label=left:{\scriptsize D\_F}] (DF) {};
\node[trstyle, right=10mm of DF,label=right:{\scriptsize D\_E}] (DE) {};
\draw[->] (Dprop) -- (DF);
\draw[->] (Dprop) -- (DE);
\draw[->] (DF) -- (CNS);
\draw[->] (DE) -- (CNS);

% ---------- ZONE E: Ankle ----------
\node[tokenstyle,tokens=1, right=36mm of CNS, yshift=-24mm] (Eprop) {};
\node[trstyle, below=10mm of Eprop,label=left:{\scriptsize E\_F}] (EF) {};
\node[trstyle, right=10mm of EF,label=right:{\scriptsize E\_E}] (EE) {};
\draw[->] (Eprop) -- (EF);
\draw[->] (Eprop) -- (EE);
\draw[->] (EF) -- (CNS);
\draw[->] (EE) -- (CNS);

% ---------- CNS-to-zone routing (examples) ----------
% (These show delivered/alert messages; keep sparse for clarity)
\draw[->, dashed] (CNS) to[bend left=20] (AF);
\draw[->, dashed] (CNS) to[bend left=12] (BE);
\draw[->, dashed] (CNS) to[bend right=15] (CE);
\draw[->, dashed] (CNS) to[bend right=18] (DF);
\draw[->, dashed] (CNS) to[bend right=10] (EE);

% ---------- Legend ----------
\node[anchor=north, align=left, below=50mm of CNS] (legend) {%
\begin{tabular}{@{}ll@{}}
\raisebox{1pt}{\tikz{\node[tokenstyle]{};}} & Place (state); dots = tokens (signals)\\
\raisebox{-1pt}{\tikz{\node[trstyle]{};}} & Transition (action zone DPU)\\
\raisebox{-1pt}{\tikz{\draw[->] (0,0) -- (0.8,0);}} & Routed token flow (firing/output)\\
\raisebox{-1pt}{\tikz{\draw[->,dashed] (0,0) -- (0.8,0);}} & Delivered/alert messages (haltability gates)\\
\end{tabular}
};

\end{tikzpicture}

\caption{Multi-zone Petri Net view of the SMN. Each antagonistic zone is modeled as a pair of \emph{transitions} (Flexor/Extensor DPUs) with a local proprioceptive \emph{place}. All zones write to a shared \emph{CNS broadcast place}, which routes messages and maintains an integrated body state. Solid arrows show token flow; dashed arrows depict routed “alert/haltability” deliveries.}
\label{fig:smn-multizone-pn}
\end{figure}

Tokens are proprioceptive/sensory messages; transitions fire when enabled by local tokens and CNS routing; dashed edges visualize haltability/alert delivery (readiness, not binary off). The shared CNS place implements broadcast/integration without central command.
\subsection{Sensory Motor Network is a Sensation Modulating Network}

In the previous sections we drew a sketch of the architecture of the cognitive agent in terms of polarized, tubular, segmented, bilaterally symmetrical, antagonistic organization. On to that architectural scaffolding, the cognitive organs are `mounted'. Consider this as another layer of describing the same body, but underlining the cognitive features of the agent.

The 3 components of concern for a cognitive architecture are (1) the sensory receptors organised as \textit{sensory boards (SB)}, (2) the motor action zones organised as \textit{motor boards (MB)} and (3) the connectors in between sensory and motor boards organised as a \textit{network}. Thus the body of a cognitive agent is modelled as a sensation modulating network (SMN). The nodes of the network include sensory and motor boards, while the edges include the connections of the network. Biologically, sensory boards are realised by sensory receptors, motor boards by the muscular and skeletal system and the edges of the network (connections) by the nervous system.

It is important to keep in mind the contrasting point of the architecture with the received view that confines network only to the nervous systems, where the body of the neuron, the soma, forms the node, and the axons and dendrites form the connections. In our model, the entire cognitive agent's body is a network, with the nervous system providing only the differential connections between different sensory-motor subsystems. 

Sensory receptors are located in a distributed way across the body in the outer, inner and middle layers. The distribution of these nodes \textit{on} and \textit{in} the body is not uniform. Polarization exists, in the sense that some nodes are arranged densely on one-side of the tube. For example, a pair of eyes are located towards one pole of the body and absent at other places. Each eye has multiple photo-receptors and motor modulators, together constituting the nodes of the network. Similarly we see such concentrations of sensory receptors for other modalities. Although tactile receptors are located in a distributed way, their density varies from place to place on and in the body. This unequal distribution in location and density of sensory receptors plays an important role in differentiation and solving the problem of location.

The sensory receptors are differentiated into specialised modalities transducing thermal, tactile, auditory, visual, olfactory, gustatory and location signals. The polarised and bilaterally symmetrical arrangement of these sensory receptors plays a significant role in constructing the picture of the internal and external world. The receptors organised as sensory boards are \textit{mounted} on the motor and mechanical boards (muscular and skeletal system). One more contrasting point of this model may be noted, that we consider the motor boards as sensory receptors of location, and not merely as actuators. As far as cognition is concerned, actuation is essentially for supporting sensation of every modality, committing to the theory of action based perception. 

On this body plan, there are also multiple action zones. It is important to note that action zones are also oriented towards the inner layers of the tubular body plan of the agent. There are more action zones at the anterior side, particularly at the inner surface of the tube, making the body polarized. Animal anatomy describes this plan often in the name of cephalization \cite{hyman1940invertebrates}, seen in almost all bilaterally symmetric beings. 

Apart from the polarised distribution of sensory organs, it is important to note that the motor organs, which are also sense organs, are located in a \textit{nested} manner across the body.

A large set of highly specialized motor receptors (location sensors), are present across the body organized in a bilaterally symmetrical and hierarchical (nested) manner, which biologists call as skeletal or voluntary muscles. In our model, insofar as cognition is concerned, skeletal muscles are sense organs, and their movement directly supports perception. Biologically, muscles are considered as organs of movement and locomotion. In conventional cognitive science, their role is considered auxiliary. In our model, movement and locomotion are primarily cognitive functions. Apart from providing location data as a sensory subsystem, their \textit{characteristic} movement provides conceptual schemes employed in cognition.

The motor receptors of the network are unique, because they are mounted/embedded in a bundle of muscle fibres organised as \textit{boards}. This specialized plan is essential for playing the role of \textit{inferring} the location. The data from location sensors is created through movement. That is to say, the position is resolved only by change-in-position (displacement). But, location cannot be determined without data from the other sensory streams. We interpret location, like consciousness, as a dispositional concept. Location is always of some thing or the other, some sensation or the other, of some phenomena or the other. Employing a Kantian aphorism, we may say: location without sensory data is empty, and sensory data without location is blind. Moreover, it is not only that location cannot be calculated without movement, the construction of an image of the world through other senses is impossible without spatio-temporal data. Action is essential for resolving both space and time.  

The source of temporal data is the serial action pattern.The body plan is not only polarised and nested, but also serial in nature. We also assume availability of repeatable actions, beats within the body, providing a temporal canvas/ stage. The availability of the temporal stage/ canvas is granted by the existence of metabolic cycles, biological clocks is well established [cite] in terms of diurnal and biurnal rhythms.

Antagonistic action is a key character of the body plan, which plays an important role in the current model. 

The body is not a complete network. The model uses the principle: ``Fire together. Wire together'' (FTWT) \cite{hebb1949organisation}. The SMNs that are recurrently active at the same time are likely to get connected over a period of time. Thus the connection `board' is plastic, not static. The connection patterns are created, reinforced and modified based on the dynamics of the body (action patterns) in its ontogeny (developmental history). The implication of the FTWT principle is that the actions determine the connections, and not the other way around. However, much of the differential connections may have been already determined by phylogenetic and prenatal action history. Though it appears as though the agent starts with an undifferentiated state, the phylogenetic and prenatal action history bestows, practically, a body with `innate' action schemas at birth.

In a model where actions are located at the interface between the body and the environment, the actions within the body (including the prenatal stage) are often ignored (e.g. suckling, swallowing action-patterns etc.). In our model, the most stable and determining action schemas are already in place by the time the agent is exposed to the external environment after birth. 

\begin{figure}[ht] 
\includegraphics[width=\textwidth]{graphics/self-modulating-perception.pdf}
\caption{\textbf{A schematic representing a cognitive agent as a network of multiple self-modulatable zones.}}
\label{smn}
\end{figure}

Mutual modulation of action within the body is an important and contrasting feature of the model, which is largely ignored in the current discussions in cognitive science. This mutual modulation, within the body, is a core cognitive mechanism in the current model, which appears very early on during the prenatal stage itself. Therefore, we locate the roots of cognition within mutual modulation within the body (internal environment).

The outgoing arrows from the zones represent the sensory modulation from an external world invoking action based perception (ABP) \cite{noe_action_2004}. According to this principle, the perception is not a passive sensing of the world out there, but an active process. The same principle is employed when an action zone modulates another action zone within the body. This implies that ABP is used for perceiving both the worlds (internal and external). The double headed arrows in the figure represent mutually modulating zones. Between Z-B and Z-C, the double-headed arrow represents mutual modulation from the outer surface. For example, consider a fidgeting action between fingers touching side-ways, the hand touching the chin in the typical thinking posture, scratching an itch on the same side of the body etc. The double-headed arrow between Z-C and Z-G represents another mutual modulation from an outer surface (e.g. clapping with hands), LHS and RHS mutually modulating.

In a well-connected body, the implementation of a single-headed arrow between two zones (like Z-E and Z-F) appears very rare. However, to understand such a possibility, we can consider touching a numb leg, where due to temporary lack of internal feedback, mutual modulation is transiently unavailable. The double-headed arrow between Z-D and Z-H could represent mutual modulation between the action zones from within the body. For example, swallowing zone and speech zone. In the case of swallowing, with or without food, the upper and lower surfaces of the buccal cavity, pharyngeal zones mutually touch and modulate each other.

Often, self modulating actions, such as thinking, are considered as recent evolutionary features, whereas in our model, the very root of cognition is due to such internal mutual modulation, though it is counter-intuitive. The world within the body is missed in the current accounts. Since this is a contrasting feature, more elaboration on this follows in the model. 

\begin{figure}[ht] 
\includegraphics[width=\textwidth]{graphics/structure-of-Zone.pdf}
\caption{\textbf{A schematic of a perception modulating action zone.}}
\label{zone}
\end{figure}

\subsection{A Network of Sensation modulating networks}

The structure of a cognitive agent's body is modeled as a nested network of Sensation Modulating Networks $\{n(SMN)\}$. And the dynamics of cognition is modeled as differentiation of Differences $\{\delta(D)\}$ in a  multi-modal sensory stream.

It is important to underline, as a point of contrast, that the Central Nervous System (CNS) is not interpreted as \textit{the} network in our model, but a part of it. The entire cognitive agent's body is a network. CNS provides only the differential connections between different DFNs and among the nodes (sensory and motor receptors). Therefore the popular schema of brain-body-world is reduced to body-world. Since the body is a network of SMN, this schema can be represented as 
\begin{equation}
[\{n(SMN)\} \rightleftharpoons \{W_I\}] \rightleftharpoons \{W_E\},
\end{equation}

where $W_I$ is the internal world and $W_E$ is the external world.  $W_I$ can be substituted with another $\{n(SMN)\}$, since it is nothing but another $W_I$ in effect. 

Thus the above representation can be re-written as 
\begin{equation}
[\{n(SMN)\}_i \rightleftharpoons \{n(SMN)\}_j] \rightleftharpoons \{W_E\}.
\end{equation}
What constitutes the internal world will become clear once we make the principle of mutual modulation clear below.

This schema can be contrasted with the brain-body-world schema, which can be represented as 

\begin{equation}
\{CNS\} \rightleftharpoons \{B\} \rightleftharpoons \{W_E\}
\end{equation}

where the CNS is the central nervous system, B is the rest of body and $W_E$ is the external world.

Consider the functional network as a $\Delta$ network of a set of sensory boards $\{S_i\}$ and motor boards $\{M_j\}$. We represent this network as 
\begin{equation}\label{delta_notation}\Delta^{\{S_i\}}_{\{M_j\}}, or simply, as  \Delta^{\{i\}}_{\{j\}},
\end{equation} 
where the subscripts always refer to motor boards, and superscripts to the sensory boards. For example, if $S_1$, $S_2$, $S_4$ and $M_1$, $M_5$, constitute a transient network at any given time $t_1$, this state of the SMN can be represented as 
\begin{equation}\label{delta_eg}SMN(t_1) = \Delta^{1,2,4}_{1,5}
\end{equation} 
This $\Delta$ network is an ephemeral functional state corresponding to a particular action performed by the cognitive agent at $t_1$. Thus differentiation of Differences is realised in $\Delta$ networks, where the motor boards differentiate on the Differences streaming from the multi-modal sensory boards. This is how we resolve every action as a functional network of sensory motor subsystems.

Differences are given and taken from the environment directly, while differentiation is an active function of the agent. This active function can resolve modalities and selectively attend to a difference in the stream.

SMNs are functional states of the cognitive agent. This identifiable functional state does differentiation and filtering. Since each SMN is a network, the body is a network of networks. We may represent the active cognitive agent as $\{n(SMN)\}$. The first order network of SMN is a functional sub-graph. By functional subgraph, we mean, it is not distinguished based on structure, but by ephemeral or transient connections. SMN is hence localizable, but ephemerally.

Each sensory board is an array of localised sensory receptors (transducers), which connect through a bundle of connectors (a bus) to a differentiating and filtering network (DFN). Each motor board is an array of location-receptors mounted on a bundle of actuators (muscle-fibers). The motor board connects through a bundle of connectors to the DFN on the one hand, and receives connections from the integrating network (IN) on the other. The DFN and IN are distinguishable only on the basis of their function and the nature of connections, and need not be localized in any specific area of the central nervous system (CNS).

The motor board can be considered as a mediating board between a DFN and IN. In other words, the motor board is sandwiched between the DFN and IN, creating dynamic and transient loops. Speaking of the network in graph-theoretic terms, we can say that, the nodes of an SMN are arrays of receptors and actuators, and the edges are dynamically distributed bundles of connections manifesting the aforementioned loop.

\begin{equation}
[\{n(SMN)\}_i \rightleftharpoons \{n(SMN)\}_j] \rightleftharpoons \{W_E\}.
\end{equation}
What constitutes the internal world will become clear once we make the principle of mutual modulation clear below.


This schema can be contrasted with the brain-body-world schema, which can be represented as 

\begin{equation}
\{CNS\} \rightleftharpoons \{B\} \rightleftharpoons \{W_E\}
\end{equation}

where the CNS is the central nervous system, B is the rest of body and $W_E$ is the external world.

%=== Empirical hooks ===
\paragraph{Empirical hooks.}
The SMN commitments align with: (i) lawful coordination dynamics and phase-transitions in interlimb control \citep{HakenKelsoBunz1985,Kelso1995}; 
(ii) nested neural oscillations that gate sensorimotor sampling and rephasing \citep{Buzsaki2006}; 
(iii) interrupt/stop paradigms demonstrating rapid, safe halting and task re-routing \citep{Aron2007}; and 
(iv) muscle-synergy composition as a substrate for modular control with low-dimensional reconfiguration \citep{BizziCheung2013}. 
Together these illustrate how interruptibility, composition, and affordance-based guarding instantiate cognitive control without presupposing disembodied symbol manipulations.

\paragraph{Antagonistic zones as distributed processors.}
The haltability relation, first illustrated with left–right coordination, applies equally to all antagonistic pairs in the body. 
Flexor–extensor pairs, for example, do not communicate through direct physical linkage but via neural mediation: proprioceptive and sensory receptors embedded in muscle fibers continuously encode state variables (e.g., length, tension, velocity) and transmit them through the central nervous system (CNS). 
In this picture, the CNS is not a hierarchical controller but a \emph{traffic router}. 
It gates the exchange between paired zones, delivers messages to other participating zones, and broadcasts state information across the body. 
This broadcast allows the whole organism to maintain a coherent map of its geometrical configuration—an integrated sense of where each zone is relative to the others.

The analogy with computer networks clarifies the architecture: each action zone functions as a distributed processing unit (DPU), capable of local computation and haltability; the CNS provides the communication backbone, like a motherboard or an internet switch, ensuring that local computations are exchanged, synchronized, and globally visible. 
Thus the SMN is neither brain-centered nor muscle-centered but a distributed system of co-regulated action zones, whose closure at the level of the whole network generates the agent’s bodily identity. 
In this way, antagonism and haltability extend beyond individual effectors to form the very principle of organizational closure that defines the cognitive agent.

\begin{figure}[t]
\begin{adjustwidth}{-2in}{0in}
\centering
\begin{tikzpicture}[
  font=\small,
  zone/.style={draw, rounded corners, inner sep=2pt, fill=black!2},
  dpu/.style={circle, draw, inner sep=1.2pt, minimum size=4mm, fill=white},
  cns/.style={draw, rounded corners=4pt, thick, fill=blue!6, inner sep=5pt},
  msg/.style={-{Latex[length=2mm]}, thick},
  halt/.style={-{Latex[length=2mm]}, dashed, gray!70, line width=0.8pt},
  broadcast/.style={draw=blue!40, thick, rounded corners, dashed},
  legend/.style={draw, rounded corners, inner sep=2pt, fill=black!2},
  >=Latex
]

% --- CNS hub (router/broadcaster) ---
\node[cns] (cns) {\textbf{CNS}: router \& broadcaster};

% broadcast "aura"
\begin{scope}[on background layer]
  \node[broadcast, fit=(cns), inner sep=10pt] (aura1) {};
  \node[broadcast, fit=(cns), inner sep=16pt] (aura2) {};
\end{scope}

% --- Antagonistic zones (flexor/extensor DPUs) ---
% Z1: Shoulder
\node[zone, left=35mm of cns, yshift=20mm] (z1) {Shoulder};
\node[dpu, above left=2mm and 1mm of z1] (z1f) {\scriptsize F};
\node[dpu, below right=2mm and 1mm of z1] (z1e) {\scriptsize E};

% Z2: Elbow
\node[zone, left=38mm of cns, yshift=-18mm] (z2) {Elbow};
\node[dpu, above left=2mm and 1mm of z2] (z2f) {\scriptsize F};
\node[dpu, below right=2mm and 1mm of z2] (z2e) {\scriptsize E};

% Z3: Wrist
\node[zone, below=22mm of cns, xshift=-14mm] (z3) {Wrist};
\node[dpu, above left=2mm and 1mm of z3] (z3f) {\scriptsize F};
\node[dpu, below right=2mm and 1mm of z3] (z3e) {\scriptsize E};

% Z4: Jaw
\node[zone, right=32mm of cns, yshift=14mm] (z4) {Jaw};
\node[dpu, above left=2mm and 1mm of z4] (z4f) {\scriptsize F};
\node[dpu, below right=2mm and 1mm of z4] (z4e) {\scriptsize E};

% Z5: Ankle
\node[zone, right=35mm of cns, yshift=-22mm] (z5) {Ankle};
\node[dpu, above left=2mm and 1mm of z5] (z5f) {\scriptsize F};
\node[dpu, below right=2mm and 1mm of z5] (z5e) {\scriptsize E};

% --- Local antagonism (reciprocal coupling inside a zone) ---
\draw[halt] (z1f) to[bend left=20] node[above] {\scriptsize haltability} (z1e);
\draw[halt] (z1e) to[bend left=20] (z1f);

\draw[halt] (z2f) to[bend left=20] (z2e);
\draw[halt] (z2e) to[bend left=20] (z2f);

\draw[halt] (z3f) to[bend left=20] (z3e);
\draw[halt] (z3e) to[bend left=20] (z3f);

\draw[halt] (z4f) to[bend left=20] (z4e);
\draw[halt] (z4e) to[bend left=20] (z4f);

\draw[halt] (z5f) to[bend left=20] (z5e);
\draw[halt] (z5e) to[bend left=20] (z5f);

% --- Routing via CNS: point-to-point messages (selected examples) ---
\draw[msg] (z1e) -- ($(z1e)!0.45!(cns)$) -- (cns);
\draw[msg] (cns) -- ($(cns)!0.55!(z3f)$) -- (z3f);

\draw[msg] (z2f) -- ($(z2f)!0.5!(cns)$) -- (cns);
\draw[msg] (cns) -- ($(cns)!0.6!(z4e)$) -- (z4e);

\draw[msg] (z5e) -- ($(z5e)!0.5!(cns)$) -- (cns);
\draw[msg] (cns) -- ($(cns)!0.6!(z1f)$) -- (z1f);

% --- Cross-zone haltability/alert cues (dashed, via CNS) ---
\draw[halt] (z4f) to[bend left=10] (cns);
\draw[halt] (cns) to[bend left=8] node[below] {\scriptsize alert} (z2e);

% --- Legend ---
\node[legend, below=28mm of cns, align=left] (leg) {
\begin{tabular}{@{}ll@{}}
\raisebox{1pt}{\tikz{\node[dpu]{};}} & Flexor/Extensor DPU (local processor) \\
\raisebox{-1pt}{\tikz{\draw[msg] (0,0) -- (0.7,0);}} & Routed message via CNS (point-to-point) \\
\raisebox{-1pt}{\tikz{\draw[halt] (0,0) -- (0.7,0);}} & Haltability/alert relation (reciprocal, dashed) \\
\raisebox{-1pt}{\tikz{\draw[broadcast] (0,0) rectangle (0.7,0.3);}} & CNS broadcast envelope (global state) \\
\end{tabular}
};

\end{tikzpicture}
\caption{SMN as a distributed network. Each antagonistic zone is a local DPU (flexor--extensor pair) with reciprocal haltability (dashed). The CNS acts as a router: mediating point-to-point exchange, enforcing gating protocols, and broadcasting integrated body state for global geometric awareness.}
\label{fig:smn-router}
\end{adjustwidth}
\end{figure}

\subsection{The Functional Model of the Cognitive Agent}


In this section we present an argument that the human cognitive life, for that matter of any other cognitive agent, cannot be explained without an asymmetrical ontology and a gap-creating epistemology.  

\emph{The asymmetry condition is satisfied by polarized body plan and a bilaterally symmetrical localization of sensory and motor boards.}  
The structural body plan provides an embodied action schema for grasping the world. It is more natural for most of the animals to walk, run, swim or fly forward than backward because of the body plan. The architecture of the limbs of all the terrestrial organisms is asymmetrical, based on a limited degrees of freedom at each joint. Polarized orientation of the body to the ground with ventral and dorsal body differentiated, facilitates dealing with the field of gravitation. 

\emph{A formatted body and its action patterns determines the structure of the phenomena.} Specialized sensory boards are also organized to suit this body plan. As a result, the possible action schemes can resolve the structure of the world. This resolution is logically impossible without movement, and most importantly the patterns of movement facilitated by the body plan. The problems of cognition cannot be resolved without addressing this basal mechanism. Mere sense organs and the nervous system are not sufficient to build a complete framework. The nervous system cannot process the data stream coming from sensory organs without a mediating and modulatory system. Data can be stored only if we have a formatted body. The format of the body is not only the structure, but also the format of actions. The action patterns depend on the body, modeled as a sensory-motor network.  We shall elaborate on how this model could ground cognition in this section.

\emph{Framing problem} A stream of uninterrupted sensations may have a pattern, and it can be processed to identify the pattern. But this is an ungrounded process, because, \textit{on its own} the machine cannot detect: Where does the stream come from, from within the machine or outside? Which pattern to attend to? Which pattern is more significant than the other? Which ones to ignore? This is the framing problem. This problem is related to how a cognitive agent understands the context/situation. 

\emph{Naming problem} Apart from the understanding of context, we need to deal with naming and referencing. Information processing is not only impossible without names but also useless if it cannot name the patterns. One may program the machine to give distinct names to distinct patterns, and can also group them nicely based on some similarity detecting algorithms. But how does the machine establish a reference, of which patterns pertains to which object in the world, or within the machine? And more importantly how could it convey to us the \textit{private} naming convention?  Can the machine at least tell them to itself?  If so how? These are three philosophical problems at one go: concept formation, symbol grounding problem and that of possibility or impossibility of a private language. Variation in the world and a capacity to detect patterns in a stream of experience is not sufficient to give names to the patterns. We need a naming mechanism in a cognitive agent, and we need a mechanism to communicate with each other, in a community of agents, through names. Let's call this \textit{the naming problem.} 

Combining the above two problems, let us call them together a \textit{naming-framing problem}, because we think that the frame problem and symbol grounding problem are intrinsically related. 
We will now argue that the naming-framing problem can be solved through modulation. In other words, referencing and distinguishing the external world from the internal world will be shown to be possible through the same mechanism. 

\marginnote{The principle of layering} Since modulating certain beats such as heartbeats is not affordable, we may consider situating actions over and above the core physiological mechanisms. In order for the actions to be affordable, the interactions of the sustaining layer  must continue, and they should generate sufficient surplus. When we say cognition is enactive, it implies that the emancipation from sustaining mechanisms is expensive. Though autopoietic mechanisms may include actions, they are uninterruptible, hence no liberty to introduce gaps here. Hence autopoiesis as a mechanism to compensate the lost energy and matter takes care of the sustaining layer and provides the necessary surplus in the system making actions possible \cite{maturana1991autopoiesis}. This is also an action, but the uninterrupted pace at which this action takes place has no liberty for introducing \textit{gaps} in this layer. In other words, the system can't physiologically afford to halt. However, it is this state that could enable ephemeral actions on the periphery of an autopoietic system whenever and wherever possible. This is made possible by a differentiated body plan that enables a division of labour. Some layers are busy in not only replenishing the loss of energy and matter but also generating surplus energy and matter, such that other layers in the body can \textit{halt}. This design now has room for free action. In this perspective, it is an uninterrupted action of some layer that grants freedom to some other layers. It is this partial break from uninterrupted work, that gives rise to the freedom to enter into the cognitive domain. It is in this subtle sense, that our model differs from Maturana and Varela's account of the connections between biology and cognition. The subtlety we introduce is haltability. 

\emph{Homologous roots of anatomical disengagement, modulation and haltability} These examples indicate how one could speculatively weave a story of the evolution as a story of decoupling the body into multiple zones, where each zone can act partially independent from another, and exhibit a distinguishable action pattern. The development of tongue, lips, jaws, pharynx, larynx, gills, lungs, fins, tails, ears, eyes, limbs, toes, fingers, neck, shoulder, hip and so on are interpreted in this story as anatomical disengagement (or decoupling). One can draw a tree of anatomical disengagement representing the epi-physiological bifurcation over and above the phylogenetic tree of evolution. 

Lets call the tightly-coupled actions as \textit{harder-actions} (e.g. the coupling between locomotion and feeding in the earthworms), and the decoupled actions as \textit{softer-actions} \cite{nagarjuna_muscularity_2005}. During the course of evolution more anatomical disengagements may have given rise to the availability of more such softer-actions. 

We need to cut the story short to revert back to how the transformation of a tightly coupled body plan into a loosely-coupled body plan, from harder-actions to softer-actions, is relevant as a context for cognitive science. The cognitive hypothesis we propose, given this transformation of the body plan, is: the greater the disengagement of differentiated action zones, the greater is the agent's capacity to modulate the incoming stream of experiences. This disengagement itself is a function of anatomical polarization and layering. 

Once we have multiple softer-action zones, it is possible to rest some while the others are active. This is the context for the genesis of \textit{haltability}. Haltable variations, one can speculate, could be sexually/culturally selected. The principle of economy also enabled the organism to perform one action while halting another. Isn't this how we describe modulation? The aspect of control we ascribe to modulation arises only when we hold one variable while modifying another. Can we use this insight to ground the regulatory actions required for cognitive processing in haltable action patterns?  We demonstrate how this can be the butterfly effect in cognition. We now move to discuss how haltable-action-patterns (HAPS) can become units of analysis for cognitive \textit{behavior}. 

\emph{Homology of FAPs and HAPs} This story has affinities with the view that dexterity and movement of the body contributes substantially to cognition \cite{bernstein2014dexterity}. And in cognitive neuro-science the emphasis has always been on how CNS or brain modulates motor actions. In the context of the current model, it is important to mention the concept of fixed action patterns (FAPs), which are no different from the multiple softer-action zones mentioned above. As Llinas argues the synergistic coordinated action of a cluster of muscles take part in \textit{fixed action patterns} (FAPs), which play an important role in his narrative of how to build mind from body \cite{llinas2002vortex}. In fact the choice of the expression haltable action patterns (HAPs) is inspired from Llinas, which suggests the contrasting feature of our model with that of Llinas. 

What is the role of halting in cognition?
Recalling the structure and dynamics of the SMNs explained in the above section (see figure\{\ref{zone}\}), we presented a view where the incoming stream of sensations go almost unnoticed without modulation. In an architecture where the sense organs are mounted on the available multiple modulators, the stream of sensations change in accordance with the action performed. This correlation binds the sensation with actions giving rise to perception. This is in line with the positions of Merleau Ponty \cite{ponty1969phenomenology} and Alva Noe \cite{noe_action_2004}, who argued for an embodied and enactive view of perception and cognition. 

In the synergistic motor assembly of FAPs, which is essentially a sensori-motor assembly, the foundations for modulating sensation may have been laid providing a mechanism for DFN. 

The fundamental question that we can ask is: What makes modulation possible? In the current views in cognitive sciences, modulation is typically managed by the brain; therefore researchers seek to locate the zones/ regions of control within the brain. By contrast, in our model, by suggesting that there is a strong link between haltability and modulation, we locate the mechanism in the zones of disengagement. In this interpretation, the anatomical disengagement, the ability to halt, and the ability to control are homologous. The terms used to describe higher order cognition --- `modulation', `regulation' and `control' --- are gross descriptions of phenomena, whereas haltability rooted in anatomical disengagement is a more nuanced and observable description of the phenomenon. Further, it avoids the need to identify a part or the system as an organ of control, instead ascribes this to the entire SMN, as a systemic ability. Therefore resolving modulation in terms of haltability adds greater rigor by providing an observable criterion for modulation. 

\emph{Rooting Epistemology in the gaps} As detailed above, the disengaged motor assemblies are capable of acting independent of others. The effective capacity emerging out of this is the potential to remain transiently inactive. The pattern of halting provides the structure for the rest of the story. 

\emph{Rooting syntax in a sequence of HAPs} As we understand from the theories of information, the logical conditions required for variations can be provided by gaps. For example, in a minimalist possible code, such as Morse code, the patterns of dots are created with interruptions (gaps). The various possible patterns of truth and falsity or 0 and 1, used as a foundation of encoding and processing in computer science, demonstrates the potential of gaps in generating variations. Introduction of \textit{gaps} makes syntax possible, which in turn enables the generation of as many patterns as required. Since the practical needs are a small subset of the logically infinite number of possible patterns, this is sufficient for encoding knowledge. In sequential patterns, syntax and pattern are identical i.e., they are two sides of the same coin. Syntax is a feature of sequential patterns, which can be grounded in the patterns of halting. 

\emph{Arbitrary mapping} However, understanding how to generate numerous action-patterns is apparently a simpler issue than decoding the action-patterns and their reference (what they stand for). Because, action-patterns can be generated arbitrarily as well. It is also possible to map any arbitrary action-pattern to an arbitrary reference. However, in a given context, the mapping is required to be \textit{conserved} for semantic coherence over time. This fixing the map between an action-pattern and what it could stand for does not make sense without reproducibility and continued conservation of the mapping of the action-pattern to the reference. 

\emph{Memories are reproducible action-patterns} Reproduction of an action-pattern is possible without binding it to a reference. For example, a melody generated either by an instrument or orally, does not have to stand for any reference, or they could stand for multiple references, making them ambiguous. It is a feature of artistic creations to escape from a stable reference. A general feature of melodies is that it is hard to forget. Given the fact that melodies are generated by action-patterns, we could consider what we remember are like melodies. We agree with the view that memory is \textit{for} action\cite{glenberg1997memory}, but we also argue, as a framework based on action ontology, that reproducible action-patterns and/or their traces \textit{are} memories. 

Enacting an action-pattern and decoding what it stands for can be distinguished. 
The capacity to decode an action pattern requires holding on or recollecting the mapping with a reference. 
The entry of reference in our discussion is necessitated by a separation between an object and its \textit{name}.
Let's use the example of talking about a cup by hand-grasping-a-cup action-pattern (gesture). 
\emph{Entry into Semiotic world} The possibility of grasping action-pattern \textit{without an object} (in this example, a cup) is a significant bifurcation point and an entry into semiotic world. 
The miming action of grasping an object can become a \textit{name} for objects. 

The mime for holding a pen, brush, liquid, cup, basket etc. could all be different, based on the affordances these objects offer to the agent. Grasping any object is saturated, while a mime of grasping without the object is unsaturated. 
Mimes can stand for not only objects but also verbs. For example, we can sign someone ``to get in'' or ``to get out'', as well as ``please come in'' or ``you may go now''

If the action patterns are always \textit{saturated} with the object or event, they can never become names for them. Naming is impossible without breaking this contingent binding. We therefore consider unsaturated action-patterns as a necessary condition for a semiotic life, where naming action-patterns are separated from the object they stand for. 

An action-pattern could be bound to a reference in a \textit{hard} or \textit{soft} manner. Harder binding specifically applies largely to gestures (inter-subjectively presented action-patterns). 
The action-pattern used as a mime, when closely related to the affordances offered by the object or event, are harder. 
Some mimes transcend the affordances of the object or an event, since they hold no morphological or functional correspondence to them. \emph{Possible relation to modal and amodal concepts} For example, in a typical Indian classroom, when a student stands-up in the middle of the class and shows his/her little finger, the teacher as well as the rest of the class understand that the student is seeking permission for a bio-break. This may not work in another culture. Because the binding between the little-finger-mime and seeking permission for a bio-break is created \textit{arbitrarily} without any match with the affordances. Whereas using a thumbs up mime to seek permission or a hydration-break is less arbitratry and matches with the affordance of drinking water. 

While the possibility of hard-mimes could be a major bifurcation point for communicating agents, the use of soft-mimes for communication is a revolutionary bifurcation point, because this breaks open a world of possibilities. 
We think that this could be the episode of punctuated equilibrium\cite{gould1977punctuated} in the evolution of homonids. The communities that could use arbitrary mimes (names) had a political and economic advantage over other communities, because arbitrary names gives rise to proprietary/ private (closed group) languages.\cite{corballis2014recursive}

Once we move from gestures as mimes to the traces of action-patterns, such as the sounds or inscriptions standing for an object or event, the separation between them is so deep that it is difficult to decode them by simple correlations without training or lived experiences. 
Affordances of objects and events, in the world of traces of action-patterns, hardly help to decode. Some traces of action-patterns, such as inscriptions that bear a similar morphology to an object or an action, e.g., smilies, are harder (closer to the reference). But enter the world of alphabets used to create names we enter into the ``software'' world of representations. 

Enter the ``software'' world, we enter the world of rule following games. Thus the mapping between the patterns and their references provides the rules. 
This jump from action-patterns to rules is not a step taken but a major leap. 
Rule following action-patterns provide a spring-board to another world. 
So, we need to halt by asking the question: what makes rule following games possible? 
This is a highly involved problem, since we have suddenly entered into a context of a community of agents, and not merely an individual agent. 
This cannot be resolved unless we demonstrate how in the proposed model, we can account for shared memory and shared experiences in a community of agents. 
We shall indicate an approach of resolving this problem, if not actually entirely solving, here. 

\subsection{Co-construction of memetat and self-identity}

\emph{Recurrent self-modulated action-patterns become habits. Every habit has an inherent syntax. These HAPs are \textit{action schemes}\cite{piaget1970genetic}.}

One of the first outcomes that an SMN needs to make in the model presented above is to differentiate the experiences into what is in the body and what is outside the body. 
The modulations of perceptions of one zone affecting the other zone happen at the same time, and therefore these zones get connected through FTWT principle. 
There exists a reinforcing loop, because of the fact that the two zones obtain stimulation at the same time. For example, thumb/toe sucking action observed in the fetus prior to birth, has simultaneous stimulation from two zones viz., the thumb/toe zone and the buccal zone. 
On the other hand, the fetus kicking action-pattern on the walls of the womb of a mother, are those action patterns where one SMN is stimulating another SMN. These are two independent networks. 
Though the stimulation happens at the same time, the reinforcing loops within the network do not exist. 

After birth when a child kicks the crib, an SMN is acting on a non-SMN. This action does not have a reinforcing loop in the SMN. So the crib is outside the body. So are the rattles and teddy bears in the vicinity of the child. 
As far as the active SMN is concerned the boundary between the body and the world is clearly drawn. No scope of solipsism here. The phenomenology of objects within the internal network and the external world are different. 

An embrace between two bodies (SMNs), say between a mother and a baby, is another case. Here two networks are mutually self-stimulating by both the SMNs enacting similar HAPs. Multiple zones are stimulated on each body at the same time, resulting in an inter-subjective feedback that reinforces the synchronous synergistic bonding. Here, even though the SMNs do not have direct connections at the network level, they are connected by synergy developed through synchronous self-stimulation and their effects. Such HAPs are repeated for the mutual positive reinforcement leading to conservation of an inter-subjective space. This is absent when a child kicks the crib. When the fetus is kicking the womb, it may appear as if two networks are acting on each other, but the only actor there is the fetus, for the mother's womb has no self-modulating capacity. Therefore, mother's body is practically an external environment to the fetus. In the above 3 cases, we have 3 different situations of the modulation --- in the crib kicking case, the baby modulates and experiences, but the crib does not; in the womb-kicking case, the fetus modulates and experiences, but the mother is only at the receiving end; whereas in the case of embrace, SMNs of both the baby and the mother mutually modulate and experience, and hence we can ascribe an inter-subjective space here.  In such an inter-subjective space the agents are not merely acting, but \textit{transacting}. This transition from action space to transaction space is the entry into \textit{memetat}. 

An infant kicking a rattle is yet another case. The generation of sound at the time of kicking the rattle is an example of self-production of sound. This also generates interest (attentional anchor), since this is in effect a zone (kicking zone) stimulating other zones (hearing zone, touching zone, visual zone etc). This exploration is motivating due to one's ability to control the production of changes in multiple sensory modalities. Similar example is when the child kicks a ball: when they kick the ball, though the ball is not part of the network, the other zones --- audio-visual-haptic zones --- provide the temporally correlated events. This action when repeated binds the experiences resulting from action. This is also a loop nevertheless. But the internal feedback loop is absent. Correlation is not a sufficient indication of causation, similarly firing together is not a sufficient indication of causal connection. In the thumb-toe sucking case on the other hand, the loop gets closed with a greater degree of certainty. Even in the case of kicking a rattle or playing drums, though there is no direct closure of the loop, there exists experiential closure leading to incorporation of the object into the agent's action-experience space. Often we enter into this recreational zone when we dissolve into a \textit{flow} \cite{Mihaly}. 

When a child holds a bat and explores the world outside though the bat is not part of the network, it provides haptic experiences extending the SMN's action space becoming part of its peri-personal space. 

The possibility of holding zones such as hands, beak, mouth, etc. is yet another major bifurcation point. The external object that the agent can hold extends the explorations and experiences, beyond what was affordable through the body itself. For example, objects that are hotter or sharper or heavier etc. can also become part of the explorable and experientiable spaces. Thus, a combination of tools with holding action zones, extends the action space leading to extended the habitat as well as perceptual space. 

The action patterns as mentioned above, become habits, and the habits become names (initially as gestures) the corresponding external objects become part of the habitat (explorable and experientiable spaces) which turn into memetat (transactable spaces). 

One may take several such examples to understand how the SMN can demarcate internal world from external world. This abstraction of internal and external is operational, and participates at the root of constructing \textit{self} and concept formation at the same time. Interestingly, the formation of self is not independent of the formation of what is not self. The context of operations/actions that are germane to the separation of the internal and external and the context of operations that are germane to the separation of the self and the-other (non-self) are not different. They are co-constructed, as a thematic pair. This is how a sensation modulating network or sensory motor network as an SMN acquires an entry into a cognitive ground. 

\subsection{Nested HAPs}

Having alluded to the possibility of naming through haltable-action-patterns, and the memet-memetat differentiation, we shall now address how nested-action-patterns can be constructed through self-modulation in an SMN with multiple zones of HAPs. 

\emph{Representing the complexity of nested HAPs} Clapping can be done, while the body is standing, sitting, walking, talking or running etc. The clapping action zone is disengaged from the other states the body could be in, i.e., it could be performed independent of the rest of the states. The nesting of action patterns can be represented as [sitting(clapping)],  [talking(clapping)],  [running(clapping)],  [walking(clapping)],  [singing(clapping)] etc. The nesting becomes complex when we keep walking, while singing and clapping [walking, (clapping (singing))]. The variations in nesting can be seen when the frequency of walking and clapping match, or some modified patterns through skip-clapping, while singing action pattern is going on. 

\emph{Iterative, recursive and alternative sequences of HAPs} In order to generate a sequence of HAPs performed at different zones, it is necessary for the zones to have the capacity to act independently. We shall use the term `iterating sequence' referring to repeated actions in the same zone. We shall use `recursive sequence' when actions involve multiple zones, and when actions can happen together. In the case of alternating action patterns involving multiple zones there is no recursion. We shall use the term `alternating sequence' for this case. Thus there can be 3 types of sequences that are distinguished:  iterative sequence, recursive sequence and alternating sequence. Recursive and alternating sequences can also be iterated. These form the syntactical aspects of HAPs. The core logical requirements of richer syntactical representation are satisfied by HAPs that can be iterative, recursive and alternative. 

Though, iteration implies a sequence of repeated actions, distinguishing an iterating sequence and a recursive sequence is significant for understanding semiotics. Alternating the iterative and recursive action patterns is the basis for the creation of rich symbolic forms. Logically, HAPs are required for creating such variations. It may appear logically sufficient to create variations in a sequence involving a single zone to generate syntax, e.g. Morse code.  However, it is cognitively insufficient, because the gaps can not be recognised without a reference clock (another iterating action sequence). 

The zones that are in a state of recursive HAPs, which are synchronous, can be considered as the roots of the nested structure. For example, a person clapping a simple rhythmic beat while tapping the feet alternately, left-right, is a very simple nested structure. Here the alternating-tapping is nested in the clap-beat. One can increase the complexity by arbitrarily moving another zone such as turning the neck left to right while also doing the alternating-tapping. The alternating-tapping and the alternate-neck-turning are nested inside the clap-beat. It is up to the creativity of the choreographer, to play with the endless possibilities of modulating haltable zones within the bounds of the body architecture. 

Let us take another example of clapping while walking forward, backward, sideways while also turning the neck, hips, shoulders etc. All these actions may happen in a sequence or may happen while holding one zone in a fixed or a stable recursive action-state as a beat. The beats need not be performed from within one body. When more than one agent is participating, then the sequences and the beats can be performed either sequentially or synchronously. The permutations and combinations of these possibilities are endless, even in this simple example. When the speech zone enters, then we add to the complexity, further. 

However, the number of such zones that are available for creating alternating nested sequence, in a context, determines how complex the symbolic life of that agent can be. Can't this be one of the comparative parameter of cognition among agents, both within and across species? These abilities can not be taken for granted in a body architecture, for they may not be realised without situating in suitable social and natural contexts. 

\emph{Creating and recreating internal world} Considering the speech as a peculiar ability of human-body, it is important to understand that it is not a single zone of action, but involves multiple zones. Apart from employing multiple zones in speech, all the actions are self-stimulating/ self-modulating as well. For example, the left and right and the top and bottom parts of the complex vocal apparatus touch each other: lips, tongue, pharynx, larynx stimulate each other in a variable sequence while modulating the inhalation and exhalation halting patterns. This is a paradigm example of generating a complex sequence of self-stimulating perceptions, because the body is not acting on an external object, but on itself. This introduces a complex phenomenology that can be generated by the body as and when intended and possible. In this example, we are not moving in the external space, but with-in. Another special feature of this speech complex is to generate traces of the possible HAPs as audible sounds and visible movements. However, it is possible to halt the generation of sound, while keeping the actions sustained. So, within this speech complex, the various ways of sequencing of HAPs, nesting some sequences or repeating a sequence of nested-sequences are possible. 

A similar account can be given for sign-language where the complex facial-zone and the pair of hands and fingers move. Thus complex and infinite nested sequences are possible within the body, without bringing in the scripting complex. The latter (scripting-complex) is the set of traces of the former (speech-complex and gesture-complex), and therefore not possible without them. 
 
% \begin{figure}[ht] 
% \includegraphics[width=\textwidth]{graphics/nesting-rHAPS.pdf}
% \caption{\textbf{A schematic representing how nested action sequences are possible in a body with independently haltable multiple zones. The schema shows the two zones A and B representing self-stimulation through the left and the right hands involved in a clapping action pattern. While the clapping is sustained, the zone-C is singing, say ``Happy Birthday'' song. The nested structure can be represented by a simple formula \{(AB)C\}. }}
% \label{nesting}
% \end{figure}

\emph{Articulating-architecture of an SMN} Prior to the development of generating sequence of visible actions and traces the SMN is also capable of using HAPs for engineering the space and objects in the environment. Let's recall that the mechanical structure of the body is a bilaterally symmetrical nested structure, which can be shown as a tree based on where the joints are located. This mechanical structure also exhibits polarity by asymmetrical joints. The available degrees of freedom provide the body with a clear demarcation of actions possible as ventral, dorsal, anterior and posterior. The zones that we discussed above are mounted on this mechanical skeleton. This structure already defines the possible embodied abstractions such as front and back, forward and backwards, up and down, top and bottom, etc. as the thematic pairs. These thematic pairs provide a conceptual scaffolding emerging out of the asymmetrical and polarised structure of the SMN itself. Beyond this foundation, the mechanical structure offers our ability to extend the possible explorations and experiences by facilitating the tool-use. The informed readers can see a number of simple and complex machines in the architecture of the body itself, which facilitates the extended explorations. 

\subsection{From HAPs to TAPs: Co-construction of Inter-subjectivity}
In the previous section, we discussed how an agent can differentiate between the external and internal objects, based on the criterion of feedback loop. Something similar happens in the construction of inter-subjective spaces between the agents. 

There are two factors involved in the construction of the subject and the construction of inter-subjective space: the asymmetry in the body plan and the variations in the external world. The available actions (FAPs) are determined by the body plan, therefore genetically determined. The variations (HAPs) in the available actions (FAPs) are determined by the affordances in the external world, which includes both physical and other agents. 

Let's recall that HAPs are possible due to the body plan of the subject that is endowed with multiple zones of FAPs. Similarity in the body plan of a community of agents in a common pool of the world will automatically lead to similar HAPs. 

The distinction drawn between interactions and actions is vital in helping us in constructing the inter-subjective space. When an agent acts on an inanimate object the response in the form of reaction we get is instantaneous. When we kick the wall, a HAP, the wall `kicks' you back instantaneously. This is the character of physical interactions.  But, what if the wall doesn't? What if the leg passes through it? What if no sound, no visual difference?  This amounts to an attempt to stimulate oneself, but couldn't. One expects a reaction, but its absence causes surprise, therefore a cognitive dissonance, and in turn a trigger for withdrawal or further exploration and learning. 

What if the wall reacts after a while. If we hear no sound instantaneously but after a while, we are not only surprised but will be motivated to enter into a transactional relation with the wall. What if the wall reverberates like a drum? Though there exists an instantaneous reaction, it is followed by an echo.  What if the echo is louder than the sound at the kicking instance? All these are unexpected. This may cause fear with demotivating effect, or may motivate to repeat the action once again. What if the wall moves away when we kick? What if the wall echos back only two times and stops? What if the wall begins to cry after the kick? What if the wall disappears soon after the kick? What if the reaction is a flash of light instead of sound? What if I kick one wall, and another wall kicks back? What if the reaction arrives from another wall after a while? What if an agent is situated in a world where no other agent exists? What if only one species of agents exist in a world? And what if only one agent of each species exists in a world? 

All these are thought experiments that will help us to understand the possibilities of variety of reactions from the objects in the world. Objects offer variations of reactions in time and space when we act on them. They help us differentiate the differences among them. Along the way we find other agents, who can act and perform HAPs. 

Inter-subjective space develops through transactional actions where the gaps and delays between action and reaction begin to have a pattern. We can call these patterns as transactional action patterns (TAPs). And as we have shown, the TAPs develop through the HAPs. The TAPs as we can see can happen with animate agents and inanimate objects. The transactional space is not only between agents of the same species. In human ontogeny, this includes toys, people, pets, plants, etc. The expectations and motivations surface in the TAP-space. Thus, the sense-making is rooted in the affective grounds of expectations and motivations. So, the meaning associated to the TAP-space arises through the transactions. 

This is the space that the artists exploit to create novel experiences. The stabilised TAPs become part of traditions and culture. The syntactical and generational features of HAPs become part of the TAPs as well. The syntax along with semantics becomes accessible inter-subjectively, though with a grain of uncertainty and inscrutability all through. The possibilities of mismatch between the expectations and motivations makes this space a permanent space of negotiations. Thus inter-subjective space becomes a space of negotiations between the agents. 

To explain the peculiar cognitive condition, the dynamic model of the body is layered with a deeper interactional space, followed by a layer of life-sustaining actions called beats, followed by a layer of FAPs, followed by a layer of HAPs, and finally a layer of TAPs. Thus the layered architecture refers to both structure and function. From the core to the periphery of this model, the degrees of regulated freedom increase with the onset of each layer. 

\begin{figure}[ht] 
\includegraphics[width=\textwidth]{graphics/SideViewLayers.pdf}
\caption{\textbf{A schematic representing layered architecture of SMN.}}
\label{side_view_lSMN}
\end{figure}

\begin{figure}[ht] 
\includegraphics[width=\textwidth]{graphics/TopviewLayers.pdf}
\caption{\textbf{A schematic representing layered architecture from a top view.}}
\label{top_view_lSMN}
\end{figure}

\subsection{Learning and Memory Grounded in Affordability}
As repeatedly mentioned here, the units of cognition are action-patterns. Patterns are available both as constellations, i.e. as pictures or snapshots, as well as melodies, i.e. sequences or streams. The former are atemporal. The so-called atemporality of constellations is due to the stable relations over time. They are differentiated based on geometrical relations. The latter, like melodies, are differentiated on the basis of how the constellations vary over time. Relations are the discernibles which form patterns and the variations are the changes in relations. The core aspect of what we can discern are relations but not things per se. What stays as memory is nothing but stability of the pattern --- the relations in the constellations and melodies. Memory is not for action; action patterns \textit{are} memories. Therefore, this model suggests that memory is not a substance that can be stored but a reproducible pattern of relations. The apparent storage property of memory arises from the potential to reproduce.
This potential is a function of re-forming and de-forming the action patterns as FAPs and HAPs. This is a dynamic negotiation space and not something that can be described in terms of the binaries of innate and learnt. 

Affordability of holding an action pattern determines the ability of storing memory. Holding action patterns is expensive than holding the traces. The traces give us the apparent character of a storable memory. Though action patterns (both HAPs and FAPs) are transient, they can be reproduced. FAPs are easily reproduced and are triggered by the immediate environment, while reproducing HAPs requires additional triggers and practice. HAPs, as discussed earlier, can be saturated or unsaturated. As HAPs evolve into TAPs, the requirements of context and practice increase, while reducing the cost of storing. The spectrum from beats to TAPs is very broad. 

In the proposed model, we assume that the polarised and asymmetric architecture and the sensory motor network of the body is evolutionarily granted. Most of these aspects can be considered as evolutionarily learnt and genetically memorised, in a broader sense of the term ``memory''. However since the body plan and SMN aspects are crucial for cognitive development we can not ignore how dependent the phenomenon is on what is biologically granted. As we distinguish between interactions (molecular and cellular level) and actions (organ level), let us focus on the learning and memory related to the latter. 

The bridge between interactions and actions that have an important role in cognition is the layer of beats. Though variations in the beating patterns (heart beat and breathing pattern) are limited, these are triggered by both internal as well as external environment. Their potential in generating cognitive experiences can not be ignored. Often we might consider such experiences only within the bracket of emotions and feelings, in our framework they play very vital role in evaluatory interpretation of other actions, and this forms the site for grounding semantics. While we perform FAPs, HAPs and TAPs, we can not afford to cross the tolerance limits of variations possible by the spectrum of beats. There for this is a significant bridge between interactions and actions. The entire spectrum of learning and memory with respect to FAPs, HAPs, and TAPs are negotiable only within this affordability space. 

The body plan (through genetic memory) and the habitat (provided by the living context) determine the FAPs such as flipping of fins and tails, swallowing and undulation, walking, running, self-pruning etc. These FAPs form the habit-habitat coupling. Based on the affordances offered by the internal and external environments, a wide range of variations are possible in these action patterns. Though largely these are determined by genetic factors and the habitat, since these actions are performed by the SMN, they play the role of differentiating the difference in the habitat.Thus the extent of variations in action patterns increases by several folds once the FAPs layer enters the scene. The reproducibility of these variations in action patterns constitutes the memory in this layer. The motor action is almost inevitable in a live SMN, as evident in pre-natal FAPs. The initial trigger for FAPs is largely internal therefore genetic. Apart from the emotions and feelings, the FAPs are the action schemas, in Piagetian terms, using which the SMN explores the world. 

Over and above the genetically determined action schemas the affordances of the habitat trigger variations introducing gaps in the FAPs. As more bifurcations become possible enabling disengagement, the spectrum of variations possible within the FAPs enhances manifold which catapults the SMN into the layer HAPs. The flexible (dexterous) motor anatomy has the potential to transform (re-form) the network following the FTWT principle, which makes the SMN plastic. Without variations in the action patterns enabled by the motor anatomy (plasticity of the action zones), the new connections in the network are not possible (plasticity of network). We tend to propose the primacy of motor plasticity resulting into neuronal plasticity. This speculation is grounded in the evolutionary and embryological history (phylogeny and ontogeny), that all cells are capable of movement and communication with each other without neurons. However once neuronal connections are available in an organic structure, it is untenable to ascribe primacy of cognitive function to any one part of the SMN --- action-over-network or network-over-action. The newfound connections and newfound HAPs are indiscernible in a network architecture. Therefore it is untenable to say the memory is exclusively in the connecting units or in the motor units. Locating memory, is hence a untenable research project, except ascribing it to the entire network. However, as we move from HAPs to TAPs the locatable, storable, and retrievable nature of the traces of HAPs, representations, become possible. Insofar as the representations are available to the cognitive agent, they are manifested organically, as a whole, and therefore cannot be ascribed to any one part of the SMN. 

Memorisation or learning of HAPs is apparently contradictory in an interesting way. The emergence of a HAP is at the breaking of a pre-formed pattern. For HAPs require gaps, halts, breakages and new variations --- a creative force. Now, when we talk about memorisation and learning of HAPs, it is about reproducing the action pattern again and again. The orientation here is to sediment the pattern to FAPs to an extreme of establishing hard connections between the zones --- a conservative force. Once you learn a HAP very well, it ceases to be a HAP. An explanation for this could be that the FAPs are the attractors ---  the conserving and reproducing tendency --- of multiple zones of SMN as dynamical systems. It is therefore a matter of habit for a biological system, to repeat almost all happenstances. It is a default self-reproducing embodied force, as vindicated by autopoetic model of life, a kind of inertia in action. 

The biological order maintains a landscape of the SMN, where the topos of the landscape is determined by the body plan (polarised or not, symmetrical or not). The dynamical nature of the FAPs generate the attractors in the landscape. The available articulatory freedom, provided by the multiplicity of the action zones, gives rise to the soundscape, so to speak of the HAPs and TAPs. The soundscape is a transient state, and exists as long as the FAPs, HAPs and TAPs persist, which is the wakeful state. When the action zones are in a restful state, e.g. in a sleep, the soundscape is like a calm surface of a lake. This analogy with the active cognitive state as a soundscape and the biological body as the landscape can be mapped to the layered architecture of the SMN where the interactional space of the SMN forms the landscape, while the action space forms the soundscape.









