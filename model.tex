\section{The Proposed Model: A Dynamic Architecture}
At the heart of our proposal is the Sensation-Modulating Network (SMN), a framework for understanding the cognitive agent as a whole, rather than as a brain-centric processing unit. The SMN is not a specific organ but the entire functional architecture of the agent, defined by a set of core biological design principles. These principles, while ubiquitous in biology, have been largely overlooked in their cognitive implications.

\subsection*{Architectural Principles of the SMN}
We model the agent's body as a topologically tubular structure possessing four key properties:
\begin{enumerate}
    \item \textbf{Polarity:} The body has a defined axis, typically from anterior to posterior, which establishes a fundamental directionality for movement and interaction with the environment.
    \item \textbf{Metameric Segmentation:} The body is composed of repeating segments or action zones (e.g., limbs, digits, vocal apparatus). Each zone is a locus of potential action.
    \item \textbf{Bilateral Symmetry:} The body is organized symmetrically around a central axis, creating pairs of coordinated structures.
    \item \textbf{Antagonistic Organization:} Action within and between zones is governed by antagonistic pairs (e.g., flexion and extension). This push-pull dynamic is the fundamental basis of control and modulation.
\end{enumerate}
This architectural plan provides the agent with a multitude of "action zones," each a dynamical system capable of generating rhythmic patterns. The core cognitive faculty, we argue, arises from the agent's ability to manage these patterns.

\subsubsection*{The SMN in a Gravitational Field}
A foundational oversight in many cognitive models, particularly those that are heavily neuro-centric, is the treatment of the environment as a passive problem-space that places a computational burden on the brain. Even ecological theories, which rightly situate the agent in its environment, have not fully accounted for the constitutive role that fundamental physical forces play in cognition. Our framework begins by asserting that the agent’s body is not merely *in* an environment but is dynamically shaped *by* it. The Sensation-Modulating Network (SMN) is therefore defined, first and foremost, as a system that has evolved to actively and continuously counteract the planet's gravitational field.

This is not a trivial point. Gravity is not a bug to be fixed or a variable to be solved for; it is a constant, predictable, and non-negotiable partner in every action. The entire architecture of the agent—its antagonistic muscle pairs, its skeletal structure, its vestibular system—is a testament to this partnership. This allows for a radical offloading of computation. The agent does not need to store vast amounts of "data" about the world. Instead, it develops and refines "data structures" in the form of action schemas. The stability and predictability of the gravitational field provide a constant, reliable feedback mechanism against which these schemas are calibrated.

This leads to a crucial distinction between biological cognition and the detached, symbolic computation of the machines we have built. An artificial system must be fed data, store it, and run explicit procedures on it—an inefficient process that requires immense energy. The SMN, by contrast, operates primarily in a "saturated mode." When an agent walks, the ground pushes back with every step; when it swims, the fluid resists and supports every movement. The rich, real-time feedback from the world is an ineliminable part of the computational loop. Because the "data" remains external, the agent's internal work is lean, efficient, and possible at room temperature.

Therefore, actions like walking and swimming are not merely locomotion; they are profound epistemic acts. They are how the agent constructs a geometric model of its world, using its own body and the constant of gravity as its measuring instruments. As these actions modulate the agent's sensory subsystems in response to the affordances of the environment, the agent "grasps" the world—not by representing it internally, but by continuously testing and refining its possibilities for action within it. The gravitational field is thus not an incidental feature of our world, but a fundamental and active component of our cognitive architecture.

\subsection*{The Primacy of Halting}
Contra the received view that the nervous system's primary role is to initiate action, we propose its crucial function for cognition is to \textit{alter} and, most importantly, \textit{halt} ongoing action patterns. Movement and rhythmic activity are default states for biological tissue; even a detached cardiac tissue beats rhythmically. The challenge for a complex agent is not to start moving, but to control, pause, and modulate its intrinsic dynamics. This capacity for halting is what provides the freedom to deviate from fixed trajectories, enabling exploration, deliberation, and the generation of phenomenological experience. A pause in an action sequence is not a lack of activity, but a cognitive act itself—one that creates a space for awareness and choice.

\subsubsection*{Haltability and Ecological Affordances}
Just as the gravitational field provides a constant, predictable partner in computation, the specific objects and surfaces within the environment provide a rich structure of opportunities for action. Following Gibson, we term these opportunities "affordances." A flat, rigid surface affords support for walking; a handle affords grasping; a gap affords leaping across. The concept of haltability is deeply intertwined with these affordances. An agent does not simply execute a fixed motor program; it initiates a Haltable Action Pattern (HAP) in the direction of an affordance, and the specific properties of the environment provide continuous feedback that shapes the action in real-time.

This creates a reciprocal, co-defining relationship. The agent’s capacity to halt and modulate its actions allows it to selectively engage with the world’s affordances, to explore them without being locked into an irreversible action sequence. In turn, the structured nature of these affordances reinforces and refines the agent’s repertoire of HAPs. For example, a child learning to grasp discovers that a soft toy affords a different kind of modulated pressure than a hard wooden block. The environment does not merely trigger an action; it actively teaches the agent *how* to halt and shape that action appropriately. The affordance, therefore, becomes an integral part of the action's control loop.

Haltability is thus not a purely internal capacity but a relational one. It is the agent's contribution to a dynamic dance with the environment. The world offers possibilities for action, and the agent's ability to pause, adjust, and sequence its HAPs is what allows it to navigate these possibilities, transforming a field of potential affordances into a meaningful, enacted world.

\subsection*{A Hierarchy of Action Patterns}
The SMN's dynamics give rise to a hierarchy of action patterns, distinguished by their degree of modulatability:
\begin{itemize}
    \item \textbf{Fixed Action Patterns (FAPs):} These are deep, phylogenetically old, and largely involuntary rhythmic patterns that form the foundation of the agent's being (e.g., heartbeat, respiration, peristalsis). They are not directly accessible to conscious modulation and constitute the stable background or "cognitive canvas" upon which experience is drawn.
    \item \textbf{Haltable Action Patterns (HAPs):} These are actions that can be consciously initiated, paused, and modulated. They are performed by the outer, more flexible layers of the SMN (e.g., reaching, grasping, vocalizing). The ability to halt a HAP is the basis for creating discrete, repeatable, and therefore tokenizable, units of action.
    \item \textbf{Transactional Action Patterns (TAPs):} When HAPs are performed in a social context, they become TAPs. These are actions that are either directed at another agent or are imitated, forming the basis of communication, shared practices, and cultural learning. TAPs are the building blocks of intersubjective meaning.
\end{itemize}

\subsection*{Layered Networks and the Emergence of Concepts}
The SMN is a layered architecture. The deeper layers, governed by FAPs, function as an \textbf{integrating network (IN)}, providing a stable, multi-dimensional background of bodily awareness. The outer layers, governed by HAPs, act as a \textbf{differentiating and filtering network (DFN)}, carving specific, meaningful patterns out of the continuous flow of experience.

This architecture provides a natural mechanism for grounding symbols and forming concepts. A HAP is initially \textit{saturated} when it is directed at and constrained by a physical object (e.g., the action of grasping a cup). However, the agent can also perform the action pattern without the object being present. This creates an \textbf{unsaturated HAP (USHAP)}—the action is delinked from the external object but still generates a phenomenological experience because it is rooted in the SMN. These USHAPs are the raw material of concepts, simulations, and imagination. They are abstract but remain grounded in the agent's bodily repertoire, thus resolving the classical symbol-grounding problem.

\subsubsection*{The Integrated Broadcasting Network}
% TODO: Elaborate on the mechanism for differentiating sensations.
