\section{The Proposed Model: A Dynamic Architecture}
At the heart of our proposal is the Sensation-Modulating Network (SMN), a framework for understanding the cognitive agent as a whole, rather than as a brain-centric processing unit. The SMN is not a specific organ but the entire functional architecture of the agent, defined by a set of core biological design principles. These principles, while ubiquitous in biology, have been largely overlooked in their cognitive implications.

\subsection*{Architectural Principles of the SMN}
We model the agent's body as a topologically tubular structure possessing four key properties:
\begin{enumerate}
    \item \textbf{Polarity:} The body has a defined axis, typically from anterior to posterior, which establishes a fundamental directionality for movement and interaction with the environment.
    \item \textbf{Metameric Segmentation:} The body is composed of repeating segments or action zones (e.g., limbs, digits, vocal apparatus). Each zone is a locus of potential action.
    \item \textbf{Bilateral Symmetry:} The body is organized symmetrically around a central axis, creating pairs of coordinated structures.
    \item \textbf{Antagonistic Organization:} Action within and between zones is governed by antagonistic pairs (e.g., flexion and extension). This push-pull dynamic is the fundamental basis of control and modulation.
\end{enumerate}
This architectural plan provides the agent with a multitude of "action zones," each a dynamical system capable of generating rhythmic patterns. The core cognitive faculty, we argue, arises from the agent's ability to manage these patterns.

\subsubsection*{The SMN in a Gravitational Field}
% TODO: Elaborate on how the SMN is defined as a system that counters gravity.

\subsection*{The Primacy of Halting}
Contra the received view that the nervous system's primary role is to initiate action, we propose its crucial function for cognition is to \textit{alter} and, most importantly, \textit{halt} ongoing action patterns. Movement and rhythmic activity are default states for biological tissue; even a detached cardiac tissue beats rhythmically. The challenge for a complex agent is not to start moving, but to control, pause, and modulate its intrinsic dynamics. This capacity for halting is what provides the freedom to deviate from fixed trajectories, enabling exploration, deliberation, and the generation of phenomenological experience. A pause in an action sequence is not a lack of activity, but a cognitive act itself—one that creates a space for awareness and choice.

\subsubsection*{Haltability and Ecological Affordances}
% TODO: Elaborate on how haltability is reinforced by environmental affordances.

\subsection*{A Hierarchy of Action Patterns}
The SMN's dynamics give rise to a hierarchy of action patterns, distinguished by their degree of modulatability:
\begin{itemize}
    \item \textbf{Fixed Action Patterns (FAPs):} These are deep, phylogenetically old, and largely involuntary rhythmic patterns that form the foundation of the agent's being (e.g., heartbeat, respiration, peristalsis). They are not directly accessible to conscious modulation and constitute the stable background or "cognitive canvas" upon which experience is drawn.
    \item \textbf{Haltable Action Patterns (HAPs):} These are actions that can be consciously initiated, paused, and modulated. They are performed by the outer, more flexible layers of the SMN (e.g., reaching, grasping, vocalizing). The ability to halt a HAP is the basis for creating discrete, repeatable, and therefore tokenizable, units of action.
    \item \textbf{Transactional Action Patterns (TAPs):} When HAPs are performed in a social context, they become TAPs. These are actions that are either directed at another agent or are imitated, forming the basis of communication, shared practices, and cultural learning. TAPs are the building blocks of intersubjective meaning.
\end{itemize}

\subsection*{Layered Networks and the Emergence of Concepts}
The SMN is a layered architecture. The deeper layers, governed by FAPs, function as an \textbf{integrating network (IN)}, providing a stable, multi-dimensional background of bodily awareness. The outer layers, governed by HAPs, act as a \textbf{differentiating and filtering network (DFN)}, carving specific, meaningful patterns out of the continuous flow of experience.

This architecture provides a natural mechanism for grounding symbols and forming concepts. A HAP is initially \textit{saturated} when it is directed at and constrained by a physical object (e.g., the action of grasping a cup). However, the agent can also perform the action pattern without the object being present. This creates an \textbf{unsaturated HAP (USHAP)}—the action is delinked from the external object but still generates a phenomenological experience because it is rooted in the SMN. These USHAPs are the raw material of concepts, simulations, and imagination. They are abstract but remain grounded in the agent's bodily repertoire, thus resolving the classical symbol-grounding problem.

\subsubsection*{The Integrated Broadcasting Network}
% TODO: Elaborate on the mechanism for differentiating sensations.
