\subsection*{Abstract}
This paper addresses the long-standing impasse between cognitivist and enactive approaches by proposing a new foundational model for cognitive science. We argue that the division stems from an incomplete understanding of the agent's biological architecture. To resolve this, we introduce the Sensation-Modulating Network (SMN), a model of the whole agent grounded in general principles of biological design, such as segmentation and antagonism. Our central thesis is that cognition is not computation but action-modulation. We demonstrate how the capacity to halt ongoing, rhythmic action patterns is the fundamental cognitive act, creating the discrete, tokenizable units required for higher cognition. This framework explains the emergence of phenomenological experience from the modulation of action, the grounding of symbols via "unsaturated" action patterns, and the origins of generative syntax from the body's own combinatorial structure. By redefining cognition in this way, the SMN provides a unified, biologically-grounded account that reconciles the core tenets of previously divided camps.
