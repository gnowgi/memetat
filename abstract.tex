\subsection*{Abstract}
This paper addresses the long-standing impasse between cognitivist and
enactive approaches by proposing a new foundational model for
cognitive science. We argue that this division stems from an
incomplete understanding of the agent's biological architecture. To
resolve this, we introduce the Sensation-Modulating Network (SMN), a
model of the whole agent grounded in a specific set of general
biological principles: a segmented, polarized, bilaterally
symmetrical, and antagonistically organized body plan. Our central
thesis is that the capacity to halt ongoing, rhythmic action
patterns—giving rise to a hierarchy of Fixed, Haltable, and
Transactional Action Patterns (FAPs, HAPs, and TAPs)—is the
fundamental cognitive act. This framework explains the emergence of
phenomenological experience, the grounding of symbols via
"unsaturated" but grounded actions, and the origins of generative
syntax from the body's own combinatorial multi-zonal structure. By
situating cognition in a core biological architecture, the SMN model
of the body provides a biologically-grounded model of cognition,
consciousness, semiotics, language and culture.

\bigskip
\noindent \textbf{Keywords:} cognitive architecture, enactivism, symbol grounding, generative syntax, phenomenology, semiotics, action patterns, Sensation-Modulating Network (SMN)
