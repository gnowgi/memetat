
\subsection*{Abstract}
This monograph addresses the long-standing divide between cognitivist and enactive approaches by proposing a new foundational model for cognitive science. We argue that the impasse stems from an incomplete account of the agent’s biological structure and its cognitive function. To resolve this, we introduce a model of the cognitive agent’s body composed of a \emph{Sensory-Motor Network} (\SMNS), which defines structural organization, and a \emph{Sensation-Modulating Network} (\SMND), which defines its dynamics. Together, these networks express the agent’s evo–devo trajectory in terms of general biological principles: a polarized, tubular, segmented, bilaterally symmetrical, and antagonistically organized body plan. 

Our central thesis is that cognition arises from the ability to halt and negotiate rhythmic action patterns, giving rise to a hierarchy of \emph{Fixed, Haltable, Negotiable, and Transactional Action Patterns} (FAPs, HAPs, NAPs, TAPs). This hierarchy grounds phenomenological experience, explains the emergence of symbols through “unsaturated” yet embodied actions, and accounts for the origin of generative syntax in the body’s combinatorial, multi-zonal structure. By situating cognition in this core biological architecture, the SMN model provides a biologically grounded account of cognition, consciousness, semiotics, language, and culture. 

Through model-based reasoning, we show: (1) how a cognitive agent develops self–world relations, (2) why such a model can support complex cognition at room temperature, (3) how representations, memory, naming, generative syntax, reasoning, and engineering competencies can be understood enactively, and (4) how long-established experimental results in cognitive neuroscience may be reinterpreted in light of this framework. We thus offer a path toward reconciling cognitivist and enactive perspectives.  

\bigskip
\noindent \textbf{Keywords:} cognitive architecture, enactivism, symbol grounding, generative syntax, phenomenology, semiotics, action patterns, Sensation-Modulating Network (SMN)
